\documentclass[a5paper]{article}
% Theorems, proofs, and other mathematical environments

\usepackage{amsthm}

\newtheoremstyle{mine} % name
  {7pt} % space above
  {5pt} % space below
  {} % body font
  {} % indent amount
  {\bfseries} % theorem head font
  {.} % punctuation after theorem head
  {.5em} % space after theorem head
  {} % theorem head spec (can be left empty, meaning no)

\theoremstyle{mine} % these are the non-italicized
\newtheorem{theorem}{Theorem}[section]
\newtheorem*{theorem*}{Theorem}
\newtheorem{lemma}[theorem]{Lemma}
\newtheorem*{lemma*}{Lemma}
\newtheorem{proposition}[theorem]{Proposition}
\newtheorem*{proposition*}{Proposition}
\newtheorem{corollary}[theorem]{Corollary}
\newtheorem*{corollary*}{Corollary}
\newtheorem{eqenv}[theorem]{Equation}
\newtheorem*{eqenv*}{Equation}
\newtheorem{remark}[theorem]{Remark}
\newtheorem*{remark*}{Remark}
\newtheorem{example}[theorem]{Example}
\newtheorem*{example*}{Example}
\newtheorem{definition}[theorem]{Definition}
\newtheorem*{definition*}{Definition}
% New environments. Syntax: \newenvironment{beforecommands}{aftercommands}
% environment for notecards. Portable accross tex files
\newenvironment{card}{\begin{framed}\begin{minipage}[t][3in][t]{5in}\noindent}{\end{minipage}\end{framed}}
\newenvironment{note}{\paragraph{Note:}}{}

% for the Anki notecard style
\newenvironment{field}{}{}
\newenvironment{conclusion}{}{}
\newenvironment{premises}{\begin{enumerate}[label=\alph*.]}{\end{enumerate}}
%% formatting

\usepackage{fontspec}
% EB Garamond (Initials - just for fun) - body text, old style, serif
% Nimbus Roman - serifed
% URW Gothic - geometric, title text
% Libre Caslon Text - absurdly well spaced
% FreeSerif - plays pretty nicely with math, looks like Times
% Liberation Serif - good alternative to FreeSerif, less TNR looking
% unclear whether to use Tinos or Liberation Serif. same design.
\setmainfont[Ligatures=TeX]{Tinos}

% date
\usepackage{polyglossia}
\usepackage[yyyymmdd]{datetime}
\renewcommand{\dateseparator}{--}

\usepackage{geometry}
\geometry{
  lmargin=4cm,
  rmargin=4cm,
  tmargin=3cm,
  bmargin=3cm,
}

\renewcommand{\arraystretch}{1.2} % Make tables a little bigger
% \usepackage{indentfirst}          % indent first \par after sections
\usepackage{parskip}              % don't indent anything
\usepackage{enumitem}             % enumerate with A), b., c) styles
\usepackage{hyperref}             % linked table of contents
\hypersetup{
    colorlinks,
    citecolor=black,
    filecolor=black,
    linkcolor=black,
    urlcolor=black
}

% graphics
\usepackage{graphicx}
\DeclareGraphicsExtensions{.pdf,.png,.jpg}
\usepackage{float} % image positioning

% semicolon is mapped to \ in my editor
\newcommand{\semicolon}{;}
%% math symbols, notation, etc
\usepackage{amssymb}       % math characters
\usepackage{mathtools}     % improvement on amsmath

% \usepackage{unicode-math}         % use OTF math fonts
% \setoperatorfont\mathup % operators should be set in math font
% \AtBeginDocument{\let\setminus\minus}

% \ExplSyntaxOn % fix xmapsto. https://github.com/wspr/unicode-math/issues/197
% \RenewDocumentCommand \xmapsto { m } {
%   \mathrel {
%     \exp_after:wN \Uoverdelimiter \cs:w sym \__um_symfont_tl \cs_end: "21A6 {
%       \mkern 5mu \scan_stop: #1 \mkern 5mu \scan_stop:
%     }
%   }
% }
% \ExplSyntaxOff

% \setmathfont[math-style=ISO]{Tex Gyre Schola Math}
% \setmathfont[range=\mathup]{Tex Gyre Schola Math}
% \setmathfont[range=\mathcal]{Latin Modern Math}
% \setmathfont[range=\setminus]{Asana Math}

% \setmathfont{Latin Modern Math}
% \setmathfont[range=\mathbb]{Tex Gyre Schola Math}
% \setmathfont[range=\setminus]{Asana Math}

% alternative characters
\AtBeginDocument{\let\epsilon\varepsilon}
\AtBeginDocument{\let\phi\varphi}
\AtBeginDocument{\let\oldemptyset\emptyset{}}
\AtBeginDocument{\let\emptyset\varnothing{}}

% paired delimiters
\DeclarePairedDelimiterX{\Abs}[1]{\lvert}{\rvert}{#1}
\DeclarePairedDelimiterX{\Angle}[1]{\langle}{\rangle}{#1}
\DeclarePairedDelimiterX{\Braces}[1]{\{}{\}}{#1}
\DeclarePairedDelimiterX{\Brackets}[1]{[}{]}{#1}
\DeclarePairedDelimiterX{\Ceiling}[1]{\lceil}{\rceil}{#1}
\DeclarePairedDelimiterX{\Floor}[1]{\lfloor}{\rfloor}{#1}
\DeclarePairedDelimiterX{\Norm}[1]{\lVert}{\rVert}{#1}
\DeclarePairedDelimiterX{\Paren}[1]{\lparen}{\rparen}{#1}

% swap the starred and unstarred versions of paired delimiters
\makeatletter % https://goo.gl/osSmHV
\def\abs{\@ifstar{\Abs}{\Abs*}}
\def\angles{\@ifstar{\Angle}{\Angle*}}
\def\braces{\@ifstar{\Braces}{\Braces*}}
\def\brackets{\@ifstar{\Brackets}{\Brackets*}}
\def\ceiling{\@ifstar{\Ceiling}{\Ceiling*}}
\def\floor{\@ifstar{\Floor}{\Floor*}}
\def\norm{\@ifstar{\Norm}{\Norm*}}
\def\paren{\@ifstar{\Paren}{\Paren*}}
\makeatother

% operators/functions
% \DeclareMathOperator{\Im}{Im}        % complex numbers: real part
% \DeclareMathOperator{\Re}{Re}        % complex numbers: imaginary part
\DeclareMathOperator{\Var}{Var}         % statistics: variance
\DeclareMathOperator{\ord}{ord}
\DeclareMathOperator{\image}{image}     % general: image
\DeclareMathOperator{\id}{id}           % general: identity function
\DeclareMathOperator{\lcm}{lcm}         % general: least common multiple
\DeclareMathOperator{\xor}{xor}         % general: exclusive or
\DeclareMathOperator{\Int}{Int}         % topology: interior
\DeclareMathOperator{\Ext}{Ext}         % topology: exterior
\DeclareMathOperator{\Lim}{Lim}
\DeclareMathOperator{\Bd}{Bd}           % topology: boundary
\DeclareMathOperator{\oball}{B}         % topology: open ball
\DeclareMathOperator{\cball}{\ov{B}}    % topology: closed ball
\DeclareMathOperator{\sphere}{S}        % topology: sphere
\DeclareMathOperator{\proj}{proj}
\DeclareMathOperator{\spanf}{span}      % linear: span
\DeclareMathOperator{\col}{Col}         % linear: column space
\DeclareMathOperator{\row}{Row}         % linear: row space
\DeclareMathOperator{\nul}{Nul}         % linear: null space
\DeclareMathOperator{\rank}{rank}       % linear: rank
\DeclareMathOperator{\determinant}{det} % linear: determinant
\DeclareMathOperator{\Alt}{Alt}         % linear: alternating group/square
\DeclareMathOperator{\Sym}{Sym}         % linear: symmetric square
\DeclareMathOperator{\Tr}{Trace}        % linear: trace of a matrix
\DeclareMathOperator{\divergence}{div}  % multi: divergence
\DeclareMathOperator{\curl}{curl}       % multi: curl
\DeclareMathOperator{\characteristic}{char} % algebra: ring/field characteristic
\DeclareMathOperator{\ann}{ann}         % algebra: the annihilator of a submodule
\DeclareMathOperator{\GL}{GL}           % algebra: general linear group
\DeclareMathOperator{\SL}{SL}           % algebra: special linear group
\DeclareMathOperator{\GF}{GF}           % algebra: galois field
\DeclareMathOperator{\Tor}{Tor}         % algebra: torsion elements/functor
\DeclareMathOperator{\Stab}{Stab}       % algebra: stabilizer
\DeclareMathOperator{\Orb}{Orb}         % algebra: orbit
\DeclareMathOperator{\Mat}{Mat}         % algebra: matrix ring
\DeclareMathOperator{\Res}{Res}         % algebra: restriction of a representation
\DeclareMathOperator{\Ind}{Ind}         % algebra: induced representation
\DeclareMathOperator{\im}{im}           % algebra: image
\DeclareMathOperator{\Aut}{Aut}         % algebra/cat: automorphisms
\DeclareMathOperator{\End}{End}         % algebra/cat: endomorphisms
\DeclareMathOperator{\Obj}{Obj}              % categories: objects of a category
\DeclareMathOperator{\Hom}{Hom}              % categories: hom-sets
\DeclareMathOperator{\Set}{\mathbf{Set}}     % categories: (small) sets
\DeclareMathOperator{\Cat}{\mathbf{Cat}}     % categories: small categories
\DeclareMathOperator{\Grpd}{\mathbf{Grpd}}   % categories: small groupoids
\DeclareMathOperator{\Mon}{\mathbf{Mon}}     % categories: monoids
\DeclareMathOperator{\Grp}{\mathbf{Grp}}     % categories: groups
\DeclareMathOperator{\AbGrp}{\mathbf{AbGrp}} % categories: abelian groups
\DeclareMathOperator{\Top}{\mathbf{Top}}     % categories: topological spaces
\DeclareMathOperator{\Man}{\mathbf{Man}}     % categories: topological manifolds
\DeclareMathOperator{\Mod}{\mathbf{-Mod}}     % categories: modules over a ring
\DeclareMathOperator{\Met}{\mathbf{Metric}}  % categories: metric spaces
\DeclareMathOperator{\Vect}{\mathbf{Vect}}   % categories: vector spaces
\DeclareMathOperator{\Rep}{\mathbf{Rep}}     % categories: representations

% make letters in different fonts (bold, blackboard, caligraphic, fraktur)
% https://tex.stackexchange.com/questions/156196/newcommand-with-variable-name
\newcommand{\mkltr}[1]{%
    \expandafter\newcommand\csname bf#1\endcsname{\ensuremath{\mathbf #1}}
    \expandafter\newcommand\csname bb#1\endcsname{\ensuremath{\mathbb #1}}
    \expandafter\newcommand\csname ca#1\endcsname{\ensuremath{\mathcal #1}}
    \expandafter\newcommand\csname fr#1\endcsname{\ensuremath{\mathfrak #1}}
}%
\mkltr A \mkltr B \mkltr C \mkltr D \mkltr E \mkltr F \mkltr G \mkltr H
\mkltr I \mkltr J \mkltr K \mkltr L \mkltr M \mkltr N \mkltr O \mkltr P
\mkltr Q \mkltr R \mkltr S \mkltr T \mkltr U \mkltr V \mkltr W \mkltr X
\mkltr Y \mkltr Z

\newcommand{\zn}{\mathbf{Z}_n}
\newcommand{\zp}{\mathbf{Z}_p}
\newcommand{\N}{\ensuremath{\mathbb{N}}}
\newcommand{\Z}{\ensuremath{\mathbb{Z}}}
\newcommand{\Q}{\ensuremath{\mathbb{Q}}}
\newcommand{\R}{\ensuremath{\mathbb{R}}}
\newcommand{\C}{\ensuremath{\mathbb{C}}}

% general stuff
\newcommand{\units}[1]{\;\mathrm{#1}}
\newcommand{\dd}{\,\mathrm{d}}
\newcommand{\ov}{\overline}

% set theory
\newcommand{\xlongmapsto}[1]{\xmapsto{\,\,\, #1\,\,\,}}
\newcommand{\xlongrightarrow}[1]{\xrightarrow{\,\,\, #1\,\,\,}}
\newcommand{\union}{\mathop{\bigcup}}
\newcommand{\disunion}{\mathop{\dot\bigcup}}
\newcommand{\intersection}{\mathop{\bigcap}}
\newcommand{\powerset}{\mathcal{P}}

% analysis & calculus
\usepackage{esint} % \oiint, etc
\newcommand{\evalat}[2]{{\bigg|_{#1}^{#2}}}
\newcommand{\ext}[1]{{#1}^{ext}}  % extension of function

% vectors
\newcommand{\ihat}{\hat{\imath}}
\newcommand{\jhat}{\hat{\jmath}}
\newcommand{\khat}{\hat{k}}
\newcommand{\lvec}[1]{\overrightarrow{#1}}
\newcommand{\len}[1]{\left\| #1 \right\|}

% topology
\newcommand{\RP}{\ensuremath{\R\hspace{-0.07em}\mathrm{P}}}

% algebra
\newcommand{\quotient}[2]{{\raisebox{.2em}{$#1$}\left/\raisebox{-.2em}{$#2$}\right.}}
\renewcommand{\S}{\mathfrak{S}}   % symmetric group
\newcommand{\ab}{\mathbf{ab}}     % algebra: abelianization

% statistics
% https://tex.stackexchange.com/questions/229023/expectation-operator#229029
% expectation and covariance
\DeclareMathOperator{\ExpOp}{Exp}
\DeclareMathOperator{\CovOp}{Cov}
\newcommand{\Exp}{\ExpOp\brackets*}
\newcommand{\Cov}{\CovOp\brackets*}
% \DeclareMathOperator{\refl}{\mathtt{refl}}
% \DeclareMathOperator{\ind}{\mathtt{ind}}

% one with more spacing. ideally, this would auto-adapt to surroundings
\renewcommand{\:}{\hspace{0.3ex}:\hspace{0.3ex}}
\newcommand{\lam}[2]{\lambda #1.#2}

% some of this is based on HoTT/HoTT

\newcommand{\emptytype}{\ensuremath{\mathbf{0}}}  % empty type
\newcommand{\unittype}{\ensuremath{\mathbf{1}}}   % unit type
\newcommand{\unitelem}{\ensuremath{1_{\unittype}}} % sole inhabitant of the unit type
\newcommand{\booltype}{\ensuremath{\mathbf{2}}}   % two element type
\newcommand{\btrue}{1_{\booltype}}                 % inhabitant of two element type
\newcommand{\bfalse}{0_{\booltype}}                % inhabitant of two element type
\newcommand{\inl}{\ensuremath{\mathsf{inl}}}
\newcommand{\inr}{\ensuremath{\mathsf{inr}}}
\newcommand{\universe}{\ensuremath{\mathcal{U}}}         % universe type
\newcommand{\universei}[1]{\ensuremath{\mathcal{U}_{#1}}}  % ith universe

\newcommand{\sigmat}[2]{\ensuremath{\sum_{\paren{#1}}#2}}    % Σ-type
\newcommand{\pr}[1]{\ensuremath{\mathsf{pr}_{#1}}}
\newcommand{\pit}[2]{\ensuremath{\prod_{\paren{#1}}#2}}      % Π-type
\newcommand{\refl}[1]{\ensuremath{\mathsf{refl}_{#1}}}      % reflexivity

%%% Natural numbers
\newcommand{\suc}{\mathsf{succ}}
\newcommand{\add}{\mathsf{add}}

%%% Lists
\newcommand{\List}[1]{\ensuremath{\mathsf{List}(#1)}}
\newcommand{\nil}{\ensuremath{\mathsf{nil}}}
\newcommand{\cons}{\ensuremath{\mathsf{cons}}}

%%% Function extensionality
\newcommand{\funext}{\mathsf{funext}}
\newcommand{\happly}{\mathsf{happly}}

\newcommand{\isContr}{\ensuremath{\mathsf{isContr}}}

%%% Relations
\newcommand{\jdeq}{\equiv}
\newcommand{\defeq}{\vcentcolon\jdeq}
\newcommand{\propeq}[3]{#2 =_{#1} #3}
\newcommand{\weq}[2]{#1 \simeq #2}
\newcommand{\homot}[2]{#1 \sim #2}

\newcommand{\apspace}{\hspace{0.3em}}
\newcommand{\apply}[2]{#1\apspace #2} % apply a function to its argument
\newcommand{\appply}[3]{#1\apspace #2 \apspace #3}
\newcommand{\appr}[2]{\apply{\pr{#1}}{#2}} % a combination of apply and pr

\newcommand{\ap}[2]{\mathsf{ap}_{#1}{\paren{#2}}} % action on paths
\newcommand{\ua}{\mathsf{ua}} % univalence axiom

%%% Transport (covariant) %%%
% \newcommand{\trans}[2]{\ensuremath{{#1}_{*}\mathopen{}\left({#2}\right)\mathclose{}}\xspace}
% \let\Trans\trans
% \newcommand{\transf}[1]{\ensuremath{{#1}_{*}}\xspace} % Without argument
\newcommand{\transpor}[2]{\ensuremath{\mathsf{transport}^{#1}\paren{#2}}}
\newcommand{\transport}[3]{\ensuremath{\mathsf{transport}^{#1}\paren{#2,#3}}}
\newcommand{\transportname}{\ensuremath{\mathsf{transport}}}
\newcounter{Problem} \stepcounter{Problem} % start at 1
\newcommand{\problem}[0]{
  \vspace{-0.8em}
  \begin{center}
    \huge
    \rule[0.25em]{0.3\textwidth}{0.6pt}
    \arabic{Problem}
    \rule[0.25em]{0.3\textwidth}{0.6pt}
  \end{center}
  \stepcounter{Problem}
  \vspace{0.5em}
}
\newcommand{\problemnum}[1]{
  \vspace{-0.5em}
  \begin{center}
    \huge
    \rule[0.25em]{0.3\textwidth}{0.5pt}
    #1
    \rule[0.25em]{0.3\textwidth}{0.5pt}
  \end{center}
  \vspace{0.5em}
}
\newcommand{\subproblem}[1]{
  \vspace{0.2em}
  \begin{center}
    \large
    \rule[0.25em]{0.2\textwidth}{0.5pt}
    #1
    \rule[0.25em]{0.2\textwidth}{0.5pt}
  \end{center}
  \vspace{0.2em}
}
% Combine the two with more pleasant vertical spacing
\newcommand{\problemsubproblem}[1]{\problem \vspace{-0.8em}\subproblem{#1}}
\newcommand{\problemnumsubproblem}[2]{\problemnum{#1}\vspace{-0.8em}\subproblem{#2}}
% set up our custom colors - these are based on "tao-yang"
\usepackage{color}
\usepackage[dvipsnames]{xcolor}
% text and background colors
\definecolor{bg}{HTML}{F1F1F1}
\definecolor{fg}{HTML}{090909}
\pagecolor{bg}
\color{fg}
\geometry{
  lmargin=1.5cm,
  rmargin=1.5cm,
  tmargin=2cm,
  bmargin=2cm
}

\usepackage{tikz}
\usetikzlibrary{cd}

\usepackage{stmaryrd}

\usepackage{titling}
\usepackage{fancyhdr}
\pagestyle{fancy}
\fancyhead[L]{\theauthor}
\fancyhead[C]{\thetitle}
\fancyhead[R]{\thedate}

\newcommand{\wt}{\widetilde}

\begin{document}
\title{Hatcher: Covering Spaces}
\author{Langston Barrett}
\date{Fall 2017}

\begin{definition*}
	A \textbf{covering space} of a topological space $(X,\tau)$ consists of the
  following data:
  \begin{itemize}
    \itemsep0em
    \item A space $(\widetilde{X},\widetilde{\tau})$,
    \item a continuous map $p:\widetilde{X}\to X$,
    \item and an open cover $\braces{U_\alpha}_{\alpha\in A}$ of $X$
  \end{itemize}
  subject to the condition that for all $\alpha\in A$, the preimage
  $p^{-1}(U_\alpha)$ is a disjoint union of open sets in $\widetilde{X}$, each of
  which is mapped homeomorphically onto $U_\alpha$ by $p$.
\end{definition*}
\problemnum{1}
\begin{proposition*}
	Let $\wt{X}$ be a covering space of $X$ with map $p:\widetilde{X}\to X$ and
  open cover $\braces{U_\alpha}_{\alpha\in A}$. If $A$ is a subspace of $X$, then
  $p^{-1}(A)$ is a covering space of $A$ with map $q:p^{-1}(A)\to A$
  the restriction of $p$ and cover
  $\braces{U_\beta}\coloneqq \braces{A\cap U_\alpha}_{\alpha\in A}$.
\end{proposition*}
\begin{proof}
  Since $\braces{U_\alpha}$ covers $X$, it certainly covers $A$. Since the
  intersection of an open set with $A$ is open in the subspace topology,
  $\braces{U_\beta}$ is an open cover of $A$.

  Consider the preimage $q^{-1}(U_\beta)$ for some $\beta$.
  Since the restriction of continuous maps to a subspace is continuous,
  $q^{-1}$ is continuous. Thus, $q^{-1}(U_\beta)$ is the disjoint union of open
  sets in $p^{-1}(A)$ and each component is mapped homeomorphically onto
  $U_\beta$ by $q$. We have given the required data and shown that it follows
  the conditions, so $p^{-1}(A)$ is a covering space of $A$.
\end{proof}

\problemnum{2}

\begin{lemma*}
  The following is a basis for a disjoint union $\coprod_i B_i$:
  \begin{equation*}
    \caB \coloneqq \braces{\iota_j(U)|U\text{ is open in }B_j}
  \end{equation*}
  where $\iota_j:B_j\hookrightarrow \coprod_i B_i$ is the canonical inclusion.
\end{lemma*}
\begin{proof}
	This is a restatement of the universal property of the coproduct.
\end{proof}

\begin{lemma*}
  In $\Top$, finite products commute with arbitrary coproducts. Namely for all
  spaces $A$ and a disjoint union $\coprod_i B_i$,
  \begin{equation*}
    A\times \coprod_i B_i\cong \coprod_i (A\times B_i)
  \end{equation*}
\end{lemma*}
\begin{proof}
  Recall that the coproduct is an indexed union, a set of pairs where the first
  element of the pair is a ``tag'' indicating which set that item originated from.
  \begin{align*}
    \phi:A\times\coprod_i B_i &\longrightarrow \coprod_i (A\times B_i) \\
    (a,(i,b))&\longmapsto (i,(a,b)) \\
  \intertext{with inverse}
    A\times\coprod_i B_i &\longleftarrow \coprod_i (A\times B_i):\phi^{-1} \\
    (a,(i,b))&\longmapsfrom (i,(a,b)) 
  \end{align*}
  To prove continuity of a map into a space, it suffices to prove that the
  inverse images of basis elements are open. As we have bases for both the
  product and coproduct topologies, it suffices to show that 
  $\phi^{-1}(j,U\times V)$ is open where $U$ is open in $A$ and $V$ is open in
  $B_j$. From the definition of $\phi$ and $\phi^{-1}$, we obtain
  \begin{equation*}
    \phi^{-1}(j,U\times V) = (U,(j,V))
  \end{equation*}
  which is open by definition of the product topology. A similar argument shows
  that $\phi^{-1}$ is continuous.
\end{proof}

\begin{proposition*}
	If $\tilde{X}$, $p:\tilde{X}\to X$, $\braces{U_\alpha}$ and $\tilde{Y}$,
  $q:\tilde{Y}\to Y$, $\braces{V_\beta}$ are covering spaces, then the product
  $\tilde{X}\times\tilde{Y}$ covers $X\times Y$.
\end{proposition*}
\begin{proof}
  If $(x,y)\in X\times Y$, then there exist $U_\alpha\ni x$ and $V_\beta\ni y$
  in the open covers of $X$ and $Y$. Their product $U_\alpha\times V_\beta$
  covers $(x,y)$ and is open in the product topology. Therefore, the pairwise
  cartesian product of the open covers on $X$ and $Y$ defines an open cover on
  $X\times Y$.

  Let $\langle p,q \rangle:\widetilde{X}\times \widetilde{Y}\to X\times Y$ be
  the product of maps $p$ and $q$, and consider the preimage
  \begin{align*}
    \langle p,q \rangle^{-1}(U_\alpha\times V_\beta)
    &= \braces{(x,y)\in \wt{X}\times\wt{Y}\Big|\langle p,q \rangle(x,y)\in U_\alpha\times V_\beta} \\
    &= \braces{(x,y)\in \wt{X}\times\wt{Y}\Big|(p(x),q(y))\in U_\alpha\times V_\beta} \\
    &= \braces{(x,y)\in \wt{X}\times\wt{Y}\Big|p(x)\in U_\alpha\text{ and }q(y)\in V_\beta} \\
    &= p^{-1}(U_\alpha)\times q^{-1}(V_\beta)
  \end{align*}
  By continuity of $p$ and $q$, this is an open set in the product topology on
  $\wt{X}\times \wt{Y}$.
  It remains to show that it is the disjoint union of sets
  mapped homeomorphically onto $U_\alpha\times V_\beta$ by $\langle p,q \rangle$.
  A slight generalization of the above lemma demonstrates that the inverse image
  is indeed a disjoint union of open sets. The definition of
  $\langle p,q \rangle$ ensures that the mapping is homeomorphic.
\end{proof}

\problemnum{3}

% \begin{lemma*}
% 	If $\wt{X}$ covers $X$ via $p:\wt{X}\to X$, then $p$ is an open map.
% \end{lemma*}
% \begin{proof}
% 	Let $\wt{U}\subseteq\wt{X}$ be open. By the definition of a covering space,
%   there is a distinguished open cover $\braces{U_\alpha}$ of $X$. Then
%   $p(\wt{U})$ is the subset of a union of some of these covering sets.
%   Additionally, the intersection
% \end{proof}

\begin{proposition*}
	Let $\wt{X}$ be a covering space with map $\wt{X}\to X$ and suppose
  $p^{-1}(x)$ is finite and nonempty for all $x\in X$. Then $X$ is compact
  Hausdorff if and only if $\wt{X}$ is.
\end{proposition*}
\begin{proof}
	($\Rightarrow$ Hausdorff) Assume $X$ is Hausdorff. Let $\wt{x},\wt{y}$ be
  distinct points in $\wt{X}$ and let $x\coloneqq p(\wt{x})$
  and $y\coloneqq p(\wt{y})$. Since $X$ is Hausdorff, there exist non-overlapping
  neighborhoods $U$ of $x$ and $V$ of $y$. Call the intersections of these
  neighborhoods with the open cover guaranteed by the covering space $U'$ and
  $V'$, respectively. Then $p^{-1}(U')$ and $p^{-1}(V')$ are disjoint unions of
  open and distinct sets in $\wt{X}$, and contain $\wt{x}$ and $\wt{y}$
  respectively.

	($\Leftarrow$ Hausdorff) Assume $\wt{X}$ is Hausdorff. Let $x,y$ be
  distinct points in $X$. By the nonemptiness assumption, there exist
  distinct $\wt{x},\wt{y}\in\wt{X}$ such that $p(\wt{x})=x$ and $p(\wt{y})=y$.
  By the axioms of covering spaces, $p^{-1}(x)$ and $p^{-1}(y)$ are the disjoint
  unions of open sets. Let $\wt{U}_x,\wt{U}_y$ be the open sets containing
  $\wt{x}$ and $\wt{y}$, respectively. Since $\wt{X}$ is Hausdorff, there exist
  non-overlapping open sets $\wt{U}_x'$ and $\wt{U}_y'$ contained within
  $\wt{U}_x$ and $\wt{U}_y$, respectively. Since homeomorphisms are open maps,
  $p(\wt{U}_x')$ and $q(\wt{U}_y')$ are open in $X$ and contain $x$ and $y$,
  respectively. 
\end{proof}

\end{document}