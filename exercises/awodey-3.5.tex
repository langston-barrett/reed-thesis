\documentclass[a5paper]{article}
% Theorems, proofs, and other mathematical environments

\usepackage{amsthm}

\newtheoremstyle{mine} % name
  {7pt} % space above
  {7pt} % space below
  {} % body font
  {} % indent amount
  {\bfseries} % theorem head font
  {.} % punctuation after theorem head
  {.5em} % space after theorem head
  {} % theorem head spec (can be left empty, meaning no)

\theoremstyle{mine} % these are the non-italicized
\newtheorem{theorem}{Theorem}[subsection]
\newtheorem*{theorem*}{Theorem}
\newtheorem{lemma}[theorem]{Lemma}
\newtheorem*{lemma*}{Lemma}
\newtheorem{proposition}[theorem]{Proposition}
\newtheorem*{proposition*}{Proposition}
\newtheorem{corollary}[theorem]{Corollary}
\newtheorem*{corollary*}{Corollary}
\newtheorem{eqenv}[theorem]{Equation}
\newtheorem*{eqenv*}{Equation}
\newtheorem{remark}[theorem]{Remark}
\newtheorem*{remark*}{Remark}
\newtheorem{example}[theorem]{Example}
\newtheorem*{example*}{Example}
\newtheorem{definition}[theorem]{Definition}
\newtheorem*{definition*}{Definition}
% New environments. Syntax: \newenvironment{beforecommands}{aftercommands}
% environment for notecards. Portable accross tex files
\newenvironment{card}{\begin{framed}\begin{minipage}[t][3in][t]{5in}\noindent}{\end{minipage}\end{framed}}
\newenvironment{note}{\paragraph{Note:}}{}

% for the Anki notecard style
\newenvironment{field}{}{}
\newenvironment{conclusion}{}{}
\newenvironment{premises}{\begin{enumerate}[label=\alph*.]}{\end{enumerate}}
%% formatting

\usepackage{fontspec}
% EB Garamond (Initials - just for fun) - body text, old style, serif
% Nimbus Roman - serifed
% URW Gothic - geometric, title text
% Libre Caslon Text - absurdly well spaced
% FreeSerif - plays pretty nicely with math, looks like Times
% Tinos - good alternative to FreeSerif, less TNR looking
% \setmainfont[Ligatures=TeX]{Tinos}
% \fontspec [Path = ../tex-preamble/fonts/, 
%            UprightFont = *-regular,
%            BoldFont    = *-bold,
%            ItalicFont  = *-italic]
%            {texgyrepagella}
\setmainfont[ Path=../tex-preamble/fonts/
            , UprightFont     = *-regular
            , BoldFont        = *-bold
            , ItalicFont      = *-italic
            , BoldItalicFont  = *-bolditalic
            ]{texgyrepagella}

\usepackage{datetime2} % use yyyy--mm--dd date with \DTMtoday
\usepackage{geometry}
\geometry{
  lmargin=4cm,
  rmargin=4cm,
  tmargin=3cm,
  bmargin=3cm,
}

\renewcommand{\arraystretch}{1.2} % Make tables a little bigger
% \usepackage{indentfirst}          % indent first \par after sections
% \usepackage{parskip}              % don't indent anything
\usepackage{enumitem}             % enumerate with A), b., c) styles
\usepackage{hyperref}             % linked table of contents
\hypersetup{
    colorlinks,
    citecolor=black,
    filecolor=black,
    linkcolor=black,
    urlcolor=black
}
\usepackage[noabbrev, capitalize]{cleveref}
\crefname{notation}{Notation}{Notations}
\Crefname{notation}{Notation}{Notations}
\crefname{rule}{Rule}{Rule}
\Crefname{rule}{Rule}{Rule}

% graphics
\usepackage{graphicx}
\DeclareGraphicsExtensions{.pdf,.png,.jpg}
\usepackage{float} % image positioning
%% math symbols, notation, etc
\usepackage{amssymb}       % math characters
\usepackage{mathtools}     % improvement on amsmath

% Use better math fonts when available
\IfFileExists{../tex-preamble/fonts/latinmodern-math.otf}
  {\newcommand{\fontpath}{../tex-preamble/fonts}}
  {\newcommand{\fontpath}{../../tex-preamble/fonts}}

\IfFileExists{\fontpath}{
  \usepackage{unicode-math}         % use OTF math fonts
  \AtBeginDocument{\let\phi\varphi} % varphi with unicode-math
  %\AtBeginDocument{\let\epsilon\varepsilon}
  % \setmathfont[Path=\fontpath]{latinmodern-math.otf}
  \setmathfont[Path=\fontpath]{Asana-Math.otf}
  \setmathfont[Path=\fontpath,range=\setminus]{Asana-Math.otf}
  \setmathfont[Path=\fontpath,range=\mathbb]{texgyretermes-math.otf}
  \setmathfont[Path=\fontpath,range=\mathcal]{latinmodern-math.otf}

  % fix xmapsto. only works with text above.
  % https://github.com/wspr/unicode-math/issues/197
  \ExplSyntaxOn
  \RenewDocumentCommand \xmapsto { m } {
    \mathrel {
      \exp_after:wN \Uoverdelimiter \cs:w sym \__um_symfont_tl \cs_end: "21A6 {
        \mkern 5mu \scan_stop: #1 \mkern 5mu \scan_stop:
      }
    }
  }
  \ExplSyntaxOff
}{}

% general stuff
\newcommand{\units}[1]{\;\mathrm{#1}}
\newcommand{\dd}{\,\mathrm{d}}
\newcommand{\ov}{\overline}
\newcommand{\paren}[1]{\left(#1\right)}
\newcommand{\braces}[1]{\left\{#1\right\}}
\newcommand{\brackets}[1]{\left[#1\right]}
\newcommand{\ceiling}[1]{\left\lceil{} #1 \right\rceil}
\newcommand{\floor}[1]{\left\lfloor{} #1 \right\rfloor}
\newcommand{\abs}[1]{\left| #1 \right|}
\newcommand{\e}[1]{\cdot 10^{#1}}
% \newcommand{\emf}{\mathcal{E}}
\newcommand{\contradiction}{\Rightarrow\Leftarrow}
\newcommand{\inv}[1]{\frac{1}{#1}}
\newcommand{\tif}{\text{ if }}
\newcommand{\totherwise}{\text{ otherwise }}
% \newcommand{\mydef}{\textbf{Definition:}}
% complex numbers
\renewcommand{\Im}{\operatorname{Im}}
\renewcommand{\Re}{\operatorname{Re}}

% operators/functions
\DeclareMathOperator{\ord}{ord}
% \DeclareMathOperator{\image}{image}
\DeclareMathOperator{\Int}{Int}
\DeclareMathOperator{\Ext}{Ext}
\DeclareMathOperator{\Lim}{Lim}
\DeclareMathOperator{\Bd}{Bd}
\DeclareMathOperator{\lcm}{lcm}
\DeclareMathOperator{\proj}{proj}
\DeclareMathOperator{\xor}{xor}
\DeclareMathOperator{\spanf}{span}
\DeclareMathOperator{\id}{id}
% linear
\DeclareMathOperator{\col}{Col}
\DeclareMathOperator{\row}{Row}
\DeclareMathOperator{\nul}{Nul}
\DeclareMathOperator{\rank}{rank}

% set theory
\newcommand{\xlongmapsto}[1]{\xmapsto{\,\,\, #1\,\,\,}}
\newcommand{\xlongrightarrow}[1]{\xrightarrow{\,\,\, #1\,\,\,}}
\newcommand{\union}{\mathop{\bigcup}}
\newcommand{\disunion}{\mathop{\dot\bigcup}}
\newcommand{\intersection}{\mathop{\bigcap}}
\newcommand{\powerset}{\mathcal{P}}

% analysis & calculus
\usepackage{esint} % \oiint, etc
\newcommand{\evalat}[2]{{\bigg|_{#1}^{#2}}}
\newcommand{\ext}[1]{{#1}^{ext}}  % extension of function
\DeclareMathOperator{\determinant}{det}
\DeclareMathOperator{\oball}{B}
\DeclareMathOperator{\cball}{\ov{B}}
\DeclareMathOperator{\sphere}{S}
\DeclareMathOperator{\divergence}{div}
\DeclareMathOperator{\curl}{curl}

% vectors
\newcommand{\ihat}{\hat{\imath}}
\newcommand{\jhat}{\hat{\jmath}}
\newcommand{\khat}{\hat{k}}
\newcommand{\lvec}[1]{\overrightarrow{#1}}
\newcommand{\len}[1]{\left\| #1 \right\|}

% topology
\newcommand{\RP}{\ensuremath{\R\hspace{-0.07em}\mathrm{P}}}

% categories
\newcommand{\I}{\ensuremath{\mathbf{I}}} % "interval groupoid"
\newcommand{\Cat}{\ensuremath{\mathcal{C}\mathrm{at}}}
\newcommand{\ComGrp}{\ensuremath{\mathcal{C}\mathrm{om}\mathcal{G}\mathrm{rp}}}
\newcommand{\AbGrp}{\ensuremath{\mathcal{A}\mathrm{b}\mathcal{G}\mathrm{rp}}}
\newcommand{\Grp}{\ensuremath{\mathcal{G}\mathrm{rp}}}
\newcommand{\Grpd}{\ensuremath{\mathcal{G}\mathrm{rpd}}}
\newcommand{\Mod}{\ensuremath{\mathcal{M}\mathrm{od}}}
\newcommand{\Man}{\ensuremath{\mathcal{M}\mathrm{an}}}
\newcommand{\Metric}{\ensuremath{\mathcal{M}\mathrm{etric}}}
\newcommand{\Set}{\ensuremath{\mathcal{S}\mathrm{et}}}
\newcommand{\Top}{\ensuremath{\mathcal{T}\!\mathrm{op}}}
\newcommand{\homoTop}{\ensuremath{\mathrm{homo}\mathcal{T}\mathrm{op}}}
\newcommand{\Vect}{\ensuremath{\mathcal{V}\mathrm{ect}}}
\renewcommand{\phi}{\varphi}

% number theory
\newcommand{\modulus}[1]{\; \left(\mathrm{mod}\; #1\right)}
\newcommand{\crash}[2]{\left(\frac{#1}{#2}\right)}
\newcommand{\QED}{\begin{flushright}QED\end{flushright}}

% make letters in different fonts (bold, blackboard, caligraphic, fraktur)
% https://tex.stackexchange.com/questions/156196/newcommand-with-variable-name
\newcommand{\mkltr}[1]{%
    \expandafter\newcommand\csname bf#1\endcsname{\ensuremath{\mathbb #1}}
    \expandafter\newcommand\csname bb#1\endcsname{\ensuremath{\mathbb #1}}
    \expandafter\newcommand\csname ca#1\endcsname{\ensuremath{\mathcal #1}}
    \expandafter\newcommand\csname fr#1\endcsname{\ensuremath{\mathfrak #1}}
}%
\mkltr A \mkltr B \mkltr C \mkltr D \mkltr E \mkltr F \mkltr G \mkltr H
\mkltr I \mkltr J \mkltr K \mkltr L \mkltr M \mkltr N \mkltr O \mkltr P
\mkltr Q \mkltr R \mkltr S \mkltr T \mkltr U \mkltr V \mkltr W \mkltr X
\mkltr Y \mkltr Z

\newcommand{\zn}{\mathbf{Z}_n}
\newcommand{\zp}{\mathbf{Z}_p}
\newcommand{\N}{\ensuremath{\mathbf{N}}}
\newcommand{\Z}{\ensuremath{\mathbf{Z}}}
\newcommand{\Q}{\ensuremath{\mathbf{Q}}}
\newcommand{\R}{\ensuremath{\mathbf{R}}}
\newcommand{\C}{\ensuremath{\mathbf{C}}}

% algebra
\newcommand{\quotient}[2]{{\raisebox{.2em}{$#1$}\left/\raisebox{-.2em}{$#2$}\right.}}
\newcommand{\GL}{\mathrm{GL}}     % general linear group
\newcommand{\SL}{\mathrm{SL}}     % special linear group
\renewcommand{\S}{\mathfrak{S}}   % symmetric group
\DeclareMathOperator{\ann}{ann}   % the annihilator of a submodule
\DeclareMathOperator{\Obj}{Obj}   % objects of a category
\DeclareMathOperator{\Hom}{Hom}   % homomorphisms between modules
\DeclareMathOperator{\Mor}{Mor}   % categorical morphisms
\DeclareMathOperator{\characteristic}{char} % ring characteristic
\DeclareMathOperator{\Tor}{Tor}   % torsion elements
\DeclareMathOperator{\Stab}{Stab} % stabilizer
\DeclareMathOperator{\Orb}{Orb}   % orbit
\DeclareMathOperator{\Aut}{Aut}   % automorphisms
\DeclareMathOperator{\End}{End}   % endomorphisms
\DeclareMathOperator{\Mat}{Mat}   % matrix ring
\DeclareMathOperator{\Tr}{Trace}  % trace of a matrix
\DeclareMathOperator{\im}{im}     % image

% statistics
\DeclareMathOperator{\Var}{Var}

% nicer empty set
\let\oldemptyset\emptyset{}
\let\emptyset\varnothing{}
\newcounter{Problem} \stepcounter{Problem} % start at 1
\newcommand{\problem}[0]{
  \vspace{-0.8em}
  \begin{center}
    \huge
    \rule[0.25em]{0.3\textwidth}{0.6pt}
    \arabic{Problem}
    \rule[0.25em]{0.3\textwidth}{0.6pt}
  \end{center}
  \stepcounter{Problem}
  \vspace{0.5em}
}
\newcommand{\problemnum}[1]{
  \vspace{-0.5em}
  \begin{center}
    \huge
    \rule[0.25em]{0.3\textwidth}{0.5pt}
    #1
    \rule[0.25em]{0.3\textwidth}{0.5pt}
  \end{center}
  \vspace{0.5em}
}
\newcommand{\subproblem}[1]{
  \vspace{0.2em}
  \begin{center}
    \large
    \rule[0.25em]{0.2\textwidth}{0.5pt}
    #1
    \rule[0.25em]{0.2\textwidth}{0.5pt}
  \end{center}
  \vspace{0.2em}
}
% Combine the two with more pleasant vertical spacing
\newcommand{\problemsubproblem}[1]{\problem \vspace{-0.8em}\subproblem{#1}}
\newcommand{\problemnumsubproblem}[2]{\problemnum{#1}\vspace{-0.8em}\subproblem{#2}}
% set up our custom colors - these are based on "tao-yang"
\usepackage{color}
\usepackage[dvipsnames]{xcolor}
% text and background colors
\definecolor{bg}{HTML}{F1F1F1}
\definecolor{fg}{HTML}{090909}
\pagecolor{bg}
\color{fg}
\geometry{
  lmargin=1.5cm,
  rmargin=1.5cm,
  tmargin=2cm,
  bmargin=2cm
}

\usepackage{tikz}
\usetikzlibrary{cd}
\usetikzlibrary{arrows,decorations.markings}

\usepackage{stmaryrd}

\usepackage{titling}
\usepackage{fancyhdr}
\pagestyle{fancy}
\fancyhead[L]{\theauthor}
\fancyhead[C]{\thetitle}
\fancyhead[R]{\thedate}

\newcommand{\wt}{\widetilde}

\begin{document}
\title{Awodey: Duality}
\author{Langston Barrett}
\date{Fall 2017}

\problemnum{2}
\begin{theorem*}
	The free monoid functor preserves binary coproducts.
  Specifically, for any sets $A,B\in\Obj(\Set)$, if $\caF:\Set\to\Mon$ denotes
  the free monoid functor, there is an isomorphism
  \begin{equation*}
    \caF(A)+\caF(B)\cong \caF(A+B)
  \end{equation*}
\end{theorem*}
\begin{proof}
  Let $f_A,f_B$ be given and label the inclusions into coproducts $i_A,i_B$ and
  the inclusions of generators $j_A,j_B,j_{A+B}$ as in the following diagram:
  \begin{center}
    \begin{tikzcd}[sep=large, row sep=large, column sep=large] 
      A \arrow[r, tail, "j_A"]
        \arrow[dr, tail, "i_A"]
        &
        \caF(A) \arrow[drr, Rightarrow, bend left=15, "f_A"]
        & {} & {}\\
      {} & A+B \arrow[r, tail, "j_{A+B}"] & {} \caF(A+B) & N\\
      B \arrow[r, tail, "j_B"]
        \arrow[ur, tail, "i_B"]
        & \caF(B) \arrow[urr, Rightarrow, bend right=15, "f_B"]
        & {} & {}
    \end{tikzcd}
  \end{center}
  Where double arrows denote arrows in $\Mon$, rather than just in $\Set$.
  The UMP of the free monoid guarantees monoid homomorphisms
  \begin{align*}
    k_A:\caF(A)\rightarrowtail \caF(A+B) &&\text{induced by}&&j_{A+B}\circ i_A \\
    k_B:\caF(B)\rightarrowtail \caF(A+B) &&\text{induced by}&&j_{A+B}\circ i_B
  \end{align*}
  such that
  \begin{align}
    |k_A| \circ j_A &= j_{A+B}\circ i_A \label{eq:ka} \\
    |k_B| \circ j_B &= j_{A+B}\circ i_B \label{eq:kb}
  \end{align}
  uniquely. They are monos because they are equal as set maps to the composition
  of two monos. The maps $|f_A|\circ j_A:A\to N$ and $|f_B|\circ j_B:B\to N$
  induce a set map $g:A+B\to N$ such that
  \begin{align}
    g\circ i_A &= |f_A|\circ j_A \label{eq:ga} \\
    g\circ i_B &= |f_B|\circ j_B \label{eq:gb}
  \end{align}
  So far, we have that the following diagram commutes:
  \begin{center}
    \begin{tikzcd}[sep=large, row sep=large, column sep=large] 
      A \arrow[r, tail, "j_A"]
        \arrow[dr, tail, "i_A"]
        &
        \caF(A) \arrow[drr, Rightarrow, bend left=15, "f_A"]
                \arrow[dr, dashed, Rightarrow, "k_A"]
        & {} & {}\\
        {} & A+B \arrow[r, tail, "j_{A+B}"] \arrow[drru, dashed, bend right=20, "g"]
        & \caF(A+B) % \arrow[r,dashed,Rightarrow, "h"]
        & N\\
      B \arrow[r, tail, "j_B"]
        \arrow[ur, tail, "i_B"]
        & \caF(B) \arrow[urr, Rightarrow, bend right=15, "f_B"]
                  \arrow[ur, dashed, Rightarrow, "k_B"]
        & {} & {}
    \end{tikzcd}
  \end{center}
  The UMP of the free monoid produces a monoid homomorphism $h:\caF(A+B)\to N$
  such that
  \begin{equation}\label{eq:umph}
    |h|\circ j_{A+B}=g
  \end{equation}
  It remains to show that
  \begin{align*}
    h\circ k_A = f_A &&\text{and}&& h\circ k_B = f_B
  \end{align*}
  We have
  \begin{align*}
    |h|\circ |k_A|\circ j_A
    &= |h|\circ j_{A+B}\circ i_A &&\quad\text{Equation \eqref{eq:ka}}\\
    &= g\circ i_A &&\quad\text{Equation \eqref{eq:umph}} \\
    &= |f_A|\circ j_A &&\quad\text{Equation \eqref{eq:ga}}
  \end{align*}
  and
  \begin{align*}
    |h|\circ |k_B|\circ j_B
    &= |h|\circ j_{A+B}\circ i_B &&\quad\text{Equation \eqref{eq:kb}}\\
    &= g\circ i_B &&\quad\text{Equation \eqref{eq:umph}} \\
    &= |f_B|\circ j_B &&\quad\text{Equation \eqref{eq:gb}}
  \end{align*}
  Which, combined with the uniqueness conditions of the respective UMPs, gives
  the required equality.
\end{proof}

\problemnum{6}

\begin{theorem*}
	The category of monoids ($\Mon$) has all equalizers.
\end{theorem*}
\begin{proof}
	(Equalizers) Let $A,B\in\Obj(\Mon)$ and
  \begin{tikzcd}[cramped,sep=small] f_1,f_2:A
    \arrow[r,yshift=2pt]\arrow[r,yshift=-2pt] & B \end{tikzcd}. We need to
  give a monoid $E$ together with an arrow $e:E\rightarrowtail A$ satisfying
  the UMP. As in $\Set$, the set $E$ is the subset of $A$ where $f_1$ and $f_2$
  agree, and $e$ is the inclusion.
  \begin{equation*}
    E\coloneqq\braces{a\in A|f_1(a)=f_2(a)}
  \end{equation*}
  By definition, $f_1\circ e=f_2\circ e$. We need to show that this is in fact
  an object of $\Mon$, i.e.\ a submonoid of $A$. Let $x,y\in E$, meaning
  $f_1(x)=f_2(x)$ and $f_1(y)=f_2(y)$.
  \begin{align*}
    f_1(xy)
    &= f_1(x)f_1(y)
    &&\quad f_1\text{ is a homomorphism} \\
    &= f_2(x)f_2(y)
    &&\quad \text{Assumption on }x,y \\
    &= f_2(xy)
    &&\quad f_2\text{ is a homomorphism}
  \end{align*}
  Additionally, $E$ is non-empty, since by the definition of a monoid morphism,
  $f_1(u_A)=f_2(u_A)=u_B$ where $u_A\in A,u_B\in B$ are the identities. It
  remains to show that any other map that equalizes $f_1$ and $f_2$ factors
  through $e$. Let $z:Z\to A$ such that $f_1\circ z=f_2\circ z$ and let
  $x\in Z$. Then $f_1(z(x))=f_2(z(x))$, so $z(x)\in E$ (more specifically, there
  is some $x'\in E$ such that $e(x')=z(x)$). Thus, $Z$ is a submonoid of $E$ and
  $z$ factors through $e$ via the inclusion $Z\rightarrowtail E$.
\end{proof}

\problemnum{7}

\begin{theorem*}
	The coproduct of projective objects is projective. 
\end{theorem*}
\begin{proof}
  Let $\bfC$ be a category with binary coproducts and let $P_1,P_2\in\Obj(\bfC)$
  be projective. Let $P_1+P_2$ denote their coproduct, and assume as given
  $f:P_1+P_2\to X$ and $e:E\twoheadrightarrow X$. By the UMP of the coproduct,
  we have
  \begin{align*}
    f_1:P_1&\longrightarrow X  &&\text{such that}&& f\circ i_1 = f_1 \\
    f_2:P_2&\longrightarrow X  &&\text{such that}&& f\circ i_2 = f_2 
  \end{align*}
  and since $P_1$ and $P_2$ are projective, we get
  \begin{align*}
    \ov{f_1}:P_1\longrightarrow E &&\text{such that}&& e\circ \ov{f_1} = f_1 \\
    \ov{f_2}:P_2\longrightarrow E &&\text{such that}&& e\circ \ov{f_2} = f_2
  \end{align*}
  which, by the UMP of the coproduct, yield
  \begin{align*}
    \ov{f}:P_1+P_2\longrightarrow E
  \end{align*}
  such that $\ov{f}\circ i_1=\ov{f_1}$ and $\ov{f}\circ i_2=\ov{f_2}$. All taken
  together, the following diagram commutes:
  \begin{center}
    \begin{tikzcd}[row sep=large]
      {} & X & {}\\
      {} & E \arrow[u, twoheadrightarrow, "e"]& \\
      P_1\arrow[r, "i_1", swap] \arrow[ur,dashed, "\ov{f_1}"]
         \arrow[uur, dashed, "f_1", bend left=15] %
      & P_1+P_2 \arrow[u, "\ov{f}"] &
      P_2\arrow[l, "i_2"]\arrow[ul,dashed, "\ov{f_2}", swap]
         \arrow[uul, dashed, "f_2", swap, bend right=15]
    \end{tikzcd}
  \end{center}
  Reading off the diagram,
  \begin{align*}
    \ov{f} \circ i_1 = \ov{f_1}
    &\implies (e\circ\ov{f})\circ i_1 = e\circ\ov{f_1}
    &&\quad\text{Left composition with }e \\
    &\implies (e\circ\ov{f})\circ i_1 = f_1
    &&\quad\text{Mapping property of }\ov{f_1} \\
    \ov{f} \circ i_2 = \ov{f_2}
    &\implies (e\circ\ov{f})\circ i_2 = e\circ\ov{f_2}
    &&\quad\text{Left composition with }e \\
    &\implies (e\circ\ov{f})\circ i_2 = f_2
    &&\quad\text{Mapping property of }\ov{f_2}
  \end{align*}
  However, by the UMP of the coproduct, $f$ is the unique map making the above
  equations commute.
  Thus, $e\circ\ov{f}=f$, which is exactly the mapping property of a
  projective.
\end{proof}

\end{document}