\documentclass{beamer}

%%% Packages
% \usepackage{hyperref}
\usepackage{./metropolis/beamerthememetropolis}

%%% Fonts
\usefonttheme{professionalfonts} % using non standard fonts for beamer
\usepackage{fontspec}
\usepackage[math]{iwona}
\setmainfont[Ligatures=TeX]{Open Sans}
\setmonofont[Scale=0.9]{DejaVu Sans Mono}
\usepackage{prftree}

%%% BibLaTeX with Biber
\usepackage[backend=bibtex, citestyle=authoryear-comp]{biblatex}
\addbibresource{references.bib}
\renewcommand*{\bibfont}{\footnotesize}
\nocite{*}

% Other packages
%% math symbols, notation, etc
\usepackage{amssymb}       % math characters
\usepackage{mathtools}     % improvement on amsmath

% Use better math fonts when available
\IfFileExists{../tex-preamble/fonts/latinmodern-math.otf}
  {\newcommand{\fontpath}{../tex-preamble/fonts}}
  {\newcommand{\fontpath}{../../tex-preamble/fonts}}

\IfFileExists{\fontpath}{
  \usepackage{unicode-math}         % use OTF math fonts
  \AtBeginDocument{\let\phi\varphi} % varphi with unicode-math
  %\AtBeginDocument{\let\epsilon\varepsilon}
  % \setmathfont[Path=\fontpath]{latinmodern-math.otf}
  \setmathfont[Path=\fontpath]{Asana-Math.otf}
  \setmathfont[Path=\fontpath,range=\setminus]{Asana-Math.otf}
  \setmathfont[Path=\fontpath,range=\mathbb]{texgyretermes-math.otf}
  \setmathfont[Path=\fontpath,range=\mathcal]{latinmodern-math.otf}

  % fix xmapsto. only works with text above.
  % https://github.com/wspr/unicode-math/issues/197
  \ExplSyntaxOn
  \RenewDocumentCommand \xmapsto { m } {
    \mathrel {
      \exp_after:wN \Uoverdelimiter \cs:w sym \__um_symfont_tl \cs_end: "21A6 {
        \mkern 5mu \scan_stop: #1 \mkern 5mu \scan_stop:
      }
    }
  }
  \ExplSyntaxOff
}{}

% general stuff
\newcommand{\units}[1]{\;\mathrm{#1}}
\newcommand{\dd}{\,\mathrm{d}}
\newcommand{\ov}{\overline}
\newcommand{\paren}[1]{\left(#1\right)}
\newcommand{\braces}[1]{\left\{#1\right\}}
\newcommand{\brackets}[1]{\left[#1\right]}
\newcommand{\ceiling}[1]{\left\lceil{} #1 \right\rceil}
\newcommand{\floor}[1]{\left\lfloor{} #1 \right\rfloor}
\newcommand{\abs}[1]{\left| #1 \right|}
\newcommand{\e}[1]{\cdot 10^{#1}}
% \newcommand{\emf}{\mathcal{E}}
\newcommand{\contradiction}{\Rightarrow\Leftarrow}
\newcommand{\inv}[1]{\frac{1}{#1}}
\newcommand{\tif}{\text{ if }}
\newcommand{\totherwise}{\text{ otherwise }}
% \newcommand{\mydef}{\textbf{Definition:}}
% complex numbers
\renewcommand{\Im}{\operatorname{Im}}
\renewcommand{\Re}{\operatorname{Re}}

% operators/functions
\DeclareMathOperator{\ord}{ord}
% \DeclareMathOperator{\image}{image}
\DeclareMathOperator{\Int}{Int}
\DeclareMathOperator{\Ext}{Ext}
\DeclareMathOperator{\Lim}{Lim}
\DeclareMathOperator{\Bd}{Bd}
\DeclareMathOperator{\lcm}{lcm}
\DeclareMathOperator{\proj}{proj}
\DeclareMathOperator{\xor}{xor}
\DeclareMathOperator{\spanf}{span}
\DeclareMathOperator{\id}{id}
% linear
\DeclareMathOperator{\col}{Col}
\DeclareMathOperator{\row}{Row}
\DeclareMathOperator{\nul}{Nul}
\DeclareMathOperator{\rank}{rank}

% set theory
\newcommand{\xlongmapsto}[1]{\xmapsto{\,\,\, #1\,\,\,}}
\newcommand{\xlongrightarrow}[1]{\xrightarrow{\,\,\, #1\,\,\,}}
\newcommand{\union}{\mathop{\bigcup}}
\newcommand{\disunion}{\mathop{\dot\bigcup}}
\newcommand{\intersection}{\mathop{\bigcap}}
\newcommand{\powerset}{\mathcal{P}}

% analysis & calculus
\usepackage{esint} % \oiint, etc
\newcommand{\evalat}[2]{{\bigg|_{#1}^{#2}}}
\newcommand{\ext}[1]{{#1}^{ext}}  % extension of function
\DeclareMathOperator{\determinant}{det}
\DeclareMathOperator{\oball}{B}
\DeclareMathOperator{\cball}{\ov{B}}
\DeclareMathOperator{\sphere}{S}
\DeclareMathOperator{\divergence}{div}
\DeclareMathOperator{\curl}{curl}

% vectors
\newcommand{\ihat}{\hat{\imath}}
\newcommand{\jhat}{\hat{\jmath}}
\newcommand{\khat}{\hat{k}}
\newcommand{\lvec}[1]{\overrightarrow{#1}}
\newcommand{\len}[1]{\left\| #1 \right\|}

% topology
\newcommand{\RP}{\ensuremath{\R\hspace{-0.07em}\mathrm{P}}}

% categories
\newcommand{\I}{\ensuremath{\mathbf{I}}} % "interval groupoid"
\newcommand{\Cat}{\ensuremath{\mathcal{C}\mathrm{at}}}
\newcommand{\ComGrp}{\ensuremath{\mathcal{C}\mathrm{om}\mathcal{G}\mathrm{rp}}}
\newcommand{\AbGrp}{\ensuremath{\mathcal{A}\mathrm{b}\mathcal{G}\mathrm{rp}}}
\newcommand{\Grp}{\ensuremath{\mathcal{G}\mathrm{rp}}}
\newcommand{\Grpd}{\ensuremath{\mathcal{G}\mathrm{rpd}}}
\newcommand{\Mod}{\ensuremath{\mathcal{M}\mathrm{od}}}
\newcommand{\Man}{\ensuremath{\mathcal{M}\mathrm{an}}}
\newcommand{\Metric}{\ensuremath{\mathcal{M}\mathrm{etric}}}
\newcommand{\Set}{\ensuremath{\mathcal{S}\mathrm{et}}}
\newcommand{\Top}{\ensuremath{\mathcal{T}\!\mathrm{op}}}
\newcommand{\homoTop}{\ensuremath{\mathrm{homo}\mathcal{T}\mathrm{op}}}
\newcommand{\Vect}{\ensuremath{\mathcal{V}\mathrm{ect}}}
\renewcommand{\phi}{\varphi}

% number theory
\newcommand{\modulus}[1]{\; \left(\mathrm{mod}\; #1\right)}
\newcommand{\crash}[2]{\left(\frac{#1}{#2}\right)}
\newcommand{\QED}{\begin{flushright}QED\end{flushright}}

% make letters in different fonts (bold, blackboard, caligraphic, fraktur)
% https://tex.stackexchange.com/questions/156196/newcommand-with-variable-name
\newcommand{\mkltr}[1]{%
    \expandafter\newcommand\csname bf#1\endcsname{\ensuremath{\mathbb #1}}
    \expandafter\newcommand\csname bb#1\endcsname{\ensuremath{\mathbb #1}}
    \expandafter\newcommand\csname ca#1\endcsname{\ensuremath{\mathcal #1}}
    \expandafter\newcommand\csname fr#1\endcsname{\ensuremath{\mathfrak #1}}
}%
\mkltr A \mkltr B \mkltr C \mkltr D \mkltr E \mkltr F \mkltr G \mkltr H
\mkltr I \mkltr J \mkltr K \mkltr L \mkltr M \mkltr N \mkltr O \mkltr P
\mkltr Q \mkltr R \mkltr S \mkltr T \mkltr U \mkltr V \mkltr W \mkltr X
\mkltr Y \mkltr Z

\newcommand{\zn}{\mathbf{Z}_n}
\newcommand{\zp}{\mathbf{Z}_p}
\newcommand{\N}{\ensuremath{\mathbf{N}}}
\newcommand{\Z}{\ensuremath{\mathbf{Z}}}
\newcommand{\Q}{\ensuremath{\mathbf{Q}}}
\newcommand{\R}{\ensuremath{\mathbf{R}}}
\newcommand{\C}{\ensuremath{\mathbf{C}}}

% algebra
\newcommand{\quotient}[2]{{\raisebox{.2em}{$#1$}\left/\raisebox{-.2em}{$#2$}\right.}}
\newcommand{\GL}{\mathrm{GL}}     % general linear group
\newcommand{\SL}{\mathrm{SL}}     % special linear group
\renewcommand{\S}{\mathfrak{S}}   % symmetric group
\DeclareMathOperator{\ann}{ann}   % the annihilator of a submodule
\DeclareMathOperator{\Obj}{Obj}   % objects of a category
\DeclareMathOperator{\Hom}{Hom}   % homomorphisms between modules
\DeclareMathOperator{\Mor}{Mor}   % categorical morphisms
\DeclareMathOperator{\characteristic}{char} % ring characteristic
\DeclareMathOperator{\Tor}{Tor}   % torsion elements
\DeclareMathOperator{\Stab}{Stab} % stabilizer
\DeclareMathOperator{\Orb}{Orb}   % orbit
\DeclareMathOperator{\Aut}{Aut}   % automorphisms
\DeclareMathOperator{\End}{End}   % endomorphisms
\DeclareMathOperator{\Mat}{Mat}   % matrix ring
\DeclareMathOperator{\Tr}{Trace}  % trace of a matrix
\DeclareMathOperator{\im}{im}     % image

% statistics
\DeclareMathOperator{\Var}{Var}

% nicer empty set
\let\oldemptyset\emptyset{}
\let\emptyset\varnothing{}
\usepackage{tikz}

\usepackage{tikz}
\usetikzlibrary{shapes.geometric}
\usetikzlibrary{cd}
\usetikzlibrary{arrows.meta}

%%% Notes
% \setbeameroption{show only notes} % show only notes works as well
\setbeamertemplate{note page}[plain]

%%% Document
\begin{document}
\title{Type Theory as Foundations for Math and Computer Science}
\author{Langston Barrett}
\institute{Reed College}
\date{\today}

\frame{\titlepage}

\begin{frame}
  \frametitle{Goals}

  Takeaways:

  \begin{itemize}
    \itemsep0.4em
    \item The distinctions between:
      \begin{itemize}
        \itemsep0em
        \item syntax and semantics
        \item object language and metalanguage
        \item propositions and judgements
        % \item first-order logic and set theory
      \end{itemize}
    \item A feel for (formal) Intuitionistic Propositional Logic
    \item A feel for the typed $\mathbf{\lambda}$-calculus
    \item That these two things aren't so different after all
    \item An understanding of the consequences of the Curry-Howard correspondence
  \end{itemize}

  \note{
    \begin{itemize}
      \itemsep0em
      \item I won't be fully formal. Don't take the systems I present and try
        and run out and reason with them! The formal systems butchered in this
        room stay in this room.
      \item I'll give philosophical background
      \item We don't learn much about fully formal reasoning, so I'll make an
        attempt to introduce the notation and style of logic
    \end{itemize}
  }
\end{frame}

\begin{frame}
  \frametitle{Formal systems}

  \begin{itemize}
    \itemsep0.4em
    \item A \textbf{deductive system} or \textbf{logic} (more generally, a
      \textbf{formal system}) is a set of rules for manipulating symbols.
    \item These symbols and their manipulations are given meaning by
      interpretation.
    \item Rules abstract common patterns of thought.
    \item Concepts like truth and rational argument are \textit{prior} to formal
      systems.
    \item We evaluate formal systems based on our experience of the world.
  \end{itemize}

  \note{
    \begin{itemize}
      \itemsep0em
      \item The formal system gives us syntax, interpretation gives us
        semantics, or meaning.
      \item Example: modus ponens. When it rains, the ground gets wet. It is
        raining. (Draw picture :)
      \item Example: trivial formal system. Everything is true/false!
      \item Example: $(\N,+)$. The statements are equalities, the rules of
        deduction are computation.
      \item Example: ZFC. This is just one formal system, subject to analysis
        like all others! We could use any other in its place! Note that we don't
        actually ``do mathematics in ZFC''.
      % \item For the next slide:
      %   \begin{itemize}
      %     \itemsep0em
      %     \item How can we define
      %   \end{itemize}
    \end{itemize}
  }
\end{frame}

\begin{frame}
  \frametitle{Object language and metalanguage}

  \begin{itemize}
    \itemsep0.4em
    \item How can we dicuss or define logic(s)? We need logic to make sensible
      definitions!
    \item Syntactic elements of a logic are part of the \textbf{object language}.
    \item When we discuss formal systems, we do so in a \textbf{metalanguage}.
  \end{itemize}

  \note{
    \begin{itemize}
      \itemsep0em
      \item Example: The statement $2+2=4$ is in the object language of the
        theory $(\N,+)$. The statement ``all statments in the formal system
        $(\N,+)$ can be proven or disproven'' is in a metalangauge.
      \item A metalanguage doesn't have to be formal, but it can be. We can talk
        about what can be proved about Peano Arithmetic in ZFC.
    \end{itemize}
  }
\end{frame}

\begin{frame}
  \frametitle{Judgements, Proofs, and Propositions}

  \begin{itemize}
    \itemsep0.4em
    \item A \textbf{judgement} (Martin-L\"of) or \textbf{assertion} (Russell) is
      something that can be known. The act of judging is that of comprehending,
      seeing, or coming to know something [\cite{martinlof1996meanings}].
    \item A \textbf{proof} is that which justifies a judgement (makes it evident).
    \item A \textbf{formal proof} is a manipulation of symbols according to the
      rules of a fixed formal system. 
    \item A \textbf{proposition} is a statement that is subject to proof
      [\cite{martinlof1984book}].
  \end{itemize}

  \begin{center}
    \begin{tikzpicture}
      \node[ellipse,draw,thick] (T) at (0,0) {\Large $A$\,\, is true};
      \node[circle,draw,thick,inner sep=0.20cm] (A) at (-0.9,0) {};
      \node[] (P) at (-3.5,0) {Proposition};
      \draw[->,>=stealth,thick] (P) to[] (A);
      \node[] (J) at (3.5,0) {Judgement};
      \draw[->,>=stealth,thick] (J) to[] (T);
    \end{tikzpicture}
  \end{center}

  \note{
    \begin{itemize}
      \itemsep0em
      \item Other judgements: ``$A$ is true right now'', ``$A$ can be achieved
        from available resources''
      \item If you're bored, ruminate on this question: Are the judgements ``$A$
        is true'' and ``$A$ has a proof'' different? (Trick question: depends on
        the formal system. What do you think the answer \textit{should} be?)
      \item Questions so far?
    \end{itemize}
  }
\end{frame}

\begin{frame}
  \frametitle{Defining Intuitionistic Propositional Logic I}
  We shall denote valid inferences like so:
  \begin{displaymath}
    \prftree[r]{}{\text{Hypotheses}}{\text{Conclusion}}
  \end{displaymath}
  The judgements of propositional logic:
  \begin{enumerate}
    \itemsep0.2em
    \item $A$ is a proposition
    \item $A$ is true
  \end{enumerate}
  Inferences will always be reflexive (and transitive). 
  \begin{center}
    \begin{minipage}[b]{0.4\linewidth}
      \begin{displaymath}
        \prftree[r]{}{A\text{ prop}}{A\text{ prop}}
      \end{displaymath}
    \end{minipage}
    \begin{minipage}[b]{0.4\linewidth}
      \begin{displaymath}
        \prftree[r]{}{A\text{ true}}{A\text{ true}}
      \end{displaymath}
    \end{minipage}
  \end{center}

  \note{
    \begin{itemize}
      \itemsep0em
      \item I've made a mistake in the second inference here. Can you see it?
        (We need $A$ prop).
      \item Note that the ``above line/below line'' distinction is implication
        on the ``meta-level'', whereas ($A$ implies $B$) is on the object level.
      \item ``A'' is a metavariable. It ranges over all statements in the
        object language.
    \end{itemize}
  }
\end{frame}

\begin{frame}
  \frametitle{Defining Intuitionistic Propositional Logic II}
  We may form (among others) the following propositions from old ones:
  \begin{center}
    \begin{minipage}[b]{0.3\linewidth}
      \begin{displaymath}
        \prftree[r]{}{A\text{ prop}}{B\text{ prop}}
          {(A\text{ and }B)\text{ prop}}
      \end{displaymath}
    \end{minipage}
    \begin{minipage}[b]{0.3\linewidth}
      \begin{displaymath}
        \prftree[r]{}{A\text{ prop}}{B\text{ prop}}
          {(A\text{ or }B)\text{ prop}}
      \end{displaymath}
    \end{minipage}
    \begin{minipage}[b]{0.3\linewidth}
      \begin{displaymath}
        \prftree[r]{}{A\text{ prop}}{B\text{ prop}}
          {(A\text{ implies }B)\text{ prop}}
      \end{displaymath}
    \end{minipage}
  \end{center}
  Two very special propositions:
  \begin{center}
    \begin{minipage}[b]{0.4\linewidth}
      \begin{displaymath}
        \prftree[r]{}{}{\text{True prop}}
      \end{displaymath}
    \end{minipage}
    \begin{minipage}[b]{0.4\linewidth}
      \begin{displaymath}
        \prftree[r]{}{}{\text{False prop}}
      \end{displaymath}
    \end{minipage}
  \end{center}
  Others may be defined.
  \begin{align*}
    \text{not }A &= A\text{ implies False} \\
    A\text{ if and only if }B &= (A\text{ implies }B)\text{ and }(B\text{ implies }A)
  \end{align*}
  % \note{
  %   \begin{itemize}
  %     \itemsep0em
  %     \item What do the variables range over?
  %   \end{itemize}
  % }
\end{frame}

\begin{frame}
  \frametitle{Defining Intuitionistic Propositional Logic III}
  Assuming $A$ and $B$ are propositions,
  \vspace{-1em}
  \begin{center}
    \begin{minipage}[b]{0.3\linewidth}
      \begin{displaymath}
        \prftree[r]{}{\text{True true}}
      \end{displaymath}
    \end{minipage}
    \begin{minipage}[b]{0.3\linewidth}
      \begin{displaymath}
        \prftree[r]{}{A\text{ true}}{B\text{ true}}
          {(A\text{ and }B)\text{ true}}
      \end{displaymath}
    \end{minipage}
    \begin{minipage}[b]{0.3\linewidth}
      \begin{displaymath}
        \prftree[r]{}{A\text{ true}}{(A\text{ or }B)\text{ true}}
      \end{displaymath}
    \end{minipage}
    \begin{minipage}[b]{0.3\linewidth}
      \begin{displaymath}
        \prftree[r]{}{B\text{ true}}{(A\text{ or }B)\text{ true}}
      \end{displaymath}
    \end{minipage}
    \begin{minipage}[b]{0.3\linewidth}
      \begin{displaymath}
        % \prftree[r]{}{A\text{ true}}{A\text{ implies }B\text{ true}}
        %   {B\text{ true}}
        \prftree[r]{}{(A\text{ and }B)\text{ true}}{A\text{ true}}
      \end{displaymath}
    \end{minipage}
    \begin{minipage}[b]{0.3\linewidth}
      \begin{displaymath}
        \prftree[r]{}{(A\text{ and }B)\text{ true}}{B\text{ true}}
      \end{displaymath}
    \end{minipage}
    \begin{minipage}[b]{0.3\linewidth}
      \begin{displaymath}
        \prftree[r]{}{\text{False true}}{A\text{ true}}
      \end{displaymath}
    \end{minipage}
    \begin{minipage}[b]{0.6\linewidth}
      \begin{displaymath}
        \prftree[r]{}{(A\text{ implies }B)\text{ true}}{A\text{
            true}}{B\text{ true}}
      \end{displaymath}
    \end{minipage}
    \begin{displaymath}
      \prftree[r]{}{(A\text{ or }B)\text{ true}}{(A\text{ implies }C)\text{
          true}}{(B\text{ implies }C)\text{ true}}
        {C\text{ true}}
    \end{displaymath}
  \end{center}

  \note{
    \begin{itemize}
      \itemsep0em
      \item Can you find modus ponens?
      \item Duality of introduction and elimination rules.
      \item short proof: ($A$ and $B$) implies ($B$ and $A$)
      \begin{displaymath}
        \prftree[r]{}{
          \prftree[r]{}{A\land B}{B}
        }{
          \prftree[r]{}{A\land B}{A}
        }{B\land A}
      \end{displaymath}
      \item Constructive mathematics
      \item Proof that $\lnot \lnot (A\lor \lnot A)$:
        \begin{itemize}
          \itemsep0em
          \item Suffices to assume $\lnot (A\lor \lnot A)$ and prove $\bot$
          \item Suffices to show $A\lor \lnot A$
          \item Suffices to show $\lnot A$
          \item Suffices to assume $A$ and prove $\bot$
          \item However, if $A$, then $A\lor \lnot A$, contradiction!
        \end{itemize}
      
      \item Questions so far?
    \end{itemize}
  }
\end{frame}

\begin{frame}
  \frametitle{Defining the $\lambda$-calculus}
  The judgements of the $\lambda$-calculus:
  \begin{enumerate}
    \itemsep0.2em
    \item $A$ is a type
    \item $a$ is a term of type $A$ (written $a:A$)
  \end{enumerate}
  We have
  \vspace{-1em}
  \begin{center}
    \begin{minipage}[b]{0.20\linewidth}
      \begin{displaymath}
        \prftree[r]{}{}{\emptyset\text{ type}}
      \end{displaymath}
    \end{minipage}
    \begin{minipage}[b]{0.15\linewidth}
      \begin{displaymath}
        \prftree[r]{}{}{\top\text{ type}}
      \end{displaymath}
    \end{minipage}
    \begin{minipage}[b]{0.30\linewidth}
      \begin{displaymath}
        \prftree[r]{}{A\text{ type}}{B\text{ type}}{A\times B\text{ type}}
      \end{displaymath}
    \end{minipage}
    \begin{minipage}[b]{0.25\linewidth}
      \begin{displaymath}
        \prftree[r]{}{A\text{ type}}{B\text{ type}}{A+B\text{ type}}
      \end{displaymath}
    \end{minipage}
    \begin{minipage}[b]{0.25\linewidth}
      \begin{displaymath}
        \prftree[r]{}{A\text{ type}}{B\text{ type}}{A\to B\text{ type}}
      \end{displaymath}
    \end{minipage}
    \begin{minipage}[b]{0.15\linewidth}
      \begin{displaymath}
        \prftree[r]{}{}{():\top}
      \end{displaymath}
    \end{minipage}
    \begin{minipage}[b]{0.25\linewidth}
      \begin{displaymath}
        \prftree[r]{}{a:A}{b:B}{(a,b):A\times B}
      \end{displaymath}
    \end{minipage}
    \begin{minipage}[b]{0.25\linewidth}
      \begin{displaymath}
        \prftree[r]{}{a:A}{\operatorname{left}(a):A+B}
      \end{displaymath}
    \end{minipage}
    \begin{minipage}[b]{0.30\linewidth}
      \begin{displaymath}
        \prftree[r]{}{a:A}{\operatorname{right}(b):A+B}
      \end{displaymath}
    \end{minipage}
    \begin{minipage}[b]{0.25\linewidth}
      \begin{displaymath}
        \prftree[r]{}{a:A}{f:A\to B}{f a:B}
      \end{displaymath}
    \end{minipage}
    \begin{minipage}[b]{0.20\linewidth}
      \begin{displaymath}
        \prftree[r]{}{p:A\times B}{\operatorname{proj}_1(p):A}
      \end{displaymath}
    \end{minipage}
    \begin{minipage}[b]{0.20\linewidth}
      \begin{displaymath}
        \prftree[r]{}{p:A\times B}{\operatorname{proj}_2(p):B}
      \end{displaymath}
    \end{minipage}
  \end{center}

  \note{
    \begin{itemize}
      \itemsep0em
      \item Example of computing with the $\lambda$-calculus: first introduce
        function introduction:
        \begin{itemize}
          \itemsep0em
          \item If $f$ is an expression containing $x$,
          \item and whenever an expression $a:A$ is substituted for $x$ in $f$,
            the result has type $B$,
          \item then $\lambda x.f:A\to B$
        \end{itemize}
      Then if $f\coloneqq x+2$, $\lambda x.f=\lambda x.x+2:\N\to\N$
      \item short proof: ($A$ and $B$) implies ($B$ and $A$)
      \begin{displaymath}
        \prftree[r]{}{
          \prftree[r]{}{p:A\times B}{\operatorname{proj}_2 p:B}
        }{
          \prftree[r]{}{p:A\times B}{\operatorname{proj}_1 p:B}
        }{\langle \operatorname{proj}_1 p, \operatorname{proj}_2p
          \rangle:A\times B}
      \end{displaymath}
      \item Flip back and forth between slides a bunch
    \end{itemize}
  }
\end{frame}

\begin{frame}
  \frametitle{The Curry-Howard Correspondence}

  \begin{itemize}
    % \itemsep0em
    \item The judgement ``$A$ type'' corresponds to ``$A'$ prop''
    \item The judgement ``$a:A$'' corresponds to ``$A'$ true'' (and $a$ is the
      proof/evidence: we've internalized the notion of proof)
  \end{itemize}
  \vspace{-1em}
  \begin{align*}
    A\times B &\longleftrightarrow A\text{ and }B \\
    A+B &\longleftrightarrow A\text{ or }B \\
    A\to B &\longleftrightarrow A\text{ implies }B \\
    \top &\longleftrightarrow \text{True} \\
    \emptyset &\longleftrightarrow \text{False}
  \end{align*}
\end{frame}

\begin{frame}
  \frametitle{Consequences}

  \begin{itemize}
    \itemsep0em
    \item The $\lambda$-calculus is a \textit{programming language}. 
    \item Type-checking in the $\lambda$-calculus is decidable!
    \item Proof assistants
    \item Extraction of witnesses
    \item Using the same language to perform and reason about computation
  \end{itemize}

\end{frame}

\begin{frame}
  \frametitle{Endnotes}
  \begin{itemize}
    \item Thank you!
    \item Questions?
    \item Extra topics:
      \begin{itemize}
        \item The question I asked about ``$A$ is true'' vs. ``$A$ has a proof''
        \item Dependent types
        \item What do the judgements \textit{mean}?
        \item Free and bound variables
      \end{itemize}
    \item You can find these slides online at
      \url{github.com/siddharthist/thesis/presentations}
  \end{itemize}
\end{frame}

\begin{frame}[allowframebreaks]
  \frametitle{References}

  \printbibliography
\end{frame}

% TODO: references, where you can find these slides online

\end{document}
