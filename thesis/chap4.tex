\documentclass[./thesis.tex]{subfiles}
\begin{document}
\chapter{Coinductive types in univalent type theory}
\label{chap:coinductive-types-in-univalent-type-theory}

The results in this chapter all appear in the \UniMath{} package
\unimathname{Induction}, so that prefix is left off
(e.g.\ \unimathname{Induction.W.Core} is shortened to
\unimathname{W.Core}).

\section{W and M}
\label{sec:w-and-m}

We now present W-types and M-types. Per Martin-Löf introduced W-types as a
way to study types with a well-ordering. In fact, W-types provide a general and
convenient setting for the study of induction. In extensional type theories,
W-types are equivalent to initial algebras (as in \cref{def:initial-alg}) for
polynomial functors (see \cref{def:polynomial-functor}), so all the examples
we've seen so far of inductive types defined as functor algebras are W-types.

Dually, M-types are non-wellfounded trees, that is, trees with potentially
infinitely long branches.

\begin{definition}\label[definition]{def:container}
  A \define{container}\index{Container} (sometimes called a
  \define{signature}\index{Signature} by analogy, see
  \cref{rmk:universal-algebra}) is a pair $(A,B)$ of a type
  $A:\universe$ and a family $B:A→\universe$.
\end{definition}
Given a container, we can form the associated W- and M-types:
\begin{gatherjot}
  \prftree[r]{W-form}
    {Γ ⊢ A:\universe}{Γ ⊢ B:A\to\universe}
    {Γ ⊢ \type{\W{a}{A}{B}}}
  \\
  \prftree[r]{M-form}
    {Γ ⊢ A:\universe}{Γ ⊢ B:A\to\universe}
    {Γ ⊢ \type{\M{a}{A}{B}}}
\end{gatherjot}
Again, these types are shaped like trees. An element of a W-type is specified by
an element $a:A$ (the ``label'' of this vertex) and a function
$b:B(a)\to \W{x}{A}{B(x)}$ which picks out its subtrees.
\begin{equation*}
  \prftree[r]{W-intro}
    {Γ ⊢ A:\universe}{Γ ⊢ B:A\to\universe}{Γ ⊢ a:A}{Γ ⊢ b:B(a)\to \W{x}{A}{B}}
    {Γ ⊢ \sup{a}{b}:\W{x}{A}{B}}
\end{equation*}
Martin-Löf points out that a ``bottom'' element can be given by taking one of
the $B(a)$ to be $\emptytype$ and the subtree-function to be the one provided by
$\rec_{\emptytype}$ (example in \cref{ex:nat-w})\TODO{reference}.

The elimination rule\index{Elimination!For W-types} for W-types encodes
transfinite induction. If, when a property $C:\W{a}{A}{B}→\universe$ holds for
all subtrees of a given node it also holds for that node, then it holds for any
element in $\W{a}{A}{B}$:
\begin{equation*}
  \prftree[r]{W-elim}
    {Γ ⊢ w:\W{a}{A}{B}}
    {\prftree[r, noline]{}
      {Γ, a:A,\quad f:\apply{B}{a}→\W{a}{A}{B},\quad g:\∏{b:\apply{B}{a}}{\apply{C}{(\apply{f}{b})}}}
      {⊢\apppply{h}{a}{f}{g}: C(\sup{a}{f})}}
    {Γ ⊢ \appply{\recW}{w}{h}:\apply{C}{w}}
\end{equation*}

\begin{example}\label[example]{ex:nat-w}
  The natural numbers are readily encoded as a W-type.
  Define $B\defeq \apppply{\rec_{\booltype}}{\universe}{\emptytype}{\unittype}$
  so that $\apply{B}{\bfalse}\jdeq \emptytype$ and
  $\apply{B}{\btrue}\jdeq\unittype$, and consider $\N'\jdeq \W{b}{A}{B}$.
  Let $!$ be the unique function out of the empty type and further define:
  \begin{align*}
    \begin{split}
      0_{\N'} &: \W{b}{\booltype}{B} \\
      0_{\N'} &\defeq \sup{\bfalse}{!} \\
    \end{split}
    \begin{split}
      1_{\N'} &: \W{b}{\booltype}{B} \\
      1_{\N'} &\defeq \sup{\btrue}{(\λ{x}{0_{\N'}})}
    \end{split}
  \end{align*}
  Following this pattern, we can define
  \begin{align*}
    \suc &: \N' \to \N' \\
    \suc &\defeq \λ{n}{\sup{\btrue}{(\λ{x}{n})}}
  \end{align*}
  so that $1_{\N'}\equiv \apply{\suc}{0_{\N'}}$. What does this tree look like?
  Well, by virtue of construction, $0_{\N'}$ is its least element. Applying
  $\suc$ gives us an element at one level higher in the tree than the one we put
  in.
  \vspace{-2em}\begin{center}
    \begin{tikzpicture}[scale=4]
      \node (A) [scale=2] {$\cdot$};
      \node     [above of=A] {$0_{\N'}$};

      \node (B) [scale=2, right of=A] {$\cdot$};
      \node     [above of=B] {$1_{\N'}$};

      \node (C) [scale=2, right of=B] {$\cdot$};
      \node     [above of=C] {$\apply{\suc}{1_{\N'}}$};

      \node (D) [scale=2, right of=C] {$\cdot$};
      \node     [above of=D] {$\cdots$};

      \node     [right of=D, scale=2.0, color=white] {$\cdot$};
      \node (E) [right of=D] {$\cdots$};
      \node     [scale=2, above of=E, color=white] {$\cdots$};

      \draw[->] (A) edge (B);
      \draw[->] (B) edge (C);
      \draw[->] (C) edge (D);
      \draw[->] (D) edge (E);
      % \draw (A) edge [in=-60,out=-20,loop] node[below] {$e$} (A);
    \end{tikzpicture}
  \end{center}

  We can derive from the induction principle of $\booltype$ that it
  only has two elements, $\bfalse$ and $\btrue$ \TODO{reference}.
  Using this (and ignoring uniquely specified terms like $!$),
  the W-elimination rule for $\N'$ can be reshaped and simplified:
  \begin{equation*}
    \prftree[r]{\N'-elim}
      {Γ ⊢ n:\N'}
      {\prftree[r, noline]{}
        {Γ⊢h_1: C(0_{\N'})}}
      {\prftree[r, noline]{}
        {Γ, f:\unittype→\N',g:\∏{\unitelem:\unittype}\apply{C}{(\apply{f}{\unitelem)}}}
        {⊢\appply{h_2}{f}{g}: C(\sup{\btrue}{f})}}
      {Γ ⊢ \apppply{\rec_{\N'}}{n}{h_1}{h_2}:\apply{C}{n}}
  \end{equation*}
  But since a function $\unittype→\N'$ is equivalent to picking out a single
  element of $\N'$ and $\sup{\btrue}{f}\equiv\apply{\suc}{f}$,
  this is just the elimination rule for the natural numbers (\TODO{ref})
  in disguise.
\end{example}

Other examples include lists, binary trees, ordinals of the second number
class, and more.\TODO{reference} In the next few sections, we will develop a
correspondence between W-types (resp.\ M-types) and initial algebras (resp.\
final coalgebras) for a general class of functor on $\universe$, which will
include all examples in \cref{sec:functors-and-their-algebras}.

\section{Preliminaries}
\label{sec:preliminaries}

\begin{definition}[\unimathname{PolynomialFunctors.polynomial\_functor}]
  \label[definition]{def:polynomial-functor}
  Given a container $S\defeq (A,B)$, its associated \define{polynomial functor}
  is the function
  \begin{gather*}
    P:\universe\to \universe \\
    \apply{P_{A,B}}{X}\defeq\∑{a:A}{B(a)\to X}
  \end{gather*}
  We will regularly leave off the subscript for $P$.
  The \define{action} of $P$ on functions is
  \begin{gather*}
    P^* : (X\to Y)\to P_{A,B}(X)\to P_{A,B}(Y) \\
    \apply{P^*f}{(a,g)}\defeq (a,g∘f).
  \end{gather*}
\end{definition}

\begin{remark}
	Technically, polynomial functors aren't functors, since $\universe$ isn't
  a category (the whole point of the univalent perspective is that it behaves
  more like a weak higher category). It has functorial properties only up to
  \textit{propositional} equality (as in \cref{def:functor-tt}), which will
  suffice for the following proofs. That is, we have
  \begin{align*}
    \apply{P^*}{g∘f} = \apply{P^*}{g}∘\apply{P^*}{f}
    &&
    \apply{P^*}{\id_X} = \apply{P^*}{\id_X}
  \end{align*}
\end{remark}

\begin{definition}\label[definition]{def:algebra-coalgebra-in-utt}
  Exactly as in \cref{def:f-coalgebra}, we define the type of (co)algebras for a
  polynomial functor $P$ associated to a container $(A,B)$ as follows:
  \begin{align*}
    \Algtype_P\≔\∑{A:\universe}{PA→A} && \Coalgtype_P\≔\∑{A:\universe}{A→PA}
  \end{align*}
  and coalgebra morphisms as
  \begin{align*}
    \ttfun{CoalgMor}((X,α),(Y,β))
    &\≔ (X,α)⇒(Y,β) \\
    &\≔ \∑{f:X→Y}{(\apply{P}{f})∘α = β∘f}
  \end{align*}
  with algebra morphisms defined dually \cite{homotopy-initial}.
\end{definition}

See \cref{ex:nat-functor-utt} for the polynomial functor for which $\ℕ$ is
initial.

\section{Internalizing M-types}
\label{sec:internalizing-m-types}

Fix a container $S\defeq (A,B)$ and let $P$ be the associated polynomial functor.
We will prove a few auxiliary lemmas on the way to the following result.
This proof appeared in the \Agda{} formalization of \cite{non-wellfounded}, this
is its first appearance ``de-formalized''.

\subsection{Uniqueness}
\label{subsec:uniqueness}

\begin{lemma}[\unimathname{M.Uniqueness.M\_coalg\_eq},
              \unimathname{M.Uniqueness.isaprop\_M}]
  \label[lemma]{lemma:final-colagebra-unique}
  Any two final $P$-coalgebras are equal. In other words, the following type is
  a proposition:
  \begin{equation*}
    \Final(S)\defeq
    \∑{(X,α):\Coalgtype_{S}}{
      \∏{(Y,β):\Coalgtype_{S}}{\isContr((Y,β)⇒ (X,α))}
    }
  \end{equation*}
\end{lemma}

First, we'll show that their carriers are equivalent:

\begin{lemma}[\unimathname{M.Uniqueness.M\_carriers\_iso}]
  \label[lemma]{lemma:algebra-iso-equiv}
	If $(X,α)$ and $(Y,β)$ are final $P$-coalgebras,
  then the first projections of the unique coalgebra morphisms
  $f:X⇒ Y$ and $g:Y⇒ X$ induce
  an equivalence of types $\weq{X}{Y}$.
\end{lemma}
\begin{proof}
  This proof is a standard categorical technique, much reminiscent of
  \cref{lemma:terminal-unique}.
	\TODO{proof}
\end{proof}

We can then invoke the characterization of paths in $Σ$-types,
\cref{lemma:path-sigma}. We'll need to demonstrate that the coalgebra map
$α:X\to FX$ is equal to $β:Y\to FY$ when transported along the path
constructed in \cref{lemma:algebra-iso-equiv}.

\begin{lemma}\label[lemma]{lemma:polynomial-functor-transport}
  For all $X,Y:\universe$, $F:\universe\to\universe$,
  $f:X\to FX$, $g:Y\to FY$, and $p:\propeq{}{X}{Y}$,
  if for all $x:X$ we have
  \begin{equation*}
    \propeq{}{
      \transport{F}{p}{\apply{f}{x}}
    }{
      g({\transport{\id_{\universe}}{p}{x}})
    }
  \end{equation*}
  then $\propeq{}{\transport{Z↦ (Z\to FZ)}{p}{f}}{g}$.
  That is, $f$ is equal to $g$ after being transported just when
  applying $f$ and transporting the result is the same as transporting the
  input and applying $g$.
\end{lemma}
\begin{proof}
  Using identity elimination (\cref{rule:id-elim}), it suffices to assume
  $X\jdeq Y$ and $p\jdeq\refl{X}$. Then by the definition of transport
  (\cref{lemma:transport}), our hypothesis becomes
  \begin{align*}
    &\propeq{}{
      \transport{F}{p}{\apply{f}{x}}
    }{
      g({\transport{\id_{\universe}}{p}{x}})
    } \\
    &\implies
    \propeq{}{
      \transport{F}{\refl{X}}{\apply{f}{x}}
    }{
      g({\transport{\id_{\universe}}{\refl{X}}{x}})
    } \\
    &\implies
    \propeq{}{\apply{f}{x}}{\apply{g}{x}}
  \end{align*}
  so by function extensionality $f=g$. Again by definition of transport,
  $\transport{Z↦ (Z\to FZ)}{\refl{X}}{f} \jdeq f=g$.
\end{proof}

To complete the proof that the transported coalgebra maps are equal, we'll need
the following auxiliary result:

\begin{lemma}\label[lemma]{lemma:polynomial-functor-transport}
  For all $X,Y:\universe$ and $p:\propeq{}{X}{Y}$,
  \begin{equation*}
    \propeq{}{
      P^*\paren{\transpor{\id_{\universe}}{p}}
    }{
      \transpor{P}{p}
    }
  \end{equation*}
  Note that $P^*$ is applied to the function $\transpor{\id_{\universe}}{p}$
  before it gets applied to its second argument.
\end{lemma}
\begin{proof}
  Note that we're using \cref{notation:transport}. By the elimination rule for
  the identity type, (\cref{rule:id-elim}), it suffices to assume that $X\jdeq
  Y$ and that $p\jdeq\refl{X}$. Then
  \begin{align*}
    P^*\paren*{\transpor{\id_{\universe}}{p}}
    &\jdeq P^*\paren*{\transpor{\id_{\universe}}{\refl{X}}}
    && \text{Identity elim.} \\
    &\jdeq P^*\paren{\id_X}
    && \text{\Cref{lemma:transport}} \\
    &\jdeq \λ{(a,f)}{(a,f∘ \id_X)}
    && \text{\Cref{def:polynomial-functor}} \\
    &\jdeq \λ{(a,f)}{(a,f)} \\
    &\jdeq \id_{PX}
    && \text{\Cref{def:id-polymorphic}} \\
    &\jdeq \transpor{\id_{\universe}}{\refl{PX}}
    && \text{\Cref{lemma:transport}} \\
    &\jdeq \transpor{\id_{\universe}}{\ap{P}{\refl{X}}}
    && \text{\Cref{lemma:ap}} \\
    &\jdeq \transpor{P}{\refl{X}}
    && \text{\Cref{lemma:transport-compose}} \\
    &\jdeq \transpor{P}{p}
    && \text{Identity elim.}
  \end{align*}
\end{proof}

\begin{proof}[Proof of \cref{lemma:final-colagebra-unique}]
  Let:
  \begin{itemize}
    \itemsep0em
    \item $(X,α)$, $(Y,β)$ be final $P$-coalgebras,
    \item $f:X⇒ Y$ and $g:Y⇒ X$ be the unique $P$-coalgebra
      morphisms between them,
    \item $q:\weq{X}{Y}$ the equivalence of types induced by $f$ and $g$
      (\cref{lemma:algebra-iso-equiv}).
  \end{itemize}
  By univalence, there is a path
    $\apply{\ua}{q}:\propeq{\universe}{X}{Y}$.\footnote{This
    is our first (but not nearly our last) crucial use of univalence. Without an
    equality, we couldn't \transportname{} $α$ to $β$ in the next step.
    This step also demonstrates that function extensionality alone doesn't
    suffice.}
  To demonstrate that $\propeq{}{(X,α)}{(Y,β)}$, we invoke the
  characterization of paths in $Σ$-types, \cref{lemma:path-sigma}. It
  remains to show
  \begin{equation*}
    \propeq{(Y\to PY)}{\transport{Z↦ (Z\to PZ)}{\apply{\ua}{q}}{α}}{β}.
  \end{equation*}
  but by \cref{lemma:polynomial-functor-transport}, it suffices to show that
  for all $x:X$,
  \begin{equation*}
    \propeq{}{
      \transport{P}{\apply{\ua}{q}}{\apply{α}{x}}
    }{
      β(\transport{\id_\universe}{p}{x})
    }.
  \end{equation*}
  % Lemma 2 in HoTT/M-types
  First, note that
  \begin{align*}
    \transpor{P}{\apply{\ua}{q}}
    &= P^*\paren{\transpor{\id_{\universe}}{\apply{\ua}{q}}}
    && \text{\Cref{lemma:polynomial-functor-transport}} \\
    &= P^*q
    && \text{\Cref{def:ua}} \\
    &= P^*(\appr{1}{f})
    && \text{\Cref{lemma:algebra-iso-equiv}}
  \end{align*}
  The last step utilizes the idea of \textit{proof-relevant
    mathematics}. Although we define $q$ within a proof, we can (without
  cheating) refer to its definition from another proof. Also note the
  use of \cref{notation:weq-coerce}. Working from the other side of the
  equation, we can use the computational rule of univalence (\cref{def:ua}):
  \begin{align*}
    β(\transport{\id_\universe}{\apply{\ua}{q}}{x})
    = β(\apply{q}{x})
    = β({\appr{1}{f}}{x})
  \end{align*}
  Thus, we now want to demonstrate that
  \begin{align*}
    P^*(\appr{1}{f})(α x) &=
    \transport{P}{\apply{\ua}{q}}{\apply{α}{x}} \\
    &= β(\transport{\id_\universe}{p}{x}) \\
    &= β({\appr{1}{f}}{x})
  \end{align*}
  However, this is exactly the condition that $f$ is a $P$-coalgebra morphism:
  \TODO{reference definition}
  \begin{center}
    \begin{tikzcd}[column sep=large]
      X  \arrow[d, "α"] \arrow[r, "\appr{1}{f}"] & Y \arrow[d, "β"] \\
      FX \arrow[r, "P^*(\appr{1}{f})"] & FY
    \end{tikzcd}
  \end{center}
\end{proof}

\begin{lemma}[\unimathname{W.Uniqueness.W\_alg\_eq},
              \unimathname{W.Uniqueness.isaprop\_W}]
  Any two initial $P$-algebras are equal.
\end{lemma}
\begin{proof}
	The above proof works with only minor modifications, see the \Coq{}
  development for details.
\end{proof}

\subsection{Fibered algebras and the natural numbers}
\label{subsec:fibered-algebras}

The following section is alarmingly technical. All but the most dedicated
readers should probably skip all theorems and definitions involving the word
``fibered''.

A functor algebra $\dpair{X}{α}$ has as its first component a type
$X:\universe$. What would it mean for an algebra to be ``fibered'' over this
one, with a first component $Y:X → \universe$? When can a \textit{dependent}
type (over an algebra) can be given the structure of a functor algebra?
Such fibered algebras generalize regular algebras for endofunctors.

The fiber-initial property of $\ℕ$
(see \cref{def:fiber-initiality} and \cref{lemma:fiber-initiality-nat})
is essential to proving a property about limits
(\cref{lemma:cochains}) that plays a pivotal role in the proof of the existence
of M-types.

\begin{definition}[\unimathname{W.Fibered.fibered\_alg},
                   \unimathname{W.Fibered.algebra\_section}]
  \label[definition]{def:algebra-fibration}
  Inspired by the homotopical semantics for \UTT{},
  if $(X,α)$ is an algebra for the functor associated to $(A,B)$,
  the type of \define{fibered $P$-algebras} over $X$ is
  \begin{equation*}
    \Fibalgtype(X)\≔
    \∑{E:X→\universe}
    {\∏{\dpair{x}{h}:P_{A,B}X}{
      \paren{\∏{b:\apply{B}{x}}{\apply{E}{(\apply{h}{b})}}}
      → \apply{E}{(\apply{\alpha}{\dpair{x}{h}})}
    }}
  \end{equation*}
  an \define{algebra fibration}.
  % ((∑ (f : forall x, E x), forall a, f (c a) = e a (f ∘ (pr2 a)))).
  An \define{algebra section} for an algebra fibration $(E,θ)$ is a pair
  \begin{equation*}
    \∑{f:\∏{x:X}{\apply{E}{x}}}
    \∏{x:P_{A,B}X}
    {\propeq
      {}
      % {\∏{x:X}{\apply{E}{(\apply{α}{x})}}}
      {\apply{f}{(\apply{α}{x})}}
      {\appply{θ}{x}{(f∘\appr{2}{x})}}}
  \end{equation*}
\end{definition}

\begin{example}[\unimathname{W.Fibered.alg2fibered\_alg},
                \unimathname{W.Fibered.fibered\_alg2alg}]
  Any $P$-algebra can be made into a fibered $P$-algebra over $(X,α)$ by using a
  constant family:
  \begin{align*}
    \Algtype_P &\longrightarrow \Fibalgtype(X) \\
    \dpair{Y}{β} &\longmapsto
                \dpair{\λ{x}{Y}} {\λ{x}{\λ{f}{\apply{β}{\dpair{\appr{1}{x}}{f}}}}}
  \end{align*}
  Since the $x$ in the argument to the second coordinate is of type
  $P_{A,B}X\jdeq \∑{a:A}\apply{B}{a}→X$ and $f:\apply{B}{(\appr{1}{x})}→Y$, the pair
  $\dpair{x}{f}$ has type $P_{A,B}Y\jdeq \∑{a:A}\apply{B}{a}→Y$, which can then be
  sent via $β$ to $Y$. This matches our expectation that a non-dependent
  $\ttfun{foo}$ should be a generalization of a dependent $\ttfun{foo}$.
  In turn, any fibered algebra with family $E:X→\universe$ can be turned into a
  non-fibered algebra over the ``total space'' $\∑{x:X}\apply{E}{x}$.
\end{example}

\begin{definition}[\unimathname{W.Fibered.is\_preinitial\_sec}]
  \label[definition]{def:fiber-preinitiality}
  An algebra $\dpair{X}{α}$ is \define{fiber-preinitial}
  if for any fibered algebra $\dpair{E}{θ}$ over
  $X$ the type of algebra sections from $X$ to $E$ is
  inhabited.\footnote{\cite{homotopy-initial} call this an ``inductive
  algebra''.\label{fn:inductive-algebra}}
\end{definition}

\begin{definition}[\unimathname{W.Fibered.fibered\_uniqueness}]
  \label[definition]{def:fiber-uniqueness}
  An algebra $\dpair{X}{α}$ has the \define{fiber uniqueness property}
  if for any fibered algebra $\dpair{E}{θ}$ over
  $X$ the type of algebra sections from $X$ to $E$ is a
  proposition.\footnote{\cite{homotopy-initial} call this the η-rule for
    inductive algebras (see \cref{fn:inductive-algebra}).}
\end{definition}

\begin{definition}\label[definition]{def:fiber-initiality}
  Combining \crefrange{def:fiber-preinitiality}{def:fiber-uniqueness}, an
  algebra $\dpair{X}{α}$ is \define{fiber-initial} if the type of algebra
  sections from $X$ to any fibered algebra $\dpair{E}{θ}$ over $X$ is
  contractible.
\end{definition}

\Cref{subsubsec:fibered-natural-numbers} explores the applications of these
functions to $\ℕ$.

\subsubsection{The natural numbers}
\label{subsubsec:fibered-natural-numbers}

We now prove a few facts about the natural numbers and the functor for which
they are initial, some of which will be familiar from
\cref{sec:functors-and-their-algebras}.\footnote{These arguments are my own,
  though some of these theorems are proved in the \Agda development of
  \cite{non-wellfounded}.}

\begin{example}[\coqname{W.Naturals.nat\_functor}]
  \label[example]{ex:nat-functor-utt}
	Consider the polynomial functor associated to $A\defeq \𝔹$ and
  $B\defeq \apppply{\rec_{\𝔹}}{\universe}{\𝟘}{\𝟙}$:
  \begin{equation*}
    PX \defeq \∑{b:\𝔹}{\appppply{\rec_{\𝔹}}{\universe}{\𝟘}{\𝟙}{b}} → X.
  \end{equation*}
  An algebra for this functor is given by a type $X:\universe$ and a function
  \begin{equation*}
    PX → X\≡ \paren{\∑{b:\𝔹}{\appppply{\rec_{\𝔹}}{\universe}{\𝟘}{\𝟙}{b}} → X} → X
  \end{equation*}
  Define:
  \begin{align*}
    &η : Pℕ → ℕ &&\\
    &\apply{η}{\dpair{\bfalse}{g}} \defeq 0
    &&\quad(\text{In this case,  }g:\𝟘 → X) \\
    &\apply{η}{\dpair{\btrue}{g}}  \defeq \apply{\suc}{(\apply{g}{\unitelem})}
    &&\quad(\text{In this case,  }g:\𝟙 → X) \\
  \end{align*}
\end{example}

While the above polynomial functor $P$ bears some relation to the definition of
$ℕ$ as an W-type (\cref{ex:nat-w}), it is not immediately clear that this is the
correct definition, nor how we can apply our intuition about $\ℕ$ to this
situation. The next lemma provides more evidence that $P$ is an analogue
of $X ↦ 1 + X$.

\begin{lemma}[\coqname{W.Naturals.nat\_functor\_equiv}]
  \label[lemma]{lemma:nat-alg-simpl}
  A $P$-algebra structure on a type $X$ is given by a point $x_0:X$ and a
  function $\suc_x:X→X$. More precisely, there is an equivalence:
  \begin{equation*}
    w:\weq{\Algtype_P}{\∑{X:\universe}{X × (X → X)}}
  \end{equation*}
\end{lemma}
\begin{proof}
	Suppose given a $P$-algebra $(X,α)$. Then define
  \begin{align*}
    \begin{split}
      x_0 &: X \\
      x_0 &\defeq \apply{α}{\dpair{\bfalse}{\apply{\rec_{\𝟘}}{X}}} \\
    \end{split}
    \begin{split}
      \suc_X  &: X → X \\
      \suc_X  &\defeq \λ{x}{\apply{α}{\dpair{\btrue}{\appply{\rec_{\𝟙}}{X}{x}}}}
    \end{split}
  \end{align*}
  Now $\dpair{X}{(x_0,\suc_X)}$ has type $\∑{X:\universe}{X × (X → X)}$ as
  desired.

  In the other direction, suppose we have some type $X$, a point $x_0:X$, and a
  function $\suc_X:X → X$. Define
  \begin{align*}
    &α:\paren{\∑{b:\𝔹}{\appppply{\rec_{\𝔹}}{\universe}{\𝟘}{\𝟙}{b}} → X} → X \\
    &\apply{α}{\dpair{\bfalse}{h}} \defeq x_0 \\
    &\apply{α}{\dpair{\btrue}{h}}  \defeq \apply{\suc_X}{(\apply{h}{\unitelem})}
  \end{align*}
  While it's not immediately that these functions are inverses, it follows from
  initiality of $\emptytype$ in $\universe$ and the contractibility of
  $\unittype$, both of which will be used extensively in the next few proofs.
\end{proof}

Applying this equivalence, we could more readily define the $P$-algebra over
$\ℕ$ as $(\appply{w}{0}{\suc})$. In fact, this gives \textit{judgmentally equal}
result. We can give a similar lemma for morphisms:

\begin{lemma}[\coqname{W.Naturals.mk\_nat\_functor\_algebra\_mor}]
  \label[lemma]{lemma:mk-nat-alg-mor}
  Let $\dpair{X}{α}$ and $\dpair{Y}{β}$ be $P$-algebras. Any function $f:X → Y$
  which sends $x_0$ to $y_0$ and intertwines $\suc_X$ with $\suc_Y$ is an
  algebra morphism.
\end{lemma}
\begin{proof}
	To show that $f$ is an algebra morphism (\cref{def:algebra-coalgebra-in-utt}),
  we must demonstrate that the following diagram commutes:
  \begin{center}
    \begin{tikzcd}
      PX  \arrow[d, "α"] \arrow[r, "P^*f"] & PY \arrow[d, "β"] \\
      X \arrow[r, "f"] & Y
    \end{tikzcd}
  \end{center}
  where $\apply{P^*f}{\dpair{b}{h}}\defeq \dpair{b}{f ∘ h}$
  (\cref{def:polynomial-functor}).
  By function extensionality (\cref{thm:funext}), it suffices to show that the
  two composites are equal on all inputs. Let $\dpair{b}{h}:PX$, and proceed by
  cases on $b:\𝔹$.

  If $b\≡\bfalse$, then $h:\𝟘 → X$. Recall that $\emptytype$ is terminal in
  $\universe$; the type of functions $\𝟘 → Z$ is contractible for any
  $Z:\universe$. Thus,
  \begin{align*}
    \apply{(β∘(P^*f))}{\dpair{\bfalse}{h}}
    &\jdeq \apply{β}{\dpair{\bfalse}{f ∘ h}} \\
    &= \apply{β}{\dpair{\bfalse}{\apply{\rec_{\𝟘}}{Y}}}
    &&\quad\text{Terminality of }\𝟘\\
    &\jdeq y_0
    &&\quad\text{Definition} \\
    &= \apply{f}{x_0}
    &&\quad\text{Hypothesis} \\
    &\jdeq \apply{(f∘α)}{\dpair{\bfalse}{\apply{\rec_{\𝟘}}{X}}} \\
    &= \apply{(f∘α)}{\dpair{\bfalse}{h}}
    &&\quad\text{Terminality of }\𝟘
  \end{align*}

  If $b\≡\btrue$, then $h:\𝟙 → X$. By the induction principle for
  $\𝟙$\TODO{reference}, for any type $Z:\universe$, any function $k:\𝟙 → Z$, and
  any $x:\𝟙$, $\apply{k}{x}=\apply{k}{\unitelem}$. In short,
  $k=\appply{\rec_{\𝟙}}{Z}{(\apply{k}{\unitelem})}$. In this case,
  \begin{align*}
    \apply{(β∘(P^*f))}{\dpair{\btrue}{h}}
    &\jdeq \apply{β}{\dpair{\btrue}{f ∘ h}} \\
    &\= \apply{β}{\dpair{\btrue}{f ∘ (\appply{\rec_{\𝟙}}{Y}{(\apply{h}{\unitelem})})}} \\
    &\= \apply{β}{\dpair{\btrue}{\appply{\rec_{\𝟙}}{Y}{(\apply{f}{(\apply{h}{\unitelem})})}}}
    &&\quad\text{Function ext.} \\
    &\jdeq \apply{\suc_Y}{(\apply{f}{(\apply{h}{\unitelem})})}
    &&\quad\text{Definition of }\suc_Y \\
    &\= \apply{f}{(\apply{\suc_X}{(\apply{h}{\unitelem})})}
    &&\quad\text{Hypothesis} \\
    &\jdeq \apply{f}{(\apply{α}{\dpair{\btrue}{\fromunit{X}{(\apply{h}{\unitelem})}}})}
    &&\quad\text{Definition of }\suc_X \\
    &\jdeq \apply{(f∘α)}{\dpair{\btrue}{\fromunit{X}{(\apply{h}{\unitelem})}}}
    &&\quad\text{Definition} \\
    &\= \apply{(f∘α)}{\dpair{\btrue}{h}}
    &&\quad\text{Function ext.}
  \end{align*}
\end{proof}

\begin{lemma}[\coqname{W.Naturals.nat\_alg\_is\_preinitial}]
	For any $P$-algebra $\dpair{X}{α}$, there is an algebra morphism
  $\dpair{ℕ}{η} ⇒ \dpair{X}{α}$.
\end{lemma}
\begin{proof}
	Define $x_0$ and $\suc_X$ as in the above proof.
  Now define a function $g$ recursively:
  \begin{alignat*}{2}
    &g : ℕ → X \\
    &\apply{g}{0}                 &&\defeq x_0 \\
    &\apply{g}{(\apply{\suc}{n})} &&\defeq \suc_X(g(n))
  \end{alignat*}
  By definition $g$ sends the distinguished point $x_0$ to $0$ and intertwines
  $\suc$ with $\suc_X$; by \cref{lemma:nat-alg-mor}, it's an algebra morphism.
\end{proof}

\begin{lemma}[\coqname{W.Naturals.nat\_alg\_func\_is\_unique}]
  \label[lemma]{lemma:nat-alg-mor-fun-unique}
  The underlying function of the above morphism is unique. That is,
	for any algebra morphism $\dpair{τ}{i}:\dpair{ℕ}{η} ⇒ \dpair{X}{α}$,
  $\propeq{}{τ}{g}$, with $g$ as in the previous proof.
\end{lemma}
\begin{proof}
  By function extensionality (\cref{thm:funext}), it suffices to show that they
  are equal on all inputs. We know that the following diagrams commute:
  \begin{center}
    \begin{minipage}[b]{0.48\linewidth}
      \centering
      \begin{tikzcd}
        P\ℕ  \arrow[d, "η"] \arrow[r, "P^*g"] & PX \arrow[d, "α"] \\
        \ℕ \arrow[r, "g"] & X
      \end{tikzcd}
    \end{minipage}
    \begin{minipage}[b]{0.48\linewidth}
      \centering
      \begin{tikzcd}
        P\ℕ  \arrow[d, "η"] \arrow[r, "P^*τ"] & PX \arrow[d, "α"] \\
        \ℕ \arrow[r, "τ"] & X
      \end{tikzcd}
    \end{minipage}
  \end{center}
  Using function extensionality, proceed
  by induction on the argument.
  % by cases on the constructor of the argument.

  % mor 0 =
  % (x ∘ polynomial_functor_arr bool (bool_rect (λ _ : bool, UU) ∅ unit) mor)
  %   (true,, fromempty)
  (Base case) In the zero case, we have
  $\dpair{\bfalse}{\apply{\rec_{\𝟘}}{\ℕ}}:P\ℕ$. Then
  % By definition of $η$ and the
  % commuting condition for $τ$ (in the diagram above),
  \begin{align*}
    \apply{τ}{0}
    &= \apply{τ}{(\apply{η}{\dpair{\bfalse}{\apply{\rec_{\𝟘}}{\ℕ}}})}
    &&\quad\text{Definition of }η \\
    &\jdeq \apply{(τ∘η)}{\dpair{\bfalse}{\apply{\rec_{\𝟘}}{\ℕ}}} \\
    &= \apply{(α∘P^*τ)}{\dpair{\bfalse}{\apply{\rec_{\𝟘}}{\ℕ}}}
    &&\quad\text{Commuting condition} \\
    &\jdeq \apply{α}{\dpair{\bfalse}{τ∘\apply{\rec_{\𝟘}}{\ℕ}}}
    &&\quad\text{Definition of }P^* \\
    &= \apply{α}{\dpair{\bfalse}{\apply{\rec_{\𝟘}}{X}}}
    &&\quad\text{Contractibility of }\𝟘 → X\\
    &\jdeq x_0 \\
    &\jdeq \apply{g}{0}.
  \end{align*}

  % (false,, (λ _ : bool_rect (λ _ : bool, UU) ∅ unit false, n0))
  (Inductive step) Suppose the hypothesis holds for $n:\ℕ$. Then
  $\dpair{\btrue}{\apply{\rec_{\𝟙}}{\ℕ}{n}}:P\ℕ$.
  Recall the definition of $f$ given in the previous proof. Similarly to the
  above reasoning,
  \begin{align*}
    \apply{τ}{(\apply{\suc}{n})}
    &= \apply{τ}{(\apply{η}{\dpair{\btrue}{\appply{\rec_{\𝟙}}{\ℕ}{n}}})}
    &&\quad\text{Definition of }η \\
    &\jdeq \apply{(τ∘η)}{\dpair{\btrue}{\appply{\rec_{\𝟙}}{\ℕ}{n}}} \\
    &= \apply{(α∘P^*τ)}{\dpair{\btrue}{\appply{\rec_{\𝟙}}{\ℕ}{n}}}
    &&\quad\text{Commuting condition} \\
    &\jdeq \apply{α}{\dpair{\btrue}{τ∘\appply{\rec_{\𝟙}}{\ℕ}{n}}}
    &&\quad\text{Definition of }P^* \\
    &= \apply{α}{\dpair{\btrue}{\appply{\rec_{\𝟙}}{X}{(\apply{τ}{n}})}} \\
    &\jdeq \apply{\suc_X}{(\apply{τ}{n})} \\
    &= \apply{\suc_X}{(\apply{g}{n})}
    &&\quad\text{Inductive hypothesis} \\
    &\jdeq \apply{g}{(\apply{\suc}{n})}.
    &&\quad\text{Definition of }g
  \end{align*}
\end{proof}

Similar to the result of \cref{lemma:nat-alg-simpl}, fibered algebras over $\ℕ$
also have a simple structure understood by an equivalence.

\begin{lemma}[\coqname{W.Naturals.fibered\_algebra\_nat}]
  \label[lemma]{lemma:nat-fib-alg-simpl}
  A fibered algebra structure (over $\ℕ$) on a family
  $E:\ℕ\to\universe$ is given by a point $e_0:E_0$
  and a family of functions $f^E:\pit{n:\ℕ}{E_n\to E_{n+1}}$. Specifically,
  there is an equivalence
  \begin{equation*}
    \weq{\Fibalgtype(\dpair{ℕ}{\suc})}
        {\∑{E:\universe}E_0×\pit{n:\ℕ}{E_n\to E_{n+1}}}
  \end{equation*}
\end{lemma}
\begin{proof}
	\TODO{proof?}
\end{proof}

And like \cref{lemma:mk-nat-alg-mor}, we can also simplify our understanding of
algebra sections.

\begin{lemma}[\coqname{W.Naturals.mk\_nat\_alg\_sec}]
  \label[lemma]{lemma:mk-nat-alg-sec}
  % (p1 : x 0 = pr1 (pr2 FA'))
  % (p2 : (∏ n, pr2 (pr2 FA') n (x n) = x (S n))),
  Let $\dpair{E}{θ}$ be a fibered $P$-algebra over $\dpair{\ℕ}{\suc}$.
  If $x:\∏{n:\ℕ}{X}$ sends $0$ to $e_0$ and for all $n$,
  $\propeq{}{\apply{f^E}{n}}{\apply{x}{(\apply{\suc}{n})}}$,
  then there is a corresponding algebra section with $x$ as the underlying
  function.
\end{lemma}
\begin{proof}
	\TODO{proof?}
\end{proof}

\begin{lemma}\label[lemma]{lemma:nat-alg-sec-equiv}
  The above function is an equivalence.
\end{lemma}

This result is proved in the \Agda{} formalization accompanying
\cite{non-wellfounded}. Its proof is outside of the scope

\begin{lemma}[\coqname{W.Naturals.nat\_alg\_is\_preinitial\_sec}]
  \label[lemma]{lemma:fiber-preinitiality-nat}
	The natural numbers are fiber-preinitial (\cref{def:fiber-preinitiality}).
\end{lemma}
\begin{proof}
	Suppose $\dpair{E}{θ}$ is a fibered algebra over $\ℕ$
  (\cref{def:algebra-fibration}) so that $E:\ℕ→\universe$ and
  \begin{equation*}
    θ:\∏{\dpair{n}{h}:P\ℕ}{\paren{\∏{b:\appppply{\rec_{\𝟚}}{\universe}{\𝟘}{\𝟙}{n}}{\apply{E}{(\apply{h}{b})}}} → \apply{E}{(\apply{η}{\dpair{n}{h}})}}.
  \end{equation*}
  The first part of an algebra section is a function $f:\∏{n:\ℕ}{\apply{E}{n}}$.
  Define $f$ by induction:
  \begin{alignat*}{2}
    &f : \∏{n:\ℕ}{\apply{E}{n}}   &&\\
    &\apply{f}{0}                 &&\defeq
    \appply{θ}{\dpair{\bfalse}{\fromempty{ℕ}}}
              {(\fromempty{(E∘\fromempty{\ℕ})})} \\
    &\apply{f}{(\apply{\suc}{n})} &&\defeq
    \appply{θ}{\dpair{\btrue}{\appply{\rec_{\𝟙}}{\ℕ}{n}}}
              {(\fromunit{(\apply{E}{n})}{(\apply{f}{n})})}
  \end{alignat*}
  Let $\dpair{b}{h}:P_{A,B}\ℕ$. It remains to show that this function is indeed
  a section, namely that
  \begin{equation*}
    \propeq
      {}
      {\apply{f}{(\apply{η}{\dpair{b}{h}})}}
      {\appply{θ}{\dpair{b}{h}}{(f∘h)}}
  \end{equation*}
  Proceed by cases on $b$.

  If $b\≡\bfalse$, then $h:\𝟘 → ℕ$. In this case,
  \begin{align*}
    \apply{f}{(\apply{η}{\dpair{\bfalse}{h}})}
    &\≡ \apply{f}{0}
    &&\quad\text{Definition of }η \\
    &\≡ \appply{θ}{\dpair{\bfalse}{\fromempty{ℕ}}}
                  {(\fromempty{(E∘\fromempty{\ℕ})})}
    &&\quad\text{Definition of }f \\
    &\= \appply{θ}{\dpair{\bfalse}{h}}{(f∘h)}
  \end{align*}
  by the contractibility of types $\𝟘 → \ℕ$ and
  $(E∘\fromempty{\ℕ}) → \ℕ$.\TODO{reference}

  If $b\≡\btrue$, then $h:\𝟙 → ℕ$.
  \begin{align*}
    \apply{f}{(\apply{η}{\dpair{\btrue}{h}})}
    &\≡ \apply{f}{(\apply{\suc}{(\apply{h}{\unitelem})})} \\
    &\≡ \appply{θ}{\dpair{\btrue}{\appply{\rec_{\𝟙}}{\ℕ}{(\apply{h}{\unitelem})}}}
                  {\Big(\fromunit{(\apply{E}{(\apply{h}{\unitelem})})}
                             {(\apply{f}{(\apply{h}{\unitelem})})}\Big)} \\
    &\= \appply{θ}{\dpair{\btrue}{h}}{(f∘h)}
  \end{align*}
  where the last equality is by two applications of function extensionality
  (\cref{thm:funext}) and the contractibility for $\𝟙$.
\end{proof}

\begin{lemma}\label[lemma]{lemma:fiber-initiality-nat}
	The natural numbers are fiber-initial (\cref{def:fiber-initiality}).
\end{lemma}

% \begin{lemma}\label[lemma]{lemma:algebra-sections}
%   Let $X$ be a $P$-algebra with $α:PX→X$, and let $C:X→\universe$. The standard
%   projection map $π_1:\∑{x:X}{Cx}→X$ has a section $s:\∏{x:X}{Cx}$
%   if $\∑{x:X}{Cx}$ also has a $P$-algebra structure, as in
%   \begin{center}
%     \begin{tikzcd}
%       {} & \∑{x:X}{Cx} \arrow[d, "π_1"] \\
%       X \arrow[ur, "s"] \arrow[r, equal] & X
%     \end{tikzcd}
%   \end{center}
%   In this case, $s$ is a $P$-algebra morphism \cite{inductive}.
% \end{lemma}

\subsection{Limits of (co)chains}
\label{subsec:limits-of-cochains}

\begin{lemma}[\coqname{Chains.shifted\_limit}]
  Let $(X,π)$
\end{lemma}
\end{document}