\documentclass[./thesis.tex]{subfiles}
\begin{document}
\chapter{Category theory}
\label{chap:category-theory}

\epigraph{Of the branches of mathematics, category theory is one which perhaps
fits the least comfortably into existing ``foundations of
mathematics''.}{\cite{rezk}}

Category theory presents a challange to set-theoretic mathematicians: the
canonical example of a category is $\Set$, the collection of all sets. In
\ZFC+\FOL{}, this category is undefinable\footnote{In
  particular, its formation is prevented by the axiom of regularity
  \cite{vonneumann}, which was included in \ZFC{} to avoid the
  paradoxes of Burali-Forti and Russell. The discovery of said paradoxes
  motivated Bertrand Russell to invent something he called the ``theory of
  types'' \cite{russell}.}.
\Crefrange{sec:basics}{sec:functors-and-their-algebras},
are willfully imprecise, we work with an abstract and undefined notion of
``collection'', and adopt set-theoretic notation. See any textbook on category
theory for information on how problems of size are dealt with set-theoretically.
We will examine category theory within \UTT{} in
\cref{sec:type-theoretic-category-theory}.

The results in this chapter all appear in the \UniMath{} package
\unimathname{CategoryTheory}, so that prefix is left off
(e.g.\ \unimathname{CategoryTheory.limits.binproducts} is shortened to
\unimathname{limits.binproducts}).

\section{Basics}
\label{sec:basics}

\begin{definition}\label[definition]{def:category}
	A \define{category}\index{Category}\footnote{In more general contexts, this
    may be called a \define{$1$-category} or even a $(1,1)$-category.}
  $\bfC$ consists of the following data:
  \begin{itemize}
    \itemsep-0.2em
    \item a collection of \define{objects}, denoted $\Obj \bfC$,
    \item for each pair of objects $A,B\in\Obj \bfC$, a collection of
      \define{arrows} (or \define{morphisms}) between them, denoted
        $\Hom_{\bfC}(A,B)$,
    \item for each object $A\in\Obj \bfC$, a distinguished arrow
      $\id_A\in\Hom_{\bfC}(A,A)$ called the
      \define{identity}\index{Identity!Morphism}, and
    \item for each triple of objects $A,B,C\in\Obj\bfC$, an operation \\
      $∘:\Hom_{\bfC}(B,C)\times\Hom_{\bfC}(A,B)\to\Hom_{\bfC}(A,C)$ called
      \define{composition}\index{Composition!In a category}.
  \end{itemize}
  These data are subject to the following axioms:
  \begin{enumerate}%[label=\Alph*.]
    \itemsep-0.2em
    \item composition is associative, and
    \item the identity acts as a unit for composition.
  \end{enumerate}
  When the category in question is clear from context, one writes
  $f:A\catarrow B$ for $f\in\Hom_{\bfC}(A,B)$.\footnote{We use $\catarrow$ to
    distinguish this from the type of functions $A→B$.}
\end{definition}

\begin{definition}\label[definition]{def:domain-and-codomain}
  If $f\in\Hom_{\bfC}(A,B)$, then $A$ is the \define{domain}\index{Domain} or
  \define{source}\index{Source} of $f$ and $B$ is the
  \define{codomain}\index{Codomain} or \define{target}\index{Target} of $f$.
\end{definition}

The presentation of the following examples owes a lot to \cite{context}.

\begin{example}
	The following categories are familiar to the student of
  mathematics:
  \begin{itemize}
    \itemsep-0.2em
    \item $\Set$: The category with sets as objects, functions as morphisms,
      the usual composition of functions, and identity functions.
    \item $\FinSet$: The category with finite sets as objects, functions as
      morphisms, the usual composition of functions, and identity functions.
    \item $\Grp$: The category of groups\index{Group} with group homomorphisms
      as morphisms. Note that the identity function of sets is the required
      identity morphism and that for any homomorphsisms $ϕ:G\to H$ and
      $ψ:H\to I$, the usual composition of functions defines a homomorphism
      $ψ∘ ϕ:G\to I$.
    \item $\AbGrp$: The category of abelian groups (this can be considered a
      \define{subcategory} of $\Grp$).
    \item $F\dVect$: The category of vector spaces\index{Vector space} over a
      field $F$ with linear transformations as morphisms.
  \end{itemize}
\end{example}

\begin{example}
  Morphisms don't have to be functions:
  \begin{itemize}
    \itemsep-0.2em
    \item For any group $(G,∘,e)$, there is a corresponding category
      $\underline{G}$ with a single object (denoted $∗$) and
      $\Hom_{\underline{G}}(∗,∗)\coloneqq G$. Composition is given by the
      group operation.
    \item For any set or type $A$, the \define{discrete category on $A$} has as
      objects the members or elements as $A$, and no other arrows except the
      requisite identities.
    \item For any preorder $(A,≤)$ (a collection with a reflexive, transitive
      binary relation), there is a category $\bfC(A)$ which has as objects the
      elements of $A$ and an arrow $a→b$ just if $a≤b$.
      Conversely, any category with at most one arrow between any two objects
      defines a preorder.
    \item The category $\ω$ is the preorder category $\bfC(\ℕ)$, that is,
      it has as objects the natural numbers, and arrows
      $n\catarrow\apply{\suc}{n}$ (plus the requisite composites $n\catarrow m$
      where $n<m$).
    \item There is a category $\bf1$ with one object and only the identity
      arrow, called the \define{unit} or (foreshadowing
      \cref{sec:limits-and-colimits}) \define{terminal} category.
  \end{itemize}
\end{example}

For the rest of the chapter, we deal only with categories in the abstract rather
than particular examples. To this end, let $\bfC$ denote an arbitrary category.

\begin{definition}\label[definition]{def:isomorphism}
	An \define{isomorphism}\index{Isomorphism!In a category} is an arrow
  $f:A\catarrow B$ such that there exists an arrow $g:B\catarrow A$
  such that $g∘f=\id_A$ and $f∘g=\id_B$. Such a $g$ is the
  \define{inverse}\index{Inverse!In a category} of $f$.
\end{definition}

Just as with functions, inverses (should they exist) are unique.

\begin{figure}[ht]
  \centering
  \begin{tikzpicture}[scale=2]
    \node (A) [] {$∗$};

    \foreach \x in {1,2,3}
      \draw (A) edge [in={50*\x},out={40+50*\x},loop] node[above] {$g_\x$} (A);

    \draw (A) edge [in=200,out=240,loop] node[below] {$\cdots$} (A);
    \draw (A) edge [in=-60,out=-20,loop] node[below] {$e$} (A);
  \end{tikzpicture}
  \caption{
    \label{fig:grp} A schematic of the category $\underline{G}$ for a
    group $G$ with identity $e$ and elements $g_1,g_2,\ldots$. Compositions and
    inverses not shown.
  }
\end{figure}

\begin{definition}\label[definition]{def:commutative-diagram}
  A commutative diagram is a way to visualize equations between arrows
  involving composition. Technically a diagram in $\bfC$ is a directed graph
  with vertices labeled by $\Obj\bfC$ and edges $e$ from $A$ to $B$
  labeled by arrows in $\Hom_\bfC(A,B)$. A diagram \define{commutes}, or is
  commutative, if the composition of the arrows labeling the edges of any two
  directed paths with the same endpoints are equal.
\end{definition}

\begin{example}\label[example]{ex:commutative-diagram}
	If $f:A\catarrow B$, $g:B\catarrow C$, $h:A\catarrow C$,
  and $g∘f=h$, then the following diagram commutes, making a
  \define{commutative triangle}:
  \begin{center}
    \begin{tikzcd}[column sep=large]
      A \arrow[r, "f"] \arrow[dr, "h"] & B \arrow[d, "g"] \\
      {}                               & C
    \end{tikzcd}
  \end{center}
	If $f:A\catarrow B$, $g:B\catarrow D$, $h:A\catarrow C$,
  $i:C\catarrow D$, and $g∘f=i∘h$, then the following diagram commutes,
  making a \define{commutative square}:
  \begin{center}
    \begin{tikzcd}[column sep=large]
      A \arrow[r, "f"]  \arrow[d, "h"] & B \arrow[d, "g"] \\
      C \arrow[r, "i"]                 & D
    \end{tikzcd}
  \end{center}
\end{example}

\section{Duality}
\label{sec:duality}

Consider a preorder $(A,≤)$ (that is, a set or type $A$ together with a
reflexive, transitive relation $≤$). Its \define{opposite preorder} is
$(A,≤^{\op})$ where
\begin{equation*}
	a ≤^{\op} b \Longleftrightarrow b ≤ a
\end{equation*}
Consider a group $(G,·,e)$. Its \define{opposite group} has the operation
\begin{equation*}
  g ·^\op h \defeq h · g
\end{equation*}
Now, suppose $(A,≤)$ has all finite meets. Immediately, $(A,≤^\op)$ has all finite
joins. In general, if we know some theorem $T$ about $(A,≤)$ or $(G,·,e)$. It
seems like there should be a purely mechanical process via which one could
discover an appropritate ``opposite theorem'' that applies to $(A,≤^\op)$ or
$(G,·^\op,e)$.

These are instances of the notion of duality in category theory. It plays an
economizing role: one \textit{can}, in fact, state ``opposite'' theorems via a
purely mechanical process, and prove them trivially.

\begin{definition}
  The opposite category of $\bfC$, denoted $\bfC^\op$ has
  \begin{itemize}
    \itemsep0em
    \item objects $\Obj\bfC^\op\defeq \Obj\bfC$,
    \item arrows $f^\op∈\Hom_{\bfC^op}(B,A)$ for each $f∈\Hom_{\bfC}(A,B)$,
    \item and composition $f^\op∘^\op g^\op\defeq (g\circ f)^\op$.
  \end{itemize}
\end{definition}

One usually identifies $(\bfC^\op)^\op$ with $\bfC$.\footnote{This is a bit more
  subtle in type theory! What one wants is a \textit{judgmental} equality
  $(\bfC^\op)^\op\jdeq \bfC$. To make this possible, one must include a
  symmetrized version of the associativity axiom.}
The following might be called ``The Fundamental Meta-theorem of Category Theory'':

\begin{theorem}\label[theorem]{thm:duality}
  If $\mvar{S}$ is a statement about $\bfC$ in the first-order language of
  category theory (i.e.\ involving only objects, arrows, composition, and
  identity), then the statement $\mvar{S}^\op$ holds of $\bfC^\op$, where
  $\mvar{S}^\op$ is obtained from $\mvar{S}$ by
  \begin{enumerate}%[label=\Alph*.]
    \itemsep0em
    \item switching the domain and codomain of all arrows, and
    \item reversing composition.
  \end{enumerate}
\end{theorem}

Usefully, isomorphisms are self-dual. This means objects are isomorphic in
$\bfC$ if and only if they are so in $\bfC^\op$.

If $D$ is a definition in the language of categories, there is a dual
definition $D^\op$, obtained in the same way as $S^\op$ above. If $D$ defines a
``foo'', then the thing that $D^\op$ defines is usually called a ``cofoo''.
From now on, dual definitions and statements will be introduced in pairs, and
typeset like so:

\dual{
  Definition
}{
  Co-definition
}

\section{Limits and colimits}
\label{sec:limits-and-colimits}

Many constructions in categories are defined via a \textit{universal property}.
Such characterizations abound in everyday mathematics, as the examples
illustrate. All of the definitions to follow are examples of limits and colimits.
Ultimately, M-types are constructed as limits.

\begin{definition}\label[definition]{def:terminal-and-initial}
  \
  \vspace{-0.3em}\dual{
    A \define{terminal object}\index{Terminal object} is an
    object $⊤\in\Obj\bfC$ such that for all $A\in\Obj\bfC$,
    there is exactly one arrow $f:A\catarrow ⊤$.
  }{
    An \define{initial object}\index{Initial object} is an
    object $⊥\in\Obj\bfC$ such that for all $A\in\Obj\bfC$,
    there is exactly one arrow $f:⊥\catarrow A$.
  }
\end{definition}

\begin{example} \
  \begin{itemize}
    \itemsep-0.2em
    \item $\Set$: 
      \begin{itemize}\renewcommand{\labelitemi}{$∘$}
        \itemsep-0.2em
        \item \vspace{-0.6em} Any singleton set $\braces{∗}$ is terminal.
        \item The empty set $\emptyset$ is initial.\footnote{This is a little
            odd: in set-theoretic foundations, functions are actually defined
            as certain kinds of relations (``functional'' ones). Accordingly,
            the unique function out of them empty set is the unique subset
            of $\emptyset×X=\emptyset$, namely $\emptyset$ itself.}
      \end{itemize}
    \item $\universe$: 
      \begin{itemize}\renewcommand{\labelitemi}{$∘$}
        \itemsep-0.2em
        \item{} \vspace{-0.6em}
          [\unimathname{categories.Types.TerminalType}]
          Any contractible type is terminal (for the
          similarity to the case of sets, note that any contractible type is
          equivalent to the canonical type with one element, $\unittype$
          (\cref{lemma:contr-weq-unit})).
        \item{} [\unimathname{categories.Types.InitialType}]
          The elimination rule for the empty type gives an arrow
          $\emptytype→X$ for any $X$. Is it unique? Apply function
          extensionality (\cref{thm:funext}), then it suffices to show that it
          has the same output as any competitor $f:\emptytype\to X$ on some
          $e:\emptytype$. But using the elimination rule for $\emptytype$ on
          $e$, conclude that it does.
      \end{itemize}
    \item $\Cat$: 
      \begin{itemize}\renewcommand{\labelitemi}{$∘$}
        \itemsep-0.2em
        \item \vspace{-0.6em} The category with one object and one arrow (its
          identity) is terminal.
        \item The empty category is initial, for the same reasons as in $\Set$
          or $\universe$ (depending on which foundational system one is working
          in).
      \end{itemize}
    \item $\Grp$, $\AbGrp$: The group with one element (``trivial group'') is
      initial \textit{and} terminal. Such an object is, in general, called a
      \define{zero object}.
    \item $F\dVect$: Similarly, the trivial vector space is a zero object (its
      underlying additive abelian group of vectors is the trivial group!)
  \end{itemize}
\end{example}

\begin{remark}\label[remark]{remark:terminal-object-id-unique}
	In particular, for a terminal object $⊤$, $\id_⊤$ is the only arrow
  $⊤\catarrow ⊤$.\footnote{As noted in \cref{sec:duality}, this statement
    holds for initial objects as well. From this point on, we will leave it
    to the reader to construct the dual of a statement and infer its truth.}
\end{remark}

\begin{lemma}\label[lemma]{lemma:terminal-unique}
  Terminal objects are unique up to a specified isomorphism.
\end{lemma}
\begin{proof}
	Suppose $A$ and $B$ are terminal objects.
  There are unique arrows $f:A\catarrow B$ and $g:B\catarrow A$.
  Then $g∘ f:A\catarrow A$ and $f∘ g:B\catarrow B$, but as per
  \cref{remark:terminal-object-id-unique}
  $g∘ f=\id_A$ and $f∘ g=\id_B$.
\end{proof}

\begin{definition}\label[definition]{def:product-and-coproduct}
  Given two objects $A,B\in\Obj\bfC$,
  \vspace{-0.3em}\dual{
    a \define{(binary) product}\index{Product!In a category} of $A$ and $B$
    consists of an object $C\in\Obj\bfC$ together with arrows $p_1:C\catarrow A$ and
    $p_2:C\catarrow B$ satisfying the following universal property:

    For any other ``candidate product'' $D\in\Obj\bfC$ with arrows
    $q_1:D\catarrow A$ and $q_2:D\catarrow B$, there is a unique arrow $u:D\catarrow C$ making the
    following diagram commute:
    \begin{center}
      \begin{tikzcd}[sep=large,ampersand replacement=\&]
        {} \& D\arrow[dl,swap,"q_1"]\arrow[dr,"q_2"]
              \arrow[d,dashed,"u"] \& {} \\
        A \& C \arrow[l, "p_1"]\arrow[r,swap, "p_2"] \& B
      \end{tikzcd}
    \end{center}
    A product of $A$ and $B$ is denoted by $A\times B$, and the unque arrow $u$
    as $\langle f,g \rangle$.
  }{
    a \define{(binary) coproduct}\index{Coproduct!In a category} of $A$ and $B$
    consists of an object $C\in\Obj\bfC$ together with arrows $i_1:A\catarrow C$ and
    $i_2:B\catarrow C$ satisfying the following universal property:

    For any other ``candidate coproduct'' $D\in\Obj\bfC$ with arrows
    $j_1:A\catarrow D$ and $j_2:B\catarrow D$, there is a unique arrow $u:C\catarrow D$ making the
    following diagram commute:
    \begin{center}
      \begin{tikzcd}[sep=large,ampersand replacement=\&]
        A \arrow[r, "i_1"{name=I1}]\arrow[dr, swap, "j_1"{name=F}]
        \& C \arrow[d,dashed,"u"]
        \& B\arrow[l, "i_2"{name=I2},swap]\arrow[dl,"j_2"{name=G}] \\
        {} \& D \& {}
      \end{tikzcd}
    \end{center}
    A coproduct $A$ and $B$ is denoted $A+B$, and the unque arrow $u$ as
    $[f,g]$.
  }
\end{definition}

\begin{remark}\label[remark]{rmk:prod-coprod-universal-adj}
	The universal properties of the product and coproduct define bijections
  (there is one and only one function that fills in the diagram)
  \begin{align*}
    \Hom(A,C)×\Hom(B,C)&≅ \Hom(A+B,C)\text{ and} \\
    \Hom(C,A)×\Hom(C,B)&≅ \Hom(C,A×B).
  \end{align*}
  With one additional criterion (called ``naturality''), this can be taken as an
  alternative definition.
\end{remark}

\begin{example} \
  \begin{itemize}
    \itemsep-0.2em
    \item $\Set$: Let $A$ and $B$ be sets.
      \begin{itemize}\renewcommand{\labelitemi}{$∘$}
        \itemsep-0.2em
        \item \vspace{-0.6em} Their Cartesian product
          $A×B\coloneqq\braces{(a,b):a∈A,b∈B}$ together with projections
          $\apply{p_1}{(a,b)}\coloneqq a$ and $\apply{p_2}{(a,b)}\coloneqq b$
          is a product.
        \item Their disjoint union
          $A\amalg B\coloneqq\braces{(0,a):a∈A}∪\braces{(1,b):b∈B}$ together
          with inclusions $\apply{i_1}{a}\coloneqq (0,a)$
          and $\apply{i_2}{b}\coloneqq (1,b)$ is a coproduct.\footnote{Of
            course, the identities of the ``labels'' $0$ and $1$ are
            unimportant, so long as they are distinct. One could equally well
            use ``$\inl$'' and ``$\inr$'' so long as $\inl≠\inr$.}
      \end{itemize}
    \item $\universe$: Let $A,B:\universe$.
      \begin{itemize}\renewcommand{\labelitemi}{$∘$}
        \itemsep-0.2em
        \item{} \vspace{-0.6em} [\unimathname{categories.Types.BinProductsType}]
          The product type $A×B$ with its projections is a product.
          (See \label[lemma]{lemma:universe-binproducts}).
        \item{} [\unimathname{categories.Types.BinCoproductsType}]
          The coproduct type $A+B$ with its injections is a coproduct.
          The universal property is guaranteed by $\rec_{A+B}$, and can be 
          expressed concisely as an equivalence $\weq{(A→C)×(B→C)}{A+B→C}$.
      \end{itemize}
    \item $\Cat$: 
      \begin{itemize}\renewcommand{\labelitemi}{$∘$}
        \itemsep-0.2em
        \item \vspace{-0.6em} See \cref{ex:diagonal}
        \item The coproduct of categories $\bfA$ and $\bfB$ is the category
          $\bfA+\bfB$ with objects
          $\Obj (\bfA+\bfB) \coloneqq (\Obj\bfA)+(\Obj\bfB)$
          and arrows
          \begin{align*}
            \Hom_{\bfA+\bfB}(\apply{\inl}{A}, \apply{\inl}{A'}) &= \Hom_{\bfA}(A,A') \\
            \Hom_{\bfA+\bfB}(\apply{\inr}{B}, \apply{\inr}{B'}) &= \Hom_{\bfB}(B,B') \\
            \Hom_{\bfA+\bfB}(\apply{\inl}{A}, \apply{\inr}{B}) &= \emptytype \\
            \Hom_{\bfA+\bfB}(\apply{\inr}{A}, \apply{\inl}{B}) &= \emptytype.
          \end{align*}
      \end{itemize}
    \item $\Grp$: Let $G,H$ be groups.
      \begin{itemize}\renewcommand{\labelitemi}{$∘$}
        \itemsep-0.2em
        \item \vspace{-0.6em} The direct product group $G×H$ is a product.
        \item The free product $G∗H$ is a coproduct.
      \end{itemize}
    \item $\AbGrp$, $R\Mod$, $F\dVect$: The direct sum $⊕$ is both a product and
      coproduct.\footnote{In fact, $⊗$ has a stronger property of being a
      \define{biproduct}. These three categories share many significant
      features; one fruitful lens through which to view them is as
      \define{abelian categories}.} 
  \end{itemize}
\end{example}

\begin{lemma}
	(Co)products are unique up to a specified isomorphism.
\end{lemma}
\begin{proof}
  Note that for any coproduct $C$ of $A$ and $B$, the identity makes the
  following diagram ($⋆$) commute, and so is necessarily unique in doing so:
  \begin{center}
    \begin{tikzcd}[sep=large,ampersand replacement=\&]
      A \arrow[r, "i_1"{name=I1}]\arrow[dr, swap, "i_1"{name=F}]
      \& C \arrow[d,"\id_C"]
      \& B\arrow[l, "i_2"{name=I2},swap]\arrow[dl,"i_2"{name=G}] \\
      {} \& C \& {}
    \end{tikzcd}
  \end{center}
	Suppose $C$ and $D$ are coproducts of $A$ and $B$. The universal
  property of the coproduct induces arrows $u$ and $u'$ making the following
  diagram commute:
  \begin{center}
    \begin{tikzcd}[column sep=large,ampersand replacement=\&]
      {} \& C \arrow[dd, dashed, "u", bend left] \& {} \\
      A \arrow[ur]\arrow[dr]
      \& {}
      \& B\arrow[ul]\arrow[dl] \\
      {} \& D \arrow[uu, dashed, "u'", bend left] \& {}
    \end{tikzcd}
  \end{center}
  However by this commutativity, the compositions $u∘u'$ and $u'∘u$ fit into the
  diagram ($⋆$), and so must be the respective identities, making $C$ and $D$
  isomorphic.
\end{proof}

\begin{definition}\label[definition]{def:equalizer-and-coequalizer}
  Given objects $A,B∈\Obj\bfC$ and arrows $f,g:A\catarrow B$,
  \vspace{-0.3em}\dual{
    an \define{equalizer}\index{Equalizer} of $f$ and $g$
    consists of an object $E$ and an arrow $e:E\catarrow A$ satisfying the
    following universal property:

    For any other ``candidate equalizer'' $C$ with arrow $c:C\catarrow A$, there
    is a unique arrow $u:C\catarrow E$ making the following diagram commute:
    \begin{center}
      \begin{tikzcd}[column sep=large, ampersand replacement=\&]
        E \arrow[r, "e"]
        \& A \arrow[r, yshift=2.5,"f"]\arrow[r, yshift=-2.5, swap,"g"]
        \& B \\
        C \arrow[ur, swap, "c"]
          \arrow[u, dashed, "u"] \& {} \& {}
      \end{tikzcd}
    \end{center}
  }{
    a \define{coequalizer}\index{Coequalizer} of $f,g$ consists of an object $Q$
    and an arrow $q:B\catarrow Q$ satisfying the following universal property:

    For any other ``candidate coequalizer'' $C$ with arrow $c:B\catarrow C$,
    there is a unique arrow $u:Q\catarrow C$ making the following diagram commute:
    \begin{center}
      \begin{tikzcd}[column sep=large, ampersand replacement=\&]
        A \arrow[r, yshift=2.5,"f"]\arrow[r, yshift=-2.5, swap,"g"]
        \& B \arrow[r, "q"] \arrow[dr, "c"]
        \& Q \arrow[d, dashed, "u"] \\
        {} \& {} \& 
        C
      \end{tikzcd}
    \end{center}
  }
\end{definition}

\begin{example}\label[example]{ex:equalizer-and-coequalizer}
  \
  \begin{itemize}
    \itemsep-0.2em
    \item $\Set$: Let $A,B$ be sets and $f,g:A→B$.
      \begin{itemize}\renewcommand{\labelitemi}{$∘$}
        \itemsep-0.2em
        \item \vspace{-0.6em} The equalizer of $f$ and $g$
          is the set $E\coloneqq\braces{a∈A:\apply{f}{a}=\apply{g}{a}}$.
          The map $E→A$ is the standard inclusion of a subset (which sends each
          element of $E$ to itself in $A$).
        \item The functions $f$ and $g$ induce an equivalence relation $∼$ on
          $B$ which is the closure of $\apply{f}{b}∼\apply{g}{b}$ for all $b∈B$.
          The quotient set $B/∼$ together with the projection $p:B→B/∼$ which
          sends each element of $b$ to its equivalence class is the coequalizer
          of $f$ and $g$.
      \end{itemize}
    \item $\universe$: Let $A,B:\universe$ and $f,g:A→B$
      \begin{itemize}\renewcommand{\labelitemi}{$∘$}
        \itemsep-0.2em
        \item{} \vspace{-0.6em} The equalizer of $f$ and $g$
          is the type $E\defeq\∑{a:A}{\propeq{}{\apply{f}{a}}{\apply{g}{a}}}$.
          The function $E→A$ is the first projection.
        \item{} The issue of general quotients of types is actually somewhat
          intricate. The univalence axiom helps to construct them, but this is
          outside of the scope of this thesis.
      \end{itemize}
    \item $F\dVect$: A coequalizer for a linear map $f:A→B$ is the quotient
      space $B/\im f$.
  \end{itemize}
\end{example}

\begin{definition}\label[definition]{def:pullback-and-pushout}
  \
  \vspace{-0.3em}\dual{
    For $A,B,C∈\Obj\bfC$ with arrows $f:A\catarrow C$ and $g:B\catarrow C$,
    a \define{pullback}\index{Pullback} of $f$ and $g$
    consists of an object $P$ together with arrows $p_1:P\catarrow A$ and
    $p_2:P\catarrow B$ satisfying the following universal property:

    For any other ``candidate pullback'' $D$ with arrows $q_1:D\catarrow A$ and
    $q_2:D\catarrow B$, there is a unique arrow $u:D\catarrow P$ making the following
    diagram commute:
    \begin{center}
      \begin{tikzcd}[sep=large, ampersand replacement=\&]
        D \arrow[dr, dashed, "u"]
          \arrow[ddr, "q_1", swap, bend right=20]
          \arrow[drr, "q_2", bend left=20] \& {} \& {} \\
        {}
        \& P \arrow[r, "p_2"] \arrow[d, swap, "p_1"]
        \& B \arrow[d, "g"] \\
        {} \& A \arrow[r, swap, "f"] \& C
      \end{tikzcd}
    \end{center}
    The object $P$ is denoted $A×_C B$, or better 
    $A×_{\langle f,g\rangle}B$.\footnote{The notation $A×_C B$
      draws an important connection to the product, but omits some the
      key information: \textit{which} arrows are being pulled back matters, 
      and there might be many choices for the same objects $A$, $B$, and $C$.}
    One says $p_2$ is the result of \textit{pulling $f$ back along $g$} and vice
    versa for $p_1$.
  }{
    For $A,B,C∈\Obj\bfC$ with arrows $f:C\catarrow A$ and $g:C\catarrow B$,
    a \define{pushout}\index{Pushout} of $f$ and $g$
    consists of an object $P$ with arrows $i_1:A\catarrow P$ and
    $i_2:B\catarrow P$ satisfying the following universal property:

    For any other ``candidate pushout'' $D$, there is a unique arrow
    $u:P\catarrow D$ making the following diagram commute:
    \begin{center}
      \begin{tikzcd}[sep=large, ampersand replacement=\&]
        C\arrow[r, "g"] \arrow[d, "f", swap]
        \& B \arrow[d, "i_2"] \arrow[ddr, "j_2", bend left=20] \& {} \\
        A \arrow[r, swap, "i_1"]
          \arrow[drr, "j_1", swap, bend right=20]
        \& P \arrow[dr, dashed, "u"] \& {} \\
        {} \& {} \& D
      \end{tikzcd}
    \end{center}
    The object $P$ is denoted $A+_C B$, or better $A+_{[f,g]}B$.
  }
\end{definition}

\begin{example}\label[example]{ex:pullback-and-pushout}
  \
  \begin{itemize}
    \itemsep-0.2em
    \item $\Set$: 
      \begin{itemize}\renewcommand{\labelitemi}{$∘$}
        \itemsep-0.1em
        \item \vspace{-0.6em}
          Let $A,B,C$ be sets, $f:A→C$, and $g:B→C$.
          The pullback of $f$ and $g$ is the set
          $P\coloneqq \braces{(a,b)∈A×B:\apply{f}{a}=\apply{g}{b}}$; 
          the maps $P→A$ and $P→B$ are just the projections.
        \item 
          Let $A,B,C$ be sets, $f:C→A$, and $g:C→B$.
          The pushout of $f$ and $g$ is the disjoint union quotiented by the
          smallest equivalence relation containing 
          $(0,\apply{f}{c})∼(1,\apply{g}{c})$; the maps $p_1$ and $p_2$ are the
          projections to equivalence classes.
      \end{itemize}
    \item $\universe$: 
      \begin{itemize}\renewcommand{\labelitemi}{$∘$}
        \itemsep-0.1em
        \item{} \vspace{-0.6em} Let $A,B,C:\universe$, $f:A→C$, and $g:B→C$.
          The pullback of $f$ and $g$ is the type 
          $\∑{a:A}{\∑{b:B} \apply{f}{a}=\apply{g}{b}}$,
          the functions $P→A$ and $P→B$ are just the projections.
      \end{itemize}
  \end{itemize}
\end{example}

\begin{example}\label[example]{ex:pullback-special-cases-set}
	The pullback and pushout are complex; it is less immediately clear where they
  arise in standard mathematics than, say, the product and coproduct.
  Three special cases of the pullback in $\Set$ help motivate these constructions.

  \begin{enumerate}%[label=\Alph*.]
    \itemsep-0.2em
    \item A pullback where one of $A$ or $B$ is the terminal object in a 
      category defines the notion of a ``fiber''. Consider the pullback $P$ in
      the following square (where $\braces{∗}$ is any one-point set, a terminal
      object):
      \begin{center}
        \begin{tikzcd}[sep=large]
          P \arrow[r, "p_2"] \arrow[d, "!"] & B \arrow[d, "g"] \\
          \braces{∗} \arrow[r, swap, "f"] & C
        \end{tikzcd}
      \end{center}
      The arrow $f$ picks out a single element of $C$, namely $\apply{f}{∗}∈C$.
      Call this $c$. Since the square commutes, $f∘!=g∘p_2$, that is,
      $\apply{(g∘p_2)}{x}=c$ for all $x∈P$. By the explicit
      construction of \cref{ex:pullback-and-pushout},
      \begin{equation*}
          P= \braces{(∗,b)∈\braces{∗}×B:\apply{f}{∗}=\apply{g}{b}}
           ≅ \braces{b∈B:c=\apply{g}{b}}.
      \end{equation*}
      This says that $P$ is (isomorphic to) the \define{fiber of $g$ above $c$}.
    \item More generally, consider the \define{inverse image} of a subset
      $S⊆C$. Again, let $B$ and $C$ be arbitrary sets and let $f:B→C$.
      Let $i:S↣C$ be the inclusion function. Then the pullback of $i$ and
      $f$ is
      \begin{align*}
        P&=\braces{(s,b)∈S×B:\apply{i}{s}=\apply{f}{b}} \\
         &=\braces{(s,b)∈S×B:s=\apply{f}{b}} \\
         &≅\braces{b∈B:\apply{f}{b}∈S},
      \end{align*}
      which is the inverse image of $S$ under $f$, denoted
      $\apply{f^{-1}}{V}$.\footnote{The notation
        $f^{-1}$ generalizes the notation for the inverse of a function.
        We can consider $f^{-1}$ to be a function $C→\caP(A)$ into the powerset---
        the set of all subsets---of $A$. In the case that $f$ is bijective,
        it sends each $c∈C$ to a singleton, and defines a function $C→A$,
        also denoted $f^{-1}$.}
      Thus, the inverse image of $C$ under $f$ is the pullback of the inclusion
      along $f$.
    \item Let $S,R⊆C$. The pullback of the inclusions $i:S↣C$ and $j:R↣C$ is
      (isomorphic to) the \define{intersection} $S∩R$:
      \begin{align*}
        P&=\braces{(s,r)∈S×R:\apply{i}{s}=\apply{j}{r}} \\
         &=\braces{(s,r)∈S×R:s=r} \\
         &≅\braces{c∈C:c∈S\text{ and }c∈R} \\
         &=S∩R
      \end{align*}
  \end{enumerate}
\end{example}

\begin{example}\label[example]{ex:pullback-mon}
  In $\Mon$, the category of monoids, the kernel of $f:M→N$
  can be represented as a pullback of the unique arrow $!:\braces{∗}→M$ along
  $f$ as in the following diagram:
  \begin{center}
      \begin{tikzcd}[sep=large]
        \ker f \arrow[r, tail] \arrow[d, swap, "!"] & M \arrow[d, "f"] \\
        \braces{∗} \arrow[r, swap, "!"] & N
      \end{tikzcd}
  \end{center}
\end{example}

\begin{example}\label[example]{ex:pushout-special-cases-top}
  In $\Top$, the wedge sum $A∨B$ is an instance of a pushout.
  Suppose $a∈A$ and $b∈B$ are the basepoints to be ``glued''.
  Define (necessarily continuous) functions
  \begin{align*}
    \begin{split}
      f:\braces{∗}&\longrightarrow A \\
      ∗&\longmapsto a
    \end{split}
    \begin{split}
      g:\braces{∗}&\longrightarrow B \\
      ∗&\longmapsto b
    \end{split}
  \end{align*}
  The equivalence relation $∼$ only relates $\apply{f}{∗}$ and 
  $\apply{g}{∗}$, that is $a∼b$. Thus,
  \begin{equation*}
    A+_{[f,g]}B 
    = A+B/(\apply{f}{∗}∼\apply{g}{∗})
    = A+B/(a∼b)
    = A∨B
  \end{equation*}
\end{example}

\Cref{def:terminal-and-initial}, \cref{def:product-and-coproduct},
\cref{def:equalizer-and-coequalizer}, and \cref{def:pullback-and-pushout}
are all instances of a general construction of \textit{limits} and
\textit{colimits}. Some preliminary constructions are necessary before these
definitions; these will help make sense of the terms ``candidate \_'' and
``candidate co\_'' used above.

These concepts can be defined more elegantly using functors
(\cref{def:functor}); here we consider only limits over graphs.

\begin{definition}\label[definition]{def:graph-diagram}
	A \define{graph}\index{Graph} $G$ consists of a (not necessarily finite)
  collection $G_0$ of vertices, and a map $G_1$ which assigns
  to each ordered pair of vertices a collection of directed edges between
  them.\footnote{This definition may seem obtuse to a classically-trained
    mathematician, but it translates more elegantly into the type-theoretic
    setting.}
\end{definition}

\begin{definition}
  A \define{diagram}\index{Diagram} $D$ for a graph $G$ is a collection
  of objects and arrows ``of shape $G$''. Specifically, $D$ consists of a map
  $D_0:G_0→\Obj\bfC$ and for each pair $(u,v)$ of vertices in $G_0$,
  a map $D_{1}^{(u,v)}:G_1(u,v)→\Hom(\apply{D_0}{u},\apply{D}_0{v})$.
  When convenient, we drop the sub- and super-scripts.
\end{definition}

\begin{definition}\label[definition]{def:cone-cocone}
  Let $D$ be a diagram in $\bfC$ for a graph $G$.
  \vspace{-0.3em}\dual{
    A \define{cone}\index{Cone} for $D$ consists of an object $C\in\Obj\bfC$
    (the \define{apex}) together with \define{projections}
    $\braces{p_{v}:C\catarrow \apply{D}{v}}_{v∈G_0}$ such that the following
    triangle commutes for all $u,v∈G_0$ and $e∈G_1(u,v)$:
    \begin{center}
      \begin{tikzcd}[row sep=large, column sep=small, ampersand replacement=\&]
        {} \& C \arrow[dl, bend right=20, swap, "p_u"] \arrow[dr, bend left=20, "p_v"] \& {} \\
        \apply{D}{u}\arrow[rr, "\apply{D}{e}"] \& {} \& \apply{D}{v}
      \end{tikzcd}
    \end{center}
  }{
    A \define{cocone}\index{Cocone} for $D$ consists of an object $C\in\Obj\bfC$
    (the \define{nadir}) together with \define{injections}
    $\braces{i_{v}:\apply{D}{v}\catarrow C}_{v∈G_0}$ such that the following
    triangle commutes for all $u,v∈G_0$ and $e∈G_1(u,v)$:
    \begin{center}
      \begin{tikzcd}[row sep=large, column sep=small, ampersand replacement=\&]
        \apply{D}{u}\arrow[dr, bend right=20, swap, "i_u"] \arrow[rr, "\apply{D}{e}"]\& {} \&
        \apply{D}{v}\arrow[dl, bend left=20, "i_v"] \\
        {} \& C \& {}
      \end{tikzcd}
    \end{center}
  }
\end{definition}

\begin{example}
  A cone for a diagram of shape $·\;·$
  picks out a ``candidate product'' in $\bfC$. The following correspondences
  illustrate other ``candidate'' constructions:
  \begin{itemize}\renewcommand{\labelitemi}{$∘$}
    \itemsep-0.2em
    \item \vspace{-0.3em} Terminal/Initial: [empty]
    \item (Co)product: $·\;·$
    \item (Co)equalizer:
      \begin{tikzcd}[cramped, sep=small]
        · \arrow[r, yshift=2.5]\arrow[r, yshift=-2.5] & · 
      \end{tikzcd}
    \item Pullback/Pushout:
      \begin{tikzcd}[cramped, sep=small] 
        · & · \arrow[l]\arrow[r] & · 
      \end{tikzcd}
  \end{itemize}
\end{example}

\begin{definition}\label[definition]{def:limit-and-colimit}
  Given a diagram $D$ for a graph $G$ in $\bfC$,
  \vspace{-0.3em}\dual{
    a \define{limit}\index{Limit} of $D$ consists of a cone $(L,\braces{p_v})$
    satisfying the following universal property:

    For any other cone $(C,\braces{q_v})$, there is a unique arrow
    $u:C\catarrow L$ making
    the following diagram commute for all $u,v∈G_0$ and $e∈G_1(u,v)$:
    \begin{center}
      \begin{tikzcd}[row sep=large, column sep=small, ampersand replacement=\&]
        {} \& C \arrow[ddl, bend right=25, swap, "q_u"]
                \arrow[ddr, bend left=25, "q_v"]
                \arrow[d, dashed, "u"]
           \& {} \\
        {} \& L
                \arrow[dl, bend right=15, "p_u"]
                \arrow[dr, bend left=15, swap, "p_v"]
           \& {} \\
        \apply{D}{u}\arrow[rr, swap, "\apply{D}{e}"] \& {} \& \apply{D}{v}
      \end{tikzcd}
    \end{center}
  }{
    a \define{colimit}\index{Colimit} of $D$
    consists of a cocone $(L,\braces{i_v})$ satisfying the following universal
    property:

    For any other cone $(C,\braces{j_v})$, there is a unique arrow
    $u:L→C$ making the following diagram commute:
    \begin{center}
      \begin{tikzcd}[row sep=large, column sep=small, ampersand replacement=\&]
        \apply{D}{u}
        \arrow[rr, "\apply{D}{e}"]
        \arrow[dr, bend right=15, "j_u"]
        \arrow[ddr, bend right=25, swap, "i_u"]
        \& {} \&
        \apply{D}{v}
        \arrow[dl, swap, bend left=15, "j_v"]
        \arrow[ddl, bend left=25, "i_v"] \\
        {} \& L \arrow[d, dashed, "u"] \& {} \\
        {} \& C \& {}
      \end{tikzcd}
    \end{center}
  }
  If $\bfC$ has a (co)limit for a diagram $D$, its apex (resp.\ nadir)
  is denoted $\apply{\Lim}{D}$ (resp.\ $\apply{\Colim}{D}$).
  The collection of (co)cones over a diagram $D$ with apex (resp.\ nadir) $X$ is
  denoted $\appply{\Cone}{D}{X}$ (resp.\ $\appply{\Cocone}{D}{X}$).
\end{definition}

\begin{remark}\label[remark]{rmk:limit-colimit-universal-adj}
  If $D$ is a diagram in $\bfC$, the above universal properties define
  bijections
  \begin{align*}
    \Hom(A,\apply{\Lim}{D}) &≅ \appply{\Cone}{D}{A}\text{ and} \\
    \Hom(\apply{\Colim}{D},A) &≅ \appply{\Cocone}{D}{A}.
  \end{align*}
  for any object $A$. Again, with one additional criterion (called
  ``naturality''), this can be taken as a alternative definitions.
  (Compare to \cref{rmk:prod-coprod-universal-adj}).
\end{remark}

\begin{definition}\label[definition]{def:complete-and-cocomplete}
  A category $\bfC$ \define{has (finite) (co)limits of shape $G$} if any
  (finite) diagram of shape $G$ has a (co)limit in $\bfC$. A category is
  \define{(finitely) (co)complete} or \define{has (finite) (co)limits} when it
  has (co)limits of every shape.
\end{definition}

Generally, one often says $\bfC$ ``has $x$'' for some object $x$ and
collection of arrows defined by a universal property, e.g.\
``$\bfC$ has (binary) products'' or ``$\bfC$ has internal homs''.

\section{Functors and their algebras}
\label{sec:functors-and-their-algebras}

\begin{definition}\label[definition]{def:functor}
	A \define{functor}\index{Functor} $F$ between categories $\bfC$ and $\bfD$
  consists of the following data:
  \begin{itemize}
    \itemsep-0.2em
    \item a map $F_0:\Obj \bfC\to\Obj \bfD$, and
    \item for each pair of objects $A,B\in\Obj C$, a map \\
      $F_1:\Hom_{\bfC}(A,B) \to\Hom_{\bfD}(F_0(A),F_0(B))$.
  \end{itemize}
  These data are subject to the following axioms:
  \begin{enumerate}%[label=\Alph*.]
    \itemsep-0.2em
    \item functors preserve composition
    \item $F(\id_A)=\id_{F(A)}$ for all $A\in\Obj C$.
  \end{enumerate}
\end{definition}

We generally leave off the subscripts and parentheses when possible, denoting
the application by simply $FA$ or $Ff$. A functor $F$ from $\bfC$ to
$\bfD$ may be denoted $F:\bfC\to\bfD$. We may define functors with or without names
using the following notation:
\begin{align*}
  \bfC &⟶ \bfD \\
  A    &⟼_0 \ldots \\
  f    &⟼_1 \ldots
\end{align*}

\begin{example}\label[example]{ex:identity-functor-cat}
  For any category $\bfC$, there is an \define{identity
  functor}\index{Identity!Functor} $\id_{\bfC}\bfC\to\bfC$ which acts as the
  identity on objects and morphisms. The composition of functors is associative,
  and there is a category $\Cat$ of ``small'' categories (it doesn't include
  itself, for instance).
\end{example}

\begin{example}\label[example]{ex:diagonal}
  There is a product in $\Cat$; the product $\bfC×\bfD$ has as object pairs of
  objects $(A,B)$ for $A∈\bfC$ and $B∈\bfD$ and similarly for arrows.
  For any category, there is a \define{diagonal functor}
  \begin{align*}
    Δ:\bfC &⟶ \bfC×\bfC \\
    A    &⟼_0 (A,A) \\
    f    &⟼_1 (f,f)
  \end{align*}
\end{example}

\begin{example}\label[example]{ex:forget}
  For each category of algebraic objects where the morphisms are the
  corresponding type of homomorphism, there is a \define{forgetful functor},
  generally denoted $U$, which takes sets with some structure to their
  underlying sets and homomorphisms to the corresponding maps of sets. For
  instance, there is a forgetful functor $U:\Grp\to\Set$.
\end{example}

\begin{example}\label[example]{ex:coproduct-functoriality}
	If $\bfC$ has binary coproducts, then
  for any fixed $A,B\in\Obj\bfC$, one can define the following
  functors:
  \begin{align*}
    \begin{split}
      \bfC &⟶ \bfC \\
      X    &⟼_0 A + X \\
      f    &⟼_1 \id_A+f = [i_1, i_2∘ f]
    \end{split}
    \begin{split}
      \bfC &⟶ \bfC \\
      Y    &⟼_0 Y + B \\
      g    &⟼_1 g+\id_B = [i_1∘ g, i_2]
    \end{split}
  \end{align*}
  These two functors interact well, meaning in part that they extend to
  a \define{bifunctor}
  \begin{align*}
    \begin{split}
      \bfC×\bfC &⟶ \bfC \\
      (X,Y)    &⟼_0 X + Y \\
      (f,g)    &⟼_1 f+g
    \end{split}
  \end{align*}
  where $(f+g)∘\inl=f$ and $(f+g)∘\inr=g$.
\end{example}

Since functors preserve sources, targets, and composition, they preserve
commutative diagrams. If $f,g,h$ form a commutative triangle in $\bfC$, then
their images under $F:\bfC\to\bfD$ do in $\bfD$:
\begin{center}
  \begin{tikzcd}[column sep=large]
    A\arrow[dr, "f"]\arrow[dd, "h"]\arrow[rr,mapsto,"F"] & {} & F A
    \arrow[dr, "F f"] \arrow[dd, near start, "F h", crossing over] & {} \\
    {} & B\arrow[dl, "g"]\arrow[rr,near start, mapsto, "F", crossing over] & {}
    & F B \arrow[dl, "F g"]\\
    C \arrow[rr,mapsto,"F"]& {} & F C &  {}
  \end{tikzcd}
\end{center}
Note however that it is possible that $FA=FB=FC$, as in a functor to a category
with a single object. However, the equalties between composites still hold.
One consequence is that functors preserve isomorphism.

\begin{definition}\label[definition]{def:endofunctor}
	An \define{endofunctor}\index{Functor!Endofunctor} is a functor with identical
  domain and codomain.
\end{definition}

Endofunctors provide a concise way to encapsulate a
\define{signature}\index{Signature}, a way of describing an object of a
category together with some arrows into it. Signatures are a basic structure
of universal algebra, the zoomed-out study of algebraic structure (as opposed to
algebra, the study of specific algebraic structures e.g.\ monoids, posets,
lattices, rings).

\begin{definition}\label[definition]{def:f-coalgebra}
  \
  \vspace{-0.3em}\dual{
    An \define{algebra}\index{(Co)algebra for a functor} for an endofunctor
    $F:\bfC\to\bfC$ (also called an $F$\define{-algebra}) is a pair
    $(A,α)$ of an object $A\in\Obj\bfC$ and an arrow $α:FA\catarrow A$.
  }{
    A \define{coalgebra} for an endofunctor
    $F:\bfC\to\bfC$ (also called an $F$\define{-coalgebra}) is a pair
    $(A,α)$ of an object $A\in\Obj\bfC$ and an arrow $α:A\catarrow FA$
    \cite{category-theory-for-computing-science}.
  }
\end{definition}

There is an unfortunate profusion of terms in mathematics using ``algebra'' in
their name. Functor algebras are unrelated to elementary algebra (the
solution of equations with unknowns). Their relation to
algebraic structures like groups is a bit more complex.

\begin{remark}\label[remark]{rmk:universal-algebra}
  By the universal property of coproducts, an algebra for a functor of the form
  $A↦(A×A)+A+1$ consists of:
  \begin{itemize}
    \itemsep0em
    \item an arrow $A×A→A$ (a binary operation on $A$),
    \item an arrow $A→A$ (a unary operation on $A$),
    \item and a distinguished element $1→A$.
  \end{itemize}
  This is exactly the signature of a \define{group}\index{Group} (the arrow
  $A×A→A$ corresponds the group operation, $A→A$ to inversion, and $1→A$ to the
  identity).

  Not every algebra for this functor will be a group. The functor doesn't
  encode the \textit{equational laws}, e.g.\ that the operation $A×A→A$ is
  associative. To represent such relations, one needs the richer
  structure of a \define{monad}\indeX{Monad}. Monads also have algebras, and it
  is difficult to say whether these generalize $F$-algebras or the other way
  around. Particular groups, rings, algebras over a field, etc.\
  are examples of $F$-algebras (for various choices of $F$), but the definition
  of a $F$-algebra don't encode sufficient information such that
  \textit{every} $F$-algebra for the ``group signature functor'' is indeed a
  group.
\end{remark}

For the remainder of this section when a definition or result is presented only
for coalgebras or algebras, it holds dually for the other.

\begin{definition}\label[definition]{def:coalgebra-morphism}
  A \define{coalgebra morphism}\index{(Co)algebra morphism} from
  $(A,α)$ to $(B,β)$ is an arrow $f:A\catarrow B$ such that the following
  diagram commutes:
  \begin{center}
    \begin{tikzcd}
      A  \arrow[d, "α"] \arrow[r, "f"] & B \arrow[d, "β"] \\
      FA \arrow[r, "Ff"] & FB
    \end{tikzcd}
  \end{center}
\end{definition}

Since $F$ is a functor, the composition of coalgebra morphisms is again a
coalgebra morphism. In fact, $F$-coalgebras have all of the structure of a
category, called $F\Coalg$.

\begin{definition}\label[definition]{def:initial-alg}
  \
  \vspace{-0.3em}\dual{
    An \define{initial $F$-algebra}\index{Initial algebra} is an initial
    object of $F\Alg$.
  }{
    An \define{terminal $F$-coalgebra}\index{Terminal coalgebra} is a terminal
    object of $F\Coalg$ (these are more commonly known as
    \define{final coalgebras} in the computer science literature).
  }
\end{definition}

The following example is crucial to
\cref{chap:coinductive-types-in-univalent-type-theory}.

\begin{example}[\unimathname{FunctorAlgebras.Nats}]\label[example]{ex:nat}
  Let $\bfC$ be a category with distinguished binary coproducts and a terminal
  object $1$. Consider the functor
  \begin{align*}
    F:\bfC &⟶ \bfC  \\
    A &⟼_0 1+A \\
    f &⟼_1 \id_1+f
  \end{align*}
  where 1 is a terminal object\index{Terminal object} and $+$ is the coproduct
  bifunctor as in \cref{ex:coproduct-functoriality}. Let's examine
  what it \textit{means} for some $F$-algebra $(N,η)$ to be initial. By
  composing with the coproduct\index{Coproduct!In category theory} injections,
  define
  \begin{align*}
    \begin{split}
      z &: 1 ⟶ N \\
      z &= η∘ i_1
    \end{split}
    \begin{split}
      s &: 1 ⟶ N \\
      s &= η∘ i_2
    \end{split}
  \end{align*}
  so that $η=[z,s]$:
  \begin{center}
    \begin{tikzcd}[sep=large,ampersand replacement=\&]
      1 \arrow[r, "i_1"{name=I1}]\arrow[dr, swap, "z"{name=F}]
      \& 1+N \arrow[d,"η"]
      \& N\arrow[l, "i_2"{name=I2},swap]\arrow[dl,"s"{name=G}] \\
      {} \& N \& {}
    \end{tikzcd}
  \end{center}
  Suppose $(A,α)$ is another $F$-algebra, and define $f,g$ by
  composition as above so that $α=[f,g]$. By initiality of $(N,η)$,
  there is a unique arrow $u$ making the following diagram commute:
  \newcommand{\eeeeeta}{[z,s]}
  \newcommand{\aaaaalpha}{[f,g]}
  \begin{center}
    \begin{tikzcd}[sep=large]
      1+N \arrow[r, dashed, "\id_1+u"]\arrow[d, "\eeeeeta"] & 1+A\arrow[d, "\aaaaalpha"]  \\
      N \arrow[r, dashed, "u"] & A
    \end{tikzcd}
  \end{center}
  By functoriality of the coproduct (\cref{ex:coproduct-functoriality}), we can
  compose along either the left- or right-hand paths in the above diagram.
  The above square states $u∘ [z,s]= [f,g]∘ (\id_1+u)$. Precomposing
  with $i_1:1\catarrow 1+N$ yields
  \begin{align*}
    u∘ z = u∘ [z,s]∘ i_1
    &= [f,g]∘ (\id_1+u) ∘ i_1
    && \text{Above diagram} \\
    &= [f,g]∘ [i_1,i_2∘ u] ∘ i_1
    && \text{Definition of }+ \\
    &= [f, g]∘ i_1 \\
    &= f
  \end{align*}
  By similar reasoning, precomposing with $i_2$ yields
  \begin{equation*}
    u∘ s = u∘ [z,s] ∘ i_2
    = [f,g]∘(\id_1+u)∘ i_2
    = g∘ u.
  \end{equation*}
  Combining the above two equations yields the following universal property
  for $(N,η)$. For any object $A$ with arrows $f:1\catarrow A$ and $g:A\catarrow A$,
  there is a unique arrow $u:N\catarrow A$ making the following diagram commute:
  \begin{center}
    \begin{tikzcd}[sep=large]
      1
        \arrow[r, "z"]
        \arrow[dr, swap, "f"]
        & N \arrow[r, "s"] \arrow[d, "u", dashed]
        & N \arrow[d, "u", dashed] \\
      {}
      & A \arrow[r, "g"]
      & A
    \end{tikzcd}
  \end{center}
  Well, we've successfully rephrased the property of initiality, but it seems
  just as cryptic now as it was then. Let's see if we can figure out what an
  type with this property would look like in our favorite category \universe, the
  universe of small types in \UTT{}!

  Let \List{\N} be the type of lists of natural numbers
  as in \cref{subsec:natural-numbers-and-lists}. To utilize the above property,
  we need to choose $A$, $f$, and $g$. Pick the following:
  \begin{itemize}
    \itemsep0em
    \item $A:\equiv \List{\N}$
    \item $f:\equiv \λ{x}{\nil}$
    \item $g:\equiv \λ{l}{\cons(5,l)}$
  \end{itemize}
  From the universal property of $(N,[z,s])$, we get a function
  $u:N\to \List{A}$ such that
  \begin{align*}
    u(z) = \nil
    &&\text{and}&&
    u(s(n)) = \cons(5, u(n))
  \end{align*}
  Look familiar yet? Indeed, one type with such a property is $\N$! In that
  case, $u$ is the function that, when given a number $n$, outputs a list of $5$s
  with length $n$.\footnote{In $\Set$, $\N$ would be an initial algebra.
    It makes sense to ask if this functor has an initial algebra in any
    category with a terminal object and binary coproducts, and in
    general, such algebras are called \define{natural number objects} (NNOs)
    \cite{sketches} \cite{lawvere-etcs}.
  }
\end{example}

\begin{example}\label[example]{ex:terminal-coalgebra}
  A coalgebra for this same functor $A↦A+1$ is a set $A$ with a partial unary
  operation (where the value $1$ represents failure/undefined).
  In $\universe$ the type $\N_∞\jdeq\N+\unittype$ is the carrier of a final
  coalgebra given by a partial operation
  \begin{alignat*}{3}
    &\ttfun{pred} : \N+\unittype &&\to(\N+\unittype)+\unittype \\
    &\apply{\ttfun{pred}}{\apinl{0}}                 &&\jdeq \apinr{(\apinr{\unitelem})}
      &\qquad\quad&\text{(undefined)}\\
    &\apply{\ttfun{pred}}{\apinl{(\apply{\suc}{n})}} &&\jdeq n
      &\qquad\quad& \\
    &\apply{\ttfun{pred}}{\apinr{\unitelem}}         &&\jdeq \apinr{\unitelem}
      &\qquad\quad&\text{($\apinr{\unitelem}$ is like $∞$)}
  \end{alignat*}
  which acts like the natural numbers with an extra ``infinite element''
  \cite{adamek}.
\end{example}

\begin{example}
	Fascinatingly, the closed real interval can be understood as a final
  coalgebra \cite{freyd}. \TODO{}
\end{example}

\begin{example}\label[example]{ex:initial-algebras}
  \
  \begin{itemize}
    \itemsep0em
    \item An initial algebra for the functor $A↦B×A+1$ has the universal property
      of a \define{list object}.
    \item The initial algebra $A↦(A×A)+1$ has a universal property expressing
      the induction principle for binary trees.
  \end{itemize}
\end{example}

These examples suggest a new perspective on inductive types: they are the
non-equational (or \textit{free}) algebraic structures generated by their
constructors (introduction rules). 

\section{Type theoretic category theory}
\label{sec:type-theoretic-category-theory}

Type theory is a great foundational system in which to do category theory,
particularly because of its hierarchy of universes, which deals elegantly with
problems of ``size''.\footnote{
  Conversely, certain categories called \textit{topoi} (singular: topos) provide
  models of type theory. In this case, we can use the logic of type theory to
  directly prove properties of such categories. In this setting, type theories
  are called \textit{internal languages} of topoi.} However, with greater
expressiveness,
issues of definition become more prominent; this section explains the relevant
considerations.

Unfortunately, terminology varies between the three predominant sources on
category theory in univalent type theory \cite{book} \cite{unimath}
\cite{hott-lib}. We follow the \UniMath{} library. The
following table offers a comparison for those already familiar with \cite{book}.
\begin{center}
  \begin{tabular}{| l | l | l | l | l |}
    \hline
    \UniMath{} & \HoTT{} book\footnote{\cite{book}} & $\Obj\bfC$ & $\Hom_{\bfC}$ & Univalence \\ \hline
    Precategory                  & n/a             & Type & Type  & No \\
    Category                     & Precategory     & Type & Set   & No \\
    Univalent category           & Category        & Type & Set   & Yes \\
    Set category                 & Strict category & Set  & Set   & No \\
    \hline
  \end{tabular}
\end{center}
For a pioneering development of univalent category theory, see \cite{rezk} (the
formalization of which served as a foundation for \UniMath{}'s
\unimathname{CategoryTheory} package). 

\subsection{Basics}
\label{subsec:basics}

Types in \UTT{} are the analogues of sets in \ZFC{}+\FOL{}, insofar as they also
represent ``collections''. Therefore, a naïve translation of the definition
would define a category to consist of a type $C:\universe$ of objects and a
family $\ttHom_{C}:C→C→\universe$ of arrows.

\begin{definition}\label[definition]{def:precategory}
  A \define{precategory}\index{Precategory} consists of the following data:
  \begin{itemize}
    \itemsep-0.2em
    \item a type $\bfC:\universe$ of \define{objects},
    \item a family $\ttHom_{\bfC}:\bfC→\bfC→\universe$ which assigns to each pair
      of types a \define{hom-type} which has \define{morphisms} as its elements, 
    \item a map $\id:\∏{c:\bfC}{\appply{\ttHom_{\bfC}}{c}{c}}$
      which assigns to each object an 
      \define{identity morphism}\index{Identity!Morphism},
    \item an operation
      \begin{equation*}
        ∘:\∏{a,b,c:\bfC}{\appply{\ttHom_{\bfC}}{b}{c}→\appply{\ttHom_{\bfC}}{a}{b}→\appply{\ttHom_{\bfC}}{a}{c}}
      \end{equation*}
      called \define{composition}\index{Composition!In a precategory}, and
  \end{itemize}
  together with maps witnessing\footnote{This is an expressly type-theoretic
    presentation; one might instead use the more classical ``for all objects
    $a,b$ of $\bfC$ and arrows $f:A$''. However, since such definitions appear
    in the previous chapter, these are phrased more idiosyncratically to help the
    reader get used to this more rigid formalism.} the following axioms:
  \begin{enumerate}%[label=\Alph*.]
    \itemsep-0.2em
    \item left and right identity:
      \begin{align*}
        \∏{a,b:\bfC}{\∏{f:\appply{\ttHom_{\bfC}}{a}{b}}{&\id_B∘f=f}} \\
        \∏{a,b:\bfC}{\∏{f:\appply{\ttHom_{\bfC}}{a}{b}}{&f∘\id_A=f}}
      \end{align*}
    \item associativity of composition:
      \begin{equation*}
        \∏{a,b,c,d:\bfC}{
          \∏{f:\appply{\ttHom_{\bfC}}{a}{b}}{
            \∏{g:\appply{\ttHom_{\bfC}}{b}{c}}{
              \∏{h:\appply{\ttHom_{\bfC}}{c}{d}}{
                \propeq{}{(h∘g)∘f}{h∘(g∘f)}
        }}}}
      \end{equation*}
  \end{enumerate} 
  We denote arrows $\appply{\ttHom}{a}{b}$ by $a\rightharpoonup b$ in an attempt
  to distinguish it from the function type $a→b$. 
  The type of all precategories is denoted $\Precat$.
\end{definition}

Many concepts from classical category theory can be translated with no further
thought into the language of precategories. 

\begin{definition}\label[definition]{def:functor-tt}
	A \define{functor}\index{Functor} $F$ between $\bfC,\bfD:\Precat$
  consists of the following data:
  \begin{itemize}
    \itemsep-0.2em
    \item a map $F_0:\bfC\to\bfD$ and
    \item a map
      $F_1:\∏{a,b:\bfC}{\appply{\Hom_{\bfD}}{(\apply{F_0}{a})}{(\apply{F_0}{b})}}$
  \end{itemize}
  together with maps witnessing the following axioms:
  \begin{enumerate}%[label=\Alph*.]
    \itemsep-0.2em
    \item preservation of identity:
      \begin{equation*}
        \∏{a:\bfC}{\apply{F_1}{\id_A}=\id_{(\apply{F_0}{A})}}
      \end{equation*}
    \item preservation of composition:
      \begin{equation*}
        \∏{a,b,c:\bfC}{
          \∏{f:\appply{\ttHom_{\bfC}}{a}{b}}{
            \∏{g:\appply{\ttHom_{\bfC}}{b}{c}}{
              \propeq{}{\apply{F_1}{(g∘f)}}{\apply{F_1}{g}∘\apply{F_1}{f}}
        }}}.
      \end{equation*}
  \end{enumerate}
  The type of functors from $\bfC$ to $\bfD$ is denoted
  $\appply{\Functor}{\bfC}{\bfD}$.
\end{definition}

\begin{example}[\unimathname{CategoryTheory.categories.Cats.precat\_precat},
                \\ \unimathname{CategoryTheory.categories.Cats.cat\_precat}]
  \label[example]{ex:precat-of-precats}
  There is a precategory of precategories, with object type
  $\Precat$ and morphism type 
  $\ttfun{Functor}$;\footnote{This is a great example of how universes deal
    elegantly with problems of size: this precategory isn't a member of itself,
    because it is really only a precategory of $\universei{i}$-small
    precategories for fixed $i$, and itself is an element of 
    $\universei{\apply{\suc}{i}}$. A good proof assistant automatically assigns
    consistent universe indices when possible.}
  there is also a precategory with object type $\ttfun{Category}$.
\end{example}

Many such ``translated'' definitions are so near to the originals that we won't
treat them here (e.g.\ commutative diagrams, isomorphisms, and the specific
kinds of (co)limits).

However, when proving analogous results about such constructions in
precategories, one soon runs into difficulty. The
root of the problem is that the hom-types may have ``higher homotopical
structure'', and to account for this structure would require infinitely many
additional axioms.\footnote{The precise nature of this structure and how it
  interferes with classical intuitions is outside the scope of this thesis,
  though certain instructional difficulties will arise in
  \cref{chap:coinductive-types-in-univalent-type-theory} (indeed, that whole
  chapter may be seen as an attempt to use 1-category theoretic-intuitions
  outside their domain of applicability).} 
For most applications, it suffices to restrict the hom-types to be hom-sets:
\begin{definition}\label[definition]{def:category-tt}
  A precategory $\bfC$ \define{has hom-sets}\index{Having hom-sets} if the type
  \begin{equation*}
    \pit{a,b:\bfC}{\apply{\isSet}{(\appply{\ttHom}{a}{b})}}
  \end{equation*}
  is inhabited. A precategory with hom-sets is called a \define{category}.
\end{definition}

\begin{example}
  The following examples illustrate to readers familiar with category theory the
  difficulty of working with precategories rather than categories. 
  \begin{itemize}
    \itemsep0em
    \item The type $\appply{\Functor}{\bfC}{\bfD}$ doesn't form a precategory
      with natural transformations as morphisms unless $\bfD$ has hom-sets. 
    \item Slices (resp.\ coslices) over (resp.\ under) an object $X$ don't form
      precategories unless the homs-types $\appply{\Hom}{-}{X}$ (resp.\
      $\appply{\Hom}{X}{-}$) are all sets.
  \end{itemize}
\end{example}

\subsection{Products in $\universe$}
\label{subsec:products-in-universe}

In \cref{chap:coinductive-types-in-univalent-type-theory}, we will be most
concerned with a precategory that is emphatically \textit{not} a category,
namely $\universe$. In particular, M-types are constructed as limits
in $\universe$. The next few lemmas explore binary products before defining and
constructing general limits in \cref{subsec:limits-in-universe}.

\begin{lemma}[\unimathname{categories.Types.BinProductsType}]
  \label[lemma]{lemma:universe-binproducts}
	The precategory of types has binary
  products.\footnote{Again, where these are defined by rote translation from
    \cref{def:product-and-coproduct}.}
\end{lemma}

While there are more direct proofs, the following is based on the
observation made in \cref{rmk:prod-coprod-universal-adj}, and will carry over to
the case of general limits. We begin a general lemma:

\begin{lemma}[\unimathname{limits.binproduct.isBinProduct'\_to\_isBinProduct}]
  \label[lemma]{lemma:isbinproduct}
  Let $\bfC$ be a precategory, let $a,b,c:\bfC$, and let $p_1:c\catarrow a$,
  $p_2:c\catarrow b$. Define a function
  \begin{gather*}
    \postcomp:\∏{d:\bfC}{(d\→ c)→ (d \→ a)×(d \→ b)
    } \\
    \appply{\ttfun{postcomp}}{f} \defeq (p_1∘f,p_2∘f)
  \end{gather*}
  If $\ttfun{postcomp}$ is an equivalence, $c$ is a product of $a$ and $b$.
\end{lemma}

\textit{Sketch}. Why is this true? If $\ttfun{postcomp}$ is an equivalence, then
its fibers are contractible. In particular, for each pair of arrows $f:d\→a$ and
$g:d\→b$, there is exactly one arrow $u:d\→c$ in the fiber over them. This means
when $u$ is composed projections $p_1$ and $p_2$, it gives back $f$ and $g$ (up
to propositional equality). This is just a backwards rephrasing of the universal
property of the product.

\begin{proof}
  Fix $d$, $q_1$, and $q_2$ as in the type of $\ttfun{postcomp}$.
  For $(c,p_1,p_2)$ to be a product of $a$ and $b$, the following type must be
  contractible:\TODO{actually just put this in its own definition}
  \begin{equation*}
    \∑{f:d\→c}{(p_1∘f=q_1)×(p_2∘f=q_2)}
  \end{equation*}
  By \cref{cor:weq-contr}, it suffices to show that the above type is equivalent
  to a contractible type. By \cref{def:weq}, the fibers of $\ttfun{postcomp}$
  are contractible, so it suffices to show that the above type is equivalent to
  the fiber over some point in $(d \→ a)×(d \→ b)$. This point is $(q_1,q_2)$:
  \begin{align*}
    \apply{\ttfun{fiber}}{(q_1,q_2)}
    &\≡ \∑{f:d\→c}{\propeq{}{\apply{\ttfun{postcomp}}{f}}{(q_1,q_2)}}
    &&\text{Definition} \\
    &\≡ \∑{f:d\→c}{\propeq{}{(p_1∘f,p_2∘f)}{(q_1,q_2)}}
    &&\text{Definition} \\
    &\≃ \∑{f:d\→c}{(p_1∘f=q_1)×(p_2∘f=q_2)}
    &&\text{\Cref{lemma:paths-dirprod-weq}}
  \end{align*}
\end{proof}

\begin{lemma}\label[lemma]{lemma:voevodsky-prod-up}
	There is an equivalence of types
  \begin{equation*}
    (C→A)×(C→B)≃(C→A×B)
  \end{equation*}
  for all $A,B,C:\universe$.\footnote{This result originally appeared in
    Voevodsky's ``Foundations'' library, a precursor to \UniMath{}.}
\end{lemma}
\begin{proof}
  Define 
  \begin{gather*}
    g : (C→A)×(C→B) → (C→A×B) \\
    \apply{g}{(h,i)} \defeq \λ{c}{(\apply{h}{c},\apply{i}{c})}
  \end{gather*}
  and let $\ttfun{postcomp}_{\universe}$ be the specialization of 
  $\ttfun{postcomp}$ to the category of types. Then
  \begin{align*}
    \apply{(g∘\ttfun{postcomp}_{\universe})}{h}
    &\≡ \apply{g}{(\pr{1}∘\apply{h},\pr{2}∘\apply{h})} \\
    &\≡ \λ{c}{(\pr{1}∘\apply{h},\pr{2}∘\apply{h})} \\
    &\= h
  \end{align*}
  where the final equality is due to function extensionality (\cref{thm:funext})
  and the η-rule for the product type \TODO{reference}. Further,
  \begin{align*}
    \apply{(\ttfun{postcomp}_{\universe}∘g)}{(h,i)}
    &\≡ \apply{\ttfun{postcomp}_{\universe}}{(\λ{c}{(\apply{h}{c},\apply{i}{c})})} \\
    &\≡ (\pr{1}∘\λ{c}{(\apply{h}{c},\apply{i}{c})},\pr{2}∘\λ{c}{(\apply{h}{c},\apply{i}{c})}) \\
    &\= (h,i)
  \end{align*}
  where the final equality is again due to function extensionality
  and the η-rule for the product type \cref{lemma:paths-dirprod-weq}.
\end{proof}

\begin{proof}[Proof of \cref{lemma:universe-binproducts}]
	\Cref{lemma:voevodsky-prod-up} proves that $\ttfun{postcomp}$ is an
  equivalence in $\universe$, so by \cref{lemma:isbinproduct}, $\universe$ has
  binary products.
\end{proof}

\subsection{General limits}
\label{subsec:general-limits}

The following definitions are mere translations of those in
\cref{sec:limits-and-colimits}, but their specific constructions will be important
when proving results about them.

\begin{definition}
  Let $G$ be a graph. A diagram for $G$ in $\bfC$ is an element of
  \begin{equation*}
    \∑{f:G_0→\bfC}{\∏{u,v:G_0} \appply{G_1}{u}{v}→(\apply{f}{u}\catarrow\apply{f}{v})}
  \end{equation*}
  If $D$ is a diagram, we denote the application of both projections by just
  $D$, as in $\apply{D}{v}$ for a vertex $v:G_0$ or $\apply{D}{e}$ for an edge $e$.
  The type of cones over a diagram $D$ with apex $X:\bfC$ is
  \begin{gather*}
    \appply{\Cone}{D}{X}\defeq
    \∑{x:\∏{v:G_0}{X\catarrow \apply{D}{v}}}
      {\∏{u,v:G_0}
        {\∏{e:\appply{G_1}{u}{v}}{\propeq{}{\apply{D}{e}∘\apply{x}{u}}{\apply{x}{v}}}}}
  \end{gather*}
  where the composition is that of $\bfC$. The application of the first
  projection of a cone is written with a subscript, as in the next definition.
\end{definition}

\begin{definition}\label[definition]{def:limit-cone}
  Let $D$ be a diagram and $\dpair{p}{P}:\appply{\Cone}{D}{L}$. There is a
  function
  \begin{gather*}
    \postcomp : \∏{X:\universe} (X\catarrow L)→\appply{\Cone}{D}{X}
  \end{gather*}
  that works by postcomposing $f$ with the projections $p$.
	Then $\dpair{p}{P}$ is a \define{limit cone} for $D$ if for all $X:\universe$,
  $\apply{\postcomp}{X}$ is an equivalence. (Compare to
  \cref{rmk:limit-colimit-universal-adj}.)
\end{definition}

In univalent type theory, it's generally useful to know the $h$-level of a given
type, especially if it is $-2$ (contractible), $-1$ (a proposition), or
$0$ (a set). The following lemma is not used after its proof, but shows us that
the property of being a limit is in some sense a ``logical'' property. 

\begin{lemma}\label[lemma]{lemma:limit-prop}
  The property of being a limit cone is a proposition. Explicitly,
  \begin{equation*}
    \apply{\isProp}
          {\paren{\∏{X:\universe}{\apply{\isEquiv}{(\apply{\postcomp}{X})}}}}
  \end{equation*}
\end{lemma}
\begin{proof}
	Apply \cref{lemma:impred} and \cref{lemma:isequiv-prop}.
\end{proof}

\begin{figure}[ht]
  \centering
  \begin{center}
    \begin{tikzcd}[row sep=large, column sep=small, ampersand replacement=\&]
      {} \& C \arrow[ddl, bend right=25, swap, "\λ{v}{p_u∘f}"]
              \arrow[ddr, bend left=25, "\λ{v}{p_v∘f}"]
              \arrow[d, "f"]
          \& {} \\
      {} \& L
              \arrow[dl, bend right=15, "p_u"]
              \arrow[dr, bend left=15, swap, "p_v"]
          \& {} \\
      \apply{D}{u}\arrow[rr, swap, "\apply{D}{e}"] \& {} \& \apply{D}{v}
    \end{tikzcd}
  \end{center}
  \caption{\label{fig:postcomp} The action of $\postcomp$}
\end{figure}

\subsection{Limits in $\universe$}
\label{subsec:limits-in-universe}

\begin{theorem}\label[theorem]{thm:universe-limits}
  The precategory $\universe$ of types has all limits, in the sense that for any
  graph $G$ and diagram $D$ of shape $G$ in $\universe$, there is a type $L$
  and an equivalence
  \begin{equation*}
    (X→L)≃\appply{\Cone}{D}{X}.
  \end{equation*}
\end{theorem}

This is a somewhat weaker condition than \cref{def:limit-cone} (we're not
claiming a particular function is an equivalence, just that there is one), but
it's all we'll need later. It corresponds directly to the proof in 
\cite{homotopy-limits}. The reason is that we'll use a different but equivalent
definition of cones, which requires a corresponding revision to $\postcomp$.

\begin{lemma}[\unimathname{Induction.M.Limits.homot\_cone\_weq\_cone}]
  \label[lemma]{lemma:weq-homot-cone}
	There is a weak equivalence between cones for a diagram $D$ with apex $X$ and 
  \define{homotopy cones} for the same diagram with the same apex:
  \begin{gather*}
    \weq{\appply{\Cone}{D}{X}}{
      \∑{x:\∏{v:G_0}{X\catarrow \apply{D}{v}}}
        {\∏{u,v:G_0}{\∏{e:\appply{G_1}{u}{v}}
                    {\homot{\apply{D}{e}∘\apply{x}{u}}{\apply{x}{v}}}}}
        \defeqback \appply{\HomotCone}{D}{X}
    }
  \end{gather*}
\end{lemma}
\begin{proof}
	The ``outer layers'' of these types are identical (see
  \cref{lemma:weqfibtototal}), and the innermost
  equivalence is the statement of function extensionality (\cref{thm:funext}).
\end{proof}

\begin{definition}\label[definition]{def:limit-homot-cone}
  Let $D$ be a diagram and $\dpair{p}{P}:\appply{\HomotCone}{D}{L}$. Define
  \begin{gather*}
    \postcomp : \∏{X:\universe} (X→ L)→\appply{\Cone}{D}{X} \\
    \appply{\postcomp}{X}{f}\defeq \dpair{\λ{v}{p_v∘f}}
           {\λ{u}{\λ{v}{\λ{e}{P_{u,v,e}∘f}}}}
  \end{gather*}
	Then $\dpair{p}{P}$ is a \define{limit homotopy cone} for $D$ if for all
  $X:\universe$, $\apply{\postcomp}{X}$ is an equivalence.
\end{definition}

Similarly to \cref{lemma:limit-prop}, the property of being a limiting homotopy
cone is also a proposition.

\begin{definition}\label[definition]{def:standard-limit}
  The \define{standard limit} of a diagram $D$ for a graph $G$ is the type
  \begin{equation*}
    \Stdlim{D}\defeq
    \∑{x:\∏{v:G_0}{\apply{D}{v}}}
      {\∏{u,v:G_0}{\∏{e:\appply{G_1}{u}{v}}{\propeq{}{\appply{D}{e}{(\apply{x}{u})}}{\apply{x}{v}}}}}
  \end{equation*}
  (Note that, since we are focusing on $\universe$, we know that edges
  $\apply{D}{e}$ are in fact functions, and so we can apply them.)
  The \define{standard limit homotopy cone} is given by
  \begin{equation*}
    \dpair{\λ{v}{\λ{L}{\apply{\appr{1}{L}}{v}}}}
          {\λ{u}{\λ{v}{\λ{e}{\λ{L}{\apppply{\appr{2}{L}}{u}{v}{e}}}}}}
    : \appply{\HomotCone}{D}{(\Stdlim{D})}
  \end{equation*}
\end{definition}

(In the \Coq{} formalization, we additionally provide a characterization of the
paths in $\Stdlim{D}$.)

\begin{proof}[Proof of \cref{thm:universe-limits}]
  Let $G$ be a graph, $D$ be a diagram of shape $G$ in $\universe$ and let
  $X:\universe$. Using \cref{lemma:weq-homot-cone} and the fact that the
  composite of two weak equivalences is a weak equivalence, it suffices to show
  \begin{equation*}
    \weq{(X→\Stdlim{D})}{\appply{\HomotCone}{D}{X}}.
  \end{equation*} 
  We construct an explicit inverse; the composites are judgmentally the
  appropriate identities.

  Let $\dpair{q}{Q}:\appply{\HomotCone}{D}{X}$ and $x:X$.
  To define a function into a Σ-type, it suffices to give both
  projections. The first projection has type $\∏{v:G_0}{\apply{D}{v}}$, define this
  to be $f\defeq\λ{v}{\apply{q_v}{x}}$ (since $q_v:X→\apply{D}{v}$). 
  Now, we must show that
  \begin{equation*}
    \∏{u,v:G_0}
      {\∏{e:\appply{G_1}{u}{v}}
         {\propeq{}{\appply{D}{e}{(\apply{f}{u})}}{\apply{f}{v}}}}
  \end{equation*} 
  This follows immediately from the fact that $q$ forms a cone. In summary,
  we have a function
  \begin{gather*}
    φ:\appply{\HomotCone}{D}{X}→(X→\Stdlim{D}) \\
    \appply{φ}{\dpair{q}{Q}}{x}\defeq\dpair{\λ{v}{\apply{q_v}{x}}}
              {\λ{u}{\λ{v}{\λ{e}{Q_{u,v,e}x}}}}
  \end{gather*}

  It remains to show that $\propeq{}{φ∘\apply{\postcomp}{X}}{\id_{X→L}}$ and
  vice versa. By function extensionality, it suffices to show
  this when they are fully applied. Let $x:X$ and $f:X→\Stdlim{D}$. 
  By definition of the standard limit homotopy cone
  and η-rules for function and Σ-types,
  \begin{align*}
    &\apply{(\apply{(φ∘\apply{\postcomp}{X})}{f})}{x} \\
    &\jdeq
      \appply{φ}
      {\dpair{\λ{v}{(\λ{L}{\apply{\appr{1}{L}}{v}})∘f}}
            {\λ{u}{\λ{v}{\λ{e}{\λ{L}{\apppply{\appr{2}{L}}{u}{v}{e}}∘f}}}}}
      {x} \\
    &\jdeq
      \dpair{\λ{v}{\apply{((\λ{L}{\apply{\appr{1}{L}}{v}})∘f)}{x}}}
            {\λ{u}{\λ{v}{\λ{e}{\apply{(\λ{L}{\apppply{\appr{2}{L}}{u}{v}{e}}∘f)}{x}}}}} \\
    &\jdeq \dpair{\λ{v}{\apply{(\appr{1}{(\apply{f}{x})})}{v}}}
                 {\λ{u}{\λ{v}{\λ{e}{\apppply{(\appr{2}{(\apply{f}{x})})}{u}{v}{e}}}}} \\
    &\jdeq \dpair{\appr{1}{(\apply{f}{x})}}{\appr{2}{(\apply{f}{x})}} \\
    &\jdeq \apply{f}{x}
  \end{align*}
  The other direction also works via such routine
  calculation.\footnote{One of the great benefits of working in a computer proof
    assistant is that such chains of judgmental equalities can be dispatched
    in a single line that essentially says: ``Calculate!''}
\end{proof}
\end{document}