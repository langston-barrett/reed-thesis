\documentclass[./thesis.tex]{subfiles}
\begin{document}

% Acknowledgements (Acceptable American spelling) are optional
% So are Acknowledgments (proper English spelling)

\chapter*{Acknowledgments}

\noindent Safia, your sage advice and unfailing support throughout this
process made it possible in the first place. Thank you for your flexibility and
your Virgilian spirit: you managed to guide me through territory unfamiliar to
both of us.

\vspace{0.6em}

\noindent Benedikt, thank you for organizing and ensuring I could attend
the \UniMath{} workshop. Your warm welcome into the study of type theory
has directed the course of this thesis. I look forward to working more with
you in the future.

\vspace{0.6em}

\noindent Dan and Anders, thank you for your perseverance and thoroughness in
proof reviews, and for your words of encouragement. I hope to collaborate
with you soon.

\vspace{0.6em}

\noindent Friends at Reed, thank you for consistently indulging my half-baked
attempts at explaining my work, and thank you for sharing your curiosities and
passions just as easily.

% The pcreface is optional
% To remove it, comment it out or delete it.
% \chapter*{Pcreface}
% This thesis delves into highly interdisciplinary territory. I'll

\chapter*{List of Abbreviations}

\begin{table}[h]
  \centering
  \begin{tabular}{ll}
    \HoTT  	&  Homotopy type theory \\
    \UTT{}  	&  univalent type theory \\
    \ITT  	&  Intensional type theory \\
    \FOL{}  	&  Classical first-order logic \\
    \IPL{}  	&  Intuitionistic propositional logic \\
    \LC{}  	  &  Church's untyped λ-calculus \\
    \STLC{}  	&  Simply-typed λ-calculus \\
    \TLC{}  	&  Typed λ-calculus
  \end{tabular}
\end{table}

% Depth to which to number and print sections in TOC
\setcounter{tocdepth}{4}
% \setcounter{secnumdepth}{2}
\tableofcontents
% if you want a list of tables, optional
% \listoftables
% if you want a list of figures, also optional
% \listoffigures

\chapter*{Abstract}

In this thesis we explain univalent type theory, a constructive and
computationally meaningful foundational system for mathematics inspired by
recent advances in the model theory of Per Martin-L\"of's intuitionistic type
theory. We first develop the classical propositions/types correspondence between
(the constructive subset of) intuitionistic natural deduction and the
λ-calculus. We proceed to explicate Martin-L\"of's theory of dependent types and
the central modern development in type theory: Vladimir Voevodsky's univalence
principle. We go on to examine some category theory and the nature of coinduction
within this theory, presenting a novel formalization of (parts of) a recent
result that M-types can be derived \textit{internally} in univalent type theory.

\chapter*{Dedication}

\mainmatter % here the regular arabic numbering starts
\pagestyle{fancyplain} % turns page numbering back on

\chapter*{Introduction}
\addcontentsline{toc}{chapter}{Introduction}
\chaptermark{Introduction}
\markboth{Introduction}{Introduction}

\setlength{\epigraphwidth}{0.8\textwidth}
\epigraph{I didn’t have the tools to explore the areas where curiosity was
  leading me and the areas that I considered to be of value and of interest and
  of beauty. So I started to look into what I could do to create such tools.}{\cite{voevodsky-ias}}

Many mathematicians err. Some acknowledge mistakes in their published work and
try to move on. For others, the experience can be life-changing. In 1998, Carlos
Simpson released a pre-print arguing that there was a major mistake in one of
Vladimir Voevodsky's 1989 papers \cite{voevodsky-presentation}. It was not
immediately clear whether Voevodsky had erred, or whether there was a flaw in
Simpson's counterexample. In 1999, Pierre Deligne found a crucial mistake in
Voevodsky's 1992/93 ``Cohomological Theory of Presheaves with Transfers'', upon
which he had based much of his work in the area of ``motivic cohomology''. As he
began to develop more and more complex arguments, Voevodsky wondered: ``And who
would ensure that I did not forget something and did not make a mistake, if even
the mistakes in much more simple arguments take years to uncover?''
\cite{voevodsky-ias}.

Elsewhere in mathematics, new developments raised questions for traditional
methodology. Complexity, specialization and sheer length made
proofs difficult to comprehend with the absolute certainty which is
supposed to characterize this discipline. In the 1970s, two teams of
topologists proved contradictory results, and neither group could find the error
in the other's proof \cite{kolata}. Wiles's famous proof of Fermat's last
theorem was utterly unintelligible to the vast majority of mathematicians
\cite{nyt}. The classification of the simple finite groups---one of the field's
crowning achievements---has a combined proof of over ten thousand pages.

The problems facing Voevodsky and his peers seemed insurmountable.
As the requisite attention span, memory, and capacity for detail required to
check new developments in higer mathematics reached inhuman proportions,
where were they to turn? In a fit of interdisciplinary collaborative spirit,
Voevodsky began to explore an avenue well-known to philosophers and computer
scientists, but relatively obscure within mathematics: type theory.

After the paradoxes and philosophical debates of the early
20\textsuperscript{th} century, most mathematicians settled on classical
first-order logic (\FOL{}) and an axiomatic set theory called \ZFC{} as the gold
standard of rigor. Virtually all of contemporary research was built on these
foundations. However, there remained some vocal opposition by logicians and
philosophers to these newly-adopted methods. Most mathematicians are dimly aware
that there's some controversy about the Axiom of Choice (the ``C'' in
\ZFC{}) \cite{martin-lof-100-years}, but one don't have to look further
than \FOL{} to find disagreements.

Brouwer, Heyting, and other intuitionists decried the use of the law
of excluded middle (or equivalently, double-negation elimination). Russell
doubted the consistency of impredicative (self-referential) definitions.
In the 1970s, philosopher and logician Per Martin-Löf created a predicative,
intuitionistically-acceptable logical system called type theory as an
alternative foundational system for ``constructive''
mathematics.\footnote{Constructive and intuitionistic mathematics differ in
their philosophical points, but mostly accord on choice of logical principles.}

Brouwer, Heyting, and Kolmogorov had all at various points suggested that
intuitionistic logic could be understood not only as a theory of truth, but also
as a theory of ``problems'' (propositions) and their ``solutions'' (proofs)
\cite{kolmogorov}. Elsewhere, Haskell Curry noticed that his formalism for
describing computations shared some surprising features with intuitionistic
logic \cite{curry-howard}. These observations were synthesized and extended by
William Howard, who noticed that intuitionistic ``natural deduction''
corresponds closely to a computational system called the λ-calculus (\LC{}).
The postcard version of the truth is that Martin-Löf's was aware of these
connections between constructive logic and computation, and his type theory
could actually serve as \textit{both} an advanced, (very) high-level programming
language \textit{and} a foundational system for mathematics (details to come in
\cref{chap:propositions-as-types}).

The computer scientist Theirry Coquand took this Curry/Howard thing
\textit{very} seriously. He developed a variant of Martin-Löf's type theory
called the Calculus of Constructions \cite{coquand}, and went on to implement it
as the programming language \Coq{}.\footnote{The French word for ``rooster'', in
  the tradition of naming software after animals (a lá \software{OCaml}).}
Computer scientists, facing the problem of ensuring the correctness of algothims
and the programs based on them, developed this and other ``proof assistants''
in order to produce machine-verifiable code.

\section*{Our contribution}

(Co)inductive types and UniMath

begin
\begin{notation}
  All of the results of this thesis have been formalized in the \Coq{} proof
  assistant \cite{coq-manual}. The names of the formal proofs
  appear in a monospaced font (e.g.\ \coqname{univalence}). Many results have
  already been reviewed (by the \UniMath{} development team \cite{unimath} and
  Anders Mörtberg) and accepted into the \UniMath{} library, these appear
  with an underline (e.g.\ \unimathname{univalence}).
\end{notation}

\end{document}
