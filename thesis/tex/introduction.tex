\chapter*{Introduction}
\addcontentsline{toc}{chapter}{Introduction}
\chaptermark{Introduction}
\markboth{Introduction}{Introduction}

\section*{A short history of discomfort in mathematics}

% \subsection*{Trouble in Cantor's Paradise}

In the early 20\textsuperscript{th} century, mathematicians had a problem. Often
held up as the pinnacle of necessary, undoubtable, \textit{a priori} truth,
their discipline was suffering from an embarrassing lack of certainty. The
intricate arguments of advanced analysis left mathematicians unable to confirm
nor deny each other's proofs. Due to the wave of paradoxes unleashed by
Cantor's na\"ive set theory, the need for a rigorous logical foundation for higher
mathematics was so great that the enterprise of axiomatic set theory was pursued
headlong until some kind of consensus was reached. To make a long, rich story
brutally short, virtually every modern paper in mathematics and computer
science uses a combination of Gentzen's first-order logic (\FOL)
and an axiomatization of sets called
\ZFC.\footnote{This ``consensus'' left out many
  prominent schools of thought, such as the Intuitionists and constructivists, a
  point we'll soon return to.}

Despite the apparent rigor of \FOL+\ZFC,
practitioners of the deductive sciences were still beset by issues of
verifiability. Complexity, specialization and sheer length made modern proofs
difficult to comprehend with the absolute certainty which is supposed to
characterize these disciplines. In the 1970s, two teams of topologists proved
contradictory results, and neither group could find the error in the other's
proof \cite{kolata}. Wiles's famous proof of Fermat's last theorem was utterly
unintelligible to the vast majority of mathematicians \cite{nyt}. The
classification of the simple finite groups, one of the crowning achievements of
modern mathematics, has a combined proof of over ten thousand pages. These
examples illustrate merely a few of the practical epistemological challenges
facing mathematicians; for a historical perspective see \cite{rigor-and-proof},
and for a general overview see \cite{fidelity}.

In 1990s, Fields medalist Vladimir Voevodsky grew concerned with the state
of mathematical knowledge. In 1998, Carlos Simpson released a pre-print arguing
that there was a major mistake in one of Voevodsky's papers. However, it was not
clear whether Voevodsky had errored, or whether there was a flaw in Simpson's
counterexample. In 1999, Pierre Deligne found a crucial mistake in Voevodsky's
``Cohomological Theory of Presheaves with Transfers'', upon which he had based
much of his work in the area of motivic cohomology. As he began to develop more
and more complex arguments, Voevodsky wondered: ``And who would ensure that I
did not forget something and did not make a mistake, if even the mistakes in
much more simple arguments take years to uncover?'' \cite{voevodsky-ias}.

The problems facing Voevodsky and his peers seemed insurmountable.
As the requisite attention span, memory, and capacity for detail required to
understand new developments in higer mathematics reached inhuman proportions,
where was he to turn? He would not find a solution in the realm of pure
mathematics, but rather in one of the finest examples of collaboration between
mathematicians, computer scientists, and philosophers: the modern proof
assistant. \TODO{reword}

\section*{TODO}

While much of the mathematical discipline was simply relieved to have ``solved''
their foundational issues with axiomatic set theory, there remained a vocal
opposition to the newly-adopted methods. Most mathematicians are dimly aware
that there's some controversy about the Axiom of Choice (the ``C'' in
\ZFC) \cite{martin-lof-100-years}, you don't have to look further
than \FOL to find disagreements.

\section*{Our contribution}

(Co)inductive types and UniMath

begin
\begin{notation}
  All of the results of this thesis have been formalized in the \Coq{} proof
  assistant \cite{coq-manual}. The names of the formal proofs
  appear in a monospaced font (e.g.\ \coqname{univalence}). Many results have
  already been reviewed (by the \UniMath{} development team \cite{unimath} and
  Anders Mörtberg) and accepted into the \UniMath{} library, these appear
  with an underline (e.g.\ \unimathname{univalence}).
\end{notation}
