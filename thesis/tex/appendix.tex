\chapter{Cross-reference of names}

The following table lists lemmas taken from \cite{book}.
\begin{table}[ht]
  \centering
  \begin{tabular}{c | c}
    This thesis & The \HoTT book \\ \hline
    \Cref{def:ua} & Axiom 2.10.3
  \end{tabular}
\end{table}

The following table compares the terminology used in this thesis to that in our
\Coq{} formalization (under the column \UniMath{}) and that of the \Agda{} development
of \cite{non-wellfounded} (under the column \MTypes{}).
\begin{table}[ht]
  \centering
  \begin{tabular}{c | c | c }
    This thesis & \UniMath{} & \MTypes{} \\ \hline
  \end{tabular}
\end{table}

\chapter{CCCs and the λ-calculus}
\label{chap:cccs-and-lc}

\Cref{chap:category-theory} provides \textit{almost} enough background to
understand an interesting connection between a certain kind of category and the
λ-calculus.

\begin{lemma}[\unimathname{CategoryTheory.categories.Types.ExponentialsType}]
	Any fixed universe $\universe$ of types has exponentials.
\end{lemma}
\begin{proof}
	
\end{proof}

\begin{definition}
	A \define{monoidal category} consists of a category $\bfC$ together with
  \begin{itemize}
    \itemsep-0.2em
    \item a bifunctor $\bfC×\bfC→\bfC$ called the \define{tensor product} or
      \define{monoidal product},
    \item an object $I∈\Obj\bfC$ called the \define{unit},
    \item for all objects $A,B,C∈\Obj\bfC$, an isomorphism
      $α_{A,B,C}:(A⊗B)⊗C≅A⊗(B⊗C)$, and
    \item for all objects $A∈\Obj\bfC$, isomorphisms $λ_A:I⊗A≅A$ and $ρ_A:A⊗I≅A$.
  \end{itemize}
  These isomorphisms are subject to additional ``coherence conditions'', which
  won't play a crucial role here. A monoidal category is
  \begin{itemize}
    \itemsep-0.2em
    \item \define{symmetric} if there are additional isomorphisms
      $s_{A,B}:A⊗B→B⊗A$ satisfying other coherence conditions; and is
    \item \define{cartesian} if the monoidal product $⊗$ coincides with the
      categorical product $×$.
  \end{itemize}
\end{definition}

Only cartesian monoidal categories play a crucial role later in this section.

\begin{remark}
	For any categorical product there is a specified isomorphism $A×B≅B×A$, so any
  cartesian monoidal category is also symmetric monoidal.
\end{remark}

\begin{example}
  \
  \begin{itemize}
    \itemsep-0.2em
    \item \vspace{-0.3em} $\Set$, $\universe$, $\Cat$, $\Grp$, and others are 
      cartesian (so necessarily symmetric) monoidal under their categorical
      products. 
    \item $F\dVect$ is symmetric monoidal under $⊗$; the trivial vector space is
      a unit.
  \end{itemize}
\end{example}

\begin{definition}\label[definition]{def:exponentials}
  An \define{internal hom} for a monoidal category $(\bfC,⊗)$
  consists of, for each pair of objects $A,B∈\Obj\bfC$, an object
  $A⇒B$ together an arrow $\eval:(A⇒B)×A→B$ satisfying the following universal
  property: for any object $C∈\Obj\bfC$ and arrow $f:C⊗A→B$, there is an arrow
  $λf:C→(A⇒B)$ making the following diagram commute:
  \begin{center}
    \begin{tikzcd}[sep=large]
      C×A \arrow[dr, "f"] \arrow[d, swap, "λf×\id_A"] & {} \\
      (A⇒B)×A \arrow[r, swap, "\eval"]      & B
    \end{tikzcd}
  \end{center}
  When $\bfC$ is cartesian, $A⇒B$ is called an \define{exponential},
  and is denoted $B^A$.
\end{definition}

\begin{definition}
  A monoidal category $(\bfC,⊗)$ is \define{closed} when it has exponentials.
\end{definition}

\begin{remark}
  (Compare to \cref{rmk:prod-coprod-universal-adj})
  In a closed monoidal category, for all $A,B,C∈\Obj\bfC$,
  $λ$ and $\eval$ define a bijection
  \begin{equation*}
    \Hom(A⊗B,C)≅\Hom(A,B⇒C).
  \end{equation*}
  This bijection can be taken as their definition
  with an additional ``naturalality'' condition.
  This bijection is often called \define{currying}.
\end{remark}

\begin{example*}
  $\Set$ and $\universe$ are closed monoidal. Define $B⇒A\coloneqq \Hom(B,A)$.
  Then this is an element of $\Set$/$\universe$. Mixing set- and type-theoretical
  notation, the bijection is given by the mutually inverse functions
  \begin{align*}
    \Hom(A×B,C) &\longrightarrow \Hom(A,\Hom(B,C)) \\
    f &\longmapsto \λ{a}{\λ{b}{\appply{f}{a}{b}}} \\
    \Hom(A,\Hom(B,C)) &\longrightarrow \Hom(A×B,C) \\
    g &\longmapsto \λ{p}{\appply{g}{(\appr{1}{a})}{(\appr{2}{b})}}
  \end{align*}
\end{example*}

The following proof will only be accessible to readers with a background in
category theory (specifically, who know what adjunctions and natural
transformations are), but isn't necessary for the rest of the appendix.

\begin{lemma}[\unimathname{CategoryTheory.categories.Cats.ExponentialsCat}]
	The precategory of categories (\cref{def:precat-of-precats}) has exponentials.
\end{lemma}
\begin{proof}
  Fix $A:\universe$. We need to define a functor $E:\ttfun{Cat}→\ttfun{Cat}$ which
  is (RIGHT/LEFT) adjoint to $B↦A×B$. Define its action on objects by
  \begin{align*}
    E : \ttfun{Cat} &\longrightarrow \ttfun{Cat} \\
    C &\longmapsto_0 [A,C]
  \end{align*}
  that is, $E$ maps each category to the category of functors $A→C$ (with
  natural transformations as morphisms).
  
  Here things get a bit tricky: the action of $E$ on arrows takes a functor in
  $[A,C]$ to one in $[A,D]$ for each functor $C→D$. This assignment must
  \textit{itself} be functorial. Fix some categories $C$ and $D$ and a functor
  $F:C→D$, and define $E'$ on objects as
  \begin{align*}
    E' : [A,C] &\longrightarrow [A,D] \\
    G &\longmapsto_0 F_0∘G \\
    η &\longmapsto_1 \braces{\apply{F_1}{η_a}}_{a:\Obj A}.
  \end{align*}
  On arrows, $E'$ maps a natural transformation $η:G⟹H$ to one
  $\apply{E'}{η}:F∘G⟹F∘H$. To show that $\apply{E'}{η}$ is indeed a natural
  transformation, let $a,b:\Obj A$ and $f:\appply{\Hom}{a}{b}$. We want to show
  that the following square commutes:
  \begin{center}
    \begin{tikzcd}
      \apply{(F∘G)}{a}
        \arrow[r, "\apply{(F∘G)}{f}"]
        \arrow[d, "\apply{F_1}{η}_a"] &
      \apply{(F∘G)}{b}
        \arrow[d, "\apply{F_1}{η}_b"] \\
      \apply{(F∘H)}{a}
        \arrow[r, "\apply{(F∘H)}{f}"] &
      \apply{(F∘H)}{b}
    \end{tikzcd}
  \end{center}
  This is essentially the diagram expressing the naturality of $η$, but with $F$
  applied everywhere. The proof that it commutes reflects this structure:
  \begin{align*}
    F_1η_b ∘ \big(\apply{(F∘G)}{f}\big)
    &\≡ F_1η_b ∘ \apply{F_1}{(\apply{G}{f})} \\
    &\= \apply{F_1}{(η_b ∘ \apply{G}{f})} 
    &&\text{Functoriality of }F \\
    &\= \apply{F_1}{(\apply{H}{f} ∘ η_a)} 
    &&\text{Naturality of }η \\
    &\= \apply{F_1}{(\apply{H}{f})} ∘ F_1η_a 
    &&\text{Functoriality of }F \\
    &\≡ \big(\apply{(F∘H)}{f}\big) ∘ F_1η_a
  \end{align*}
\end{proof}

\begin{definition}\label[definition]{def:cartesian-cat}
  A category is \define{Cartesian closed}
  when it is closed, cartesian monoidal, and has a terminal object.
  Equivalently, it is cartesian closed when it has
  \begin{itemize}
    \itemsep-0.2em
    \item \vspace{-0.1em} a terminal object (\cref{def:terminal-and-initial}),
    \item binary products (\cref{def:product-and-coproduct}), and
    \item exponentials (\cref{def:exponentials}).
  \end{itemize}
  ``CCC'' abbreviates ``Cartesian closed category''.
\end{definition}

\begin{remark}\label[remark]{rmk:lambek}
	We now have the vocabulary to (informally) extend the correspondence of
  \cref{sec:propositions-and-types} to statements about CCCs.
  \cite{lambek}
  \TODO{}
\end{remark}

\begin{remark}
  The Curry-Howard-Lambek correspondence gives the barest hint of insight into
  Voevodsky's innovation. He also gave a categorically-based model of a version
  of the typed λ-calculus, but of full Martin-Löf dependent type theory. The
  structure of higher/nested identity types
  ($\propeq{\propeq{\cdots}{a}{b}}{p}{q}$) recalled ideas of homotopy theory,
  where paths are considered identical up to the existence of ``higher
  paths''.\TODO{}

  For the initial development of the connection between locally Cartesian closed
  categories (LCCCs) and type theory, see \cite{seely}.
\end{remark}
