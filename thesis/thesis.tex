\documentclass[12pt,twoside]{reedthesis}
\usepackage{graphicx} 
\usepackage{booktabs,setspace} 
% \usepackage{natbib}
% Comment out the natbib line above and uncomment the following two lines to use the new 
% biblatex-chicago style, for Chicago A. Also make some changes at the end where the 
% bibliography is included. 
%\usepackage{biblatex-chicago}
%\bibliography{thesis}

%%%%%%%%%%%%%%%%%%%%%%%%%%%%%%% While drafting:
\usepackage{showlabels}
% \usepackage{showidx}
\let\oldindex\index
\renewcommand{\index}[1]{\oldindex{#1}\marginpar{\footnotesize idx: #1}}
%%%%%%%%%%%%%%%%%%%%%%%%%%%%%%% My stuff:
%% math symbols, notation, etc
\usepackage{amssymb}       % math characters
\usepackage{mathtools}     % improvement on amsmath

% Use better math fonts when available
\IfFileExists{../tex-preamble/fonts/latinmodern-math.otf}
  {\newcommand{\fontpath}{../tex-preamble/fonts}}
  {\newcommand{\fontpath}{../../tex-preamble/fonts}}

\IfFileExists{\fontpath}{
  \usepackage{unicode-math}         % use OTF math fonts
  \AtBeginDocument{\let\phi\varphi} % varphi with unicode-math
  %\AtBeginDocument{\let\epsilon\varepsilon}
  % \setmathfont[Path=\fontpath]{latinmodern-math.otf}
  \setmathfont[Path=\fontpath]{Asana-Math.otf}
  \setmathfont[Path=\fontpath,range=\setminus]{Asana-Math.otf}
  \setmathfont[Path=\fontpath,range=\mathbb]{texgyretermes-math.otf}
  \setmathfont[Path=\fontpath,range=\mathcal]{latinmodern-math.otf}

  % fix xmapsto. only works with text above.
  % https://github.com/wspr/unicode-math/issues/197
  \ExplSyntaxOn
  \RenewDocumentCommand \xmapsto { m } {
    \mathrel {
      \exp_after:wN \Uoverdelimiter \cs:w sym \__um_symfont_tl \cs_end: "21A6 {
        \mkern 5mu \scan_stop: #1 \mkern 5mu \scan_stop:
      }
    }
  }
  \ExplSyntaxOff
}{}

% general stuff
\newcommand{\units}[1]{\;\mathrm{#1}}
\newcommand{\dd}{\,\mathrm{d}}
\newcommand{\ov}{\overline}
\newcommand{\paren}[1]{\left(#1\right)}
\newcommand{\braces}[1]{\left\{#1\right\}}
\newcommand{\brackets}[1]{\left[#1\right]}
\newcommand{\ceiling}[1]{\left\lceil{} #1 \right\rceil}
\newcommand{\floor}[1]{\left\lfloor{} #1 \right\rfloor}
\newcommand{\abs}[1]{\left| #1 \right|}
\newcommand{\e}[1]{\cdot 10^{#1}}
% \newcommand{\emf}{\mathcal{E}}
\newcommand{\contradiction}{\Rightarrow\Leftarrow}
\newcommand{\inv}[1]{\frac{1}{#1}}
\newcommand{\tif}{\text{ if }}
\newcommand{\totherwise}{\text{ otherwise }}
% \newcommand{\mydef}{\textbf{Definition:}}
% complex numbers
\renewcommand{\Im}{\operatorname{Im}}
\renewcommand{\Re}{\operatorname{Re}}

% operators/functions
\DeclareMathOperator{\ord}{ord}
% \DeclareMathOperator{\image}{image}
\DeclareMathOperator{\Int}{Int}
\DeclareMathOperator{\Ext}{Ext}
\DeclareMathOperator{\Lim}{Lim}
\DeclareMathOperator{\Bd}{Bd}
\DeclareMathOperator{\lcm}{lcm}
\DeclareMathOperator{\proj}{proj}
\DeclareMathOperator{\xor}{xor}
\DeclareMathOperator{\spanf}{span}
\DeclareMathOperator{\id}{id}
% linear
\DeclareMathOperator{\col}{Col}
\DeclareMathOperator{\row}{Row}
\DeclareMathOperator{\nul}{Nul}
\DeclareMathOperator{\rank}{rank}

% set theory
\newcommand{\xlongmapsto}[1]{\xmapsto{\,\,\, #1\,\,\,}}
\newcommand{\xlongrightarrow}[1]{\xrightarrow{\,\,\, #1\,\,\,}}
\newcommand{\union}{\mathop{\bigcup}}
\newcommand{\disunion}{\mathop{\dot\bigcup}}
\newcommand{\intersection}{\mathop{\bigcap}}
\newcommand{\powerset}{\mathcal{P}}

% analysis & calculus
\usepackage{esint} % \oiint, etc
\newcommand{\evalat}[2]{{\bigg|_{#1}^{#2}}}
\newcommand{\ext}[1]{{#1}^{ext}}  % extension of function
\DeclareMathOperator{\determinant}{det}
\DeclareMathOperator{\oball}{B}
\DeclareMathOperator{\cball}{\ov{B}}
\DeclareMathOperator{\sphere}{S}
\DeclareMathOperator{\divergence}{div}
\DeclareMathOperator{\curl}{curl}

% vectors
\newcommand{\ihat}{\hat{\imath}}
\newcommand{\jhat}{\hat{\jmath}}
\newcommand{\khat}{\hat{k}}
\newcommand{\lvec}[1]{\overrightarrow{#1}}
\newcommand{\len}[1]{\left\| #1 \right\|}

% topology
\newcommand{\RP}{\ensuremath{\R\hspace{-0.07em}\mathrm{P}}}

% categories
\newcommand{\I}{\ensuremath{\mathbf{I}}} % "interval groupoid"
\newcommand{\Cat}{\ensuremath{\mathcal{C}\mathrm{at}}}
\newcommand{\ComGrp}{\ensuremath{\mathcal{C}\mathrm{om}\mathcal{G}\mathrm{rp}}}
\newcommand{\AbGrp}{\ensuremath{\mathcal{A}\mathrm{b}\mathcal{G}\mathrm{rp}}}
\newcommand{\Grp}{\ensuremath{\mathcal{G}\mathrm{rp}}}
\newcommand{\Grpd}{\ensuremath{\mathcal{G}\mathrm{rpd}}}
\newcommand{\Mod}{\ensuremath{\mathcal{M}\mathrm{od}}}
\newcommand{\Man}{\ensuremath{\mathcal{M}\mathrm{an}}}
\newcommand{\Metric}{\ensuremath{\mathcal{M}\mathrm{etric}}}
\newcommand{\Set}{\ensuremath{\mathcal{S}\mathrm{et}}}
\newcommand{\Top}{\ensuremath{\mathcal{T}\!\mathrm{op}}}
\newcommand{\homoTop}{\ensuremath{\mathrm{homo}\mathcal{T}\mathrm{op}}}
\newcommand{\Vect}{\ensuremath{\mathcal{V}\mathrm{ect}}}
\renewcommand{\phi}{\varphi}

% number theory
\newcommand{\modulus}[1]{\; \left(\mathrm{mod}\; #1\right)}
\newcommand{\crash}[2]{\left(\frac{#1}{#2}\right)}
\newcommand{\QED}{\begin{flushright}QED\end{flushright}}

% make letters in different fonts (bold, blackboard, caligraphic, fraktur)
% https://tex.stackexchange.com/questions/156196/newcommand-with-variable-name
\newcommand{\mkltr}[1]{%
    \expandafter\newcommand\csname bf#1\endcsname{\ensuremath{\mathbb #1}}
    \expandafter\newcommand\csname bb#1\endcsname{\ensuremath{\mathbb #1}}
    \expandafter\newcommand\csname ca#1\endcsname{\ensuremath{\mathcal #1}}
    \expandafter\newcommand\csname fr#1\endcsname{\ensuremath{\mathfrak #1}}
}%
\mkltr A \mkltr B \mkltr C \mkltr D \mkltr E \mkltr F \mkltr G \mkltr H
\mkltr I \mkltr J \mkltr K \mkltr L \mkltr M \mkltr N \mkltr O \mkltr P
\mkltr Q \mkltr R \mkltr S \mkltr T \mkltr U \mkltr V \mkltr W \mkltr X
\mkltr Y \mkltr Z

\newcommand{\zn}{\mathbf{Z}_n}
\newcommand{\zp}{\mathbf{Z}_p}
\newcommand{\N}{\ensuremath{\mathbf{N}}}
\newcommand{\Z}{\ensuremath{\mathbf{Z}}}
\newcommand{\Q}{\ensuremath{\mathbf{Q}}}
\newcommand{\R}{\ensuremath{\mathbf{R}}}
\newcommand{\C}{\ensuremath{\mathbf{C}}}

% algebra
\newcommand{\quotient}[2]{{\raisebox{.2em}{$#1$}\left/\raisebox{-.2em}{$#2$}\right.}}
\newcommand{\GL}{\mathrm{GL}}     % general linear group
\newcommand{\SL}{\mathrm{SL}}     % special linear group
\renewcommand{\S}{\mathfrak{S}}   % symmetric group
\DeclareMathOperator{\ann}{ann}   % the annihilator of a submodule
\DeclareMathOperator{\Obj}{Obj}   % objects of a category
\DeclareMathOperator{\Hom}{Hom}   % homomorphisms between modules
\DeclareMathOperator{\Mor}{Mor}   % categorical morphisms
\DeclareMathOperator{\characteristic}{char} % ring characteristic
\DeclareMathOperator{\Tor}{Tor}   % torsion elements
\DeclareMathOperator{\Stab}{Stab} % stabilizer
\DeclareMathOperator{\Orb}{Orb}   % orbit
\DeclareMathOperator{\Aut}{Aut}   % automorphisms
\DeclareMathOperator{\End}{End}   % endomorphisms
\DeclareMathOperator{\Mat}{Mat}   % matrix ring
\DeclareMathOperator{\Tr}{Trace}  % trace of a matrix
\DeclareMathOperator{\im}{im}     % image

% statistics
\DeclareMathOperator{\Var}{Var}

% nicer empty set
\let\oldemptyset\emptyset{}
\let\emptyset\varnothing{}
%% formatting

\usepackage{fontspec}
% EB Garamond (Initials - just for fun) - body text, old style, serif
% Nimbus Roman - serifed
% URW Gothic - geometric, title text
% Libre Caslon Text - absurdly well spaced
% FreeSerif - plays pretty nicely with math, looks like Times
% Tinos - good alternative to FreeSerif, less TNR looking
% \setmainfont[Ligatures=TeX]{Tinos}
% \fontspec [Path = ../tex-preamble/fonts/, 
%            UprightFont = *-regular,
%            BoldFont    = *-bold,
%            ItalicFont  = *-italic]
%            {texgyrepagella}
\setmainfont[ Path=../tex-preamble/fonts/
            , UprightFont     = *-regular
            , BoldFont        = *-bold
            , ItalicFont      = *-italic
            , BoldItalicFont  = *-bolditalic
            ]{texgyrepagella}

\usepackage{datetime2} % use yyyy--mm--dd date with \DTMtoday
\usepackage{geometry}
\geometry{
  lmargin=4cm,
  rmargin=4cm,
  tmargin=3cm,
  bmargin=3cm,
}

\renewcommand{\arraystretch}{1.2} % Make tables a little bigger
% \usepackage{indentfirst}          % indent first \par after sections
% \usepackage{parskip}              % don't indent anything
\usepackage{enumitem}             % enumerate with A), b., c) styles
\usepackage{hyperref}             % linked table of contents
\hypersetup{
    colorlinks,
    citecolor=black,
    filecolor=black,
    linkcolor=black,
    urlcolor=black
}
\usepackage[noabbrev, capitalize]{cleveref}
\crefname{notation}{Notation}{Notations}
\Crefname{notation}{Notation}{Notations}
\crefname{rule}{Rule}{Rule}
\Crefname{rule}{Rule}{Rule}

% graphics
\usepackage{graphicx}
\DeclareGraphicsExtensions{.pdf,.png,.jpg}
\usepackage{float} % image positioning
% Theorems, proofs, and other mathematical environments

\usepackage{amsthm}

\newtheoremstyle{mine} % name
  {7pt} % space above
  {7pt} % space below
  {} % body font
  {} % indent amount
  {\bfseries} % theorem head font
  {.} % punctuation after theorem head
  {.5em} % space after theorem head
  {} % theorem head spec (can be left empty, meaning no)

\theoremstyle{mine} % these are the non-italicized
\newtheorem{theorem}{Theorem}[subsection]
\newtheorem*{theorem*}{Theorem}
\newtheorem{lemma}[theorem]{Lemma}
\newtheorem*{lemma*}{Lemma}
\newtheorem{proposition}[theorem]{Proposition}
\newtheorem*{proposition*}{Proposition}
\newtheorem{corollary}[theorem]{Corollary}
\newtheorem*{corollary*}{Corollary}
\newtheorem{eqenv}[theorem]{Equation}
\newtheorem*{eqenv*}{Equation}
\newtheorem{remark}[theorem]{Remark}
\newtheorem*{remark*}{Remark}
\newtheorem{example}[theorem]{Example}
\newtheorem*{example*}{Example}
\newtheorem{definition}[theorem]{Definition}
\newtheorem*{definition*}{Definition}
\newcounter{Problem} \stepcounter{Problem} % start at 1
\newcommand{\problem}[0]{
  \vspace{-0.8em}
  \begin{center}
    \huge
    \rule[0.25em]{0.3\textwidth}{0.6pt}
    \arabic{Problem}
    \rule[0.25em]{0.3\textwidth}{0.6pt}
  \end{center}
  \stepcounter{Problem}
  \vspace{0.5em}
}
\newcommand{\problemnum}[1]{
  \vspace{-0.5em}
  \begin{center}
    \huge
    \rule[0.25em]{0.3\textwidth}{0.5pt}
    #1
    \rule[0.25em]{0.3\textwidth}{0.5pt}
  \end{center}
  \vspace{0.5em}
}
\newcommand{\subproblem}[1]{
  \vspace{0.2em}
  \begin{center}
    \large
    \rule[0.25em]{0.2\textwidth}{0.5pt}
    #1
    \rule[0.25em]{0.2\textwidth}{0.5pt}
  \end{center}
  \vspace{0.2em}
}
% Combine the two with more pleasant vertical spacing
\newcommand{\problemsubproblem}[1]{\problem \vspace{-0.8em}\subproblem{#1}}
\newcommand{\problemnumsubproblem}[2]{\problemnum{#1}\vspace{-0.8em}\subproblem{#2}}
\newcommand{\abbreviation}[1]{\textbf{#1}} % abbreviations
\usepackage{makeidx}

\title{Deriving Coinductive Types in Univalent Type Theory}
% \title{Bruh: $(A\simeq B)\simeq (A=_{\mathcal{U}}B)$}
\author{Langston Barrett}
% The month and year that you submit your FINAL draft TO THE LIBRARY (May or December)
\date{May 2018}
\division{Mathematics and Natural Sciences}
\advisor{Safia Chettih}
\department{Mathematics}
% if you're writing a thesis in an interdisciplinary major,
% uncomment the line below and change the text as appropriate.
% check the Senior Handbook if unsure.
\thedivisionof{The Established Interdisciplinary Committee for}

\setlength{\parskip}{0pt}
\begin{document}

\maketitle
\frontmatter % this stuff will be roman-numbered
\pagestyle{empty} % this removes page numbers from the frontmatter

% Acknowledgements (Acceptable American spelling) are optional
% So are Acknowledgments (proper English spelling)
\chapter*{Acknowledgments}

% The preface is optional
% To remove it, comment it out or delete it.
% \chapter*{Preface}
% This thesis delves into highly interdisciplinary territory. I'll

\chapter*{List of Abbreviations}

\begin{table}[h]
  \centering
  \begin{tabular}{ll}
    \abbreviation{HoTT}  	&  Homotopy type theory \\
    \abbreviation{UTT}  	&  Univalent type theory \\
    \abbreviation{ITT}  	&  Intensional type theory \\
    \abbreviation{FOL}  	&  First-order logic (Gentzen's natural deduction) \\
  \end{tabular}
\end{table}
	
\tableofcontents
% if you want a list of tables, optional
% \listoftables
% if you want a list of figures, also optional
% \listoffigures

% If your abstract is longer than a page, there may be a formatting issue.
\chapter*{Abstract}

In this thesis, we explain Univalent type theory, a constructive and
computationally meaningful foundational system for mathematics inspired by
recent advances in the semantics of Per Martin-L\"of's intentional type theory.
We first develop the classical Curry-Howard correspondence between (the
constructive subset of) Gentzen's natural deducation and the $\lambda$-calculus.
We proceed to explicate Martin-L\"of's theory of dependent types and the central
modern development in type theory: Vladimir Voevodsky's Univalence principle. 
We go on to examine the nature of coinduction within this theory, presenting a
novel formalization of a recent result that M-types can be derived
\textit{internally} in Univalent type theory.

\chapter*{Dedication}

\mainmatter % here the regular arabic numbering starts
\pagestyle{fancyplain} % turns page numbering back on

\chapter*{Introduction}
\addcontentsline{toc}{chapter}{Introduction}
\chaptermark{Introduction}
\markboth{Introduction}{Introduction}

\section*{A short history of discomfort in mathematics}

% \subsection*{Trouble in Cantor's Paradise}

In the early 20\textsuperscript{th} century, mathematicians had a problem. Often
held up as the pinnacle of necessary, undoubtable, \textit{a priori} truth,
their discipline was suffering from an embarrassing lack of certainty. The
intricate arguments of advanced analysis left mathematicians unable to confirm
nor deny each other's proofs. While Cantor's na\"ive set theory had unleashed a
wave of paradoxes, the need for a rigorous logical foundation for higher
mathematics was so great that the enterprise of axiomatic set theory was pursued
headlong until some kind of consensus was reached. To make a long, rich story
brutally short, virtually every modern paper in mathematics and computer
science uses a combination of Gentzen's first-order logic (\abbreviation{FOL})
and an axiomatization of sets called
\abbreviation{ZFC}\index{ZFC}.\footnote{This ``consensus'' left out many
  prominent schools of thought, such as the Intuitionists and constructivists, a
  point we'll soon return to.} 

Despite the apparent rigor of \abbreviation{FOL}+\abbreviation{ZFC},
practitioners of the deductive sciences were still beset by issues of
verifiability. Complexity, specialization and sheer length made modern proofs
difficult to comprehend with the absolute certainty which is supposed to
characterize these disciplines. In the 1970s, two teams of topologists proved
contradictory results, and neither group could find the error in the other's
proof \cite{kolata}. Wiles's famous proof of Fermat's last theorem was utterly
unintelligible to the vast majority of mathematicians \cite{nyt}. The
classification of the simple finite groups, one of the crowning achievements of
modern mathematics, has a combined proof of over ten thousand pages. These
examples illustrate merely a few of the practical epistemological challenges
facing mathematicians, for a historical perspective see \cite{rigor-and-proof},
and for a general overview see \cite{fidelity}.

In 1990s, Fields medalist Vladimir Voevodsky grew concerned with the state
of mathematical knowledge. In 1998, Carlos Simpson released a pre-print arguing
that there was a major mistake in one of Voevodsky's papers. However, it was not
clear whether Voevodsky had errored, or whether there was a flaw in Simpson's
counterexample. In 1999, Pierre Deligne found a crucial mistake in Voevodsky's
``Cohomological Theory of Presheaves with Transfers'', upon which he had based
much of his work in the area of motivic cohomology. As he began to develop more
and more complex arguments, Voevodsky wondered: ``And who would ensure that I
did not forget something and did not make a mistake, if even the mistakes in
much more simple arguments take years to uncover?'' \cite{voevodsky-ias}. 

The problems facing Voevodsky and his peers seemed insurmountable.
As the requisite attention span, memory, and capacity for detail required to
understand new developments in higer mathematics reached inhuman proportions,
where was he to turn? He would not find a solution in the realm of pure
mathematics, but rather in one of the finest examples of collaboration between
mathematicians, computer scientists, and philosophers: the modern proof
assistant. % TODO: reword

\section*{TODO}

While much of the mathematical discipline was simply relieved to have ``solved''
their foundational issues with axiomatic set theory, there remained a vocal
opposition to the newly-adopted methods. Most mathematicians are dimly aware
that there's some controversy about the Axiom of Choice (the ``C'' in
\abbreviation{ZFC}) \cite{martin-lof-100-years}, you don't have to look further
than \abbreviation{FOL} to find disagreements.

\section*{Our contribution}

(Co)inductive types and UniMath

\chapter{Propositions and Types}
\label{chap:propositions-and-types}

\section{Natural deduction}
\label{sec:natural-deduction}

Formal systems, proof trees

\section{The $\lambda$-calculus}
\label{sec:the-lambda-calculus}

\subsection{The untyped $\lambda$-calculus and computation}
\label{subsec:the-untyped-lambda-calculus}

The Church-Turing thesis

$\beta$-reduction

\subsection{The simply-typed $\lambda$-calculus}
\label{subsec:the-simply-typed-lambda-calculus}

Two judgements

The strong normalization property

\section{The Curry-Howard correspondence}
\label{sec:the-curry-howard-correspondence}

\subsection{The BHK interpretation}
\label{subsec:the-bhk-interpretation}

\subsection{On the law of the excluded middle}
\label{subsec:on-the-law-of-the-excluded-middle}

\chapter{Type theory}
\label{chap:type-theory}

\section{The identity type}
\label{sec:the-identity-type}

\section{Dependent types}
\label{sec:dependent-types}

\section{$h$-levels and truncation}
\label{sec:$h$-levels and truncation}

\section{Univalence}
\label{sec:univalence}

\subsection{Weak equivalences}
\label{subsec:weak-equivalences}

\subsection{The univalence axiom}
\label{subsec:the-univalence-axiom}

\chapter{Coinductive types in univalent type theory}
\label{chap:coinductive-types-in-univalent-type-theory}

\section{W and M}
\label{sec:w-and-m}

\section{Category theory}
\label{sec:category-theory}

\subsection{Basics}
\label{subsec:basics}

\subsection{Functor algebras}
\label{subsec:functor-algebras}

\section{Internalizing M-types}
\label{sec:internalizing-m-types}

\chapter*{Conclusion}
\addcontentsline{toc}{chapter}{Conclusion}
\chaptermark{Conclusion}
\markboth{Conclusion}{Conclusion}
\setcounter{chapter}{4}
\setcounter{section}{0}


%If you feel it necessary to include an appendix, it goes here.
% \appendix
% \chapter{The First Appendix}
% \chapter{The Second Appendix, for Fun}

\backmatter % backmatter makes the index and bibliography appear properly in the TOC

% if you're using bibtex, the next line forces every entry in the bibtex file to be included
% in your bibliography, regardless of whether or not you've cited it in the thesis.
\nocite{*}

%  \bibliographystyle{bsts/mla-good} % there are a variety of styles available; 
%  \bibliographystyle{plainnat}
% replace ``plainnat'' with the style of choice. You can refer to files in the bsts or APA 
% subfolder, e.g. 
% \bibliographystyle{APA/apa-good}  % or
\bibliography{thesis}
\bibliographystyle{plain}
% Comment the above two lines and uncomment the next line to use biblatex-chicago.
%\printbibliography[heading=bibintoc]

% Finally, an index would go here... but it is also optional.
\printindex
\end{document}
