\documentclass[12pt,twoside]{reedthesis}
\usepackage{graphicx}
\usepackage{booktabs,setspace}
% \usepackage{natbib}
% Comment out the natbib line above and uncomment the following two lines to use the new
% biblatex-chicago style, for Chicago A. Also make some changes at the end where the
% bibliography is included.
%\usepackage{biblatex-chicago}
%\bibliography{thesis}

%%%%%%%%%%%%%%%%%%%%%%%%%%%% While drafting:
\makeatletter
\def\ifdraft{\ifdim\overfullrule>\z@
  \expandafter\@firstoftwo\else\expandafter\@secondoftwo\fi}
\makeatother

\ifdraft{
  \usepackage{showlabels}
  % \usepackage{showidx}
  \let\oldindex\index
  \definecolor{index}{HTML}{0088EE}
  \renewcommand{\index}[1]
               {\oldindex{#1}\marginpar{\footnotesize\color{index}index: #1}}
  \newcommand{\indeX}{\oldindex}
}{
  \newcommand{\indeX}{\index}
}

\usepackage{color}
\usepackage[dvipsnames]{xcolor}
\definecolor{TODO}{HTML}{EE8800}
\newcommand{\TODO}[1]{\marginpar{\footnotesize\color{TODO}todo: #1}}

%%%%%%%%%%%%%%%%%%%%%%%%%%%%%%% My stuff:
%% math symbols, notation, etc
\usepackage{amssymb}       % math characters
\usepackage{mathtools}     % improvement on amsmath

% Use better math fonts when available
\IfFileExists{../tex-preamble/fonts/latinmodern-math.otf}
  {\newcommand{\fontpath}{../tex-preamble/fonts}}
  {\newcommand{\fontpath}{../../tex-preamble/fonts}}

\IfFileExists{\fontpath}{
  \usepackage{unicode-math}         % use OTF math fonts
  \AtBeginDocument{\let\phi\varphi} % varphi with unicode-math
  %\AtBeginDocument{\let\epsilon\varepsilon}
  % \setmathfont[Path=\fontpath]{latinmodern-math.otf}
  \setmathfont[Path=\fontpath]{Asana-Math.otf}
  \setmathfont[Path=\fontpath,range=\setminus]{Asana-Math.otf}
  \setmathfont[Path=\fontpath,range=\mathbb]{texgyretermes-math.otf}
  \setmathfont[Path=\fontpath,range=\mathcal]{latinmodern-math.otf}

  % fix xmapsto. only works with text above.
  % https://github.com/wspr/unicode-math/issues/197
  \ExplSyntaxOn
  \RenewDocumentCommand \xmapsto { m } {
    \mathrel {
      \exp_after:wN \Uoverdelimiter \cs:w sym \__um_symfont_tl \cs_end: "21A6 {
        \mkern 5mu \scan_stop: #1 \mkern 5mu \scan_stop:
      }
    }
  }
  \ExplSyntaxOff
}{}

% general stuff
\newcommand{\units}[1]{\;\mathrm{#1}}
\newcommand{\dd}{\,\mathrm{d}}
\newcommand{\ov}{\overline}
\newcommand{\paren}[1]{\left(#1\right)}
\newcommand{\braces}[1]{\left\{#1\right\}}
\newcommand{\brackets}[1]{\left[#1\right]}
\newcommand{\ceiling}[1]{\left\lceil{} #1 \right\rceil}
\newcommand{\floor}[1]{\left\lfloor{} #1 \right\rfloor}
\newcommand{\abs}[1]{\left| #1 \right|}
\newcommand{\e}[1]{\cdot 10^{#1}}
% \newcommand{\emf}{\mathcal{E}}
\newcommand{\contradiction}{\Rightarrow\Leftarrow}
\newcommand{\inv}[1]{\frac{1}{#1}}
\newcommand{\tif}{\text{ if }}
\newcommand{\totherwise}{\text{ otherwise }}
% \newcommand{\mydef}{\textbf{Definition:}}
% complex numbers
\renewcommand{\Im}{\operatorname{Im}}
\renewcommand{\Re}{\operatorname{Re}}

% operators/functions
\DeclareMathOperator{\ord}{ord}
% \DeclareMathOperator{\image}{image}
\DeclareMathOperator{\Int}{Int}
\DeclareMathOperator{\Ext}{Ext}
\DeclareMathOperator{\Lim}{Lim}
\DeclareMathOperator{\Bd}{Bd}
\DeclareMathOperator{\lcm}{lcm}
\DeclareMathOperator{\proj}{proj}
\DeclareMathOperator{\xor}{xor}
\DeclareMathOperator{\spanf}{span}
\DeclareMathOperator{\id}{id}
% linear
\DeclareMathOperator{\col}{Col}
\DeclareMathOperator{\row}{Row}
\DeclareMathOperator{\nul}{Nul}
\DeclareMathOperator{\rank}{rank}

% set theory
\newcommand{\xlongmapsto}[1]{\xmapsto{\,\,\, #1\,\,\,}}
\newcommand{\xlongrightarrow}[1]{\xrightarrow{\,\,\, #1\,\,\,}}
\newcommand{\union}{\mathop{\bigcup}}
\newcommand{\disunion}{\mathop{\dot\bigcup}}
\newcommand{\intersection}{\mathop{\bigcap}}
\newcommand{\powerset}{\mathcal{P}}

% analysis & calculus
\usepackage{esint} % \oiint, etc
\newcommand{\evalat}[2]{{\bigg|_{#1}^{#2}}}
\newcommand{\ext}[1]{{#1}^{ext}}  % extension of function
\DeclareMathOperator{\determinant}{det}
\DeclareMathOperator{\oball}{B}
\DeclareMathOperator{\cball}{\ov{B}}
\DeclareMathOperator{\sphere}{S}
\DeclareMathOperator{\divergence}{div}
\DeclareMathOperator{\curl}{curl}

% vectors
\newcommand{\ihat}{\hat{\imath}}
\newcommand{\jhat}{\hat{\jmath}}
\newcommand{\khat}{\hat{k}}
\newcommand{\lvec}[1]{\overrightarrow{#1}}
\newcommand{\len}[1]{\left\| #1 \right\|}

% topology
\newcommand{\RP}{\ensuremath{\R\hspace{-0.07em}\mathrm{P}}}

% categories
\newcommand{\I}{\ensuremath{\mathbf{I}}} % "interval groupoid"
\newcommand{\Cat}{\ensuremath{\mathcal{C}\mathrm{at}}}
\newcommand{\ComGrp}{\ensuremath{\mathcal{C}\mathrm{om}\mathcal{G}\mathrm{rp}}}
\newcommand{\AbGrp}{\ensuremath{\mathcal{A}\mathrm{b}\mathcal{G}\mathrm{rp}}}
\newcommand{\Grp}{\ensuremath{\mathcal{G}\mathrm{rp}}}
\newcommand{\Grpd}{\ensuremath{\mathcal{G}\mathrm{rpd}}}
\newcommand{\Mod}{\ensuremath{\mathcal{M}\mathrm{od}}}
\newcommand{\Man}{\ensuremath{\mathcal{M}\mathrm{an}}}
\newcommand{\Metric}{\ensuremath{\mathcal{M}\mathrm{etric}}}
\newcommand{\Set}{\ensuremath{\mathcal{S}\mathrm{et}}}
\newcommand{\Top}{\ensuremath{\mathcal{T}\!\mathrm{op}}}
\newcommand{\homoTop}{\ensuremath{\mathrm{homo}\mathcal{T}\mathrm{op}}}
\newcommand{\Vect}{\ensuremath{\mathcal{V}\mathrm{ect}}}
\renewcommand{\phi}{\varphi}

% number theory
\newcommand{\modulus}[1]{\; \left(\mathrm{mod}\; #1\right)}
\newcommand{\crash}[2]{\left(\frac{#1}{#2}\right)}
\newcommand{\QED}{\begin{flushright}QED\end{flushright}}

% make letters in different fonts (bold, blackboard, caligraphic, fraktur)
% https://tex.stackexchange.com/questions/156196/newcommand-with-variable-name
\newcommand{\mkltr}[1]{%
    \expandafter\newcommand\csname bf#1\endcsname{\ensuremath{\mathbb #1}}
    \expandafter\newcommand\csname bb#1\endcsname{\ensuremath{\mathbb #1}}
    \expandafter\newcommand\csname ca#1\endcsname{\ensuremath{\mathcal #1}}
    \expandafter\newcommand\csname fr#1\endcsname{\ensuremath{\mathfrak #1}}
}%
\mkltr A \mkltr B \mkltr C \mkltr D \mkltr E \mkltr F \mkltr G \mkltr H
\mkltr I \mkltr J \mkltr K \mkltr L \mkltr M \mkltr N \mkltr O \mkltr P
\mkltr Q \mkltr R \mkltr S \mkltr T \mkltr U \mkltr V \mkltr W \mkltr X
\mkltr Y \mkltr Z

\newcommand{\zn}{\mathbf{Z}_n}
\newcommand{\zp}{\mathbf{Z}_p}
\newcommand{\N}{\ensuremath{\mathbf{N}}}
\newcommand{\Z}{\ensuremath{\mathbf{Z}}}
\newcommand{\Q}{\ensuremath{\mathbf{Q}}}
\newcommand{\R}{\ensuremath{\mathbf{R}}}
\newcommand{\C}{\ensuremath{\mathbf{C}}}

% algebra
\newcommand{\quotient}[2]{{\raisebox{.2em}{$#1$}\left/\raisebox{-.2em}{$#2$}\right.}}
\newcommand{\GL}{\mathrm{GL}}     % general linear group
\newcommand{\SL}{\mathrm{SL}}     % special linear group
\renewcommand{\S}{\mathfrak{S}}   % symmetric group
\DeclareMathOperator{\ann}{ann}   % the annihilator of a submodule
\DeclareMathOperator{\Obj}{Obj}   % objects of a category
\DeclareMathOperator{\Hom}{Hom}   % homomorphisms between modules
\DeclareMathOperator{\Mor}{Mor}   % categorical morphisms
\DeclareMathOperator{\characteristic}{char} % ring characteristic
\DeclareMathOperator{\Tor}{Tor}   % torsion elements
\DeclareMathOperator{\Stab}{Stab} % stabilizer
\DeclareMathOperator{\Orb}{Orb}   % orbit
\DeclareMathOperator{\Aut}{Aut}   % automorphisms
\DeclareMathOperator{\End}{End}   % endomorphisms
\DeclareMathOperator{\Mat}{Mat}   % matrix ring
\DeclareMathOperator{\Tr}{Trace}  % trace of a matrix
\DeclareMathOperator{\im}{im}     % image

% statistics
\DeclareMathOperator{\Var}{Var}

% nicer empty set
\let\oldemptyset\emptyset{}
\let\emptyset\varnothing{}
%% formatting

\usepackage{fontspec}
% EB Garamond (Initials - just for fun) - body text, old style, serif
% Nimbus Roman - serifed
% URW Gothic - geometric, title text
% Libre Caslon Text - absurdly well spaced
% FreeSerif - plays pretty nicely with math, looks like Times
% Tinos - good alternative to FreeSerif, less TNR looking
% \setmainfont[Ligatures=TeX]{Tinos}
% \fontspec [Path = ../tex-preamble/fonts/, 
%            UprightFont = *-regular,
%            BoldFont    = *-bold,
%            ItalicFont  = *-italic]
%            {texgyrepagella}
\setmainfont[ Path=../tex-preamble/fonts/
            , UprightFont     = *-regular
            , BoldFont        = *-bold
            , ItalicFont      = *-italic
            , BoldItalicFont  = *-bolditalic
            ]{texgyrepagella}

\usepackage{datetime2} % use yyyy--mm--dd date with \DTMtoday
\usepackage{geometry}
\geometry{
  lmargin=4cm,
  rmargin=4cm,
  tmargin=3cm,
  bmargin=3cm,
}

\renewcommand{\arraystretch}{1.2} % Make tables a little bigger
% \usepackage{indentfirst}          % indent first \par after sections
% \usepackage{parskip}              % don't indent anything
\usepackage{enumitem}             % enumerate with A), b., c) styles
\usepackage{hyperref}             % linked table of contents
\hypersetup{
    colorlinks,
    citecolor=black,
    filecolor=black,
    linkcolor=black,
    urlcolor=black
}
\usepackage[noabbrev, capitalize]{cleveref}
\crefname{notation}{Notation}{Notations}
\Crefname{notation}{Notation}{Notations}
\crefname{rule}{Rule}{Rule}
\Crefname{rule}{Rule}{Rule}

% graphics
\usepackage{graphicx}
\DeclareGraphicsExtensions{.pdf,.png,.jpg}
\usepackage{float} % image positioning
% Theorems, proofs, and other mathematical environments

\usepackage{amsthm}

\newtheoremstyle{mine} % name
  {7pt} % space above
  {7pt} % space below
  {} % body font
  {} % indent amount
  {\bfseries} % theorem head font
  {.} % punctuation after theorem head
  {.5em} % space after theorem head
  {} % theorem head spec (can be left empty, meaning no)

\theoremstyle{mine} % these are the non-italicized
\newtheorem{theorem}{Theorem}[subsection]
\newtheorem*{theorem*}{Theorem}
\newtheorem{lemma}[theorem]{Lemma}
\newtheorem*{lemma*}{Lemma}
\newtheorem{proposition}[theorem]{Proposition}
\newtheorem*{proposition*}{Proposition}
\newtheorem{corollary}[theorem]{Corollary}
\newtheorem*{corollary*}{Corollary}
\newtheorem{eqenv}[theorem]{Equation}
\newtheorem*{eqenv*}{Equation}
\newtheorem{remark}[theorem]{Remark}
\newtheorem*{remark*}{Remark}
\newtheorem{example}[theorem]{Example}
\newtheorem*{example*}{Example}
\newtheorem{definition}[theorem]{Definition}
\newtheorem*{definition*}{Definition}
\newcounter{Problem} \stepcounter{Problem} % start at 1
\newcommand{\problem}[0]{
  \vspace{-0.8em}
  \begin{center}
    \huge
    \rule[0.25em]{0.3\textwidth}{0.6pt}
    \arabic{Problem}
    \rule[0.25em]{0.3\textwidth}{0.6pt}
  \end{center}
  \stepcounter{Problem}
  \vspace{0.5em}
}
\newcommand{\problemnum}[1]{
  \vspace{-0.5em}
  \begin{center}
    \huge
    \rule[0.25em]{0.3\textwidth}{0.5pt}
    #1
    \rule[0.25em]{0.3\textwidth}{0.5pt}
  \end{center}
  \vspace{0.5em}
}
\newcommand{\subproblem}[1]{
  \vspace{0.2em}
  \begin{center}
    \large
    \rule[0.25em]{0.2\textwidth}{0.5pt}
    #1
    \rule[0.25em]{0.2\textwidth}{0.5pt}
  \end{center}
  \vspace{0.2em}
}
% Combine the two with more pleasant vertical spacing
\newcommand{\problemsubproblem}[1]{\problem \vspace{-0.8em}\subproblem{#1}}
\newcommand{\problemnumsubproblem}[2]{\problemnum{#1}\vspace{-0.8em}\subproblem{#2}}
% requires

% one with more spacing. ideally, this would auto-adapt to surroundings
\renewcommand{\:}{\hspace{0.3ex}:\hspace{0.3ex}}

\newcommand{\mvar}[1]{\ensuremath{\dot{#1}}}

\newcommand{\FV}{\ensuremath{\mathsf{FV}}}
\newcommand{\BV}{\ensuremath{\mathsf{BV}}}

% application
\newcommand{\apspace}{\hspace{0.25em}}
\newcommand{\apply}[2]{#1\apspace #2} % apply a function to its argument
\newcommand{\appply}[3]{#1\apspace #2 \apspace #3}
\newcommand{\apppply}[4]{#1\apspace #2 \apspace #3 \apspace #4}
\newcommand{\appppply}[5]{#1\apspace #2 \apspace #3 \apspace #4 \apspace #5}
\newcommand{\apppppply}[6]{#1\apspace #2 \apspace #3 \apspace #4 \apspace #5 \apspace #6}
\newcommand{\appppppply}[7]{#1\apspace #2 \apspace #3 \apspace #4 \apspace #5 \apspace #6 \apspace #7}

% LC
\newcommand{\lam}[2]{\uplambda #1.\apspace#2}                % λ-abstraction
\DeclareRobustCommand\longtwoheadrightarrow
     {\relbar\joinrel\twoheadrightarrow}
\newcommand{\betato}{\overset{\upbeta}
  {\;\vphantom{\rule{0pt}{.3ex}}\smash{{\longrightarrow}}}\;}
\newcommand{\betatoo}{\overset{\upbeta}
  {\;\vphantom{\rule{0pt}{.3ex}}\smash{{\longtwoheadrightarrow}}}\;}
\newcommand{\groundtype}{\mathbb{T}}

% Judgments
\newcommand{\judgment}[2]{#2\hspace{0.4em}\text{#1}}
\newcommand{\prop}[1]{\judgment{prop}{#1}}
\newcommand{\type}[1]{\judgment{type}{#1}}
\newcommand{\true}[1]{\judgment{true}{#1}}
\newcommand{\term}[1]{\judgment{term}{#1}}
\newcommand{\valid}[1]{\judgment{valid}{#1}}

% Names of formal systems
\newcommand{\formalsystem}[1]{\textbf{#1}\indeX{#1@\textbf{#1}}}
\newcommand{\HoTT}[0]{\formalsystem{HoTT}}  % homotopy type theory
\newcommand{\LC}[0]{\formalsystem{LC}}      % λ-calculus
\newcommand{\STLC}[0]{\formalsystem{STLC}}  % simply-typed λ-calculus
\newcommand{\TLC}[0]{\formalsystem{TLC}}    % typed λ-calculus
\newcommand{\ITT}[0]{\formalsystem{ITT}}    % intuitionistic type theory
\newcommand{\UTT}[0]{\formalsystem{UTT}}    % univalent type theory
\newcommand{\IPL}[0]{\formalsystem{IPL}}    % intuitionistic propositional logic
\newcommand{\FOL}[0]{\formalsystem{FOL}}    % first-order logic
\newcommand{\ZF}[0]{\formalsystem{ZF}}      % Zermelo–Fraenkel  set theory
\newcommand{\ZFC}[0]{\formalsystem{ZFC}}    % Zermelo-... with choice
% requires logic.tex
% some of this is based on HoTT/HoTT

\newcommand{\ttfun}[1]{\ensuremath{\mathsf{#1}}}  % type-theoretic functions

\newcommand{\eval}{\ttfun{eval}}

% recurors
\newcommand{\case}{\ttfun{case}}
\newcommand{\rec}{\ttfun{rec}}

\newcommand{\emptytype}{\ensuremath{\mathbf{0}}}  % empty type
\newcommand{\unittype}{\ensuremath{\mathbf{1}}}   % unit type
% \newcommand{\unitelem}{\ensuremath{1_{\unittype}}} % sole inhabitant of the unit type
\newcommand{\unitelem}{\star} % sole inhabitant of the unit type
\newcommand{\booltype}{\ensuremath{\mathbf{2}}}   % two element type
\newcommand{\btrue}{1_{\booltype}}                 % inhabitant of two element type
\newcommand{\bfalse}{0_{\booltype}}                % inhabitant of two element type
\newcommand{\inl}{\ttfun{inl}}
\newcommand{\inr}{\ttfun{inr}}
\newcommand{\universe}{\ensuremath{\mathcal{U}}}         % universe type
\newcommand{\universei}[1]{\ensuremath{\mathcal{U}_{#1}}}  % ith universe

\newcommand{\sigmat}[2]{\ensuremath{\sum_{\paren*{#1}}#2}}    % Σ-type
\newcommand{\dpair}[2]{\ensuremath{\paren{#1\hspace{0.1em};#2}}} % Σ-pair
\newcommand{\pr}[1]{\ttfun{pr}_{#1}}
\newcommand{\pit}[2]{\ensuremath{\prod_{\paren*{#1}}#2}}      % Π-type
\newcommand{\refl}[1]{\ttfun{refl}_{#1}}      % reflexivity
\newcommand{\W}[3]{\ttfun{W}_{(#1:#2)}\apply{#3}{#1}}      % W-types
\renewcommand{\M}[3]{\ttfun{M}_{(#1:#2)}\apply{#3}{#1}}      % M-types
\renewcommand{\sup}[2]{\appply{\ttfun{sup}}{#1}{#2}}      % supremum
\newcommand{\recW}{\rec_{\mathsf{W}}}                   % W-elim

\newcommand{\fromempty}[1]{\apply{\rec_{\emptytype}}{#1}}
\newcommand{\fromunit}[2]{\appply{\rec_{\unittype}}{#1}{#2}}


%%% Natural numbers
\newcommand{\suc}{\mathsf{succ}}
\newcommand{\add}{\mathsf{add}}

%%% Lists
\newcommand{\List}[1]{\ttfun{List}(#1)}
\newcommand{\nil}{\ttfun{nil}}
\newcommand{\cons}{\ttfun{cons}}

%%% Function extensionality
\newcommand{\funext}{\mathsf{funext}}
\newcommand{\happly}{\mathsf{happly}}

%%% hlevels
\newcommand{\isContr}{\ttfun{isContr}}
\newcommand{\isProp}{\ttfun{isProp}}
\newcommand{\isSet}{\ttfun{isSet}}
\newcommand{\hlevel}{\ttfun{hlevel}}

%%% categorytheory

\newcommand{\ttHom}{\ttfun{Hom}}
\newcommand{\catarrow}{\rightharpoonup}
\newcommand{\Precat}{\ttfun{Precat}}
\newcommand{\Functor}{\ttfun{Functor}}
\newcommand{\Cocone}{\ttfun{Cocone}}
\newcommand{\Cone}{\ttfun{Cone}}
\newcommand{\Lim}{\ttfun{Lim}}
\newcommand{\Colim}{\ttfun{Colim}}

%%% Relations

% Propositional equality
\newcommand{\propeqsign}{=}
\renewcommand{\=}{\propeqsign}
\newcommand{\propeq}[3]{#2 \propeqsign_{#1} #3}

% Others
\newcommand{\jdeq}{\equiv}
\newcommand{\defeq}{\vcentcolon\jdeq}
\newcommand{\weqsign}{\simeq}
\newcommand{\weq}[2]{#1 \weqsign #2}
\newcommand{\homot}[2]{#1 \sim #2}

\newcommand{\appr}[2]{\apply{\pr{#1}}{#2}} % a combination of apply and pr
\newcommand{\apinl}[1]{\apply{\inl}{#1}}   % a combination of apply and inl
\newcommand{\apinr}[1]{\apply{\inr}{#1}}   % a combination of apply and inl

\newcommand{\ap}[2]{\mathsf{ap}_{#1}{\paren{#2}}} % action on paths

% univalence
\newcommand{\ua}{\mathsf{ua}} % univalence axiom

%%% Transport (covariant) %%%
% \newcommand{\trans}[2]{\ensuremath{{#1}_{*}\mathopen{}\left({#2}\right)\mathclose{}}\xspace}
% \let\Trans\trans
% \newcommand{\transf}[1]{\ensuremath{{#1}_{*}}\xspace} % Without argument
\newcommand{\transpor}[2]{\ttfun{transport}^{#1}\paren{#2}}
\newcommand{\transport}[3]{\ttfun{transport}^{#1}\paren{#2,#3}}
\newcommand{\transportname}{\ttfun{transport}}

\input{../tex-preamble/unicode.tex}
\usepackage{epigraph}
\usepackage{rotating}
\usepackage{makeidx}
\usepackage{prftree}\setlength{\prfinterspace}{1.2em}

\newcommand{\software}[1]{{\textsc{#1}}\indeX{#1}}
\newcommand{\Agda}{\software{Agda}}
\newcommand{\UniMath}{\software{UniMath}}
\newcommand{\Coq}{\software{Coq}}
\newcommand{\MTypes}{\software{HoTT/M-Types}}

\definecolor{accepted}{HTML}{0088EE}
\definecolor{notaccepted}{HTML}{EE8800}
% \newcommand{\coqname}{\texttt}
% \newcommand{\unimathname}[1]{\underline{\texttt{#1}}}
\newcommand{\coqname}[1]{\texttt{\footnotesize\color{notaccepted} #1}}
\newcommand{\unimathname}[1]{\texttt{\footnotesize\color{accepted} #1}}

\usepackage{tikz}
\usetikzlibrary{cd,arrows.meta}
\tikzcdset{
  arrow style=tikz,
  arrows={line width=0.65pt},
  >={stealth}
}
\tikzset{every path/.append style={line width=0.65pt}}
\newcommand{\comma}{,}
% transport diagrams, etc
\usetikzlibrary{decorations}
\usetikzlibrary{decorations.pathmorphing}

% Gather environments with more interline spacing
\usepackage{environ}
\newlength{\oldjot}
\NewEnviron{gatherjot}{%
  \setlength{\oldjot}{\jot}\addtolength{\jot}{1em}
  \begin{gather*}
    \BODY
  \end{gather*}
  \setlength{\jot}{\oldjot}
}

\newcommand{\dual}[2]{
  \begin{itemize}\renewcommand{\labelitemi}{$∘$}
    \itemsep0em
    \item #1
    \item #2
   \end{itemize}
}

\newcommand{\define}[1]{\textbf{#1}} % term being defined
\newtheorem{notation}[theorem]{Notation}
\newtheorem{tt-rule}[theorem]{Rule}

\newcommand{\Algtype}{\ensuremath{\ttfun{Alg}}}
\newcommand{\Fibalgtype}{\ensuremath{\ttfun{FiberedAlg}}}
\newcommand{\Coalgtype}{\ensuremath{\ttfun{Coalg}}}
\newcommand{\Final}{\ensuremath{\ttfun{Final}}}
\newcommand{\Stdlim}[1]{\ensuremath{\apply{\ttfun{Stdlim}}{#1}}}
\newcommand{\postcomp}{\ensuremath{\ttfun{postcomp}}}
\newcommand{\HomotCone}{\ensuremath{\ttfun{HomotCone}}}

\usepackage{subfiles}

%%%%%%%%%%%%%%%%%%%%%%%%%%%%%%% My stuff:

\title{Deriving Coinductive Types in Univalent Type Theory}
\author{Langston Barrett}
\date{May 2018}
\division{Mathematics and Natural Sciences}
\advisor{Safia Chettih}
\department{Mathematics}
\thedivisionof{The Established Interdisciplinary Committee for}

\setlength{\parskip}{0pt}
\begin{document}

\maketitle
\frontmatter % this stuff will be roman-numbered
\pagestyle{empty} % this removes page numbers from the frontmatter

% Acknowledgements (Acceptable American spelling) are optional
% So are Acknowledgments (proper English spelling)

% \chapter*{Acknowledgments}

\noindent Safia, your sage advice and unfailing support throughout this
process made it possible in the first place. Thank you for your flexibility and
your Virgilian spirit: you managed to guide me through territory unfamiliar to
both of us.

\vspace{0.6em}

\noindent Benedikt, thank you for organizing and ensuring I could attend
the \UniMath{} workshop. Your warm welcome into the study of type theory
has directed the course of this thesis. I look forward to working more with
you in the future.

\vspace{0.6em}

\noindent Dan and Anders, thank you for your perseverance and thoroughness in
proof reviews, and for your words of encouragement. I hope to collaborate
with you soon.

\vspace{0.6em}

\noindent Friends at Reed, thank you for consistently indulging my half-baked
attempts at explaining my work, and thank you for sharing your curiosities and
passions just as easily.

% The pcreface is optional
% To remove it, comment it out or delete it.
% \chapter*{Pcreface}
% This thesis delves into highly interdisciplinary territory. I'll

\chapter*{List of Abbreviations}

\begin{table}[h]
  \centering
  \begin{tabular}{ll}
    \HoTT  	&  Homotopy type theory \\
    \UTT{}  	&  univalent type theory \\
    \ITT  	&  Intensional type theory \\
    \FOL{}  	&  Classical first-order logic \\
    \IPL{}  	&  Intuitionistic propositional logic \\
    \LC{}  	  &  Church's untyped λ-calculus \\
    \STLC{}  	&  Simply-typed λ-calculus \\
    \TLC{}  	&  Typed λ-calculus
  \end{tabular}
\end{table}

% Depth to which to number and print sections in TOC
\setcounter{tocdepth}{4}
% \setcounter{secnumdepth}{2}
\tableofcontents
% if you want a list of tables, optional
% \listoftables
% if you want a list of figures, also optional
% \listoffigures

\chapter*{Abstract}

In this thesis we explain univalent type theory, a constructive and
computationally meaningful foundational system for mathematics inspired by
recent advances in the model theory of Per Martin-L\"of's intuitionistic type
theory. We first develop the classical propositions/types correspondence between
(the constructive subset of) intuitionistic natural dedication and the
λ-calculus. We proceed to explicate Martin-L\"of's theory of dependent types and
the central modern development in type theory: Vladimir Voevodsky's univalence
principle. We go on to examine some category theory and the nature of coinduction
within this theory, presenting a novel formalization of (parts of) a recent
result that M-types can be derived \textit{internally} in univalent type theory.


\chapter*{Dedication}

\mainmatter % here the regular arabic numbering starts
\pagestyle{fancyplain} % turns page numbering back on

\chapter*{Introduction}
\addcontentsline{toc}{chapter}{Introduction}
\chaptermark{Introduction}
\markboth{Introduction}{Introduction}

\section*{A short history of discomfort in mathematics}

% \subsection*{Trouble in Cantor's Paradise}

In the early 20\textsuperscript{th} century, mathematicians had a problem. Often
held up as the pinnacle of necessary, undoubtable, \textit{a priori} truth,
their discipline was suffering from an embarrassing lack of certainty. The
intricate arguments of advanced analysis left mathematicians unable to confirm
nor deny each other's proofs. Due to the wave of paradoxes unleashed by
Cantor's na\"ive set theory, the need for a rigorous logical foundation for higher
mathematics was so great that the enterprise of axiomatic set theory was pursued
headlong until some kind of consensus was reached. To make a long, rich story
brutally short, virtually every modern paper in mathematics and computer
science uses a combination of Gentzen's first-order logic (\FOL)
and an axiomatization of sets called
\ZFC.\footnote{This ``consensus'' left out many
  prominent schools of thought, such as the Intuitionists and constructivists, a
  point we'll soon return to.}

Despite the apparent rigor of \FOL+\ZFC,
practitioners of the deductive sciences were still beset by issues of
verifiability. Complexity, specialization and sheer length made modern proofs
difficult to comprehend with the absolute certainty which is supposed to
characterize these disciplines. In the 1970s, two teams of topologists proved
contradictory results, and neither group could find the error in the other's
proof \cite{kolata}. Wiles's famous proof of Fermat's last theorem was utterly
unintelligible to the vast majority of mathematicians \cite{nyt}. The
classification of the simple finite groups, one of the crowning achievements of
modern mathematics, has a combined proof of over ten thousand pages. These
examples illustrate merely a few of the practical epistemological challenges
facing mathematicians; for a historical perspective see \cite{rigor-and-proof},
and for a general overview see \cite{fidelity}.

In 1990s, Fields medalist Vladimir Voevodsky grew concerned with the state
of mathematical knowledge. In 1998, Carlos Simpson released a pre-print arguing
that there was a major mistake in one of Voevodsky's papers. However, it was not
clear whether Voevodsky had errored, or whether there was a flaw in Simpson's
counterexample. In 1999, Pierre Deligne found a crucial mistake in Voevodsky's
``Cohomological Theory of Presheaves with Transfers'', upon which he had based
much of his work in the area of motivic cohomology. As he began to develop more
and more complex arguments, Voevodsky wondered: ``And who would ensure that I
did not forget something and did not make a mistake, if even the mistakes in
much more simple arguments take years to uncover?'' \cite{voevodsky-ias}.

The problems facing Voevodsky and his peers seemed insurmountable.
As the requisite attention span, memory, and capacity for detail required to
understand new developments in higer mathematics reached inhuman proportions,
where was he to turn? He would not find a solution in the realm of pure
mathematics, but rather in one of the finest examples of collaboration between
mathematicians, computer scientists, and philosophers: the modern proof
assistant. \TODO{reword}

\section*{TODO}

While much of the mathematical discipline was simply relieved to have ``solved''
their foundational issues with axiomatic set theory, there remained a vocal
opposition to the newly-adopted methods. Most mathematicians are dimly aware
that there's some controversy about the Axiom of Choice (the ``C'' in
\ZFC) \cite{martin-lof-100-years}, you don't have to look further
than \FOL to find disagreements.

\section*{Our contribution}

(Co)inductive types and UniMath

begin
\begin{notation}
  All of the results of this thesis have been formalized in the \Coq{} proof
  assistant \cite{coq-manual}. The names of the formal proofs
  appear in a monospaced font (e.g.\ \coqname{univalence}). Many results have
  already been reviewed (by the \UniMath{} development team \cite{unimath} and
  Anders Mörtberg) and accepted into the \UniMath{} library, these appear
  with an underline (e.g.\ \unimathname{univalence}).
\end{notation}


\subfile{chap1}
\subfile{chap2}
\subfile{chap3}
\subfile{chap4}

\chapter*{Conclusion}
\addcontentsline{toc}{chapter}{Conclusion}
\chaptermark{Conclusion}
\markboth{Conclusion}{Conclusion}
\setcounter{chapter}{5}
\setcounter{section}{0}

This thesis has presented univalent type theory, a modified version of Martin-Löf
dependent type theory inspired by a homotopy-theoretic semantics. The
presentation began somewhat historically, first discussing intuitionistic
logic and the λ-calculus, describing their interplay via the BHK interpretation
and Curry-Howard correspondence, and going on to describe how enrichments of
these systems lead to full-blown type theory. This perspective emphasizes
the harmonic interplay between the logical and computational aspects of this
theory. 

In \cref{chap:type-theory}, we saw some organizational principles such as
hlevels, weak equivalences, and characterization of paths inspired by the
homotopy model. Notably, this conceptual framework, motivated by spatial
intuition, greatly eased the decidedly non-topological work of later chapters. 
The univalent perspective ends up being useful in practice for logicians and
computer scientists, and not just topologists.

\Crefrange{chap:category-theory}{chap:coinductive-types-in-univalent-type-theory}
describe applications of \UTT{}. The development of (higher) category theory in
type theory continues to be an area of active research with rich results.
In keeping a well-understood pattern in the practice of constructive mathematics,
\UTT{} allows for the definition and exploration of classically-equivalent
concepts (e.g.\ categories and precategories), resulting in a more fine-grained
and general theory. Type theory is not ``just another'' foundational system in
which to do the exact same mathematics one might do in set theory, but indeed
gives rise to interesting, nontrivial, classically-inexpressible results.

While all results in this thesis were formalized in the computer proof assistant
\Coq{}, the workings, benefits, and limitations of such systems were mostly
relegated to footnotes when present at all. The methodology of computerized proof
undergoes an extended renaissance as computer scientists become more
interested in formally verified code and a wider community of mathematicians
begin to worry about the robustness of current (informal) proof-checking
procedures or get excited about \HoTT{}.

The type theory research community is small, yet vibrant. There are incredibly
important questions in the foundations of the theory itself (Is there a
constructive model of \UTT{}? Voevodsky's construction uses the axiom of
choice.), in its use as a framework for mathematics (How can one work
effectively with structures like $\R$ within a constructive theory?), and
in its implementations in proof assistants (How can we effectively use
automation in proofs?). Due to the proliferation of such concerns and the
ability to utilize intuitions from functional programming or algebraic topology,
type theory is a field in which even undergraduates can work on significant
research-level questions. 

%If you feel it necessary to include an appendix, it goes here.
\appendix

% \chapter{Cross-reference of names}

The following table lists lemmas taken from \cite{book}.
\begin{table}[ht]
  \centering
  \begin{tabular}{c | c}
    This thesis & The \HoTT book \\ \hline
    \Cref{def:ua} & Axiom 2.10.3
  \end{tabular}
\end{table}

The following table compares the terminology used in this thesis to that in our
\Coq{} formalization (under the column \UniMath{}) and that of the \Agda{} development
of \cite{non-wellfounded} (under the column \MTypes{}).
\begin{table}[ht]
  \centering
  \begin{tabular}{c | c | c }
    This thesis & \UniMath{} & \MTypes{} \\ \hline
  \end{tabular}
\end{table}

\chapter{CCCs and the λ-calculus}
\label{chap:cccs-and-lc}

\Cref{chap:category-theory} provides \textit{almost} enough background to
understand an interesting connection between a certain kind of category and the
λ-calculus.

\begin{lemma}[\unimathname{CategoryTheory.categories.Types.ExponentialsType}]
	Any fixed universe $\universe$ of types has exponentials.
\end{lemma}
\begin{proof}
	
\end{proof}

\begin{definition}
	A \define{monoidal category} consists of a category $\bfC$ together with
  \begin{itemize}
    \itemsep-0.2em
    \item a bifunctor $\bfC×\bfC→\bfC$ called the \define{tensor product} or
      \define{monoidal product},
    \item an object $I∈\Obj\bfC$ called the \define{unit},
    \item for all objects $A,B,C∈\Obj\bfC$, an isomorphism
      $α_{A,B,C}:(A⊗B)⊗C≅A⊗(B⊗C)$, and
    \item for all objects $A∈\Obj\bfC$, isomorphisms $λ_A:I⊗A≅A$ and $ρ_A:A⊗I≅A$.
  \end{itemize}
  These isomorphisms are subject to additional ``coherence conditions'', which
  won't play a crucial role here. A monoidal category is
  \begin{itemize}
    \itemsep-0.2em
    \item \define{symmetric} if there are additional isomorphisms
      $s_{A,B}:A⊗B→B⊗A$ satisfying other coherence conditions; and is
    \item \define{cartesian} if the monoidal product $⊗$ coincides with the
      categorical product $×$.
  \end{itemize}
\end{definition}

Only cartesian monoidal categories play a crucial role later in this section.

\begin{remark}
	For any categorical product there is a specified isomorphism $A×B≅B×A$, so any
  cartesian monoidal category is also symmetric monoidal.
\end{remark}

\begin{example}
  \
  \begin{itemize}
    \itemsep-0.2em
    \item \vspace{-0.3em} $\Set$, $\universe$, $\Cat$, $\Grp$, and others are 
      cartesian (so necessarily symmetric) monoidal under their categorical
      products. 
    \item $F\dVect$ is symmetric monoidal under $⊗$; the trivial vector space is
      a unit.
  \end{itemize}
\end{example}

\begin{definition}\label[definition]{def:exponentials}
  An \define{internal hom} for a monoidal category $(\bfC,⊗)$
  consists of, for each pair of objects $A,B∈\Obj\bfC$, an object
  $A⇒B$ together an arrow $\eval:(A⇒B)×A→B$ satisfying the following universal
  property: for any object $C∈\Obj\bfC$ and arrow $f:C⊗A→B$, there is an arrow
  $λf:C→(A⇒B)$ making the following diagram commute:
  \begin{center}
    \begin{tikzcd}[sep=large]
      C×A \arrow[dr, "f"] \arrow[d, swap, "λf×\id_A"] & {} \\
      (A⇒B)×A \arrow[r, swap, "\eval"]      & B
    \end{tikzcd}
  \end{center}
  When $\bfC$ is cartesian, $A⇒B$ is called an \define{exponential},
  and is denoted $B^A$.
\end{definition}

\begin{definition}
  A monoidal category $(\bfC,⊗)$ is \define{closed} when it has exponentials.
\end{definition}

\begin{remark}
  (Compare to \cref{rmk:prod-coprod-universal-adj})
  In a closed monoidal category, for all $A,B,C∈\Obj\bfC$,
  $λ$ and $\eval$ define a bijection
  \begin{equation*}
    \Hom(A⊗B,C)≅\Hom(A,B⇒C).
  \end{equation*}
  This bijection can be taken as their definition
  with an additional ``naturalality'' condition.
  This bijection is often called \define{currying}.
\end{remark}

\begin{example*}
  $\Set$ and $\universe$ are closed monoidal. Define $B⇒A\coloneqq \Hom(B,A)$.
  Then this is an element of $\Set$/$\universe$. Mixing set- and type-theoretical
  notation, the bijection is given by the mutually inverse functions
  \begin{align*}
    \Hom(A×B,C) &\longrightarrow \Hom(A,\Hom(B,C)) \\
    f &\longmapsto \λ{a}{\λ{b}{\appply{f}{a}{b}}} \\
    \Hom(A,\Hom(B,C)) &\longrightarrow \Hom(A×B,C) \\
    g &\longmapsto \λ{p}{\appply{g}{(\appr{1}{a})}{(\appr{2}{b})}}
  \end{align*}
\end{example*}

The following proof will only be accessible to readers with a background in
category theory (specifically, who know what adjunctions and natural
transformations are), but isn't necessary for the rest of the appendix.

\begin{lemma}[\unimathname{CategoryTheory.categories.Cats.ExponentialsCat}]
	The precategory of categories (\cref{def:precat-of-precats}) has exponentials.
\end{lemma}
\begin{proof}
  Fix $A:\universe$. We need to define a functor $E:\ttfun{Cat}→\ttfun{Cat}$ which
  is (RIGHT/LEFT) adjoint to $B↦A×B$. Define its action on objects by
  \begin{align*}
    E : \ttfun{Cat} &\longrightarrow \ttfun{Cat} \\
    C &\longmapsto_0 [A,C]
  \end{align*}
  that is, $E$ maps each category to the category of functors $A→C$ (with
  natural transformations as morphisms).
  
  Here things get a bit tricky: the action of $E$ on arrows takes a functor in
  $[A,C]$ to one in $[A,D]$ for each functor $C→D$. This assignment must
  \textit{itself} be functorial. Fix some categories $C$ and $D$ and a functor
  $F:C→D$, and define $E'$ on objects as
  \begin{align*}
    E' : [A,C] &\longrightarrow [A,D] \\
    G &\longmapsto_0 F_0∘G \\
    η &\longmapsto_1 \braces{\apply{F_1}{η_a}}_{a:\Obj A}.
  \end{align*}
  On arrows, $E'$ maps a natural transformation $η:G⟹H$ to one
  $\apply{E'}{η}:F∘G⟹F∘H$. To show that $\apply{E'}{η}$ is indeed a natural
  transformation, let $a,b:\Obj A$ and $f:\appply{\Hom}{a}{b}$. We want to show
  that the following square commutes:
  \begin{center}
    \begin{tikzcd}
      \apply{(F∘G)}{a}
        \arrow[r, "\apply{(F∘G)}{f}"]
        \arrow[d, "\apply{F_1}{η}_a"] &
      \apply{(F∘G)}{b}
        \arrow[d, "\apply{F_1}{η}_b"] \\
      \apply{(F∘H)}{a}
        \arrow[r, "\apply{(F∘H)}{f}"] &
      \apply{(F∘H)}{b}
    \end{tikzcd}
  \end{center}
  This is essentially the diagram expressing the naturality of $η$, but with $F$
  applied everywhere. The proof that it commutes reflects this structure:
  \begin{align*}
    F_1η_b ∘ \big(\apply{(F∘G)}{f}\big)
    &\≡ F_1η_b ∘ \apply{F_1}{(\apply{G}{f})} \\
    &\= \apply{F_1}{(η_b ∘ \apply{G}{f})} 
    &&\text{Functoriality of }F \\
    &\= \apply{F_1}{(\apply{H}{f} ∘ η_a)} 
    &&\text{Naturality of }η \\
    &\= \apply{F_1}{(\apply{H}{f})} ∘ F_1η_a 
    &&\text{Functoriality of }F \\
    &\≡ \big(\apply{(F∘H)}{f}\big) ∘ F_1η_a
  \end{align*}
\end{proof}

\begin{definition}\label[definition]{def:cartesian-cat}
  A category is \define{Cartesian closed}
  when it is closed, cartesian monoidal, and has a terminal object.
  Equivalently, it is cartesian closed when it has
  \begin{itemize}
    \itemsep-0.2em
    \item \vspace{-0.1em} a terminal object (\cref{def:terminal-and-initial}),
    \item binary products (\cref{def:product-and-coproduct}), and
    \item exponentials (\cref{def:exponentials}).
  \end{itemize}
  ``CCC'' abbreviates ``Cartesian closed category''.
\end{definition}

\begin{remark}\label[remark]{rmk:lambek}
	We now have the vocabulary to (informally) extend the correspondence of
  \cref{sec:propositions-and-types} to statements about CCCs.
  \cite{lambek}
  \TODO{}
\end{remark}

\begin{remark}
  The Curry-Howard-Lambek correspondence gives the barest hint of insight into
  Voevodsky's innovation. He also gave a categorically-based model of a version
  of the typed λ-calculus, but of full Martin-Löf dependent type theory. The
  structure of higher/nested identity types
  ($\propeq{\propeq{\cdots}{a}{b}}{p}{q}$) recalled ideas of homotopy theory,
  where paths are considered identical up to the existence of ``higher
  paths''.\TODO{}

  For the initial development of the connection between locally Cartesian closed
  categories (LCCCs) and type theory, see \cite{seely}.
\end{remark}


\backmatter % backmatter makes the index and bibliography appear properly in the TOC

%  \bibliographystyle{bsts/mla-good} % there are a variety of styles available;
%  \bibliographystyle{plainnat}
% replace ``plainnat'' with the style of choice. You can crefer to files in the bsts or APA
% subfolder, e.g.
% \bibliographystyle{APA/apa-good}  % or

% uncomment me!
\nocite{*}
\bibliographystyle{apalike}
\bibliography{thesis}

% Comment the above two lines and uncomment the next line to use biblatex-chicago.
%\printbibliography[heading=bibintoc]

% Finally, an index would go here... but it is also optional.
\index{$\id$|see {Identity}}
\index{$F$-algebra|see {Algebra for a functor}}
\printindex
\end{document}
