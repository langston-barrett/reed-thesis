\documentclass[12pt,twoside]{reedthesis}
\usepackage{graphicx}
\usepackage{booktabs,setspace}
% \usepackage{natbib}
% Comment out the natbib line above and uncomment the following two lines to use the new
% biblatex-chicago style, for Chicago A. Also make some changes at the end where the
% bibliography is included.
%\usepackage{biblatex-chicago}
%\bibliography{thesis}

%%%%%%%%%%%%%%%%%%%%%%%%%%%% While drafting:
\makeatletter
\def\ifdraft{\ifdim\overfullrule>\z@
  \expandafter\@firstoftwo\else\expandafter\@secondoftwo\fi}
\makeatother

\ifdraft{
  \usepackage{showlabels}
  % \usepackage{showidx}
  \let\oldindex\index
  \definecolor{index}{HTML}{0088EE}
  \renewcommand{\index}[1]
               {\oldindex{#1}\marginpar{\footnotesize\color{index}index: #1}}
  \newcommand{\indeX}{\oldindex}
}{
  \newcommand{\indeX}{\index}
}

\usepackage{color}
\usepackage[dvipsnames]{xcolor}
\definecolor{TODO}{HTML}{EE8800}
\newcommand{\TODO}[1]{\marginpar{\footnotesize\color{TODO}todo: #1}}

%%%%%%%%%%%%%%%%%%%%%%%%%%%%%%% My stuff:
%% math symbols, notation, etc
\usepackage{amssymb}       % math characters
\usepackage{mathtools}     % improvement on amsmath

% \usepackage{unicode-math}         % use OTF math fonts
% \setoperatorfont\mathup % operators should be set in math font
% \AtBeginDocument{\let\setminus\minus}

% \ExplSyntaxOn % fix xmapsto. https://github.com/wspr/unicode-math/issues/197
% \RenewDocumentCommand \xmapsto { m } {
%   \mathrel {
%     \exp_after:wN \Uoverdelimiter \cs:w sym \__um_symfont_tl \cs_end: "21A6 {
%       \mkern 5mu \scan_stop: #1 \mkern 5mu \scan_stop:
%     }
%   }
% }
% \ExplSyntaxOff

% \setmathfont[math-style=ISO]{Tex Gyre Schola Math}
% \setmathfont[range=\mathup]{Tex Gyre Schola Math}
% \setmathfont[range=\mathcal]{Latin Modern Math}
% \setmathfont[range=\setminus]{Asana Math}

% \setmathfont{Latin Modern Math}
% \setmathfont[range=\mathbb]{Tex Gyre Schola Math}
% \setmathfont[range=\setminus]{Asana Math}

% alternative characters
\AtBeginDocument{\let\epsilon\varepsilon}
\AtBeginDocument{\let\phi\varphi}
\AtBeginDocument{\let\oldemptyset\emptyset{}}
\AtBeginDocument{\let\emptyset\varnothing{}}

% paired delimiters
\DeclarePairedDelimiterX{\Abs}[1]{\lvert}{\rvert}{#1}
\DeclarePairedDelimiterX{\Angle}[1]{\langle}{\rangle}{#1}
\DeclarePairedDelimiterX{\Braces}[1]{\{}{\}}{#1}
\DeclarePairedDelimiterX{\Brackets}[1]{[}{]}{#1}
\DeclarePairedDelimiterX{\Ceiling}[1]{\lceil}{\rceil}{#1}
\DeclarePairedDelimiterX{\Floor}[1]{\lfloor}{\rfloor}{#1}
\DeclarePairedDelimiterX{\Norm}[1]{\lVert}{\rVert}{#1}
\DeclarePairedDelimiterX{\Paren}[1]{\lparen}{\rparen}{#1}

% swap the starred and unstarred versions of paired delimiters
\makeatletter % https://goo.gl/osSmHV
\def\abs{\@ifstar{\Abs}{\Abs*}}
\def\angles{\@ifstar{\Angle}{\Angle*}}
\def\braces{\@ifstar{\Braces}{\Braces*}}
\def\brackets{\@ifstar{\Brackets}{\Brackets*}}
\def\ceiling{\@ifstar{\Ceiling}{\Ceiling*}}
\def\floor{\@ifstar{\Floor}{\Floor*}}
\def\norm{\@ifstar{\Norm}{\Norm*}}
\def\paren{\@ifstar{\Paren}{\Paren*}}
\makeatother

% operators/functions
% \DeclareMathOperator{\Im}{Im}        % complex numbers: real part
% \DeclareMathOperator{\Re}{Re}        % complex numbers: imaginary part
\DeclareMathOperator{\Var}{Var}         % statistics: variance
\DeclareMathOperator{\ord}{ord}
\DeclareMathOperator{\image}{image}     % general: image
\DeclareMathOperator{\id}{id}           % general: identity function
\DeclareMathOperator{\lcm}{lcm}         % general: least common multiple
\DeclareMathOperator{\xor}{xor}         % general: exclusive or
\DeclareMathOperator{\Int}{Int}         % topology: interior
\DeclareMathOperator{\Ext}{Ext}         % topology: exterior
\DeclareMathOperator{\Lim}{Lim}
\DeclareMathOperator{\Bd}{Bd}           % topology: boundary
\DeclareMathOperator{\oball}{B}         % topology: open ball
\DeclareMathOperator{\cball}{\ov{B}}    % topology: closed ball
\DeclareMathOperator{\sphere}{S}        % topology: sphere
\DeclareMathOperator{\proj}{proj}
\DeclareMathOperator{\spanf}{span}      % linear: span
\DeclareMathOperator{\col}{Col}         % linear: column space
\DeclareMathOperator{\row}{Row}         % linear: row space
\DeclareMathOperator{\nul}{Nul}         % linear: null space
\DeclareMathOperator{\rank}{rank}       % linear: rank
\DeclareMathOperator{\determinant}{det} % linear: determinant
\DeclareMathOperator{\Alt}{Alt}         % linear: alternating group/square
\DeclareMathOperator{\Sym}{Sym}         % linear: symmetric square
\DeclareMathOperator{\Tr}{Trace}        % linear: trace of a matrix
\DeclareMathOperator{\divergence}{div}  % multi: divergence
\DeclareMathOperator{\curl}{curl}       % multi: curl
\DeclareMathOperator{\characteristic}{char} % algebra: ring/field characteristic
\DeclareMathOperator{\ann}{ann}         % algebra: the annihilator of a submodule
\DeclareMathOperator{\GL}{GL}           % algebra: general linear group
\DeclareMathOperator{\SL}{SL}           % algebra: special linear group
\DeclareMathOperator{\GF}{GF}           % algebra: galois field
\DeclareMathOperator{\Tor}{Tor}         % algebra: torsion elements/functor
\DeclareMathOperator{\Stab}{Stab}       % algebra: stabilizer
\DeclareMathOperator{\Orb}{Orb}         % algebra: orbit
\DeclareMathOperator{\Mat}{Mat}         % algebra: matrix ring
\DeclareMathOperator{\Res}{Res}         % algebra: restriction of a representation
\DeclareMathOperator{\Ind}{Ind}         % algebra: induced representation
\DeclareMathOperator{\im}{im}           % algebra: image
\DeclareMathOperator{\Aut}{Aut}         % algebra/cat: automorphisms
\DeclareMathOperator{\End}{End}         % algebra/cat: endomorphisms
\DeclareMathOperator{\Obj}{Obj}              % categories: objects of a category
\DeclareMathOperator{\Hom}{Hom}              % categories: hom-sets
\DeclareMathOperator{\Set}{\mathbf{Set}}     % categories: (small) sets
\DeclareMathOperator{\Cat}{\mathbf{Cat}}     % categories: small categories
\DeclareMathOperator{\Grpd}{\mathbf{Grpd}}   % categories: small groupoids
\DeclareMathOperator{\Mon}{\mathbf{Mon}}     % categories: monoids
\DeclareMathOperator{\Grp}{\mathbf{Grp}}     % categories: groups
\DeclareMathOperator{\AbGrp}{\mathbf{AbGrp}} % categories: abelian groups
\DeclareMathOperator{\Top}{\mathbf{Top}}     % categories: topological spaces
\DeclareMathOperator{\Man}{\mathbf{Man}}     % categories: topological manifolds
\DeclareMathOperator{\Mod}{\mathbf{-Mod}}     % categories: modules over a ring
\DeclareMathOperator{\Met}{\mathbf{Metric}}  % categories: metric spaces
\DeclareMathOperator{\Vect}{\mathbf{Vect}}   % categories: vector spaces
\DeclareMathOperator{\Rep}{\mathbf{Rep}}     % categories: representations

% make letters in different fonts (bold, blackboard, caligraphic, fraktur)
% https://tex.stackexchange.com/questions/156196/newcommand-with-variable-name
\newcommand{\mkltr}[1]{%
    \expandafter\newcommand\csname bf#1\endcsname{\ensuremath{\mathbf #1}}
    \expandafter\newcommand\csname bb#1\endcsname{\ensuremath{\mathbb #1}}
    \expandafter\newcommand\csname ca#1\endcsname{\ensuremath{\mathcal #1}}
    \expandafter\newcommand\csname fr#1\endcsname{\ensuremath{\mathfrak #1}}
}%
\mkltr A \mkltr B \mkltr C \mkltr D \mkltr E \mkltr F \mkltr G \mkltr H
\mkltr I \mkltr J \mkltr K \mkltr L \mkltr M \mkltr N \mkltr O \mkltr P
\mkltr Q \mkltr R \mkltr S \mkltr T \mkltr U \mkltr V \mkltr W \mkltr X
\mkltr Y \mkltr Z

\newcommand{\zn}{\mathbf{Z}_n}
\newcommand{\zp}{\mathbf{Z}_p}
\newcommand{\N}{\ensuremath{\mathbb{N}}}
\newcommand{\Z}{\ensuremath{\mathbb{Z}}}
\newcommand{\Q}{\ensuremath{\mathbb{Q}}}
\newcommand{\R}{\ensuremath{\mathbb{R}}}
\newcommand{\C}{\ensuremath{\mathbb{C}}}

% general stuff
\newcommand{\units}[1]{\;\mathrm{#1}}
\newcommand{\dd}{\,\mathrm{d}}
\newcommand{\ov}{\overline}

% set theory
\newcommand{\xlongmapsto}[1]{\xmapsto{\,\,\, #1\,\,\,}}
\newcommand{\xlongrightarrow}[1]{\xrightarrow{\,\,\, #1\,\,\,}}
\newcommand{\union}{\mathop{\bigcup}}
\newcommand{\disunion}{\mathop{\dot\bigcup}}
\newcommand{\intersection}{\mathop{\bigcap}}
\newcommand{\powerset}{\mathcal{P}}

% analysis & calculus
\usepackage{esint} % \oiint, etc
\newcommand{\evalat}[2]{{\bigg|_{#1}^{#2}}}
\newcommand{\ext}[1]{{#1}^{ext}}  % extension of function

% vectors
\newcommand{\ihat}{\hat{\imath}}
\newcommand{\jhat}{\hat{\jmath}}
\newcommand{\khat}{\hat{k}}
\newcommand{\lvec}[1]{\overrightarrow{#1}}
\newcommand{\len}[1]{\left\| #1 \right\|}

% topology
\newcommand{\RP}{\ensuremath{\R\hspace{-0.07em}\mathrm{P}}}

% algebra
\newcommand{\quotient}[2]{{\raisebox{.2em}{$#1$}\left/\raisebox{-.2em}{$#2$}\right.}}
\renewcommand{\S}{\mathfrak{S}}   % symmetric group
\newcommand{\ab}{\mathbf{ab}}     % algebra: abelianization

% statistics
% https://tex.stackexchange.com/questions/229023/expectation-operator#229029
% expectation and covariance
\DeclareMathOperator{\ExpOp}{Exp}
\DeclareMathOperator{\CovOp}{Cov}
\newcommand{\Exp}{\ExpOp\brackets*}
\newcommand{\Cov}{\CovOp\brackets*}
%% formatting

\usepackage{fontspec}
% EB Garamond (Initials - just for fun) - body text, old style, serif
% Nimbus Roman - serifed
% URW Gothic - geometric, title text
% Libre Caslon Text - absurdly well spaced
% FreeSerif - plays pretty nicely with math, looks like Times
% Liberation Serif - good alternative to FreeSerif, less TNR looking
% unclear whether to use Tinos or Liberation Serif. same design.
\setmainfont[Ligatures=TeX]{Tinos}

% date
\usepackage{polyglossia}
\usepackage[yyyymmdd]{datetime}
\renewcommand{\dateseparator}{--}

\usepackage{geometry}
\geometry{
  lmargin=4cm,
  rmargin=4cm,
  tmargin=3cm,
  bmargin=3cm,
}

\renewcommand{\arraystretch}{1.2} % Make tables a little bigger
% \usepackage{indentfirst}          % indent first \par after sections
\usepackage{parskip}              % don't indent anything
\usepackage{enumitem}             % enumerate with A), b., c) styles
\usepackage{hyperref}             % linked table of contents
\hypersetup{
    colorlinks,
    citecolor=black,
    filecolor=black,
    linkcolor=black,
    urlcolor=black
}

% graphics
\usepackage{graphicx}
\DeclareGraphicsExtensions{.pdf,.png,.jpg}
\usepackage{float} % image positioning

% semicolon is mapped to \ in my editor
\newcommand{\semicolon}{;}
% Theorems, proofs, and other mathematical environments

\usepackage{amsthm}

\newtheoremstyle{mine} % name
  {7pt} % space above
  {5pt} % space below
  {} % body font
  {} % indent amount
  {\bfseries} % theorem head font
  {.} % punctuation after theorem head
  {.5em} % space after theorem head
  {} % theorem head spec (can be left empty, meaning no)

\theoremstyle{mine} % these are the non-italicized
\newtheorem{theorem}{Theorem}[section]
\newtheorem*{theorem*}{Theorem}
\newtheorem{lemma}[theorem]{Lemma}
\newtheorem*{lemma*}{Lemma}
\newtheorem{proposition}[theorem]{Proposition}
\newtheorem*{proposition*}{Proposition}
\newtheorem{corollary}[theorem]{Corollary}
\newtheorem*{corollary*}{Corollary}
\newtheorem{eqenv}[theorem]{Equation}
\newtheorem*{eqenv*}{Equation}
\newtheorem{remark}[theorem]{Remark}
\newtheorem*{remark*}{Remark}
\newtheorem{example}[theorem]{Example}
\newtheorem*{example*}{Example}
\newtheorem{definition}[theorem]{Definition}
\newtheorem*{definition*}{Definition}
\newcounter{Problem} \stepcounter{Problem} % start at 1
\newcommand{\problem}[0]{
  \vspace{-0.8em}
  \begin{center}
    \huge
    \rule[0.25em]{0.3\textwidth}{0.6pt}
    \arabic{Problem}
    \rule[0.25em]{0.3\textwidth}{0.6pt}
  \end{center}
  \stepcounter{Problem}
  \vspace{0.5em}
}
\newcommand{\problemnum}[1]{
  \vspace{-0.5em}
  \begin{center}
    \huge
    \rule[0.25em]{0.3\textwidth}{0.5pt}
    #1
    \rule[0.25em]{0.3\textwidth}{0.5pt}
  \end{center}
  \vspace{0.5em}
}
\newcommand{\subproblem}[1]{
  \vspace{0.2em}
  \begin{center}
    \large
    \rule[0.25em]{0.2\textwidth}{0.5pt}
    #1
    \rule[0.25em]{0.2\textwidth}{0.5pt}
  \end{center}
  \vspace{0.2em}
}
% Combine the two with more pleasant vertical spacing
\newcommand{\problemsubproblem}[1]{\problem \vspace{-0.8em}\subproblem{#1}}
\newcommand{\problemnumsubproblem}[2]{\problemnum{#1}\vspace{-0.8em}\subproblem{#2}}
% requires

% one with more spacing. ideally, this would auto-adapt to surroundings
\renewcommand{\:}{\hspace{0.3ex}:\hspace{0.3ex}}

\definecolor{mvar}{HTML}{22CC00}
\newcommand{\mvar}[1]{{\color{mvar} #1}}          % TODO: capitalize

\newcommand{\FV}{\ensuremath{\mathsf{FV}}}
\newcommand{\BV}{\ensuremath{\mathsf{BV}}}

% application
\newcommand{\apspace}{\hspace{0.25em}}
\newcommand{\apply}[2]{#1\apspace #2} % apply a function to its argument
\newcommand{\appply}[3]{#1\apspace #2 \apspace #3}
\newcommand{\apppply}[4]{#1\apspace #2 \apspace #3 \apspace #4}
\newcommand{\appppply}[5]{#1\apspace #2 \apspace #3 \apspace #4 \apspace #5}

% application
\newcommand{\lam}[2]{\lambda #1.\apspace#2}               % λ-abstraction
\newcommand{\λ}{\lam}                                     % λ-abstraction

% Judgments
\newcommand{\judgment}[2]{#2\hspace{0.4em}\text{#1}}
\newcommand{\prop}[1]{\judgment{prop}{#1}}
\newcommand{\type}[1]{\judgment{type}{#1}}
\newcommand{\true}[1]{\judgment{true}{#1}}
\newcommand{\valid}[1]{\judgment{valid}{#1}}
\newcommand{\intro}[1]{#1\text{-intro}}
\newcommand{\intror}[1]{#1\text{-intro}_{\text{r}}}
\newcommand{\elim}[1]{#1\text{-elim}}
\newcommand{\eliml}[1]{#1\text{-elim}_{\text{l}}}
\newcommand{\elimr}[1]{#1\text{-elim}_{\text{r}}}
% \DeclareMathOperator{\refl}{\mathtt{refl}}
% \DeclareMathOperator{\ind}{\mathtt{ind}}

% one with more spacing. ideally, this would auto-adapt to surroundings
\renewcommand{\:}{\hspace{0.3ex}:\hspace{0.3ex}}
\newcommand{\lam}[2]{\lambda #1.#2}

% some of this is based on HoTT/HoTT

\newcommand{\emptytype}{\ensuremath{\mathbf{0}}}  % empty type
\newcommand{\unittype}{\ensuremath{\mathbf{1}}}   % unit type
\newcommand{\unitelem}{\ensuremath{1_{\unittype}}} % sole inhabitant of the unit type
\newcommand{\booltype}{\ensuremath{\mathbf{2}}}   % two element type
\newcommand{\btrue}{1_{\booltype}}                 % inhabitant of two element type
\newcommand{\bfalse}{0_{\booltype}}                % inhabitant of two element type
\newcommand{\inl}{\ensuremath{\mathsf{inl}}}
\newcommand{\inr}{\ensuremath{\mathsf{inr}}}
\newcommand{\universe}{\ensuremath{\mathcal{U}}}         % universe type
\newcommand{\universei}[1]{\ensuremath{\mathcal{U}_{#1}}}  % ith universe

\newcommand{\sigmat}[2]{\ensuremath{\sum_{\paren{#1}}#2}}    % Σ-type
\newcommand{\pr}[1]{\ensuremath{\mathsf{pr}_{#1}}}
\newcommand{\pit}[2]{\ensuremath{\prod_{\paren{#1}}#2}}      % Π-type
\newcommand{\refl}[1]{\ensuremath{\mathsf{refl}_{#1}}}      % reflexivity

%%% Natural numbers
\newcommand{\suc}{\mathsf{succ}}
\newcommand{\add}{\mathsf{add}}

%%% Lists
\newcommand{\List}[1]{\ensuremath{\mathsf{List}(#1)}}
\newcommand{\nil}{\ensuremath{\mathsf{nil}}}
\newcommand{\cons}{\ensuremath{\mathsf{cons}}}

%%% Function extensionality
\newcommand{\funext}{\mathsf{funext}}
\newcommand{\happly}{\mathsf{happly}}

\newcommand{\isContr}{\ensuremath{\mathsf{isContr}}}

%%% Relations
\newcommand{\jdeq}{\equiv}
\newcommand{\defeq}{\vcentcolon\jdeq}
\newcommand{\propeq}[3]{#2 =_{#1} #3}
\newcommand{\weq}[2]{#1 \simeq #2}
\newcommand{\homot}[2]{#1 \sim #2}

\newcommand{\apspace}{\hspace{0.3em}}
\newcommand{\apply}[2]{#1\apspace #2} % apply a function to its argument
\newcommand{\appply}[3]{#1\apspace #2 \apspace #3}
\newcommand{\appr}[2]{\apply{\pr{#1}}{#2}} % a combination of apply and pr

\newcommand{\ap}[2]{\mathsf{ap}_{#1}{\paren{#2}}} % action on paths
\newcommand{\ua}{\mathsf{ua}} % univalence axiom

%%% Transport (covariant) %%%
% \newcommand{\trans}[2]{\ensuremath{{#1}_{*}\mathopen{}\left({#2}\right)\mathclose{}}\xspace}
% \let\Trans\trans
% \newcommand{\transf}[1]{\ensuremath{{#1}_{*}}\xspace} % Without argument
\newcommand{\transpor}[2]{\ensuremath{\mathsf{transport}^{#1}\paren{#2}}}
\newcommand{\transport}[3]{\ensuremath{\mathsf{transport}^{#1}\paren{#2,#3}}}
\newcommand{\transportname}{\ensuremath{\mathsf{transport}}}
\input{./tex-preamble/unicode.tex}
\usepackage{epigraph}
\usepackage{rotating}
\usepackage{makeidx}
\usepackage{prftree}\setlength{\prfinterspace}{1.2em}

\newcommand{\software}[1]{{\textsc{#1}}\indeX{#1}}
\newcommand{\Agda}{\software{Agda}}
\newcommand{\UniMath}{\software{UniMath}}
\newcommand{\Coq}{\software{Coq}}
\newcommand{\MTypes}{\software{HoTT/M-Types}}

\definecolor{accepted}{HTML}{0088EE}
\definecolor{notaccepted}{HTML}{EE8800}
% \newcommand{\coqname}{\texttt}
\newcommand{\unimathname}[1]{\texttt{#1}}

\usepackage{tikz}
\usetikzlibrary{cd,arrows.meta}
\tikzcdset{
  arrow style=tikz,
  arrows={line width=0.65pt},
  >={stealth}
}
\tikzset{every path/.append style={line width=0.65pt}}
\newcommand{\comma}{,}
% transport diagrams, etc
\usetikzlibrary{decorations}
\usetikzlibrary{decorations.pathmorphing}

% Gather environments with more interline spacing
\usepackage{environ}
\newlength{\oldjot}
\NewEnviron{gatherjot}{%
  \setlength{\oldjot}{\jot}\addtolength{\jot}{1em}
  \begin{gather*}
    \BODY
  \end{gather*}
  \setlength{\jot}{\oldjot}
}

\newcommand{\dual}[2]{
  \begin{itemize}\renewcommand{\labelitemi}{$∘$}
    \itemsep0em
    \item #1
    \item #2
   \end{itemize}
}

\newcommand{\define}[1]{\textbf{#1}} % term being defined
\newtheorem{notation}[theorem]{Notation}
\newtheorem{tt-rule}[theorem]{Rule}

\newcommand{\Algtype}{\ensuremath{\ttfun{Alg}}}
\newcommand{\Fibalgtype}{\ensuremath{\ttfun{FiberedAlg}}}
\newcommand{\Coalgtype}{\ensuremath{\ttfun{Coalg}}}
\newcommand{\Final}{\ensuremath{\ttfun{Final}}}
\newcommand{\Stdlim}[1]{\ensuremath{\apply{\ttfun{Stdlim}}{#1}}}
\newcommand{\postcomp}{\ensuremath{\ttfun{postcomp}}}
\newcommand{\HomotCone}{\ensuremath{\ttfun{HomotCone}}}

\usepackage{subfiles}

%%%%%%%%%%%%%%%%%%%%%%%%%%%%%%% My stuff:

\title{Deriving Coinductive Types in Univalent Type Theory}
\author{Langston Barrett}
\date{May 2018}
\approvedforthe{Committee}
\thedivisionof{The Established Interdisciplinary Committee for Computer Science and Mathematics}
\department{Computer Science and Mathematics}
\advisor{Safia Chettih}

\setlength{\parskip}{0pt}
\begin{document}

\maketitle
\frontmatter % this stuff will be roman-numbered
\pagestyle{empty} % this removes page numbers from the frontmatter

% Acknowledgements (Acceptable American spelling) are optional
% So are Acknowledgments (proper English spelling)

\chapter*{Acknowledgments}

\noindent Safia, your sage advice and unfailing support throughout this
process made it possible in the first place. You managed to guide me through
territory unfamiliar to both of us.

\vspace{0.6em}

\noindent Angélica, thank you for talking to me about anything and everything
mathematical in your office hours this year. You encouraged me to pursue
my learning well beyond the classroom.

\vspace{0.6em}

\noindent Benedikt, thank you for organizing and ensuring I could attend
the \UniMath{} workshop. Your warm welcome into the study of type theory
has directed the course of this thesis. I look forward to working more with
you in the future.

\vspace{0.6em}

\noindent Dan and Anders, thank you for your perseverance and thoroughness in
proof reviews, and for your words of encouragement. 

\vspace{0.6em}

\noindent Friends at Reed, thank you for consistently indulging my half-baked
attempts at explaining my work, and thank you for sharing your curiosities and
passions just as easily.

% The pcreface is optional
% To remove it, comment it out or delete it.
% \chapter*{Pcreface}

% \chapter*{List of Abbreviations}

% \begin{table}[h]
%   \centering
%   \begin{tabular}{ll}
%     \FOL{}  	&  Classical first-order logic \\
%     \HoTT  	  &  Homotopy type theory \\
%     \IPL{}  	&  Intuitionistic propositional logic \\
%     \ITT  	  &  Intensional type theory \\
%     \LC{}  	  &  Church's untyped λ-calculus \\
%     \STLC{}  	&  Simply-typed λ-calculus \\
%     \TLC{}  	&  Typed λ-calculus \\
%     \UTT{}  	&  Univalent type theory
%   \end{tabular}
% \end{table}

% Depth to which to number and print sections in TOC
% \setcounter{tocdepth}{4}
% \setcounter{secnumdepth}{2}
\tableofcontents
% if you want a list of tables, optional
% \listoftables
% if you want a list of figures, also optional
% \listoffigures

\chapter*{Abstract}

In this thesis I explain univalent type theory, a constructive and
computationally meaningful foundational system for mathematics inspired by
recent advances in the model theory of Per Martin-L\"of's intuitionistic type
theory. I first develop the classical propositions/types correspondence between
intuitionistic natural deduction and the λ-calculus. I proceed to explicate
Martin-L\"of's theory of dependent types and the central modern development in
type theory: Vladimir Voevodsky's univalence principle. I go on to examine some
category theory and the nature of coinduction within this theory, presenting a
novel formalization of (parts of) a recent result that M-types can be derived
\textit{internally} in univalent type theory.

\chapter*{Dedication}

% \setlength{\epigraphwidth}{0.60\textwidth}
\setlength{\epigraphwidth}{0.25\textwidth}
\epigraph{
  % spring!may -- \\
  % everywhere's here \\
  % (with a low high low \\
  % and the bird on the bough) \\
  % how?why \\
  % -- we never we know \\
  % (so kiss me) shy sweet eagerly my

  (2 little ams \\
  and over them this \\
  aflame with dreams \\
  incredible is)
}{E.E.\ Cummings}

For Mar.

\mainmatter % here the regular arabic numbering starts
\pagestyle{fancyplain} % turns page numbering back on

\chapter*{Introduction}
\addcontentsline{toc}{chapter}{Introduction}
\chaptermark{Introduction}
\markboth{Introduction}{Introduction}

\setlength{\epigraphwidth}{0.8\textwidth}
\epigraph{I didn’t have the tools to explore the areas where curiosity was
  leading me and the areas that I considered to be of value and of interest and
  of beauty. So I started to look into what I could do to create such tools.}{\cite{voevodsky-ias}}

Mathematicians err. Some acknowledge mistakes in their published work and
try to move on. For others, the experience can be life-changing. In 1998, Carlos
Simpson released a pre-print arguing that there was a major mistake in one of
Vladimir Voevodsky's 1989 papers \cite{voevodsky-presentation}. It was not
immediately clear whether Voevodsky had erred, or whether there was a flaw in
Simpson's counterexample. In 1999, Pierre Deligne found a crucial mistake in
Voevodsky's 1992/93 ``Cohomological Theory of Presheaves with Transfers'', upon
which he had based much of his work in the area of ``motivic cohomology''. As he
began to develop more and more complex arguments, Voevodsky wondered: ``And who
would ensure that I did not forget something and did not make a mistake, if even
the mistakes in much more simple arguments take years to uncover?''
\cite{voevodsky-ias}.

Elsewhere in mathematics, new developments raised questions for traditional
methodology. Complexity, specialization and sheer length made
proofs difficult to comprehend with the absolute certainty which is
supposed to characterize this discipline. In the 1970s, two teams of
topologists proved contradictory results, and neither group could find the error
in the other's proof \cite{kolata}. Wiles's famous proof of Fermat's last
theorem was utterly unintelligible to the vast majority of mathematicians
\cite{nyt}. The classification of the simple finite groups---one of the field's
crowning achievements---has a combined proof of over ten thousand pages.

The problems facing Voevodsky and his peers seemed insurmountable.
As the requisite attention span, memory, and capacity for detail required to
check new developments in higer mathematics reached inhuman proportions,
where were they to turn? In a fit of interdisciplinary collaborative spirit,
Voevodsky began to explore an avenue well-known to philosophers and computer
scientists, but relatively obscure within mathematics: type theory.

\paragraph{An abbreviated history of modern type theory.} After the paradoxes
and philosophical debates of the early 20\textsuperscript{th} century, most
mathematicians settled on classical first-order logic (\FOL{}) and an axiomatic
set theory called \ZFC{} as the gold standard of rigor. Virtually all of
contemporary research was built on these foundations. However, there remained
some vocal opposition by logicians and philosophers to these newly-adopted
methods. Most mathematicians are dimly aware that there's some controversy about
the Axiom of Choice (the ``C'' in \ZFC{}) \cite{martin-lof-100-years}, but one
doesn't have to look further than \FOL{} to find disagreements.

Brouwer, Heyting, and other intuitionists decried the use of the law
of excluded middle (or equivalently, double-negation elimination). Russell
doubted the consistency of impredicative (self-referential) definitions.
In the 1970s, philosopher and logician Per Martin-Löf created a predicative,
intuitionistically-acceptable logical system called type theory as an
alternative foundational system for ``constructive''
mathematics.\footnote{Constructive and intuitionistic mathematics differ in
their philosophical points, but mostly accord on choice of logical principles.}

Brouwer, Heyting, and Kolmogorov had all at various points suggested that
intuitionistic logic could be understood not only as a theory of truth, but also
as a theory of ``problems'' (propositions) and their ``solutions'' (proofs)
\cite{kolmogorov}. Elsewhere, Haskell Curry noticed that his formalism for
describing computations shared some surprising features with intuitionistic
logic \cite{curry-howard}. These observations were synthesized and extended by
William Howard, who noticed that intuitionistic ``natural deduction''
corresponds closely to a computational system called the λ-calculus (\LC{}).
The postcard version of the truth is that Martin-Löf was aware of these
connections between constructive logic and computation, and his type theory
could actually serve as \textit{both} an advanced, (very) high-level programming
language \textit{and} a foundational system for mathematics (details to come in
\cref{chap:propositions-and-types}).

Computer scientists Theirry Coquand and Christine Pauline took this Curry/Howard
thing very seriously. They developed a variant of Martin-Löf's type theory called
the Calculus of Constructions \cite{coquand}, and it was implemented as the
programming language \Coq{} \cite{coq-manual}.\footnote{The French word for
``rooster'', in the tradition of naming software after animals (a lá
\software{OCaml}).} Computer scientists, facing the problem of ensuring the
correctness of algorithms and the programs based on them, developed this and
other ``proof assistants'' in order to produce machine-verifiable code. These
languages became Voevodsky's ``tools''.

\paragraph{The simplicial set model.}
Voevodsky was an algebraic topologist and
geometer by training. In 2011 in the unpublished \textit{Notes on type systems},
and later in summaries and reformulations, Voevodsky discovered a new model of
\Coq{}'s (core) type theory. In this model, types are interpreted as Kan complexes,
a highly technical and abstract representation of spaces. Voevodsky and others
were thus able to utilize intuitions from algebraic topology and category theory
to develop proofs in type theory.

The model inspired two additional logical principles for type theory. The
Calculus of Constructions together with these ``higher inductive types''
and the ``univalence axiom'' was called homotopy type theory (\HoTT{}),
referring explicitly to its newfound semantics.

\paragraph{The \UniMath{} library.} Voevodsky's initial experiments with
formalizing proofs in \Coq{} turned into a library called
\software{Foundations}. This and other libraries of code utilizing the
univalence axiom were merged into the \UniMath{} library. The development of
this formalism in \Coq{} faces one fundamental barrier: When assuming axioms in
type theory, one loses the computational behavior.

Developing a ``computational interpretation'' of the univalence axiom, that is,
the creation of a (computationally meaningful) type theory in which it is a
theorem rather than an axiom, remains an area of active research. In the
expectation that there will someday be a proof assistant implementing such a
type theory (and for a simpler metatheory), the \UniMath{} library uses a
minimal type theory that we'll develop in this thesis called univalent type
theory, or \UTT{}.

\paragraph{(Co)induction.} ``Inductive'' types such as the natural numbers $\ℕ$,
lists, and binary trees, and their ``coinductive'' versions such as streams (not
necesssarily finite lists) and infinite binary trees, are essential to any
modern proof assistant (or more broadly, functional programming language).
However, their presence is generally postulated, rather than derived. Thus,
\UniMath{} disallows such definitions.

However, recent results have shown that one can derive the presence of
coinductive types within \UTT{} while only postulating the natural numbers.
This result has been formalized within the proof assistant \Agda{}, one goal of
the present work is to prove (some of) that result in \Coq{}, and integrate it
into \UniMath{} so that it may be practically utilized by a wider research
community. 

Hence, the results in this thesis have been formalized in the \Coq{} proof
assistant. They were subsequently reviewed (by the \UniMath{}
development team \cite{unimath} and Anders Mörtberg) and accepted into the
\UniMath{} library. Their names appear in a monospaced font (e.g.\
\unimathname{univalence}). 

\subfile{chap1}
\subfile{chap2}
\subfile{chap3}
\subfile{chap4}

\chapter*{Conclusion}
\addcontentsline{toc}{chapter}{Conclusion}
\chaptermark{Conclusion}
\markboth{Conclusion}{Conclusion}
\setcounter{chapter}{5}
\setcounter{section}{0}

This thesis has presented univalent type theory, a modified version of Martin-Löf
dependent type theory inspired by a homotopy-theoretic semantics. The
presentation began somewhat historically, first discussing intuitionistic
logic and the λ-calculus, describing their interplay via the BHK interpretation
and Curry-Howard correspondence, and going on to describe how enrichments of
these systems lead to full-blown type theory. This perspective emphasizes
the harmonic interplay between the logical and computational aspects of this
theory. 

In \cref{chap:type-theory}, we saw some organizational principles such as
hlevels, weak equivalences, and characterization of paths inspired by the
homotopy model. Notably, this conceptual framework, motivated by spatial
intuition, greatly eased the decidedly non-topological work of later chapters. 
The univalent perspective ends up being useful in practice for logicians and
computer scientists, and not just topologists.

\Crefrange{chap:category-theory}{chap:coinductive-types-in-univalent-type-theory}
describe applications of \UTT{}. The development of (higher) category theory in
type theory continues to be an area of active research with rich results.
In keeping a well-understood pattern in the practice of constructive mathematics,
\UTT{} allows for the definition and exploration of classically-equivalent
concepts (e.g.\ categories and precategories), resulting in a more fine-grained
and general theory. Type theory is not ``just another'' foundational system in
which to do the exact same mathematics one might do in set theory, but indeed
gives rise to interesting, nontrivial, classically-inexpressible results.

While all results in this thesis were formalized in the computer proof assistant
\Coq{}, the workings, benefits, and limitations of such systems were mostly
relegated to footnotes when present at all. The methodology of computerized proof
undergoes an extended renaissance as computer scientists become more
interested in formally verified code and a wider community of mathematicians
begin to worry about the robustness of current (informal) proof-checking
procedures or get excited about \HoTT{}.

The type theory research community is small, yet vibrant. There are incredibly
important questions in the foundations of the theory itself (Is there a
constructive model of \UTT{}? Voevodsky's construction uses the axiom of
choice.), in its use as a framework for mathematics (How can one work
effectively with structures like $\R$ within a constructive theory?), and
in its implementations in proof assistants (How can we effectively use
automation in proofs?). Due to the proliferation of such concerns and the
ability to utilize intuitions from functional programming or algebraic topology,
type theory is a field in which even undergraduates can work on significant
research-level questions. 

%If you feel it necessary to include an appendix, it goes here.
\appendix

% \chapter{Cross-reference of names}

The following table lists lemmas taken from \cite{book}.
\begin{table}[ht]
  \centering
  \begin{tabular}{c | c}
    This thesis & The \HoTT book \\ \hline
    \Cref{def:ua} & Axiom 2.10.3
  \end{tabular}
\end{table}

The following table compares the terminology used in this thesis to that in our
\Coq{} formalization (under the column \UniMath{}) and that of the \Agda{} development
of \cite{non-wellfounded} (under the column \MTypes{}).
\begin{table}[ht]
  \centering
  \begin{tabular}{c | c | c }
    This thesis & \UniMath{} & \MTypes{} \\ \hline
  \end{tabular}
\end{table}

\chapter{CCCs and the λ-calculus}
\label{chap:cccs-and-lc}

\Cref{chap:category-theory} provides \textit{almost} enough background to
understand an interesting connection between a certain kind of category and the
λ-calculus.

\begin{lemma}[\unimathname{CategoryTheory.categories.Types.ExponentialsType}]
	Any fixed universe $\universe$ of types has exponentials.
\end{lemma}
\begin{proof}
	
\end{proof}

\begin{definition}
	A \define{monoidal category} consists of a category $\bfC$ together with
  \begin{itemize}
    \itemsep-0.2em
    \item a bifunctor $\bfC×\bfC→\bfC$ called the \define{tensor product} or
      \define{monoidal product},
    \item an object $I∈\Obj\bfC$ called the \define{unit},
    \item for all objects $A,B,C∈\Obj\bfC$, an isomorphism
      $α_{A,B,C}:(A⊗B)⊗C≅A⊗(B⊗C)$, and
    \item for all objects $A∈\Obj\bfC$, isomorphisms $λ_A:I⊗A≅A$ and $ρ_A:A⊗I≅A$.
  \end{itemize}
  These isomorphisms are subject to additional ``coherence conditions'', which
  won't play a crucial role here. A monoidal category is
  \begin{itemize}
    \itemsep-0.2em
    \item \define{symmetric} if there are additional isomorphisms
      $s_{A,B}:A⊗B→B⊗A$ satisfying other coherence conditions; and is
    \item \define{cartesian} if the monoidal product $⊗$ coincides with the
      categorical product $×$.
  \end{itemize}
\end{definition}

Only cartesian monoidal categories play a crucial role later in this section.

\begin{remark}
	For any categorical product there is a specified isomorphism $A×B≅B×A$, so any
  cartesian monoidal category is also symmetric monoidal.
\end{remark}

\begin{example}
  \
  \begin{itemize}
    \itemsep-0.2em
    \item \vspace{-0.3em} $\Set$, $\universe$, $\Cat$, $\Grp$, and others are 
      cartesian (so necessarily symmetric) monoidal under their categorical
      products. 
    \item $F\dVect$ is symmetric monoidal under $⊗$; the trivial vector space is
      a unit.
  \end{itemize}
\end{example}

\begin{definition}\label[definition]{def:exponentials}
  An \define{internal hom} for a monoidal category $(\bfC,⊗)$
  consists of, for each pair of objects $A,B∈\Obj\bfC$, an object
  $A⇒B$ together an arrow $\eval:(A⇒B)×A→B$ satisfying the following universal
  property: for any object $C∈\Obj\bfC$ and arrow $f:C⊗A→B$, there is an arrow
  $λf:C→(A⇒B)$ making the following diagram commute:
  \begin{center}
    \begin{tikzcd}[sep=large]
      C×A \arrow[dr, "f"] \arrow[d, swap, "λf×\id_A"] & {} \\
      (A⇒B)×A \arrow[r, swap, "\eval"]      & B
    \end{tikzcd}
  \end{center}
  When $\bfC$ is cartesian, $A⇒B$ is called an \define{exponential},
  and is denoted $B^A$.
\end{definition}

\begin{definition}
  A monoidal category $(\bfC,⊗)$ is \define{closed} when it has exponentials.
\end{definition}

\begin{remark}
  (Compare to \cref{rmk:prod-coprod-universal-adj})
  In a closed monoidal category, for all $A,B,C∈\Obj\bfC$,
  $λ$ and $\eval$ define a bijection
  \begin{equation*}
    \Hom(A⊗B,C)≅\Hom(A,B⇒C).
  \end{equation*}
  This bijection can be taken as their definition
  with an additional ``naturalality'' condition.
  This bijection is often called \define{currying}.
\end{remark}

\begin{example*}
  $\Set$ and $\universe$ are closed monoidal. Define $B⇒A\coloneqq \Hom(B,A)$.
  Then this is an element of $\Set$/$\universe$. Mixing set- and type-theoretical
  notation, the bijection is given by the mutually inverse functions
  \begin{align*}
    \Hom(A×B,C) &\longrightarrow \Hom(A,\Hom(B,C)) \\
    f &\longmapsto \λ{a}{\λ{b}{\appply{f}{a}{b}}} \\
    \Hom(A,\Hom(B,C)) &\longrightarrow \Hom(A×B,C) \\
    g &\longmapsto \λ{p}{\appply{g}{(\appr{1}{a})}{(\appr{2}{b})}}
  \end{align*}
\end{example*}

The following proof will only be accessible to readers with a background in
category theory (specifically, who know what adjunctions and natural
transformations are), but isn't necessary for the rest of the appendix.

\begin{lemma}[\unimathname{CategoryTheory.categories.Cats.ExponentialsCat}]
	The precategory of categories (\cref{def:precat-of-precats}) has exponentials.
\end{lemma}
\begin{proof}
  Fix $A:\universe$. We need to define a functor $E:\ttfun{Cat}→\ttfun{Cat}$ which
  is (RIGHT/LEFT) adjoint to $B↦A×B$. Define its action on objects by
  \begin{align*}
    E : \ttfun{Cat} &\longrightarrow \ttfun{Cat} \\
    C &\longmapsto_0 [A,C]
  \end{align*}
  that is, $E$ maps each category to the category of functors $A→C$ (with
  natural transformations as morphisms).
  
  Here things get a bit tricky: the action of $E$ on arrows takes a functor in
  $[A,C]$ to one in $[A,D]$ for each functor $C→D$. This assignment must
  \textit{itself} be functorial. Fix some categories $C$ and $D$ and a functor
  $F:C→D$, and define $E'$ on objects as
  \begin{align*}
    E' : [A,C] &\longrightarrow [A,D] \\
    G &\longmapsto_0 F_0∘G \\
    η &\longmapsto_1 \braces{\apply{F_1}{η_a}}_{a:\Obj A}.
  \end{align*}
  On arrows, $E'$ maps a natural transformation $η:G⟹H$ to one
  $\apply{E'}{η}:F∘G⟹F∘H$. To show that $\apply{E'}{η}$ is indeed a natural
  transformation, let $a,b:\Obj A$ and $f:\appply{\Hom}{a}{b}$. We want to show
  that the following square commutes:
  \begin{center}
    \begin{tikzcd}
      \apply{(F∘G)}{a}
        \arrow[r, "\apply{(F∘G)}{f}"]
        \arrow[d, "\apply{F_1}{η}_a"] &
      \apply{(F∘G)}{b}
        \arrow[d, "\apply{F_1}{η}_b"] \\
      \apply{(F∘H)}{a}
        \arrow[r, "\apply{(F∘H)}{f}"] &
      \apply{(F∘H)}{b}
    \end{tikzcd}
  \end{center}
  This is essentially the diagram expressing the naturality of $η$, but with $F$
  applied everywhere. The proof that it commutes reflects this structure:
  \begin{align*}
    F_1η_b ∘ \big(\apply{(F∘G)}{f}\big)
    &\≡ F_1η_b ∘ \apply{F_1}{(\apply{G}{f})} \\
    &\= \apply{F_1}{(η_b ∘ \apply{G}{f})} 
    &&\text{Functoriality of }F \\
    &\= \apply{F_1}{(\apply{H}{f} ∘ η_a)} 
    &&\text{Naturality of }η \\
    &\= \apply{F_1}{(\apply{H}{f})} ∘ F_1η_a 
    &&\text{Functoriality of }F \\
    &\≡ \big(\apply{(F∘H)}{f}\big) ∘ F_1η_a
  \end{align*}
\end{proof}

\begin{definition}\label[definition]{def:cartesian-cat}
  A category is \define{Cartesian closed}
  when it is closed, cartesian monoidal, and has a terminal object.
  Equivalently, it is cartesian closed when it has
  \begin{itemize}
    \itemsep-0.2em
    \item \vspace{-0.1em} a terminal object (\cref{def:terminal-and-initial}),
    \item binary products (\cref{def:product-and-coproduct}), and
    \item exponentials (\cref{def:exponentials}).
  \end{itemize}
  ``CCC'' abbreviates ``Cartesian closed category''.
\end{definition}

\begin{remark}\label[remark]{rmk:lambek}
	We now have the vocabulary to (informally) extend the correspondence of
  \cref{sec:propositions-and-types} to statements about CCCs.
  \cite{lambek}
\end{remark}

\begin{remark}
  The Curry-Howard-Lambek correspondence gives the barest hint of insight into
  Voevodsky's innovation. He also gave a categorically-based model of a version
  of the typed λ-calculus, but of full Martin-Löf dependent type theory. The
  structure of higher/nested identity types
  ($\propeq{\propeq{\cdots}{a}{b}}{p}{q}$) recalled ideas of homotopy theory,
  where paths are considered identical up to the existence of ``higher
  paths''.

  For the initial development of the connection between locally Cartesian closed
  categories (LCCCs) and type theory, see \cite{seely}.
\end{remark}


\backmatter % backmatter makes the index and bibliography appear properly in the TOC

%  \bibliographystyle{bsts/mla-good} % there are a variety of styles available;
%  \bibliographystyle{plainnat}
% replace ``plainnat'' with the style of choice. You can crefer to files in the bsts or APA
% subfolder, e.g.
% \bibliographystyle{APA/apa-good}  % or

% uncomment me!
\nocite{*}
\bibliographystyle{apalike}
\bibliography{thesis}

% Comment the above two lines and uncomment the next line to use biblatex-chicago.
%\printbibliography[heading=bibintoc]

% Finally, an index would go here... but it is also optional.
\index{$\id$|see {Identity}}
\index{$F$-algebra|see {Algebra for a functor}}
\printindex
\end{document}
