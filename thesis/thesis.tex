\documentclass[12pt,twoside,draft]{reedthesis}
\usepackage{graphicx} 
\usepackage{booktabs,setspace} 
% \usepackage{natbib}
% Comment out the natbib line above and uncomment the following two lines to use the new 
% biblatex-chicago style, for Chicago A. Also make some changes at the end where the 
% bibliography is included. 
%\usepackage{biblatex-chicago}
%\bibliography{thesis}

%%%%%%%%%%%%%%%%%%%%%%%%%%%%%%% While drafting:
\usepackage{showlabels}

\usepackage{color}
\usepackage[dvipsnames]{xcolor}
\definecolor{TODO}{HTML}{EE8800}
\newcommand{\TODO}[1]{\marginpar{\footnotesize\color{TODO}todo: #1}}

% \usepackage{showidx}
\let\oldindex\index
\definecolor{index}{HTML}{0088EE}
\renewcommand{\index}[1]{\oldindex{#1}\marginpar{\footnotesize\color{index}index: #1}}
\newcommand{\indeX}[1]{\oldindex{#1}}
%%%%%%%%%%%%%%%%%%%%%%%%%%%%%%% My stuff:
%% math symbols, notation, etc
\usepackage{amssymb}       % math characters
\usepackage{mathtools}     % improvement on amsmath

% Use better math fonts when available
\IfFileExists{../tex-preamble/fonts/latinmodern-math.otf}
  {\newcommand{\fontpath}{../tex-preamble/fonts}}
  {\newcommand{\fontpath}{../../tex-preamble/fonts}}

\IfFileExists{\fontpath}{
  \usepackage{unicode-math}         % use OTF math fonts
  \AtBeginDocument{\let\phi\varphi} % varphi with unicode-math
  %\AtBeginDocument{\let\epsilon\varepsilon}
  % \setmathfont[Path=\fontpath]{latinmodern-math.otf}
  \setmathfont[Path=\fontpath]{Asana-Math.otf}
  \setmathfont[Path=\fontpath,range=\setminus]{Asana-Math.otf}
  \setmathfont[Path=\fontpath,range=\mathbb]{texgyretermes-math.otf}
  \setmathfont[Path=\fontpath,range=\mathcal]{latinmodern-math.otf}

  % fix xmapsto. only works with text above.
  % https://github.com/wspr/unicode-math/issues/197
  \ExplSyntaxOn
  \RenewDocumentCommand \xmapsto { m } {
    \mathrel {
      \exp_after:wN \Uoverdelimiter \cs:w sym \__um_symfont_tl \cs_end: "21A6 {
        \mkern 5mu \scan_stop: #1 \mkern 5mu \scan_stop:
      }
    }
  }
  \ExplSyntaxOff
}{}

% general stuff
\newcommand{\units}[1]{\;\mathrm{#1}}
\newcommand{\dd}{\,\mathrm{d}}
\newcommand{\ov}{\overline}
\newcommand{\paren}[1]{\left(#1\right)}
\newcommand{\braces}[1]{\left\{#1\right\}}
\newcommand{\brackets}[1]{\left[#1\right]}
\newcommand{\ceiling}[1]{\left\lceil{} #1 \right\rceil}
\newcommand{\floor}[1]{\left\lfloor{} #1 \right\rfloor}
\newcommand{\abs}[1]{\left| #1 \right|}
\newcommand{\e}[1]{\cdot 10^{#1}}
% \newcommand{\emf}{\mathcal{E}}
\newcommand{\contradiction}{\Rightarrow\Leftarrow}
\newcommand{\inv}[1]{\frac{1}{#1}}
\newcommand{\tif}{\text{ if }}
\newcommand{\totherwise}{\text{ otherwise }}
% \newcommand{\mydef}{\textbf{Definition:}}
% complex numbers
\renewcommand{\Im}{\operatorname{Im}}
\renewcommand{\Re}{\operatorname{Re}}

% operators/functions
\DeclareMathOperator{\ord}{ord}
% \DeclareMathOperator{\image}{image}
\DeclareMathOperator{\Int}{Int}
\DeclareMathOperator{\Ext}{Ext}
\DeclareMathOperator{\Lim}{Lim}
\DeclareMathOperator{\Bd}{Bd}
\DeclareMathOperator{\lcm}{lcm}
\DeclareMathOperator{\proj}{proj}
\DeclareMathOperator{\xor}{xor}
\DeclareMathOperator{\spanf}{span}
\DeclareMathOperator{\id}{id}
% linear
\DeclareMathOperator{\col}{Col}
\DeclareMathOperator{\row}{Row}
\DeclareMathOperator{\nul}{Nul}
\DeclareMathOperator{\rank}{rank}

% set theory
\newcommand{\xlongmapsto}[1]{\xmapsto{\,\,\, #1\,\,\,}}
\newcommand{\xlongrightarrow}[1]{\xrightarrow{\,\,\, #1\,\,\,}}
\newcommand{\union}{\mathop{\bigcup}}
\newcommand{\disunion}{\mathop{\dot\bigcup}}
\newcommand{\intersection}{\mathop{\bigcap}}
\newcommand{\powerset}{\mathcal{P}}

% analysis & calculus
\usepackage{esint} % \oiint, etc
\newcommand{\evalat}[2]{{\bigg|_{#1}^{#2}}}
\newcommand{\ext}[1]{{#1}^{ext}}  % extension of function
\DeclareMathOperator{\determinant}{det}
\DeclareMathOperator{\oball}{B}
\DeclareMathOperator{\cball}{\ov{B}}
\DeclareMathOperator{\sphere}{S}
\DeclareMathOperator{\divergence}{div}
\DeclareMathOperator{\curl}{curl}

% vectors
\newcommand{\ihat}{\hat{\imath}}
\newcommand{\jhat}{\hat{\jmath}}
\newcommand{\khat}{\hat{k}}
\newcommand{\lvec}[1]{\overrightarrow{#1}}
\newcommand{\len}[1]{\left\| #1 \right\|}

% topology
\newcommand{\RP}{\ensuremath{\R\hspace{-0.07em}\mathrm{P}}}

% categories
\newcommand{\I}{\ensuremath{\mathbf{I}}} % "interval groupoid"
\newcommand{\Cat}{\ensuremath{\mathcal{C}\mathrm{at}}}
\newcommand{\ComGrp}{\ensuremath{\mathcal{C}\mathrm{om}\mathcal{G}\mathrm{rp}}}
\newcommand{\AbGrp}{\ensuremath{\mathcal{A}\mathrm{b}\mathcal{G}\mathrm{rp}}}
\newcommand{\Grp}{\ensuremath{\mathcal{G}\mathrm{rp}}}
\newcommand{\Grpd}{\ensuremath{\mathcal{G}\mathrm{rpd}}}
\newcommand{\Mod}{\ensuremath{\mathcal{M}\mathrm{od}}}
\newcommand{\Man}{\ensuremath{\mathcal{M}\mathrm{an}}}
\newcommand{\Metric}{\ensuremath{\mathcal{M}\mathrm{etric}}}
\newcommand{\Set}{\ensuremath{\mathcal{S}\mathrm{et}}}
\newcommand{\Top}{\ensuremath{\mathcal{T}\!\mathrm{op}}}
\newcommand{\homoTop}{\ensuremath{\mathrm{homo}\mathcal{T}\mathrm{op}}}
\newcommand{\Vect}{\ensuremath{\mathcal{V}\mathrm{ect}}}
\renewcommand{\phi}{\varphi}

% number theory
\newcommand{\modulus}[1]{\; \left(\mathrm{mod}\; #1\right)}
\newcommand{\crash}[2]{\left(\frac{#1}{#2}\right)}
\newcommand{\QED}{\begin{flushright}QED\end{flushright}}

% make letters in different fonts (bold, blackboard, caligraphic, fraktur)
% https://tex.stackexchange.com/questions/156196/newcommand-with-variable-name
\newcommand{\mkltr}[1]{%
    \expandafter\newcommand\csname bf#1\endcsname{\ensuremath{\mathbb #1}}
    \expandafter\newcommand\csname bb#1\endcsname{\ensuremath{\mathbb #1}}
    \expandafter\newcommand\csname ca#1\endcsname{\ensuremath{\mathcal #1}}
    \expandafter\newcommand\csname fr#1\endcsname{\ensuremath{\mathfrak #1}}
}%
\mkltr A \mkltr B \mkltr C \mkltr D \mkltr E \mkltr F \mkltr G \mkltr H
\mkltr I \mkltr J \mkltr K \mkltr L \mkltr M \mkltr N \mkltr O \mkltr P
\mkltr Q \mkltr R \mkltr S \mkltr T \mkltr U \mkltr V \mkltr W \mkltr X
\mkltr Y \mkltr Z

\newcommand{\zn}{\mathbf{Z}_n}
\newcommand{\zp}{\mathbf{Z}_p}
\newcommand{\N}{\ensuremath{\mathbf{N}}}
\newcommand{\Z}{\ensuremath{\mathbf{Z}}}
\newcommand{\Q}{\ensuremath{\mathbf{Q}}}
\newcommand{\R}{\ensuremath{\mathbf{R}}}
\newcommand{\C}{\ensuremath{\mathbf{C}}}

% algebra
\newcommand{\quotient}[2]{{\raisebox{.2em}{$#1$}\left/\raisebox{-.2em}{$#2$}\right.}}
\newcommand{\GL}{\mathrm{GL}}     % general linear group
\newcommand{\SL}{\mathrm{SL}}     % special linear group
\renewcommand{\S}{\mathfrak{S}}   % symmetric group
\DeclareMathOperator{\ann}{ann}   % the annihilator of a submodule
\DeclareMathOperator{\Obj}{Obj}   % objects of a category
\DeclareMathOperator{\Hom}{Hom}   % homomorphisms between modules
\DeclareMathOperator{\Mor}{Mor}   % categorical morphisms
\DeclareMathOperator{\characteristic}{char} % ring characteristic
\DeclareMathOperator{\Tor}{Tor}   % torsion elements
\DeclareMathOperator{\Stab}{Stab} % stabilizer
\DeclareMathOperator{\Orb}{Orb}   % orbit
\DeclareMathOperator{\Aut}{Aut}   % automorphisms
\DeclareMathOperator{\End}{End}   % endomorphisms
\DeclareMathOperator{\Mat}{Mat}   % matrix ring
\DeclareMathOperator{\Tr}{Trace}  % trace of a matrix
\DeclareMathOperator{\im}{im}     % image

% statistics
\DeclareMathOperator{\Var}{Var}

% nicer empty set
\let\oldemptyset\emptyset{}
\let\emptyset\varnothing{}
%% formatting

\usepackage{fontspec}
% EB Garamond (Initials - just for fun) - body text, old style, serif
% Nimbus Roman - serifed
% URW Gothic - geometric, title text
% Libre Caslon Text - absurdly well spaced
% FreeSerif - plays pretty nicely with math, looks like Times
% Tinos - good alternative to FreeSerif, less TNR looking
% \setmainfont[Ligatures=TeX]{Tinos}
% \fontspec [Path = ../tex-preamble/fonts/, 
%            UprightFont = *-regular,
%            BoldFont    = *-bold,
%            ItalicFont  = *-italic]
%            {texgyrepagella}
\setmainfont[ Path=../tex-preamble/fonts/
            , UprightFont     = *-regular
            , BoldFont        = *-bold
            , ItalicFont      = *-italic
            , BoldItalicFont  = *-bolditalic
            ]{texgyrepagella}

\usepackage{datetime2} % use yyyy--mm--dd date with \DTMtoday
\usepackage{geometry}
\geometry{
  lmargin=4cm,
  rmargin=4cm,
  tmargin=3cm,
  bmargin=3cm,
}

\renewcommand{\arraystretch}{1.2} % Make tables a little bigger
% \usepackage{indentfirst}          % indent first \par after sections
% \usepackage{parskip}              % don't indent anything
\usepackage{enumitem}             % enumerate with A), b., c) styles
\usepackage{hyperref}             % linked table of contents
\hypersetup{
    colorlinks,
    citecolor=black,
    filecolor=black,
    linkcolor=black,
    urlcolor=black
}
\usepackage[noabbrev, capitalize]{cleveref}
\crefname{notation}{Notation}{Notations}
\Crefname{notation}{Notation}{Notations}
\crefname{rule}{Rule}{Rule}
\Crefname{rule}{Rule}{Rule}

% graphics
\usepackage{graphicx}
\DeclareGraphicsExtensions{.pdf,.png,.jpg}
\usepackage{float} % image positioning
% Theorems, proofs, and other mathematical environments

\usepackage{amsthm}

\newtheoremstyle{mine} % name
  {7pt} % space above
  {7pt} % space below
  {} % body font
  {} % indent amount
  {\bfseries} % theorem head font
  {.} % punctuation after theorem head
  {.5em} % space after theorem head
  {} % theorem head spec (can be left empty, meaning no)

\theoremstyle{mine} % these are the non-italicized
\newtheorem{theorem}{Theorem}[subsection]
\newtheorem*{theorem*}{Theorem}
\newtheorem{lemma}[theorem]{Lemma}
\newtheorem*{lemma*}{Lemma}
\newtheorem{proposition}[theorem]{Proposition}
\newtheorem*{proposition*}{Proposition}
\newtheorem{corollary}[theorem]{Corollary}
\newtheorem*{corollary*}{Corollary}
\newtheorem{eqenv}[theorem]{Equation}
\newtheorem*{eqenv*}{Equation}
\newtheorem{remark}[theorem]{Remark}
\newtheorem*{remark*}{Remark}
\newtheorem{example}[theorem]{Example}
\newtheorem*{example*}{Example}
\newtheorem{definition}[theorem]{Definition}
\newtheorem*{definition*}{Definition}
\newcounter{Problem} \stepcounter{Problem} % start at 1
\newcommand{\problem}[0]{
  \vspace{-0.8em}
  \begin{center}
    \huge
    \rule[0.25em]{0.3\textwidth}{0.6pt}
    \arabic{Problem}
    \rule[0.25em]{0.3\textwidth}{0.6pt}
  \end{center}
  \stepcounter{Problem}
  \vspace{0.5em}
}
\newcommand{\problemnum}[1]{
  \vspace{-0.5em}
  \begin{center}
    \huge
    \rule[0.25em]{0.3\textwidth}{0.5pt}
    #1
    \rule[0.25em]{0.3\textwidth}{0.5pt}
  \end{center}
  \vspace{0.5em}
}
\newcommand{\subproblem}[1]{
  \vspace{0.2em}
  \begin{center}
    \large
    \rule[0.25em]{0.2\textwidth}{0.5pt}
    #1
    \rule[0.25em]{0.2\textwidth}{0.5pt}
  \end{center}
  \vspace{0.2em}
}
% Combine the two with more pleasant vertical spacing
\newcommand{\problemsubproblem}[1]{\problem \vspace{-0.8em}\subproblem{#1}}
\newcommand{\problemnumsubproblem}[2]{\problemnum{#1}\vspace{-0.8em}\subproblem{#2}}
% requires logic.tex
% some of this is based on HoTT/HoTT

\newcommand{\ttfun}[1]{\ensuremath{\mathsf{#1}}}  % type-theoretic functions

\newcommand{\eval}{\ttfun{eval}}

% recurors
\newcommand{\case}{\ttfun{case}}
\newcommand{\rec}{\ttfun{rec}}

\newcommand{\emptytype}{\ensuremath{\mathbf{0}}}  % empty type
\newcommand{\unittype}{\ensuremath{\mathbf{1}}}   % unit type
% \newcommand{\unitelem}{\ensuremath{1_{\unittype}}} % sole inhabitant of the unit type
\newcommand{\unitelem}{\star} % sole inhabitant of the unit type
\newcommand{\booltype}{\ensuremath{\mathbf{2}}}   % two element type
\newcommand{\btrue}{1_{\booltype}}                 % inhabitant of two element type
\newcommand{\bfalse}{0_{\booltype}}                % inhabitant of two element type
\newcommand{\inl}{\ttfun{inl}}
\newcommand{\inr}{\ttfun{inr}}
\newcommand{\universe}{\ensuremath{\mathcal{U}}}         % universe type
\newcommand{\universei}[1]{\ensuremath{\mathcal{U}_{#1}}}  % ith universe

\newcommand{\sigmat}[2]{\ensuremath{\sum_{\paren*{#1}}#2}}    % Σ-type
\newcommand{\dpair}[2]{\ensuremath{\paren{#1\hspace{0.1em};#2}}} % Σ-pair
\newcommand{\pr}[1]{\ttfun{pr}_{#1}}
\newcommand{\pit}[2]{\ensuremath{\prod_{\paren*{#1}}#2}}      % Π-type
\newcommand{\refl}[1]{\ttfun{refl}_{#1}}      % reflexivity
\newcommand{\W}[3]{\ttfun{W}_{(#1:#2)}\apply{#3}{#1}}      % W-types
\renewcommand{\M}[3]{\ttfun{M}_{(#1:#2)}\apply{#3}{#1}}      % M-types
\renewcommand{\sup}[2]{\appply{\ttfun{sup}}{#1}{#2}}      % supremum
\newcommand{\recW}{\rec_{\mathsf{W}}}                   % W-elim

\newcommand{\fromempty}[1]{\apply{\rec_{\emptytype}}{#1}}
\newcommand{\fromunit}[2]{\appply{\rec_{\unittype}}{#1}{#2}}


%%% Natural numbers
\newcommand{\suc}{\mathsf{succ}}
\newcommand{\add}{\mathsf{add}}

%%% Lists
\newcommand{\List}[1]{\ttfun{List}(#1)}
\newcommand{\nil}{\ttfun{nil}}
\newcommand{\cons}{\ttfun{cons}}

%%% Function extensionality
\newcommand{\funext}{\mathsf{funext}}
\newcommand{\happly}{\mathsf{happly}}

%%% hlevels
\newcommand{\isContr}{\ttfun{isContr}}
\newcommand{\isProp}{\ttfun{isProp}}
\newcommand{\isSet}{\ttfun{isSet}}
\newcommand{\hlevel}{\ttfun{hlevel}}

%%% categorytheory

\newcommand{\ttHom}{\ttfun{Hom}}
\newcommand{\catarrow}{\rightharpoonup}
\newcommand{\Precat}{\ttfun{Precat}}
\newcommand{\Functor}{\ttfun{Functor}}
\newcommand{\Cocone}{\ttfun{Cocone}}
\newcommand{\Cone}{\ttfun{Cone}}
\newcommand{\Lim}{\ttfun{Lim}}
\newcommand{\Colim}{\ttfun{Colim}}

%%% Relations

% Propositional equality
\newcommand{\propeqsign}{=}
\renewcommand{\=}{\propeqsign}
\newcommand{\propeq}[3]{#2 \propeqsign_{#1} #3}

% Others
\newcommand{\jdeq}{\equiv}
\newcommand{\defeq}{\vcentcolon\jdeq}
\newcommand{\weqsign}{\simeq}
\newcommand{\weq}[2]{#1 \weqsign #2}
\newcommand{\homot}[2]{#1 \sim #2}

\newcommand{\appr}[2]{\apply{\pr{#1}}{#2}} % a combination of apply and pr
\newcommand{\apinl}[1]{\apply{\inl}{#1}}   % a combination of apply and inl
\newcommand{\apinr}[1]{\apply{\inr}{#1}}   % a combination of apply and inl

\newcommand{\ap}[2]{\mathsf{ap}_{#1}{\paren{#2}}} % action on paths

% univalence
\newcommand{\ua}{\mathsf{ua}} % univalence axiom

%%% Transport (covariant) %%%
% \newcommand{\trans}[2]{\ensuremath{{#1}_{*}\mathopen{}\left({#2}\right)\mathclose{}}\xspace}
% \let\Trans\trans
% \newcommand{\transf}[1]{\ensuremath{{#1}_{*}}\xspace} % Without argument
\newcommand{\transpor}[2]{\ttfun{transport}^{#1}\paren{#2}}
\newcommand{\transport}[3]{\ttfun{transport}^{#1}\paren{#2,#3}}
\newcommand{\transportname}{\ttfun{transport}}

\newcommand{\abbreviation}[1]{\textbf{#1}\indeX{#1@\textbf{#1}}} % abbreviations
% Judgments
\newcommand{\prop}{\hspace{0.4em}\text{prop}}
\newcommand{\type}{\hspace{0.4em}\text{type}}
\newcommand{\true}{\hspace{0.4em}\text{true}}

\usepackage{makeidx}
\usepackage{prftree}
\usepackage{tikz}
\usetikzlibrary{cd}
\tikzcdset{
  arrow style=tikz,
  arrows={line width=0.65pt},
  >={stealth}
}
\newcommand{\software}[1]{{\textsc{#1}}\indeX{#1}}
\newcommand{\Agda}{\software{Agda}}
\newcommand{\UniMath}{\software{UniMath}}
\newcommand{\Coq}{\software{Coq}}
\newcommand{\MTypes}{\software{HoTT/M-Types}}

\newcommand{\dual}[2]{
  \begin{itemize}\renewcommand{\labelitemi}{$\circ$}
    \itemsep0em
    \item #1
    \item #2
   \end{itemize}
}

\newcommand{\define}[1]{\textbf{#1}} % term being defined
\newtheorem{notation}[theorem]{Notation}
\newtheorem{tt-rule}[theorem]{Rule}

\newcommand{\Coalgtype}{\ensuremath{\mathsf{Coalg}}}
\newcommand{\Final}{\ensuremath{\mathsf{Final}}}

%%%%%%%%%%%%%%%%%%%%%%%%%%%%%%% My stuff:

\title{Deriving Coinductive Types in Univalent Type Theory}
% \title{Bruh: $(A\simeq B)\simeq (A=_{\mathcal{U}}B)$}
\author{Langston Barrett}
% The month and year that you submit your FINAL draft TO THE LIBRARY (May or December)
\date{May 2018}
\division{Mathematics and Natural Sciences}
\advisor{Safia Chettih}
\department{Mathematics}
% if you're writing a thesis in an interdisciplinary major,
% uncomment the line below and change the text as appropriate.
% check the Senior Handbook if unsure.
\thedivisionof{The Established Interdisciplinary Committee for}

\setlength{\parskip}{0pt}
\begin{document}

\maketitle
\frontmatter % this stuff will be roman-numbered
\pagestyle{empty} % this removes page numbers from the frontmatter

% Acknowledgements (Acceptable American spelling) are optional
% So are Acknowledgments (proper English spelling)
\chapter*{Acknowledgments}

% The pcreface is optional
% To remove it, comment it out or delete it.
% \chapter*{Pcreface}
% This thesis delves into highly interdisciplinary territory. I'll

\chapter*{List of Abbreviations}

\begin{table}[h]
  \centering
  \begin{tabular}{ll}
    \textbf{HoTT}  	&  Homotopy type theory \\
    \textbf{UTT}  	&  Univalent type theory \\
    \textbf{ITT}  	&  Intensional type theory \\
    \textbf{FOL}  	&  Classical first-order logic \\
    \textbf{IFOL}  	&  First-order logic (Gentzen's natural deduction) \\
    \textbf{LC}  	  &  Church's untyped $\lambda$-calculus
  \end{tabular}
\end{table}
	
% Depth to which to number and print sections in TOC
\setcounter{tocdepth}{4}
% \setcounter{secnumdepth}{2}
\tableofcontents
% if you want a list of tables, optional
% \listoftables
% if you want a list of figures, also optional
% \listoffigures

% If your abstract is longer than a page, there may be a formatting issue.
\chapter*{Abstract}

In this thesis, we explain Univalent type theory, a constructive and
computationally meaningful foundational system for mathematics inspired by
recent advances in the semantics of Per Martin-L\"of's intentional type theory.
We first develop the classical Curry-Howard correspondence between (the
constructive subset of) Gentzen's natural deducation and the $\lambda$-calculus.
We proceed to explicate Martin-L\"of's theory of dependent types and the central
modern development in type theory: Vladimir Voevodsky's Univalence principle. 
We go on to examine the nature of coinduction within this theory, presenting a
novel formalization of a recent result that M-types can be derived
\textit{internally} in Univalent type theory.

\chapter*{Dedication}

\mainmatter % here the regular arabic numbering starts
\pagestyle{fancyplain} % turns page numbering back on

\chapter*{Introduction}
\addcontentsline{toc}{chapter}{Introduction}
\chaptermark{Introduction}
\markboth{Introduction}{Introduction}

\section*{A short history of discomfort in mathematics}

% \subsection*{Trouble in Cantor's Paradise}

In the early 20\textsuperscript{th} century, mathematicians had a problem. Often
held up as the pinnacle of necessary, undoubtable, \textit{a priori} truth,
their discipline was suffering from an embarrassing lack of certainty. The
intricate arguments of advanced analysis left mathematicians unable to confirm
nor deny each other's proofs. While Cantor's na\"ive set theory had unleashed a
wave of paradoxes, the need for a rigorous logical foundation for higher
mathematics was so great that the enterprise of axiomatic set theory was pursued
headlong until some kind of consensus was reached. To make a long, rich story
brutally short, virtually every modern paper in mathematics and computer
science uses a combination of Gentzen's first-order logic (\abbreviation{FOL})
and an axiomatization of sets called
\abbreviation{ZFC}.\footnote{This ``consensus'' left out many
  prominent schools of thought, such as the Intuitionists and constructivists, a
  point we'll soon return to.} 

Despite the apparent rigor of \abbreviation{FOL}+\abbreviation{ZFC},
practitioners of the deductive sciences were still beset by issues of
verifiability. Complexity, specialization and sheer length made modern proofs
difficult to comprehend with the absolute certainty which is supposed to
characterize these disciplines. In the 1970s, two teams of topologists proved
contradictory results, and neither group could find the error in the other's
proof \cite{kolata}. Wiles's famous proof of Fermat's last theorem was utterly
unintelligible to the vast majority of mathematicians \cite{nyt}. The
classification of the simple finite groups, one of the crowning achievements of
modern mathematics, has a combined proof of over ten thousand pages. These
examples illustrate merely a few of the practical epistemological challenges
facing mathematicians, for a historical perspective see \cite{rigor-and-proof},
and for a general overview see \cite{fidelity}.

In 1990s, Fields medalist Vladimir Voevodsky grew concerned with the state
of mathematical knowledge. In 1998, Carlos Simpson released a pre-print arguing
that there was a major mistake in one of Voevodsky's papers. However, it was not
clear whether Voevodsky had errored, or whether there was a flaw in Simpson's
counterexample. In 1999, Pierre Deligne found a crucial mistake in Voevodsky's
``Cohomological Theory of Presheaves with Transfers'', upon which he had based
much of his work in the area of motivic cohomology. As he began to develop more
and more complex arguments, Voevodsky wondered: ``And who would ensure that I
did not forget something and did not make a mistake, if even the mistakes in
much more simple arguments take years to uncover?'' \cite{voevodsky-ias}. 

The problems facing Voevodsky and his peers seemed insurmountable.
As the requisite attention span, memory, and capacity for detail required to
understand new developments in higer mathematics reached inhuman proportions,
where was he to turn? He would not find a solution in the realm of pure
mathematics, but rather in one of the finest examples of collaboration between
mathematicians, computer scientists, and philosophers: the modern proof
assistant. \TODO{reword}

\section*{TODO}

While much of the mathematical discipline was simply relieved to have ``solved''
their foundational issues with axiomatic set theory, there remained a vocal
opposition to the newly-adopted methods. Most mathematicians are dimly aware
that there's some controversy about the Axiom of Choice (the ``C'' in
\abbreviation{ZFC}) \cite{martin-lof-100-years}, you don't have to look further
than \abbreviation{FOL} to find disagreements.

\section*{Our contribution}

(Co)inductive types and UniMath

\chapter{Propositions and Types}
\label[chapter]{chap:propositions-and-types}

In this chapter, we will present several logical
and computational frameworks of increasing complexity.
We will begin with general remarks on discussing and defining formal systems in
\cref{sec:discussing-logic}. \Cref{sec:natural-deduction} begins with an
intuitionistic version of Gentzen's natural deduction
\cite{gentzen1935untersuchungen} that we will call \abbreviation{IFOL}. As most
mathematicians, computer scientists, and philosophers are quite familiar with
some version of first-order logic, this section will focus on the introduction
of notation and meta-logical concerns. In \cref{sec:the-lambda-calculus}, we
will introduce Church's $\lambda$-calculus (typed and untyped). Finally, in
\cref{sec:the-curry-howard-correspondence}, we will discuss the fundamental and
harmonious relation between these systems known as the Curry-Howard
correspondence.

\section{Discussing logic}
\label{sec:discussing-logic}

The observant student of logic may notice a problem in the usual definition of
implication: one usually defines $P\implies Q$ as ``if whenever $P$ is true,
$Q$ is true, then $P\implies Q$ is true''. It seems that in order to understand
implication, one must first understand implication! More perniciously,
philosophers and mathematicians alike have argued that Skolem's
``paradox''\footnote{The L\"owenheim-Skolem theorem states that
  \abbreviation{ZFC} has a countable model. The paradox is that the statement
  ``there are uncountable sets'' is (famously) a theorem of \abbreviation{ZFC},
  Cantor's.} has deep and significant consequences \cite{skolem}. These
misunderstandings both arise from a common root: the failure to distinguish
between object language and metalanguage. In this section, we will attempt to
mitigate some of the difficulties that arise when discussing and defining formal
systems for logic and computation.

\begin{definition}\label[definition]{def:object-meta-language}
  When discussing or defining a formal language (most commonly, a
  logico-deductive framework), we call that language the
  \define{object language}\index{object language}. Our discussion takes place in
  a \define{metalanguage}.
\end{definition}

\begin{example}
  \
  \begin{itemize}
    \itemsep0em
    \item In the statement ``\abbreviation{FOL} is complete, \abbreviation{FOL}
      is the object language, and the metalanguage is English.
    \item G\"odel's second incompleteness theorem is a statement in the
      metalanguage of \abbreviation{FOL}+\abbreviation{ZFC}, about the object
      language of Peano arithmetic.
  \end{itemize}
\end{example}

\begin{definition}\label[definition]{def:metavariable}
	A \define{metavariable}\index{Metavariable} is a variable meant to stand for
  any well-formed expression of our object language.
\end{definition}

\begin{notation}
  In imitation of various proof assistants, we will prefix metavariables with
  question mark, e.g.\ $?a$.
\end{notation}

We will discuss metavariables and see many examples in the next section. 
We will follow Martin-L\"of in defining our object languages in terms of
judgments \cite{martin-lof-meanings}.

\begin{definition}\label[definition]{def:judgment}
	A \define{judgment}\index{Judgment} is something that may be known. A judgment
  is \define{evident} if one does, in fact, know it.
  A \define{hypothetical judgment}\index{Judgment!Hypothetical} is one that
  holds under the assumption that some other judgments hold.
\end{definition}

\begin{example}
  Judgment is inherently a meta-theoretical notion.
	We can (and will) make the following kinds of judgments:
  \begin{itemize}
    \itemsep0em
    \item $P$ is a well-formed formula 
    \item if $P$ and $Q$ are well-formed formulae, then $P\land Q$ is a
      well-formed formula (this judgment is hypothetical)
    \item $P$ is true
    \item the variable $x$ is free (resp.\ bound) in $P$ 
  \end{itemize}
  If one is working in a formal metalanguage, judgments may be defined
  inductively in it.
\end{example}

\begin{notation}\label[notation]{notation:proof-tree}
  The premises and conclusions of proofs or rules of deduction are always
  judgments. If we can deduce a conclusion $B$ from premises $A_1,\ldots,A_n$
  via some rule R (again: $A_1,\ldots,A_n$ and $B$ are judgments, not terms of
  an object language), we will write
  \begin{displaymath}
    \prftree[r]{\footnotesize R}
      {A_1}{A_2}{\ldots}{A_n}
      {B}
  \end{displaymath}
  In this way, we can write proofs as wellfounded trees with the conclusion as
  their root.
\end{notation}

\begin{notation}\label[notation]{notation:sequent}
  If $B$ is a hypothetical judgment holding under assumptions $A_1,\ldots,A_n$,
  we denote this by $A_1,\ldots,A_n\vdash B$ (this is called Gentzen's
  \define{sequent notation}, $\vdash$ can be read ``entails''). A hypothetical
  judgment of this form is distinguished from the deduction of $B$ from premises
  $A_1,\ldots,A_n$ (shown in \cref{notation:proof-tree}).

  To denote some arbitrary set of hypotheses, we will use capital Greek letters
  $\Gamma$ and $\Theta$. Our notions of entailment will always allow for
  \define{reflexivity} and \define{weakening}, which are the following
  meta-rules (where $J$ and $K$ stand for judgments):
  \begin{align*}
    \prftree[r]{}{}
      {J\vdash J}
    &&
    \prftree[r]{}
      {\Gamma\vdash J}
      {\Gamma,K\vdash J}
  \end{align*}
  When something is derivable under no (and so by weakening, any) hypotheses, we
  may leave off the $\vdash$. 
\end{notation}

\begin{notation}
  We denote the substitution of a term $t$ for all free occurrences variable $v$
  in an expression $e$ by $e[v\coloneqq t]$. We will not deal directly with the
  issues of variable capture, scoping, and substitution here, for a rigorous
  treatment see any thorough textbook on logic.
\end{notation}

\section{Intuitionistic natural deduction}
\label{sec:natural-deduction}

With these notations, we can quickly and uniformly define \abbreviation{IFOL}. 
Let $v,w,\ldots$ denote a countable set of
\define{variables}\index{Variable!In \abbreviation{IFOL}}. The first judgment of
natural deduction is that some term $?a$ represents a proposition (more
narrowly, is a well-formed formula). We will denote this $?a\prop$.

The following are the \define{formation rules}\index{Formation!In
\abbreviation{IFOL}}, they tell us what counts as a well-formed
formula:\footnote{For the less logically-versed reader:
  $\top$ is read ``true'', $\bot$ is read ``false'',
  $\lnot$ is read ``not'', $\lor$ is read ``or'',
  $\land$ is read ``and'', $\implies$ is read ``implies'', $\forall$ is read
  ``for all'', and $\exists$ is read ``there exists''.
  The rules for deriving truth will justify these readings.}\addtolength{\jot}{1em}
\begin{gather*}
  \prftree[r]{}{}{\vdash\top\prop}
  \qquad
  \prftree[r]{}{}{\vdash\bot\prop}
  \qquad
  \prftree[r]{}
    {?a\prop}
    {\lnot ?a\prop} \\
  \prftree[r]{}
     {?a\prop}{?b\prop}
     {?a\implies ?b\prop}
  \qquad
  \prftree[r]{}
     {?a\prop}{?b\prop}
     {?a\land ?b\prop}
  \qquad
  \prftree[r]{}
     {?a\prop}{?b\prop}
     {?a\lor ?b\prop} \\
  \qquad
  \prftree[r]{}
     {?a\prop}{v\notin\FV(?a)}
     {\forall v.?a}
  \qquad
  \prftree[r]{}
     {?a\prop}{v\notin\FV(?a)}
     {\exists v.?a}
\end{gather*}
The hypothesis $v\notin\FV(?a)$ involves another kind of judgment: that $v$ is
not a free variable of $?a$. Intuitively, this just means that if $v$ appears at
all in $?a$, it is already ``bound'' by appearing just after a $\forall$ or
$\exists$. Again, we won't deal with such issues with much care here.

Our final judgment is $?a\true$, and represents the truth of a formula.
The following are the \define{introduction rules}\index{Introduction rules!In
  \abbreviation{IFOL}}, they tell us how to prove formulae.
\begin{gather*}
  \prftree[r]{\footnotesize $\top$-intro}
    {}
    {\top\true}
  \qquad
  \prftree[r]{\footnotesize $\bot$-intro}
    {?a\prop}{?a\true}{\lnot ?a\true}
    {\bot\true} \\
  \prftree[r]{\footnotesize $\lor$-intro-l}
    {?a\prop}{?b\prop}{?a\true}
    {?a\lor?b\true}
  \qquad
  \prftree[r]{\footnotesize $\lor$-intro-r}
    {?a\prop}{?b\prop}{?b\true}
    {?a\lor?b\true} \\
  \prftree[r]{\footnotesize $\land$-intro}
    {?a\prop}{?b\prop}{?a\true}{?b\true}
    {?a\land?b\true} \\
  \prftree[r]{\footnotesize $\Rightarrow$-intro}
    {?a\prop}{?b\prop}{?a\true\vdash ?b\true}
    {?a\implies?b\true} \\
  \qquad
  \prftree[r]{\footnotesize $\exists$-intro}
    {?a\prop}{?a\true}{v\in\BV(a)}
    {\exists w.?a[v\coloneqq w]}
\end{gather*}
Since the introduction rule for $\forall$ involves a complex hypothetical
judgment, we leave it out here.

% To define the rules for introducing quantifiers, we must deal with the issue of
% free and bound variables. Our second judgment expresses that $v$ is a free variable
% of $?a$, denoted $v\in\FV(?a)$. For the sake of brevity, metavariables in the
% following are implicitly assumed to be propositions:
% \begin{gather*}
%   \prftree[r]{}
%      {v\in \FV(?a)}
%      {v\in \FV(?a\land ?b\prop)}
%   \qquad
%   \prftree[r]{}
%      {v\in \FV(?b)}
%      {v\in \FV(?a\land ?b\prop)} \\
%   \prftree[r]{}
%      {v\in \FV(?a)}
%      {v\in \FV(?a\lor ?b\prop)}
%   \qquad
%   \prftree[r]{}
%      {v\in \FV(?b)}
%      {v\in \FV(?a\lor ?b\prop)}
% \end{gather*}

The second (and final) judgment in \abbreviation{IFOL} is that a sentence is
true. 

\section{The $\lambda$-calculus}
\label{sec:the-lambda-calculus}

\subsection{The untyped $\lambda$-calculus and computation}
\label{subsec:the-untyped-lambda-calculus}

The Church-Turing thesis

$\beta$-reduction

\subsection{The simply-typed $\lambda$-calculus}
\label{subsec:the-simply-typed-lambda-calculus}

Two judgments

The strong normalization property

\subsection{More complex types}
\label{subsec:more-complex-types}

\subsubsection{Product types}
\label{subsubsec:product-types}

\subsubsection{Coproduct types}
\label{subsubsec:coproduct-types}

\subsubsection{Natural numbers}
\label{subsubsec:natural-numbers}

\subsubsection{Lists}
\label{subsubsec:lists}

\section{The Curry-Howard correspondence}
\label{sec:the-curry-howard-correspondence}

\cite{curry-howard}

\subsection{The BHK interpretation}
\label{subsec:the-bhk-interpretation}

\chapter{Type theory}
\label{chap:type-theory}

\section{The identity type}
\label{sec:the-identity-type}

\begin{tt-rule}\label[rule]{rule:id-elim}
  The rule of \define{identity elimination} (in the \abbreviation{HoTT}
  community, known as \define{path induction}) is as follows:\TODO{define id-elim}
\end{tt-rule}

\begin{lemma}[$\mathsf{ap}$]\label[lemma]{lemma:ap}
	\TODO{define ap}
\end{lemma}

\begin{lemma}[Transport]\label[lemma]{lemma:transport}
	\TODO{define transport}
\end{lemma}

\begin{notation}\label[notation]{notation:transport}
  We will often curry\index{Curry} \transportname. If we have a family
  $P:A\to\universe$ and a path $p:\propeq{A}{x}{y}$, we write $\transpor{P}{p}$
  for $\lam{x}\transport{P}{p}{x}$.
\end{notation}

\begin{lemma}\label[lemma]{lemma:transport-compose}
	For $f:A\to B$, $P:B\to\universe$, and $p:\propeq{A}{x}{y}$, 
  \begin{equation*}
    \propeq{}{
      \transpor{P\circ f}{p}
    }{
      \transpor{P}{\ap{f}{p}}
    }
  \end{equation*}
\end{lemma}

\section{Dependent types}
\label{sec:dependent-types}

Up to now, we have not considered the type of types. However, types are terms
like any other, and every term has a type.

\begin{definition}\label[definition]{def:universes}
  All the types we have considered up to this point are members of a universe
  \universei{0}.
\end{definition}

We now have the power to write an \textit{single} identity function that works
for \textit{all} types.

\begin{definition}\label[definition]{def:id-polymorphic}
  The identity function is
	\begin{align*}
    \id :\pit{A:\universe}{A\to A} &&
    \id \defeq \lam{A}\lam{a}a
  \end{align*}
  We will write its first argument as a subscript, as in $\id_{A}$.
\end{definition}

\begin{lemma}[Paths in $\Sigma$-types]\label[lemma]{lemma:path-sigma}
	If $x,y:\sigmat{a:A}{B(a)}$, then
  \begin{equation*}
    \weq{
      (\propeq{}{x}{y})
    }{
      \sigmat{
        p:\propeq{}{\appr{1}{x}}{\appr{1}{y}}
      }{
        \propeq{}{
          \transport{}{p}{\appr{2}{x}}
        }{
          \appr{2}{y}
        }
      }
    }.
  \end{equation*}
  \TODO{cite with page number}
\end{lemma}

\subsection{$\Pi$-types}
\label{subsec:pi-types}

\subsubsection{On the law of the excluded middle}
\label{subsec:on-the-law-of-the-excluded-middle}

\TODO{introduce the topic based on previous sections}

Consider functions with the following types:
\begin{itemize}
  \itemsep0em
  \item $\mathsf{dne}:\pit{A:\universe}{(A\to\bot\to\bot)\to A}$
  \item $\mathsf{lem}:\pit{A:\universe}{A+(A\to\bot)}$
\end{itemize}
What can we tell about these functions from their type signatures?
The term $\mathsf{dne}$ takes as argument a term $x:A\to\bot\to\bot$,
that is, a term showing that $A$ is not the empty type\footnote{More precisely,
  $x$ demonstrates that $A$ is not \textit{isomorphic} or \textit{equivalent} to
  $\bot$, in the sense of \cref{subsec:weak-equivalences}. If $A$ were isomorphic
  to $\bot$, then that isomorphism would be an inhabitant of $A\to\bot$.}, and
produces some element of $A$. This seems like a very tricky function to write:
how can we give a term of type $A$ just by knowing $A$ has terms? We don't
know what form data of type $A$ have! The function $\mathsf{lem}$ seems
similarly quagmired. Given a type $A$, this function either has to produce an
element of it, or demonstrate that it is uninhabited. 

As you may have already guessed, under the Curry-Howard correspondence, these
functions correspond to the rule of double negation elimination
($\mathsf{dne}$)\index{Double negation elimination}
and the law of excluded middle ($\mathsf{lem}$)\index{Law of excluded middle}:

\begin{minipage}[b]{0.50\linewidth}
  \centering
  \begin{displaymath}
    \prftree[r]{\footnotesize DNE}
      {P\prop}{\lnot \lnot P\true}
      {P\true}
  \end{displaymath}
\end{minipage}
\begin{minipage}[b]{0.50\linewidth}
  \centering
  \begin{displaymath}
    \prftree[r]{\footnotesize LEM}
      {P\prop}
      {P\lor\lnot P\true}
  \end{displaymath}
\end{minipage}
\vspace{-0.3em}

\noindent These are logically equivalent. To demonstrate with finality that
these are not definable terms in any consistent type theory, consider the
application of $\mathsf{lem}$ to a type $\mathsf{pnp}$ that corresponds to the
$\textsc{P}\neq\textsc{NP}$ conjecture under the types-as-propositions
interpretation. If $\mathsf{lem}$ existed, we could trivially solve this problem
and make a million dollars! We need no recourse to philosophy to justify the
constructive/intuitionistic logic of proof assistants: nothing else computes.

\section{$h$-levels and truncation}
\label{sec:$h$-levels and truncation}

\section{Univalence}
\label{sec:univalence}

\subsection{Weak equivalences}
\label{subsec:weak-equivalences}

\begin{notation}\label[notation]{notation:weq-coerce}
  We may treat a weak equivalence as if it were a function and apply it to an
  argument. This amounts to implicitly appling $\pr{1}$.
\end{notation}

\subsection{The univalence axiom}
\label{subsec:the-univalence-axiom}

\begin{definition}\label[definition]{def:ua}
  \TODO{definintion}
  It has the following computation rule:

  Note that this rule holds up to \textit{propositional equality}. This is
  because in \abbreviation{UTT}, univalence is indeed an \textit{axiom}. It has
  no computational meaning. \TODO{mention CTT}
\end{definition}

\begin{theorem}[Function extensionality]\label[theorem]{thm:funext}
	Under the hypothesis of the univalence axiom, function extensionality holds,
  i.e.\ there is a term
  \begin{equation*}\label{eq:funext}
    \funext:\pit{A,B\:\universe}{(\homot{f}{g})\to \propeq{}{f}{g}}
  \end{equation*}
\end{theorem}

\chapter{Category theory}
\label{chap:category-theory}

Category theory presents a challange to set-theoretic mathematicians: the
canonical example of a category is $\Set$, the collection of all sets. In
\abbreviation{ZFC}+\abbreviation{FOL}, this category is undefinable\footnote{In
  particular, its formation is prevented by the axiom of regularity
  \cite{vonneumann}, which was included in \abbreviation{ZFC} to avoid the
  paradoxes of Burali-Forti and Russell. The discovery of said paradoxes
  motivated Bertrand Russell to invent something he called the ``theory of
  types'' \cite{russell}.}.
\Crefrange{sec:basics}{sec:functors-and-their-algebras}, 
are willfully imprecise, we work with an abstract and undefined notion of
``collection'', and adopt set-theoretic notation. See any textbook on category
theory for information on how problems of size are dealt with set-theoretically.
We will examine category theory within \abbreviation{UTT} in
\cref{sec:type-theoretic-category-theory}.

\section{Basics}
\label{sec:basics}


\TODO{Type theory and category theory are intimately connected via the
      discipline of \textit{topos theory}}%\index{Topos theory}.}

\begin{definition}
	A \define{category}\index{Category} $\bfC$ consists of the following data:
  \begin{itemize}
    \itemsep-0.2em
    \item a collection of \define{objects}, denoted $\Obj \bfC$,
    \item for each pair of objects $A,B\in\Obj \bfC$, a collection of
      \define{arrows} (or \define{morphisms}) between them, denoted
        $\Hom_{\bfC}(A,B)$,
    \item for each object $A\in\Obj \bfC$, a distinguished arrow
      $\id_A\in\Hom_{\bfC}(A,A)$ called the
      \define{identity}\index{Identity!Morphism}, and
    \item for each triple of objects $A,B,C\in\Obj\bfC$, an operation \\
      $\circ:\Hom_{\bfC}(B,C)\times\Hom_{\bfC}(A,B)\to\Hom_{\bfC}(A,C)$ called
      \define{composition}\index{Composition!In a category}.
  \end{itemize}
  These data are subject to the following axioms:
  \begin{enumerate}%[label=\Alph*.]
    \itemsep-0.2em
    \item composition is associative, and
    \item the identity acts as a unit for composition.
  \end{enumerate}
  When the category in question is clear from context, one writes $f:A\to B$ for
  $f\in\Hom_{\bfC}(A,B)$. 
\end{definition}

\begin{definition}\label[definition]{def:domain-and-codomain}
  If $f\in\Hom_{\bfC}(A,B)$, then $A$ is the \define{domain}\index{Domain} or
  \define{source}\index{Source} of $f$ and $B$ is the
  \define{codomain}\index{Codomain} or \define{target}\index{Target} of $f$.
\end{definition}

\begin{example}
  A student of mathematics will be familiar with the following categories:
  \begin{itemize}
    \itemsep-0.2em
    \item $\Set$: The category with sets as objects, functions as morphisms, 
      the usual composition of functions, and identity functions.
    \item $\Grp$: The category of groups\index{Group} with group homomorphisms
      as morphisms. Note that the identity function of sets is the required
      identity morphism and that for any homomorphsisms $\phi:G\to H$ and
      $\psi:H\to I$, the usual composition of functions defines a homomorphism
      $\psi\circ \phi:G\to I$.
    \item $\AbGrp$: The category of abelian groups (this can be considered a
      \define{subcategory} of $\Grp$).
    \item $F\dVect$: The category of vector spaces\index{Vector space} over a
      field $F$ with linear transformations as morphisms.
    \item For any group $(G,\circ,e)$, there is a corresponding category
      $\underline{G}$ with a single object (denoted $\ast$) and
      $\Hom_{\underline{G}}(\ast,\ast)\coloneqq G$. Composition is given by the
      group operation.
  \end{itemize}
\end{example}

\begin{definition}\label[definition]{def:isomorphism}
	An \define{isomorphism}\index{Isomorphism!In a category} in $\bfC$ is an arrow
  $f\in\Hom_{\bfC}(A,B)$ such that there exists an arrow $g\in\Hom_{\bfC}(B,A)$
  such that $g\circ f=\id_A$ and $f\circ g=\id_B$. We call such a $g$ the
  \define{inverse}\index{Inverse!In a category} of $f$.
\end{definition}

Just as with functions, inverses (should they exist) are unique.

\begin{figure}[ht]
  \centering
  \begin{tikzpicture}[scale=2]
    \node (A) [] {$\ast$};

    \foreach \x in {1,2,3} 
      \draw (A) edge [in={50*\x},out={40+50*\x},loop] node[above] {$g_\x$} (A);

    \draw (A) edge [in=200,out=240,loop] node[below] {$\cdots$} (A);
    \draw (A) edge [in=-60,out=-20,loop] node[below] {$e$} (A);
  \end{tikzpicture}
  \caption{
    \label{fig:grp} A schematic of the category $\underline{G}$ for a
    group $G$ with identity $e$ and elements $g_1,g_2,\ldots$. Compositions and
    inverses not shown.
  }
\end{figure}

\begin{definition}\label[definition]{def:commutative-diagram}
  A commutative diagram is a way to visualize equations between arrows
  involving composition. Technically a diagram in $\bfC$ is a directed graph
  with vertices labeled by $\Obj\bfC$. If $e$ is an edge from $A$ to $B$, then it
  is labeled by an arrow in $\Hom_\bfC(A,B)$. A diagram \define{commutes}, or is
  commutative, if the composition of the arrows labeling the edges of any two
  directed paths with the same endpoints are equal.
  \TODO{better wording}
\end{definition}

\begin{example}\label[example]{ex:commutative-diagram}
	If $f\in\Hom_{\bfC}(A,B)$, $g\in\Hom_{\bfC}(B,C)$, $h\in\Hom_{\bfC}(A,C)$,
  and $g\circ f=h$, then we may draw the following
  \define{commutative triangle}:
  \begin{center}
    \begin{tikzcd}[column sep=large]
      A \arrow[r, "f"] \arrow[dr, "h"] & B \arrow[d, "g"] \\
      {}                               & C
    \end{tikzcd}
  \end{center}
	If $f\in\Hom_{\bfC}(A,B)$, $g\in\Hom_{\bfC}(B,D)$, $h\in\Hom_{\bfC}(A,C)$,
  $i\in\Hom_{\bfC}(C,D)$, and $g\circ f=i\circ h$, then we may draw the
  following \define{commutative square}:
  \begin{center}
    \begin{tikzcd}[column sep=large]
      A \arrow[r, "f"]  \arrow[d, "h"] & B \arrow[d, "g"] \\
      C \arrow[r, "i"]                 & D
    \end{tikzcd}
  \end{center}
\end{example}

\section{Duality}
\label{sec:duality}

From now on, dual concepts and statements will be introduced in pairs, and
typeset like so:

\dual{
  Concept
}{
  Co-concept
}

\section{Limits and colimits}
\label{sec:limits-and-colimits}

For this section, fix an arbitrary category $\bfC$.\TODO{elsewhere?}

\begin{definition}\label[definition]{def:terminal-and-initial}
  \
  \vspace{-0.3em}\dual{
    A \define{terminal object}\index{Terminal object} is an
    object $\top\in\Obj\bfC$ such that for all $A\in\Obj\bfC$,
    there is exactly one arrow $f:A\to\top$.
  }{
    An \define{initial object}\index{Initial object} is an
    object $\bot\in\Obj\bfC$ such that for all $A\in\Obj\bfC$,
    there is exactly one arrow $f:\bot\to A$.
  }
\end{definition}

\begin{remark}\label[remark]{remark:terminal-object-id-unique}
	In particular, for a terminal object $\top$, $\id_\top$ is the only arrow
  $\top\to\top$.\footnote{As noted in \cref{sec:duality}, this statement
    holds for initial objects as well. From this point on, we will leave it
    to the reader to construct the dual of a statement and infer its truth.}
\end{remark}

\begin{lemma}\label[lemma]{lemma:terminal-unique}
  Terminal objects are unique up to\footnote{For the non-mathematical reader:
    When a mathematician says ``X is true up to $R$'' for some (equivalence)
    relation $R$, it means that all objects related by $R$ are considered
    equivalent. In this case, ``unique up to isomorphism'' means that there may
    be other terminal objects, but all of them are mutually isomorphic.} a
  specified isomorphism.
\end{lemma}
\begin{proof}
	Suppose $A$ and $B$ are terminal objects.
  There are unique arrows $f:A\to B$ and $g:B\to A$.
  Then $g\circ f:A\to A$ and $f\circ g:B\to B$, but as per
  \cref{remark:terminal-object-id-unique}\TODO{how to make this ``remark''?},
  $g\circ f=\id_A$ and $f\circ g=\id_B$.
\end{proof}

\begin{definition}\label[definition]{def:product-and-coproduct}
  Given two objects $A,B\in\Obj\bfC$,
  \vspace{-0.3em}\dual{
    a \define{product}\index{Product!In a category} of $A$ and $B$
    consists of an object $C\in\Obj\bfC$ together with arrows $p_1:C\to A$ and
    $p_2:C\to B$ satisfying the following universal property:

    For any other ``candidate product'' $D\in\Obj\bfC$ with arrows
    $q_1:D\to A$ and $q_2:D\to B$, there is a unique arrow $u:D\to C$ making the
    following diagram commute:
    \begin{center}
      \begin{tikzcd}[sep=large,ampersand replacement=\&]
        {} \& D\arrow[dl,swap,"q_1"]\arrow[dr,"q_2"]
              \arrow[d,dashed,"u"] \& {} \\
        A \& C \arrow[l, "p_1"]\arrow[r,swap, "p_2"] \& B
      \end{tikzcd}
    \end{center}
    If $C$ is a product of $A$ and $B$, we will denote it by $A\times B$, and
    the unque arrow $u$ as $\langle f,g \rangle$.
  }{
    a \define{coproduct}\index{Coproduct!In a category} of $A$ and $B$
    consists of an object $C\in\Obj\bfC$ together with arrows $i_1:A\to C$ and
    $i_2:B\to C$ satisfying the following universal property:

    For any other ``candidate coproduct'' $D\in\Obj\bfC$ with arrows
    $j_1:A\to D$ and $j_2:B\to D$, there is a unique arrow $u:C\to D$ making the
    following diagram commute:
    \begin{center}
      \begin{tikzcd}[sep=large,ampersand replacement=\&]
        A \arrow[r, "i_1"{name=I1}]\arrow[dr, swap, "j_1"{name=F}]
        \& C \arrow[d,dashed,"u"]
        \& B\arrow[l, "i_2"{name=I2},swap]\arrow[dl,"j_2"{name=G}] \\
        {} \& D \& {}
      \end{tikzcd}
    \end{center}
    If $C$ is a coproduct of $A$ and $B$, we will denote it by $A+B$, and
    the unque arrow $u$ as $[f,g]$.
  }
\end{definition}

\section{Functors and their algebras}
\label{sec:functors-and-their-algebras}

\begin{definition}\label[definition]{def:functor}
	A \define{functor}\index{Functor} $F$ between categories $\bfC$ and $\bfD$
  consists of the following data:
  \begin{itemize}
    \itemsep-0.2em
    \item a map $F_0:\Obj \bfC\to\Obj \bfD$ 
    \item for each pair of objects $A,B\in\Obj C$, a map \\
      $F_1:\Hom_{\bfC}(A,B) \to\Hom_{\bfD}(F_0(A),F_0(B))$ 
  \end{itemize}
  These data are subject to the following axioms:
  \begin{enumerate}%[label=\Alph*.]
    \itemsep-0.2em
    \item functors preserve composition
    \item $F(\id_A)=\id_{F(A)}$ for all $A\in\Obj C$.
  \end{enumerate}
\end{definition}

We generally leave off the subscripts and parentheses when possible, denoting
the application by simply $FA$ or $Ff$. A functor $F$ from $\bfC$ to
$\bfD$ may be denoted $F:\bfC\to\bfD$. We may define functors without names
using the following notation:
\begin{align*}
  \bfC &\longrightarrow \bfD \\
  A    &\longmapsto_0 \ldots \\
  f    &\longmapsto_1 \ldots
\end{align*}

\begin{example}\label[example]{ex:identity-functor-cat}
  For any category $\bfC$, there is an \define{identity
  functor}\index{Identity!Functor} $\id_{\bfC}\bfC\to\bfC$ which acts as the
  identity on objects and morphisms. The composition of functors is associative,
  and there is a category $\Cat$ of ``small'' categories (it doesn't include
  itself, for instance).
\end{example}

\begin{example}\label[example]{ex:forget}
  For each category of algebraic objects where the morphisms are the
  corresponding type of homomorphism, there is a \define{forgetful functor},
  generally denoted $U$, which takes sets with some structure to their
  underlying sets and homomorphisms to the corresponding maps of sets. For
  instance, there is a forgetful functor $U:\Grp\to\Set$.
\end{example}

\begin{example}\label[example]{ex:coproduct-functoriality}
	If $\bfC$ has distinguished binary coproducts, then 
  for any fixed $A,B\in\Obj\bfC$, one can define the following
  functors:\footnote{Really, this is a consequence of the fact that $+$ is
    what's called a \define{bifunctor}\indeX{Bifunctor}, but we don't need the
    full power of that statement for our purposes.}
  \begin{align*}
    \begin{split}
      \bfC &\longrightarrow \bfC \\
      X    &\longmapsto_0 A + X \\
      f    &\longmapsto_1 \id_A+f = [i_1, i_2\circ f]
    \end{split}
    \begin{split}
      \bfC &\longrightarrow \bfC \\
      Y    &\longmapsto_0 Y + B \\
      g    &\longmapsto_1 g+\id_B = [i_1\circ g, i_2]
    \end{split}
  \end{align*}
\end{example}

Since functors preserve sources, targets, and composition, they preserve
commutative diagrams. If $f,g,h$ form a commutative triangle in $\bfC$, then
their images under $F:\bfC\to\bfD$ do in $\bfD$:
\begin{center}
  \begin{tikzcd}[column sep=large]
    A\arrow[dr, "f"]\arrow[dd, "h"]\arrow[rr,mapsto,"F"] & {} & F A
    \arrow[dr, "F f"] \arrow[dd, near start, "F h", crossing over] & {} \\
    {} & B\arrow[dl, "g"]\arrow[rr,near start, mapsto, "F", crossing over] & {}
    & F B \arrow[dl, "F g"]\\
    C \arrow[rr,mapsto,"F"]& {} & F C &  {}
  \end{tikzcd}
\end{center}
Note however that it is possible that $FA=FB=FC$, as in a functor to a category
with a single object. However, the equalties between composites still hold.
One consequence is that functors preserve isomorphism.

\begin{definition}\label[definition]{def:endofunctor}
	An \define{endofunctor}\index{Functor!Endofunctor} is a functor with identical
  domain and codomain.
\end{definition}

\begin{definition}\label[definition]{def:f-coalgebra}
  \
  \vspace{-0.3em}\dual{
    An \define{algebra}\index{(Co)algebra for a functor} for an endofunctor
    $F:\bfC\to\bfC$ (also called an $F$\define{-algebra}) is a pair
    $(A,\alpha)$ of an object $A\in\Obj\bfC$ and an arrow $\alpha:FA\to A$.
  }{
    A \define{coalgebra} for an endofunctor
    $F:\bfC\to\bfC$ (also called an $F$\define{-coalgebra}) is a pair
    $(A,\alpha)$ of an object $A\in\Obj\bfC$ and an arrow $\alpha:A\to FA$
    \cite{category-theory-for-computing-science}.
  }
\end{definition}

For now, we will concentrate on properties of functor coalgebras, though all
constructions in this section hold dually for algebras.

\begin{definition}\label[definition]{def:coalgebra-morphism}
  A \define{coalgebra morphism}\index{(Co)algebra morphism} from
  $(A,\alpha)$ to $(B,\beta)$ is an arrow $f:A\to B$ such that the following
  diagram commutes:
  \begin{center}
    \begin{tikzcd}
      A  \arrow[d, "\alpha"] \arrow[r, "f"] & B \arrow[d, "\beta"] \\
      FA \arrow[r, "Ff"] & FB
    \end{tikzcd}
  \end{center}
\end{definition}

Since $F$ is a functor, the composition of coalgebra morphisms is again a
coalgebra morphism. In fact, $F$-coalgebras have all of the structure of a
category, which we will call $F\Coalg$.\TODO{Proof?}

\begin{definition}
  \
  \vspace{-0.3em}\dual{
    An \define{initial $F$-algebra}\index{Initial algebra} is an initial
    object of $F\Alg$.
  }{
    An \define{final $F$-coalgebra}\index{Final coalgebra} is a terminal
    object of $F\Coalg$.
  }
\end{definition}

The following example is crucial to
\cref{chap:coinductive-types-in-univalent-type-theory}.

\begin{example}\label[example]{ex:nat}
  Let $\bfC$ be a category with distinguished binary coproducts and a terminal
  object $1$. Consider the functor
  \begin{align*}
    F:\bfC &\longrightarrow \bfC  \\
    A &\longmapsto_0 1+A \\
    f &\longmapsto_1 \id_1+f
  \end{align*}
  where 1 is a terminal object\index{Terminal object} and $+$ is the coproduct
  bifunctor as in \cref{ex:coproduct-functoriality}. Let's examine
  what it \textit{means} for some $F$-algebra $(N,\eta)$ to be initial. By
  composing with the coproduct\index{Coproduct!In category theory} injections,
  we can define
  \begin{align*}
    \begin{split}
      z &: 1 \longrightarrow N \\
      z &= \eta\circ i_1
    \end{split}
    \begin{split}
      s &: 1 \longrightarrow N \\
      s &= \eta\circ i_2
    \end{split}
  \end{align*}
  so that $\eta=[z,s]$:
  \begin{center}
    \begin{tikzcd}[sep=large,ampersand replacement=\&]
      1 \arrow[r, "i_1"{name=I1}]\arrow[dr, swap, "z"{name=F}]
      \& 1+N \arrow[d,"\eta"]
      \& N\arrow[l, "i_2"{name=I2},swap]\arrow[dl,"s"{name=G}] \\
      {} \& N \& {}
    \end{tikzcd}
  \end{center}
  Suppose we have another $F$-algebra $(A,\alpha)$, and we define $f,g$ by
  composition as above so that $\alpha=[f,g]$. By initiality of $(N,\eta)$,
  there is a unique arrow $u$ making the following diagram commute:
  \newcommand{\eeeeeta}{[z,s]}
  \newcommand{\aaaaalpha}{[f,g]}
  \begin{center}
    \begin{tikzcd}[sep=large]
      1+N \arrow[r, dashed, "\id_1+u"]\arrow[d, "\eeeeeta"] & 1+A\arrow[d, "\aaaaalpha"]  \\
      N \arrow[r, dashed, "u"] & A
    \end{tikzcd}
  \end{center}
  By functoriality of the coproduct (\ref{example:coproduct-functoriality}), we can
  compose along either the left- or right-hand paths in the above diagram.
  The above square states $u\circ [z,s]= [f,g]\circ (\id_1+u)$. Precomposing
  with $i_1:1\to 1+N$ yields
  \begin{align*}
    u\circ z = u\circ [z,s]\circ i_1
    &= [f,g]\circ (\id_1+u) \circ i_1 
    &&\quad\text{Above diagram} \\
    &= [f,g]\circ [i_1,i_2\circ u] \circ i_1 
    &&\quad\text{Definition of }+ \\
    &= [f, g]\circ i_1 \\
    &= f
  \end{align*}
  By similar reasoning, precomposing with $i_2$ yields
  \begin{equation*}
    u\circ s = u\circ [z,s] \circ i_2
    = [f,g]\circ(\id_1+u)\circ i_2
    = g\circ u.
  \end{equation*}
  Combining the above two equations, we have the following universal property
  for $(N,\eta)$. For any object $A$ with arrows $f:1\to A$ and $g:A\to A$,
  there is a unique arrow $u:N\to A$ making the following diagram commute:
  \begin{center}
    \begin{tikzcd}[sep=large]
      1
        \arrow[r, "z"]
        \arrow[dr, swap, "f"]
        & N \arrow[r, "s"] \arrow[d, "u", dashed]
        & N \arrow[d, "u", dashed] \\
      {}
      & A \arrow[r, "g"]
      & A
    \end{tikzcd}
  \end{center}
  Well, we've successfully rephrased the property of initiality, but it seems
  just as cryptic now as it was then. Let's see if we can figure out what an
  type with this property would look like in our favorite category \universe, the
  universe of small types in \abbreviation{ITT}!

  Let \List{\N} be the type of lists of natural numbers
  as in \cref{subsubsec:lists}. To utilize the above property, we need
  to choose $A$, $f$, and $g$. Pick the following:
  \begin{itemize}
    \itemsep0em
    \item $A:\equiv \List{\N}$
    \item $f:\equiv \lam{x}{\nil}$
    \item $g:\equiv \lam{l}{\cons(5,l)}$
  \end{itemize}
  From the universal property of $(N,[z,s])$, we get a function
  $u:N\to \List{A}$ such that
  \begin{align*}
    u(z) = \nil
    &&\text{and}&&
    u(s(n)) = \cons(5, u(n))
  \end{align*}
  Look familiar yet? Indeed, one type with such a property is $\N$! In that
  case, $u$ is the function that, when given a number $n$, outputs a list of $5$s
  with length $n$.\footnote{In $\Set$, $\N$ would be an initial algebra.
    It makes sense to ask if this functor has an initial algebra in any
    category with a terminal object and chosen binary coproducts, and in
    general, such algebras are called \define{natural number objects} (NNOs) 
    \cite{sketches}.
  }
\end{example}

\section{Type theoretic category theory}
\label{sec:type-theoretic-category-theory}

Unfortunately, terminology varies between the three predominant sources
on category theory in univalent type theory \cite{book} \cite{unimath}
\cite{hott-lib}. 

\chapter{Coinductive types in univalent type theory}
\label{chap:coinductive-types-in-univalent-type-theory}

\begin{notation}
  Following \cite{non-wellfounded}, we will adopt the notation
  $(X,\alpha)\Rightarrow (Y,\beta)$ or $X\Rightarrow Y$ to denote the type of
  coalgebra morphisms from $(X,\alpha)$ to $(Y,\beta)$.
\end{notation}

\begin{definition}\label[definition]{def:polynomial-functor}
  Given a signature $S\defeq (A,B)$, its associated \define{polynomial functor}
  is the function
  \begin{gather*}
    P:\universe\to \universe \\
    \apply{P_{A,B}}{X}\defeq\sigmat{a:A}{B(a)\to X}
  \end{gather*}
  We will regularly leave off the subscript for $P$.
  The \define{action} of $P$ on functions is
  \begin{gather*}
    P^* : (X\to Y)\to P_{A,B}(X)\to P_{A,B}(Y) \\
    \apply{P^*f}{(a,g)}\defeq (a,g\circ f).
  \end{gather*}
  \TODO{this isn't a functor}
\end{definition}

Fix a signature $S\defeq (A,B)$ and let $P$ be the associated polynomial functor.
We will prove a few auxiliary lemmas on the way to the following result.
This proof appeared in the \Agda{} formalization of \cite{non-wellfounded}, this
is its first appearance ``de-formalized''.

\begin{lemma}\label[lemma]{lemma:final-colagebra-unique}
  Any two final $P$-coalgebras are equal. In other words, the following type is
  a proposition:
  \begin{equation*}
    \Final(S)\defeq
    \sigmat{(X,\alpha):\Coalgtype(S)}{
      \pit{(Y,\beta):\Coalgtype(S)}{\isContr((Y,\beta)\Rightarrow (X,\alpha))}
    }
  \end{equation*}
\end{lemma}

First, we'll show that their carriers are equivalent:

\begin{lemma}\label[lemma]{lemma:algebra-iso-equiv}
	If $(X,\alpha)$ and $(Y,\beta)$ are final $P$-coalgebras,
  then the first projections of the unique coalgebra morphisms
  $f:X\Rightarrow Y$ and $g:Y\Rightarrow X$ induce
  an equivalence of types $\weq{X}{Y}$.
\end{lemma}
\begin{proof}
  This proof is a standard categorical technique, much reminiscent of
  \cref{lemma:terminal-unique}.
	\TODO{proof}
\end{proof}

We can then invoke the characterization of paths in $\Sigma$-types,
\cref{lemma:path-sigma}. We'll need to demonstrate that the coalgebra map
$\alpha:X\to FX$ is equal to $\beta:Y\to FY$ when transported along the path
constructed in \cref{lemma:algebra-iso-equiv}.

\begin{lemma}\label[lemma]{lemma:polynomial-functor-transport}
  For all $X,Y:\universe$, $F:\universe\to\universe$,
  $f:X\to FX$, $g:Y\to FY$, and $p:\propeq{}{X}{Y}$, 
  if for all $x:X$ we have
  \begin{equation*}
    \propeq{}{
      \transport{F}{p}{\apply{f}{x}}
    }{
      g({\transport{\id_{\universe}}{p}{x}})
    }
  \end{equation*}
  then $\propeq{}{\transport{Z\mapsto (Z\to FZ)}{p}{f}}{g}$.
  That is, $f$ is equal to $g$ after being transported just when
  applying $f$ and transporting the result is the same as transporting the 
  input and applying $g$.
\end{lemma}
\begin{proof}
  Using identity elimination (\cref{rule:id-elim}), it suffices to assume
  $X\jdeq Y$ and $p\jdeq\refl{X}$. Then by the definition of transport
  (\cref{lemma:transport}), our hypothesis becomes
  \begin{align*}
    &\propeq{}{
      \transport{F}{p}{\apply{f}{x}}
    }{
      g({\transport{\id_{\universe}}{p}{x}})
    } \\
    &\implies
    \propeq{}{
      \transport{F}{\refl{X}}{\apply{f}{x}}
    }{
      g({\transport{\id_{\universe}}{\refl{X}}{x}})
    } \\
    &\implies
    \propeq{}{\apply{f}{x}}{\apply{g}{x}}
  \end{align*}
  so by function extensionality $f=g$. Again by definition of transport,
  $\transport{Z\mapsto (Z\to FZ)}{\refl{X}}{f} \jdeq f=g$.
\end{proof}

To complete the proof that the transported coalgebra maps are equal, we'll need
the following auxiliary result:

\begin{lemma}\label[lemma]{lemma:polynomial-functor-transport}
  For all $X,Y:\universe$ and $p:\propeq{}{X}{Y}$, 
  \begin{equation*}
    \propeq{}{
      P^*\paren{\transpor{\id_{\universe}}{p}}
    }{
      \transpor{P}{p}
    }
  \end{equation*}
  Note that $P^*$ is applied to the function $\transpor{\id_{\universe}}{p}$
  before it gets applied to its second argument.
\end{lemma}
\begin{proof}
  Note that we're using \cref{notation:transport}. By the elimination rule for
  the identity type, (\cref{rule:id-elim}), it suffices to assume that $X\jdeq
  Y$ and that $p\jdeq\refl{X}$. Then
  \begin{align*}
    P^*\paren*{\transpor{\id_{\universe}}{p}}
    &\jdeq P^*\paren*{\transpor{\id_{\universe}}{\refl{X}}}
    &&\quad\text{Identity elim.} \\
    &\jdeq P^*\paren{\id_X}
    &&\quad\text{\Cref{lemma:transport}} \\
    &\jdeq \lam{(a,f)}{(a,f\circ \id_X)}
    &&\quad\text{\Cref{def:polynomial-functor}} \\
    &\jdeq \lam{(a,f)}{(a,f)} \\
    &\jdeq \id_{PX}
    &&\quad\text{\Cref{def:id-polymorphic}} \\
    &\jdeq \transpor{\id_{\universe}}{\refl{PX}}
    &&\quad\text{\Cref{lemma:transport}} \\
    &\jdeq \transpor{\id_{\universe}}{\ap{P}{\refl{X}}}
    &&\quad\text{\Cref{lemma:ap}} \\
    &\jdeq \transpor{P}{\refl{X}}
    &&\quad\text{\Cref{lemma:transport-compose}} \\
    &\jdeq \transpor{P}{p}
    &&\quad\text{Identity elim.}
  \end{align*}
\end{proof}

\begin{proof}[Proof of \cref{lemma:final-colagebra-unique}]
  Let:
  \begin{itemize}
    \itemsep0em
    \item $(X,\alpha)$, $(Y,\beta)$ be final $P$-coalgebras,
    \item $f:X\Rightarrow Y$ and $g:Y\Rightarrow X$ be the unique $P$-coalgebra
      morphisms between them,
    \item $q:\weq{X}{Y}$ the equivalence of types induced by $f$ and $g$
      (\cref{lemma:algebra-iso-equiv}).
  \end{itemize}
  By univalence, there is a path
    $\apply{\ua}{q}:\propeq{\universe}{X}{Y}$.\footnote{This
    is our first (but not nearly our last) crucial use of univalence. Without an
    equality, we couldn't \transportname{} $\alpha$ to $\beta$ in the next step.
    This step also demonstrates that function extensionality alone doesn't
    suffice.} 
  To demonstrate that $\propeq{}{(X,\alpha)}{(Y,\beta)}$, we invoke the
  characterization of paths in $\Sigma$-types, \cref{lemma:path-sigma}. It
  remains to show
  \begin{equation*}
    \propeq{(Y\to PY)}{\transport{Z\mapsto (Z\to PZ)}{\apply{\ua}{q}}{\alpha}}{\beta}.
  \end{equation*}
  but by \cref{lemma:polynomial-functor-transport}, it suffices to show that
  for all $x:X$,
  \begin{equation*}
    \propeq{}{
      \transport{P}{\apply{\ua}{q}}{\apply{\alpha}{x}}
    }{
      \beta(\transport{\id_\universe}{p}{x})
    }.
  \end{equation*}
  % Lemma 2 in HoTT/M-types
  First, note that
  \begin{align*}
    \transpor{P}{\apply{\ua}{q}}
    &= P^*\paren{\transpor{\id_{\universe}}{\apply{\ua}{q}}}
    &&\quad\text{\Cref{lemma:polynomial-functor-transport}} \\
    &= P^*q
    &&\quad\text{\Cref{def:ua}} \\
    &= P^*(\appr{1}{f})
    &&\quad\text{\Cref{lemma:algebra-iso-equiv}}
  \end{align*}
  The last step utilizes the idea of \textit{proof-relevant
    mathematics}. Although we define $q$ within a proof, we can (without
  cheating) refer to its definition from another proof. Also note the
  use of \cref{notation:weq-coerce}. Working from the other side of the
  equation, we can use the computational rule of univalence (\cref{def:ua}):
  \begin{align*}
    \beta(\transport{\id_\universe}{\apply{\ua}{q}}{x})
    = \beta(\apply{q}{x})
    = \beta({\appr{1}{f}}{x})
  \end{align*}
  Thus, we now want to demonstrate that
  \begin{align*}
    P^*(\appr{1}{f})(\alpha x) &=
    \transport{P}{\apply{\ua}{q}}{\apply{\alpha}{x}} \\
    &= \beta(\transport{\id_\universe}{p}{x}) \\
    &= \beta({\appr{1}{f}}{x})
  \end{align*}
  However, this is exactly the condition that $f$ is a $P$-coalgebra morphism:
  \TODO{reference definition}
  \begin{center}
    \begin{tikzcd}[column sep=large]
      X  \arrow[d, "\alpha"] \arrow[r, "\appr{1}{f}"] & Y \arrow[d, "\beta"] \\
      FX \arrow[r, "P^*(\appr{1}{f})"] & FY
    \end{tikzcd}
  \end{center}
\end{proof}

\begin{definition}\label[definition]{def:limitt}
	Given a family of types $X:\N\to\universe$ and a family of functions
  $\pi_n:X_{n+1}\to X_n$, the \define{limit} of $(X,\pi)$ is the type
  \begin{equation*}
    L(X,\pi)\defeq \sigmat{x:\pit{n:\N}{X_n}}{\pit{n:\N}\pi_nx_{n+1}=x_n}
  \end{equation*}
  We give special names to the projection maps for limits: $p \defeq \pr{1}$ and
  $\beta \defeq \pr{2}$.
\end{definition}

\begin{lemma}\label[lemma]{lemma:limitt-universal}
	There is an equivalence of types
  \begin{equation*}
    \weq{(A\to L(X,\pi))}{
      \sigmat{f:\pit{n:\N}A\to X_n}{\pit{n:\N}{\pi_n\circ f_{n+1}=f_n}}
    }
  \end{equation*}
\end{lemma}
\begin{proof}
	To give an equivalence, it suffices to give functions back and forth that
  compose to the respective identities.\TODO{reference definition}
  First, define
  \begin{gather*}
    \phi :(A\to L(X,\pi)) \longrightarrow \sigmat{f:\pit{n:\N}A\to X_n}{\pit{n:\N}{\pi_n\circ f_{n+1}=f_n}} \\
    \phi(g) \defeq
    \dpair{\lam{n}\lam{a}\appply{p}{(\apply{g}{a})}{n}}
          {\lam{n}\appply{\beta}{(\apply{g}{a})}{n})}
  \intertext{and}
    \psi : \paren*[\bigg]{\sigmat{f:\pit{n:\N}A\to X_n}{\pit{n:\N}{\pi_n\circ f_{n+1}=f_n}}}
          \to A\to L(X,\pi) \\
    \psi((f; h)) \defeq \lam{a:A} \dpair{\lam{n}\appply{f}{n}{a}}{h}
  \end{gather*}
  Their composites are judgmentally equal to the appropriate identities.
  \TODO{should I include proof?}
\end{proof}

\begin{lemma}\label[lemma]{lemma:cochains}
	If $X:\N\to\universe$ is a family of types, $\rho:\pit{n:\N}{X_n\to X_{n+1}}$ is a
  family of functions, and
  \begin{equation*}
    Z\defeq \sigmat{x:\pit{n:\N}X_n}{\pit{n:\N}x_{n+1}=\rho_n(x_n)},
  \end{equation*}
  then $\weq{Z}{X_0}$, that is, limits of cochains are entirely determined by
  the first element.
\end{lemma}
\begin{proof}
  Consider again the functor of \cref{ex:nat}, specialized to the case of
  $\bfC\defeq\universe$:
  \begin{gather*}
    G:\universe \longrightarrow \universe  \\
    G(W) \defeq 1+W
  \end{gather*}
  A $G$-algebra is given by a type $W:\universe$, a point in $W$, and a function
  $W\to W$.
\end{proof}

\section{W and M}
\label{sec:w-and-m}

\section{Internalizing M-types}
\label{sec:internalizing-m-types}

\chapter*{Conclusion}
\addcontentsline{toc}{chapter}{Conclusion}
\chaptermark{Conclusion}
\markboth{Conclusion}{Conclusion}
\setcounter{chapter}{4}
\setcounter{section}{0}


%If you feel it necessary to include an appendix, it goes here.
\appendix
\chapter{Cross-reference of names}

The following table lists lemmas taken from \cite{book}.
\begin{table}[ht]
  \centering
  \begin{tabular}{c | c}
    This thesis & The \abbreviation{HoTT} book \\ \hline
    \Cref{def:ua} & Axiom 2.10.3
  \end{tabular}
\end{table}

The following table compares the terminology used in this thesis to that in our
\Coq{} formalization (under the column \UniMath{}) and that of the \Agda{} development
of \cite{non-wellfounded} (under the column \MTypes{}).
\begin{table}[ht]
  \centering
  \begin{tabular}{c | c | c }
    This thesis & \UniMath{} & \MTypes{} \\ \hline
  \end{tabular}
\end{table}
% \chapter{The Second Appendix, for Fun}

\backmatter % backmatter makes the index and bibliography appear properly in the TOC

% if you're using bibtex, the next line forces every entry in the bibtex file to be included
% in your bibliography, regardless of whether or not you've cited it in the thesis.
\nocite{*}

%  \bibliographystyle{bsts/mla-good} % there are a variety of styles available; 
%  \bibliographystyle{plainnat}
% replace ``plainnat'' with the style of choice. You can crefer to files in the bsts or APA 
% subfolder, e.g. 
% \bibliographystyle{APA/apa-good}  % or
\bibliography{thesis}
\bibliographystyle{plain}
% Comment the above two lines and uncomment the next line to use biblatex-chicago.
%\printbibliography[heading=bibintoc]

% Finally, an index would go here... but it is also optional.
\index{$\id$|see {Identity}}
\index{$F$-algebra|see {Algebra for a functor}}
\printindex
\end{document}
