\documentclass[./thesis.tex]{subfiles}
\begin{document}
\chapter{Propositions and Types}
\label{chap:propositions-and-types}

% \epigraph{Now, it  is  the  contention of the  intuitionists […] that the basic mathematical notions […] ought to be interpreted in  such a way that the cleavage between mathematics […] and programming that we are witnessing at present disappears.}{\cite{martin-lof-constructive}}

\epigraph{The intuitionistic type theory, which I began to develop solely with the philosophical motive of clarifying the syntax and semantics of intuitionistic mathematics, may equally well be viewed as a programming language.}{\cite{martin-lof-constructive}}

The goal of this chapter is to transform the Curry-Howard correspondence from
a loose metaphor into an unavoidable identification. \Cref{sec:ipl} discusses
intuitionistic propositional logic (\IPL{}) a weaker, constructive,
proof-relevant variant of standard propositional logic. In
\cref{sec:the-lambda-calculus}, I introduce Church's λ-calculus (\LC{})
(and the simply-typed and typed versions \STLC{} and \TLC{}). Finally, in
\cref{sec:propositions-and-types}, I discuss the fundamental and
harmonious relation between these systems. A full understanding of \LC{},
\IPL{}, and their relation via Curry-Howard requires a degree of formality, as
spelled out in \cref{sec:discussing-logic}.

The vocabulary of \cref{sec:discussing-logic} is due to Martin-Löf, who
emphasized the importance of the notion of judgment (\cref{def:judgment})
\cite{martin-lof-meanings}. It may be helpful to see \cite{modal-judgment} for a
similar application of Martin-L\"of vocabulary to modal logic.

While it is in general unclear how to justify logical rules, I attempt to
provide some informal defense of \IPL{}. One approach is to use 
meaning explanations \cite{martin-lof-constructive}, where
introduction and elimination rules are justified by explaining how they match
with our intuitive ideas of what they mean.\footnote{See \cite{hott-meaning} for a
  justification of \HoTT{} using meaning explanations.}
We also explore a variant of Dummett's notions of harmony, in their ``local''
versions in \cref{subsec:proof-terms}. 

The presentation of \crefrange{sec:ipl}{sec:the-lambda-calculus} is influenced by
Philip Wadler \cite{wadler-propositions}, Frank Pfenning's lecture notes for
``Constructive Logic'', and Robert Harper's lectures on \HoTT{} \cite{harper}.

Two recent Reed graduates wrote theses on the Curry-Howard correspondence, both
taking a different approach in presentation that may complement this work,
\cite{curry-howard-reed-thesis} and \cite{process-calculi-reed-thesis}.

\begin{notation}\label[notation]{notation:parens}
  Throughout this thesis, I will omit parentheses when applying function-like
  constructions wherever this results in no ambiguity. Function application is
  left-associative. For example, for function-like constructions $f$, $g$, $h$,
  and $i$ we have
  \begin{align*}
    \apply{f}{h}=f(h) &&
    f(\apply{g}{h})=f(g(h)) && \appply{f}{(\apply{g}{i})}{h}=f(g(i))(h)
  \end{align*}
  and so on.
\end{notation}

\section{Discussing logic}
\label{sec:discussing-logic}

The observant student of logic may notice a problem in the usual definition of
implication: one usually defines $P→ Q$ as ``if whenever $P$ is true,
$Q$ is true, then $P→ Q$ is true''. It seems that in order to understand
implication, one must first understand implication! More perniciously,
philosophers and mathematicians alike have argued that Skolem's
``paradox''\footnote{(An application of) the L\"owenheim-Skolem theorem states
  that if there is a model of \ZFC, there is a countable one.
  The paradox is that the statement ``there are uncountable sets'' is (famously)
  a theorem of \ZFC, Cantor's.}
has deep and significant consequences for the philosophy of
mathematics \cite{skolem}. These misunderstandings both arise from a common
root: the failure to distinguish between object language and metalanguage. In
this section, I will attempt to mitigate some of the difficulties that arise
when discussing and defining formal systems for logic and computation.

\begin{definition}\label[definition]{def:object-meta-language}
  When discussing or defining a formal language (most commonly, a
  logico-deductive framework), we call that language the
  \define{object language}\index{object language}. Our discussion takes place in
  a \define{metalanguage}.
\end{definition}

\begin{example}
  \
  \begin{itemize}
    \itemsep0em
    \item In the statement ``\FOL{} is complete'', \FOL{}
      is the object language, and the metalanguage is English.
    \item G\"odel's second incompleteness theorem is a statement in the
      metalanguage of \FOL{}+\ZFC{}, about the object
      language of Peano arithmetic.
  \end{itemize}
\end{example}

\begin{figure}
  \centering
  \includegraphics[scale=1.1]{figures/nested-languages.pdf}
  \caption{\label{fig:nested}Nested languages, formal and informal. Outer boxes
    are meta-languages, inner boxes are object languages.}
\end{figure}

\begin{definition}\label[definition]{def:metavariable}
	A \define{metavariable}\index{Metavariable} (also called a \define{schematic
  variable}) is a variable in the metalanguage meant to stand for any expression
  of our object language.
\end{definition}

\begin{notation}
  Metavariables are formatted like so: $\mvar{a}$.
\end{notation}

See \cref{fig:structure}, \cref{ex:judgments}, and \cref{sec:ipl} for examples
of uses of metavariables.

\begin{definition}\label[definition]{def:judgment}
	A \define{judgment}\index{Judgment} is something that could be known. A
  judgment is \define{evident} if one does, in fact, know it.
  A \define{proof} is (sufficient, undeniable, etc.) evidence for a judgment.
  A \define{hypothetical judgment}\index{Judgment!Hypothetical} is one that
  is evident under the assumption that some other judgments are evident.
\end{definition}

\begin{example}\label[example]{ex:judgments}
  Here are some examples of common judgments in logic (where the metavariables
  in these examples are to be understood as ranging over expressions of some
  unspecified object language):
  \begin{itemize}
    \itemsep0em
    \item $\mvar{a}$ is a well-formed formula
    \item $\mvar{a}$ is a proposition
    \item if $\mvar{a}$ and $\mvar{b}$ are propositions, then
      $\mvar{a}\land \mvar{b}$ is a proposition (this judgment is hypothetical)
    \item $\mvar{a}$ will happen in the future
    \item $\mvar{a}$ is a program with type $\mvar{t}$
    \item $\mvar{a}$ is true
    \item the variable $v$ is free (resp.\ bound) in $\mvar{a}$
  \end{itemize}
  If one is working in a formal metalanguage, judgments may be defined
  inductively in it.
\end{example}

\begin{figure}
  \centering
  \begin{equation*}
    \overbrace{\text{I know} \overbrace{\underbrace{\mvar{A}}_{\text{Expression}}\text{ is true}}^{\text{Judgment}}}^{\text{Evident judgment}}
  \end{equation*}
  \caption{\label{fig:structure}The structure of a judgment (transcribed
    from \cite{martin-lof-meanings})}
\end{figure}

% Since this thesis will define several different formal systems, it
% will be useful to establish some uniform notation for such definitions.
% Therefore, the following definitions and notations are part of our
% meta-language, which is plain English.

\begin{notation}\label[notation]{notation:proof-tree}
  The premises and conclusions of proofs or rules of deduction are always
  judgments. A deduction of a conclusion $K$ from premises $J_1,\ldots,J_n$
  via some rule R (again: $J_1,\ldots,J_n$ and $K$ are \textit{judgments},
  phrased in plain Enlgish, not terms of an object language)\footnote{These
    could be called meta-metavariables.},
  is written
  \begin{align*}
    \prftree[r]{\footnotesize R}
      {J_1}{J_2}{\ldots}{J_n}
      {K}
    &&\text{or}&&
    \prftree[r]{\footnotesize R}
      {\prftree[r, noline]{}
        {J_1}
        {J_2}}
      {\prftree[r, noline]{}
        {J_3}
        {J_4}}
      {\ldots}{J_n}
      {K}.
  \end{align*}
  Iterating this notation, one can write long derivations as trees with the
  conclusion as their root.
\end{notation}

\begin{notation}\label[notation]{notation:sequent}
  If $K$ is a hypothetical judgment holding under assumptions $J_1,\ldots,J_n$,
  we denote this by $J_1,\ldots,J_n⊢ K$ (``$⊢$'' can be read ``entails'').
  There are two (seemingly) related notions that entailment is distinct from:
  \begin{enumerate}
    \itemsep-0.5em
    \item the deduction of $K$ from premises $J_1,\ldots,J_n$ (shown in
      \cref{notation:proof-tree}), and
    \item implication (see \cref{subsec:ipl-intro} for more discussion of this
      distinction).
  \end{enumerate}
  Capital Greek letters $Γ$, $Δ$, and $Θ$ will denote some arbitrary sequence of
  hypotheses.\footnote{Again, these are sorts of meta-metavariables.}
\end{notation}

\begin{definition}\label[definition]{def:entailment-rules}
  Our notions of entailment will always allow for \define{reflexivity},
  \define{weakening}, \define{contraction}, \define{substitution}, and
  \IPL{} will allow for \define{exchange}, which are the following
  ``meta-rules'' (where $J$, $K$, and $L$ stand for judgments):
  \begin{gatherjot}
    \prftree[r]{\footnotesize refl}{}
      {Γ,J,Δ ⊢ J}
    \qquad
    \prftree[r]{\footnotesize weak}
      {Γ ⊢ J}
      {Γ,K ⊢ J}
    \qquad
    \prftree[r]{\footnotesize contr}
      {Γ,K,Δ,K,Θ ⊢ J}
      {Γ,K,Δ,Θ ⊢ J} \\
    \prftree[r]{\footnotesize subst}
      {Γ,K,Δ ⊢ J}{Γ ⊢ K}
      {Γ,Δ ⊢ J}
    \qquad
    \prftree[r]{\footnotesize ex}
      {Γ,J,Δ,K,Θ ⊢ L}
      {Γ,K,Δ,J,Θ ⊢ L}
  \end{gatherjot}
\end{definition}

\begin{notation}\label[notation]{notation:substitution}
  The substitution of a term $t$ for all free occurrences of a variable $v$
  in an expression $e$ is denoted by $e[v\≔ t]$. I will not deal directly with the
  issues of variable capture, scoping, and substitution here, for a rigorous
  treatment see any thorough textbook on logic.
\end{notation}

\begin{sidewaystable}
  \centering
  \begin{tabular}{c | l | l | l}
    Symbol & First used & Language & Meaning \\ \hline
    $J,K,L$
      & \cref{notation:proof-tree}
      & English
      & Some arbitrary judgment \\
    $Γ,Δ,Θ$
      & \cref{notation:proof-tree}
      & English
      & Some arbitrary ordered sequence of judgments \\
    $Γ ⊢ J$
      & \cref{notation:sequent}
      & English
      & Under the hypotheses $Γ$, $J$ holds \\
    $\mvar{a},\mvar{b}$
      & \cref{def:metavariable}
      & English
      & Metavariable: a term in the object language \\
    $x\≔\mvar{a}$
      & \cref{subsec:ipl-intro}
      & English
      & ``$x$'' is a shorthand for the expression $\mvar{a}$ \\
    $\mvar{a}[\mvar{b} \≔ \mvar{c}]$
      & \cref{notation:substitution}
      & English
      & Substitute the expression $\mvar{c}$ for $\mvar{b}$ in $\mvar{a}$ \\
    $\prop{\mvar{a}}$
      & \cref{subsec:ipl-intro}
      & English
      & Judgment: $\mvar{a}$ is a proposition \\
    $\mvar{a}:\mvar{p}$
      & \cref{subsec:ipl-intro}
      & English
      & Judgment: $\mvar{a}$ is a proof of (or has type) $\mvar{p}$ \\
    $\mvar{a}\≡\mvar{b}:\mvar{p}$
      & \cref{def:jdeq-ipl}
      & English
      & Judgment: $\mvar{a}$ is equal to $\mvar{b}$ as proofs/elements of $\mvar{p}$ \\
    %%%%%%%%%%%%%%%%%%%%%%%%%%%%%%%%%%%%%%%%%%%%%%%%%%%%%%%%%%%%%%%%
    $\apply{\mvar{a}}{\mvar{b}}$
      & \cref{notation:parens}
      &  All
      & Application of $\mvar{a}$ to $\mvar{b}$ \\
    $v,w,x,y,z$
      & \cref{subsec:ipl-intro}
      & All
      & Variables (free or bound) \\
    % $c_0,c_1,\ldots$
    %   &
    %   & \(S)TLC
    %   & Constants \\
    %%%%%%%%%%%%%%%%%%%%%%%%%%%%%%%%%%%%%%%%%%%%%%%%%%%%%%%%%%%%%%%%
    $→$
      & \cref{subsec:ipl-form}
      & \IPL{}, \formalsystem{(S)TLC}, \UTT{}
      & Material implication or function type \\
    $×$
      & \cref{subsec:ipl-form}
      & \ZFC{}+\FOL{}, \TLC{}, \UTT{}
      & Cartesian or categorical product, product type \\
    $+$
      & \cref{subsec:ipl-form}
      & \TLC{}, \UTT{}
      & Coproduct type, categorical coproduct \\
    %%%%%%%%%%%%%%%%%%%%%%%%%%%%%%%%%%%%%%%%%%%%%%%%%%%%%%%%%%%%%%%%
    $\λ{x}\mvar{a}$
      & \cref{subsec:ipl-intro}
      & \IPL{}, \formalsystem{((S)T)LC}, \UTT{}
      & A function that takes an input $x$ \\
    $(a,b)$
      & \cref{subsec:ipl-intro}
      & \IPL{}, \TLC{}, \UTT{}
      & Constructor of the pair type \\
    $\inl$, $\inr$
      & \cref{subsec:ipl-intro}
      & \IPL{}, \TLC{}, \UTT{}
      & Constructor of the coproduct type \\
    %%%%%%%%%%%%%%%%%%%%%%%%%%%%%%%%%%%%%%%%%%%%%%%%%%%%%%%%%%%%%%%%
    $\pr{1}$, $\pr{2}$
      & \cref{subsec:ipl-elim}
      & \IPL{}, \TLC{}, \UTT{}
      & Destructor of the product type \\
    $\case$
      & \cref{subsec:ipl-elim}
      & \IPL{}, \TLC{}
      & Destructor of the coproduct type \\
    $\rec$
      & Various
      & \IPL{}, \TLC{}, \UTT{}
      & Recursion principle or elimination rule \\
    %%%%%%%%%%%%%%%%%%%%%%%%%%%%%%%%%%%%%%%%%%%%%%%%%%%%%%%%%%%%%%%%
    $\universe$, $\universei{0}$
      & \cref{subsec:the-universe}
      & \UTT{}
      & Universe type \\
    $\∏{\mvar{a}:\mvar{p}}{\apply{\mvar{b}}{\mvar{a}}}$
      & \cref{subsec:pi-types}
      & \UTT{}
      & Π or dependent function type \\
    $\∑{\mvar{a}:\mvar{p}}{\apply{\mvar{b}}{\mvar{a}}}$
      & \cref{subsec:sigma-types}
      & \UTT{}
      & Σ or dependent pair type \\
    %%%%%%%%%%%%%%%%%%%%%%%%%%%%%%%%%%%%%%%%%%%%%%%%%%%%%%%%%%%%%%%%
    $\catarrow$, $\Rightarrow$
      & \cref{def:category}
      & \UTT{}
      & The type $\appply{\Hom}{A}{B}$ of arrows from $A$ to $B$ \\
  \end{tabular}
  \caption{\label{tab:symbols}Symbols and their interpretations}
\end{sidewaystable}

\section{Intuitionistic propositional logic}
\label{sec:ipl}

Intuitionistic propositional logic is a variant of classical sentential logic
with some rules omitted due to philosophical (or, for computer scientists,
pragmatic) concerns. As discussed in the introduction, it is commonly understood
as both a theory of truth, and as a theory of ``problems'' and their ``solutions''.

\Crefrange{sec:ipl}{sec:the-lambda-calculus} present these
systems in some degree of formality. Proofs in later chapters are not written in
this style, but it is important to see precisely specified versions of these
logics in order to understand \cref{sec:propositions-and-types}.

\subsection{Formation rules}
\label{subsec:ipl-form}

Since \IPL{} is supposed to be a logic, it better have a notion of a
proposition.\footnote{In the intuitionistically-motivated style of Robert
  Harper, I avoid specifying the notion of ``well-formed formulae'', opting
  instead for a more general partial specification of ``propositions''.
  In this same spirit, I refrain from specifying our syntax via formal BNF (or
  similar) grammars.}
Indeed, we denote the judgment that some term $\mvar{a}$ represents a
proposition by $\prop{\mvar{a}}$. The following are the \define{formation
rules}\index{Formation!In \IFOL{}}:\footnote{For the less
  logically-versed reader: $⊤$ is read ``true'', $⊥$ is read ``false'',
  $\lnot$ is read ``not'', $\lor$ is read ``or'',
  $\land$ is read ``and'', $→$ is read ``implies'', $∀$ is read
  ``for all'', and $∃$ is read ``there exists''.
  \Cref{subsec:ipl-intro} attempts to justify these readings.}
\begin{gatherjot}
  \prftree[r]{}{}{Γ⊢\prop{⊤}}
  \qquad
  \prftree[r]{}{}{Γ⊢\prop{⊥}}
  \qquad
  \prftree[r]{}
    {\prop{Γ⊢\mvar{a}}}
    {\prop{Γ⊢\lnot \mvar{a}}} \\
  \prftree[r]{}
    {\prftree[r, noline]{}
      {Γ ⊢ \prop{\mvar{a}}}
      {Γ ⊢ \prop{\mvar{b}}}}
    {Γ⊢\prop{\mvar{a}→\mvar{b}}}
  \qquad
  \prftree[r]{}
    {\prftree[r, noline]{}
      {Γ ⊢ \prop{\mvar{a}}}
      {Γ ⊢ \prop{\mvar{b}}}}
    {Γ⊢\prop{\mvar{a}\land \mvar{b}}}
  \qquad
  \prftree[r]{}
    {\prftree[r, noline]{}
      {Γ ⊢ \prop{\mvar{a}}}
      {Γ ⊢ \prop{\mvar{b}}}}
    {Γ⊢\prop{\mvar{a}\lor\mvar{b}}}
\end{gatherjot}
We add parentheses where necessary to disambiguate compound expressions.
As noted in \cref{tab:symbols}, ``$Γ$'', ``$⊢$'', and the horizontal line
``$\frac{\hspace{1em}}{\hspace{1em}}$'' are symbols in our metalanguage
(English), whereas ``$⊤,⊥,¬,→,∧,∨$'' are symbols in the object language
(\IPL{}). See \cref{tab:incoherent} for examples of some valid
expressions we could form with the above rules. Such valid expressions (in the
left column) are things that could be substituted in for our metavariables like
$\mvar{a}$.

\begin{table}[ht]
  \centering
  \begin{tabular}{l | l}
    \IPL{}                      & Incoherent \\ \hline
    $(\mvar{a}∧\mvar{b})∨\mvar{c}$          & $(Γ ⊢ \prop{\mvar{a}})∨\mvar{b}$ \\
    $(\mvar{a}→\mvar{b})→\mvar{c}$          & $∧∨→$ \\
    $⊤∨\mvar{a}_1∨\mvar{a}_2∨⋯∨\mvar{a}_n$  & $Γ⊢Θ$ \\
    $⊤→⊥$                                   & $\prop{\mvar{a}}→\mvar{b}∧\mvar{c}$ \\
  \end{tabular}
  \caption{\label{tab:incoherent}Valid \IPL{} expressions and incoherent
    nonsense.}
\end{table}

\subsection{Introduction rules}
\label{subsec:ipl-intro}

Martin-L\"of said ``The meaning of a proposition is determined by [...] what
counts as a verification of it'' \cite{martin-lof-meanings}. To
understand the meanings of the formation rules, we must define the judgment
``$\mvar{p}:\mvar{a}$'', read ``$\mvar{p}$ is a proof of $\mvar{a}$''. The
ways of proving a proposition are called \define{introduction rules}.

The symbol $⊤$ represents the trivially true proposition. Accordingly, we
give it a trivial proof:
\begin{equation*}
  \prftree[r]{}{}{Γ ⊢ \unitelem:⊤}.
\end{equation*}

How do we know a conjunction? Intuitively, we know (have a proof of) $\mvar{a}$
and $\mvar{b}$ (written $\mvar{a}\land\mvar{b}$) just when we know (have a proof
of) both $\mvar{a}$ and $\mvar{b}$. In symbols,\footnote{Note that we're
  currently defining the notation ``$(,)$'', as we defined ``$\star$'' above and
  will define ``$\inl$'' below. These aren't imported from (though they are
  inspired by) some other area of math.}
\begin{equation*}
  \prftree[r]{}
    {Γ ⊢ \prop{\mvar{a}}}{Γ ⊢ \prop{\mvar{b}}}
    {Γ ⊢ \mvar{p}:\mvar{a}}{Γ ⊢ \mvar{q}:\mvar{b}}
    {Γ ⊢ (\mvar{p},\mvar{q}):\mvar{a}\land\mvar{b}}.
\end{equation*}

What about disjunction? There are two ways to know $\mvar{a}$ or
$\mvar{b}$ (written $\mvar{a}\lor\mvar{b}$). We can either know $\mvar{a}$ or
know $\mvar{b}$. Correspondingly, we have two introduction rules:
\begin{align*}
  \prftree[r]{}
    {\prftree[r, noline]{}
      {Γ ⊢ \prop{\mvar{a}}}
      {Γ ⊢ \prop{\mvar{b}}}}
    {Γ ⊢ \mvar{p}:\mvar{a}}
    {Γ ⊢ \apply{\inl}{\mvar{p}}:\mvar{a}\lor\mvar{b}}
  &&\text{and}&&
  \prftree[r]{}
    {\prftree[r, noline]{}
      {Γ ⊢ \prop{\mvar{a}}}
      {Γ ⊢ \prop{\mvar{b}}}}
    {Γ ⊢ \mvar{p}:\mvar{b}}
    {Γ ⊢ \apply{\inr}{\mvar{p}}:\mvar{a}\lor\mvar{b}}.
\end{align*}
The symbol $⊥$ represents falsehood. A logic is
\define{consistent}\index{Consistency} if it cannot prove falsehood (a highly
desirable property!). To this end, there is no introduction rule for $⊥$.

What does it mean to know a negation? We'll define
$\lnot\mvar{a}\defeq \mvar{a}→⊥$ and instead worry about when we
know an implication.\footnote{This definition suggests the following ``derived
  introduction rule'' for $⊥$:
  \begin{equation*}
    \prftree[r]{}
      {Γ ⊢ \prop{\mvar{a}}}{Γ ⊢ \mvar{p}:\mvar{a}}{Γ ⊢ \mvar{q}:\lnot\mvar{a}}
      {Γ ⊢ \ttfun{whoops!}(\mvar{p},\mvar{q}):⊥}.
  \end{equation*}
  The derivation of this rule from the elimination rule for $→$
  (\cref{subsec:ipl-elim}) is immediate.}
An implication is true just when we have a hypothetical judgment where the
hypothesis is the antecedent and the conclusion is the consequent. Implication
allows us to \textit{internalize}\index{Internalizing!Hypothetical judgments}
the meta-theoretical notion of hypothetical judgments by turning them into
non-hypothetical ones. To define the introduction rule for $→$, we
assume the presence of a countably-infinite set of \define{variables}
$v,w,\ldots$
\begin{equation*}
  \prftree[r]{}
    {Γ ⊢ \prop{\mvar{a}}}{Γ ⊢ \prop{\mvar{b}}}
    {Γ,\mvar{p}:\mvar{a},Δ⊢ \mvar{q}:\mvar{b}}
    {Γ,Δ⊢\λ{v}{\mvar{q}[\mvar{p}\≔ v]}:\mvar{a}→\mvar{b}}.
\end{equation*}
(Generally, $\mvar{q}$ will be some expression containing $\mvar{p}$ so that the
substitution $\mvar{q}[\mvar{p}\≔ v]$ is different from $\mvar{q}$.)

\begin{definition}\label[definition]{def:valid-theorem}
  If know that under no hypotheses $\mvar{a}$ is a proposition and has a proof,
  then we may judge that $\mvar{a}$ is \define{valid}\index{Valid!In
  \IPL{}}:
  \begin{equation*}
    \prftree[r]{}
      {⊢\prop{\mvar{a}}}{⊢\mvar{p}:\mvar{a}}
      {⊢\valid{\mvar{a}}}.
  \end{equation*}
  In this case, $\mvar{a}$ is a \define{theorem}\index{Theorem!Of \IPL{}} of
  \IPL{}.
\end{definition}

\subsection{Elimination rules}
\label{subsec:ipl-elim}

Once we know a judgment, how do we use it in a derivation?
We need \define{elimination rules}. How do we know what they should be?
In a sense to be made precise in \crefrange{subsec:proof-terms}{subsec:ipl-uni},
they should be dual to our introduction rules: we should be able to extract the
same amount of information from a proposition that went into its proof.

For conjunction, we should be able to recover proofs of both conjuncts:
\begin{align*}
  \prftree[r]{}
    {\prftree[r, noline]{}
      {Γ ⊢ \prop{\mvar{a}}}
      {Γ ⊢ \prop{\mvar{b}}}}
    {Γ ⊢ \mvar{p}:\mvar{a}\land\mvar{b}}
    {Γ ⊢ \appr{1}\mvar{p}:\mvar{a}}
  &&\text{and}&&
  \prftree[r]{}
    {\prftree[r, noline]{}
      {Γ ⊢ \prop{\mvar{a}}}
      {Γ ⊢ \prop{\mvar{b}}}}
    {Γ ⊢ \mvar{p}:\mvar{a}\land\mvar{b}}
    {Γ ⊢ \appr{2}\mvar{p}:\mvar{b}}.
\end{align*}
If both disjuncts imply a hypothesis, so does their disjunction:
\begin{align*}
  \prftree[r]{}
    {\prftree[r, noline]{}
      {\prftree[r, noline]{}
        {Γ ⊢ \prop{\mvar{a}}}
        {Γ ⊢ \prop{\mvar{b}}}}
      {Γ ⊢ \prop{\mvar{c}}}}
    {\prftree[r, noline]{}
      {Γ ⊢ \mvar{p}:\mvar{a}→\mvar{c}}
      {Γ ⊢ \mvar{q}:\mvar{b}→\mvar{c}}}
    {Γ ⊢ \mvar{r}:\mvar{a}\lor\mvar{b}}
    {Γ ⊢ \apppply{\case}{\mvar{p}}{\mvar{q}}{\mvar{r}}:\mvar{c}}.
\end{align*}
The elimination rule for implication has the common name
\textit{modus ponens}\index{Modus ponens}, and expresses the idea that, if we
had a hypothetical judgment and then come to know all of the hypotheses, we can
deduce the consequence:
\begin{align*}
  \prftree[r]{}
    {Γ ⊢ \prop{\mvar{a}}}{Γ ⊢ \prop{\mvar{b}}}
    {Γ ⊢ \mvar{p}:\mvar{a}→ \mvar{b}}
    {Γ ⊢ \mvar{q}:\mvar{a}}
    {Γ ⊢ \apply{\mvar{p}}{\mvar{q}}:\mvar{b}}.
\end{align*}
The elimination rule for falsehood comes from the principle
\textit{ex falso quod libet}, ``from falsehood anything follows'' (also more
dramatically called the ``principle of explosion''). If we've been
able to prove $⊥$, our whole logic is bankrupt (or we're doing a proof by
contradiction), we can derive anything we please:
\begin{align*}
  \prftree[r]{}
    {Γ ⊢ \prop{\mvar{a}}}
    {Γ ⊢ \mvar{p}:⊥}
    {Γ ⊢ \apply{\rec_{⊥}}{\mvar{p}}:\mvar{a}}.
\end{align*}

At this point, we'll begin to see some full proofs. Since these quickly become
unmanageably large, we'll implicitly hypothesize that all metavariables involved
represent propositions.

\begin{example}\label[example]{ex:ipl-and-comm}
  The following proves the classical tautology that $\land$ is commutative:
  $\mvar{a}\land\mvar{b}⊢ \mvar{b}\land\mvar{a}$.
  \begin{equation*}
    \prftree[r]{}
      {\prftree[r]{}
        {Γ ⊢ \mvar{p}:\mvar{a}\land\mvar{b}}
        {Γ ⊢ \appr{2}{\mvar{p}}:\mvar{b}}}
      {\prftree[r]{}
        {Γ ⊢ \mvar{p}:\mvar{a}\land\mvar{b}}
        {Γ ⊢ \appr{1}{\mvar{p}}:\mvar{a}}}
      {Γ ⊢ (\appr{2}{\mvar{p}},\appr{1}{\mvar{p}}):\mvar{b}\land\mvar{a}}
  \end{equation*}
  This derivation serves as a small-scale verification that the introduction and
  elimination rules for conjunction complement one another well.
\end{example}

\begin{definition}\label[definition]{def:lem-dne}
  The \define{law of the excluded middle}\index{Law of excluded middle} (from
  the syllogistic principle \textit{tertium non datur}, ``no third is given'')
  is the following rule of deduction:
  \begin{equation*}
    \prftree[r]{\footnotesize LEM}
      {Γ ⊢ \prop{\mvar{a}}}
      {Γ ⊢ \apply{\ttfun{lem}}{\mvar{a}}:\true{\mvar{a}\lor\lnot\mvar{a}}}.
  \end{equation*}
  It is inter-derivable with the rule of \define{double negation elimination}:
  \index{Double negation elimination}
  \begin{equation*}
    \prftree[r]{\footnotesize DNE}
      {Γ ⊢ \prop{\mvar{a}}}
      {Γ ⊢ \mvar{p}:\lnot\lnot\mvar{a}}
      {Γ ⊢ \apply{\ttfun{dne}}{\mvar{p}}:\mvar{a}}.
  \end{equation*}
  that is, using the rule LEM, you can construct a proof of DNE and vice versa.
  These are \textit{not} rules of \IPL{}.
\end{definition}

\begin{example}\label[example]{ex:ipl-not-not-lem}
  A crucial consequence of the definition of \IPL{} is that the law
  of excluded middle isn't a theorem. However, its negation isn't either. While
  reading this proof, keep the following in mind:
  \begin{itemize}
    \itemsep0em
    \item I abbreviate the hypothesis
      $\mvar{h}:\lnot(\mvar{a}\lor\lnot \mvar{a})$ as $\mvar{h}:H$
    \item $\lnot\mvar{a}\≔ \mvar{a}→⊥$
  \end{itemize}
  \begin{center}
  \noindent\makebox[\textwidth]{%
    \prftree[r]{}
      {\prftree[r]{}
        {\prftree[r]{}
          {\prftree[r]{}
            {\prftree[r]{}
              {\prftree[r]{\footnotesize weak}
                {\prftree[r]{}
                  {\prftree[r]{\footnotesize refl}
                    {\mvar{p}:\mvar{a}⊢\mvar{p}:\mvar{a}}}
                  {\mvar{p}:\mvar{a}⊢\apply{\inl}{\mvar{p}}:\mvar{a}\lor\lnot\mvar{a}}}
                {\mvar{p}:\mvar{a},\mvar{h}:H⊢\apply{\inl}{\mvar{p}}:\mvar{a}\lor\lnot\mvar{a}}}
              {\prftree[r]{\footnotesize ex}
                {\prftree[r]{\footnotesize weak}
                  {\prftree[r]{\footnotesize refl}
                    {\mvar{h}:H⊢\mvar{h}:H}}
                  {\mvar{h}:H,\mvar{p}:\mvar{a}⊢\mvar{h}:H}}
                {\mvar{p}:\mvar{a},\mvar{h}:H⊢\mvar{h}:H}}
              {\mvar{p}:\mvar{a},\mvar{h}:H⊢\apply{\mvar{h}}{(\apply{\inl}{\mvar{p}})}:⊥}}
            {\mvar{h}:H⊢\λ{v}\apply{\mvar{h}}{(\apply{\inl}{v})}:\lnot\mvar{a}}}
          {\mvar{h}:H⊢\apply{\inr}{(\λ{v}\apply{\mvar{h}}{(\apply{\inl}{v})})}:\mvar{a}\lor\lnot \mvar{a}}}
        {\prftree[r]{\footnotesize refl}
          {\mvar{h}:H⊢\mvar{h}:H}}
        {\mvar{h}:H⊢ \apply{\mvar{h}}{(\inr(\λ{v}\apply{\mvar{h}}{(\apply{\inl}{v})}))}:⊥}}
      {⊢\λ{w}\apply{w}{(\inr(\λ{v}\apply{w}{(\apply{\inl}{v})}))}:\lnot (\lnot(\mvar{a}\lor\lnot \mvar{a}))}
  }
  \end{center}
  The law of the excluded middle will be discussed further in
  \cref{subsec:on-lem} \cite{harper}.
\end{example}

\subsection{Proof terms and harmony}
\label{subsec:proof-terms}

As we can see in \cref{ex:ipl-and-comm}, and \cref{ex:ipl-not-not-lem},
derivations in \IPL{} end with a \define{proof term} which
summarizes the whole proof tree. To demonstrate this, let's attempt to
reconstruct a proof tree based on a judgment including such a term. Suppose your
friend tells you they apprehended the following hypothetical judgment
in a dream, but upon waking, couldn't recall the derivation:
\begin{equation*}
    {Γ ⊢ \apply{\mvar{r}}{(\appr{1}{(\apply{\mvar{p}}{\mvar{q}})})} : \mvar{d}}.
\end{equation*}
Well, it certainly had to end with an application of implication elimination
to hypotheses of the form $\mvar{b}$ and $\mvar{b}→ \mvar{d}$:
\begin{equation*}
  \prftree[r]{}
    {Γ ⊢ \mvar{r}:\mvar{b}→\mvar{d}}
    {\prftree[r]{}
      {?}
      {Γ ⊢ \appr{1}{(\apply{\mvar{p}}{\mvar{q}})}:\mvar{b}}}
    {Γ ⊢ \apply{\mvar{r}}{(\appr{1}{(\apply{\mvar{p}}{\mvar{q}})})} : \mvar{d}}.
\end{equation*}
and if the term of type $\mvar{b}$ was produced from the application of
$\pr{1}$, the term it was applied to had to be a proof of
$\mvar{b}\land \mvar{c}$ for some $\mvar{c}$:
\begin{equation*}
  \prftree[r]{}
    {Γ ⊢ \mvar{r}:\mvar{b}→\mvar{d}}
    {\prftree[r]{}
      {\prftree[r]{}
        {?}
        {Γ ⊢ \apply{\mvar{p}}{\mvar{q}}:\mvar{b}\land\mvar{c}}}
      {Γ ⊢ \appr{1}{(\apply{\mvar{p}}{\mvar{q}})}:\mvar{b}}}
    {Γ ⊢ \apply{\mvar{r}}{(\appr{1}{(\apply{\mvar{p}}{\mvar{q}})})} : \mvar{d}}.
\end{equation*}
which itself had to be applied to proofs of $\mvar{a}$ and
$\mvar{a}→\mvar{b}\land\mvar{c}$ for some $\mvar{a}$:
\begin{equation*}
  \prftree[r]{}
    {Γ ⊢ \mvar{r}:\mvar{b}→\mvar{d}}
    {\prftree[r]{}
      {\prftree[r]{}
        {Γ ⊢ \mvar{p}:\mvar{a}→(\mvar{b}\land\mvar{c})}
        {Γ ⊢ \mvar{q}:\mvar{a}}
        {Γ ⊢ \apply{\mvar{p}}{\mvar{q}}:\mvar{b}\land\mvar{c}}}
      {Γ ⊢ \appr{1}{(\apply{\mvar{p}}{\mvar{q}})}:\mvar{b}}}
    {Γ ⊢ \apply{\mvar{r}}{(\appr{1}{(\apply{\mvar{p}}{\mvar{q}})})} : \mvar{d}}.
\end{equation*}
At this point, we can (un)deduce no further. The form of the judgment allowed
us no discretion; this is the only derivation (up to\footnote{For the
  non-mathematical reader: When a mathematician says ``X is true up to $R$''
  for some (equivalence) relation $R$, it means that all objects related by $R$
  are considered equivalent.}
renaming of metavariables) that could have produced that conclusion.

The proof term in longer arguments is often cumbersome, and we might wonder if
there are any benefits to internalizing\index{Internalizing!Proofs} proof. The
answer to this question will become clear in
\crefrange{subsec:ipl-compute}{subsec:ipl-uni}, and will play a critical role in
\cref{sec:propositions-and-types}. In short, they will allow us to
verify the \define{harmony}\index{Harmony} of our logic: that the introduction
rules complement the elimination rules with respect to two specific
criteria.\footnote{On the philosophical side, Martin-L\"of made an interesting
  argument that judgments involving proof terms are analytic, whereas
  proof-irrelevant mathematical arguments are synthetic: one must go
  beyond conceptual analysis and view the proof to be convinced of them
  \cite{martin-lof-analytic}.}

\begin{definition}\label[definition]{def:local-soundness-completeness}
  \define{Local soundness}\index{Soundness!Local} is the condition that
  elimination rules can't extract more information than was put into a
  conclusion by introduction rules; they aren't too strong.

  \define{Local completeness}\index{Completeness!Local} is the condition that
  elimination rules can recover all the information that was put into a conclusion
  by introduction rules; they aren't too weak.
\end{definition}

\subsection{Computation rules and judgmental equality}
\label{subsec:ipl-compute}

Some proofs are needlessly roundabout, and can be shortened.
Consider:
\begin{equation*}
  {\prftree[r]{}
    {\prftree[r]{}
      {Γ ⊢ \mvar{p}:\mvar{a}}
      {Γ ⊢ \mvar{q}:\mvar{b}}
      {Γ ⊢ (\mvar{p},\mvar{q}):\mvar{a}\land\mvar{b}}}
    {Γ ⊢ \appr{1}{(\mvar{p},\mvar{q})}:\mvar{a}}}.
\end{equation*}
This derivation could be eliminated entirely: we had a proof of $\mvar{a}$ to
begin with! This is an instance of a general pattern: whenever we have an
elimination rule for a connective just after its introduction, we can avoid the
circumlocution and cut straight to the conclusion.

Do such shortcuts result in \textit{the same} proof? To phrase this
question precisely, one needs a notion of identity for proof terms.
This notion is constructed just so we can answer this question in the
affirmative.

\begin{definition}\label[definition]{def:jdeq-ipl}
	\define{Judgmental equality} of proof terms, denoted
  $\mvar{p}\jdeq\mvar{q}:\mvar{a}$, is the least congruence closed under the
  following
  \define{computation rules}\index{Computation rules!In \IPL{}}
  (also called \define{β-rules}):
  \begin{gatherjot}
    {\prftree[r]{\footnotesize β$∧$\textsubscript{l}}
      {Γ ⊢ \mvar{p}:\mvar{a}}{\mvar{q}:\mvar{b}}
      {Γ ⊢ \appr{1}{(\mvar{p},\mvar{q})}\jdeq \mvar{p}:\mvar{a}}}
    \qquad
    {\prftree[r]{\footnotesize β$∧$\textsubscript{r}}
      {Γ ⊢ \mvar{p}:\mvar{a}}{\mvar{q}:\mvar{b}}
      {Γ ⊢ \appr{2}{(\mvar{p},\mvar{q})} \jdeq \mvar{q}:\mvar{b}}} \\
    {\prftree[r]{\footnotesize β$→$}
      {Γ, \mvar{p}:\mvar{a}⊢\mvar{q}:\mvar{b}}
      {Γ ⊢ \mvar{r}:\mvar{a}}
      {Γ ⊢ \apply{(\λ{v}\mvar{q})}{\mvar{r}}\jdeq
        \mvar{q}[\mvar{p}\≔\mvar{r}]:\mvar{b}}} \\
    {\prftree[r]{\footnotesize β$∨$\textsubscript{l}}
      {\prftree[r, noline]{}
        {Γ ⊢ \mvar{p}:\mvar{a}→\mvar{c}}
        {Γ ⊢ \mvar{q}:\mvar{b}→\mvar{c}}}
      {Γ ⊢ \mvar{r}:\mvar{a}}
      {Γ ⊢ \apppply{\case}{\mvar{p}}{\mvar{q}}{(\apply{\inl}{\mvar{r}})}\jdeq
        \apply{\mvar{p}}{\mvar{r}}:\mvar{c}}}
    \qquad
    {\prftree[r]{\footnotesize β$∨$\textsubscript{r}}
      {\prftree[r, noline]{}
        {Γ ⊢ \mvar{p}:\mvar{a}→\mvar{c}}
        {Γ ⊢ \mvar{q}:\mvar{b}→\mvar{c}}}
      {Γ ⊢ \mvar{r}:\mvar{b}}
      {Γ ⊢ \apppply{\case}{\mvar{p}}{\mvar{q}}{(\apply{\inr}{\mvar{r}})}\jdeq
        \apply{\mvar{q}}{\mvar{r}}:\mvar{c}}}.
  \end{gatherjot}
  (Recall \cref{notation:substitution} for the meaning of
  $\mvar{q}[\mvar{p}\≔\mvar{r}]$).\footnote{
    Since the elimination rule for falsehood (\cref{subsec:ipl-elim}) gives back a
    proof of an arbitrary proposition (which is not in general one of the ``inputs''
    to falsehood introduction), there's no corresponding β-rule.}
\end{definition}

Judgmental equality, as the name implies, is a meta-level concept. In
particular, ``$\≡$'', like ``$:$'', is not part of the syntax of \IPL{}. 
The term $\mvar{p}\≡\mvar{q}:\mvar{a}$ should never be read as a proof of
$\mvar{a}$. Since it is a congruence, none of the rules of \IPL{} can distinguish
judgmentally equal terms. Thus, if $\mvar{p}\jdeq\mvar{q}:\mvar{a}$, then we can
write $\mvar{q}$ for $\mvar{p}$ anywhere in an \IPL{} derivation without
changing the meaning.\footnote{This ``indiscerniblility'' is implemented as a
  rule in \UTT{} in \cref{subsec:the-identity-type}, after we
  \textit{internalize} the notion of judgmental equality.}

Besides allowing us to write shorter proofs, these rules witness local
soundness; they show that the elimination rules never prove any conclusion that
wasn't given to an introduction rule.

\subsection{Unicity rules}
\label{subsec:ipl-uni}

% Just because a logic doesn't give us an easy way to do something unexpected
% doesn't mean it can't be done. The prime example is proving a contradiction.
% History is filled with deductive systems that were thought to be sound, but
% were shown to be useless by some clever argument. Russell's paradox dispensed
% with Frege's \textit{Begriffsschrift}, the Burali-Forti antimony unraveled
% na\"ive set theories, and Girard's paradox revealed a fatal flaw in
% Martin-L\"of's early attempts at dependent type theory.

One worry we might have when designing our logic is that our propositions have
proofs of a non-standard form. We might \textit{think} that every proof of
$\mvar{a}∧\mvar{b}$ has the form $(\mvar{p},\mvar{q})$ for $\mvar{p}:\mvar{a}$ and
$\mvar{q}:\mvar{b}$, but how do we \textit{know}? The answer was is in
following \define{unicity rules}\index{Unicity rules}, (also known as
\define{η-rules} or \define{uniqueness principles}):
\begin{gatherjot}
  {\prftree[r]{\footnotesize η$⊤$}
    {Γ ⊢ \mvar{p}:⊤}
    {Γ ⊢ \mvar{p}\jdeq \unitelem : ⊤}}
  \qquad
  {\prftree[r]{\footnotesize η$∧$}
    {Γ ⊢ \mvar{p}:\mvar{a}∧\mvar{b}}
    {Γ ⊢ (\appr{1}{\mvar{p}},\appr{2}{\mvar{p}})\jdeq
      \mvar{p}:\mvar{a}∧\mvar{b}}} \\
  {\prftree[r]{\footnotesize η$→$}
    {Γ ⊢ \mvar{p}:\mvar{a}→\mvar{b}}
    {Γ ⊢ \λ{v}{(\apply{\mvar{p}}{v})}\jdeq \mvar{p}:\mvar{a}→\mvar{b}}}
  \qquad
  {\prftree[r]{\footnotesize η$∨$}
    {Γ ⊢ \mvar{p}:\mvar{a}∨\mvar{b}}
    {Γ ⊢ \apppply{\case}{\inl}{\inr}{\mvar{p}}\jdeq \mvar{p}:\mvar{a}∨\mvar{b}}}.
\end{gatherjot}
These rules witness local completeness; after applying elimination
rules and then introduction rules we can recover our original
proofs.\footnote{There's also no η-rule for $\bot$.
  After internalizing equality (\cref{subsec:the-identity-type}),
  we'll be able to derive that any two elements of (the analogue of) $\bot$ are
  equal.}

% \subsection{Diagrams}
% \label{subsec:ipl-diagrams}

% One of the ways that we can understand our computation rules is via a clever
% sort of diagram. This section foreshadows some connections between
% \IPL{} and the content of \cref{chap:category-theory}.

% \begin{definition}\label[definition]{def:implication-composition}
%   Suppose given terms $\mvar{p}:\mvar{a}→\mvar{b}$ and
%   $\mvar{q}:\mvar{b}→\mvar{c}$ under some hypotheses $\Gamma$. We define the
%   \define{composition}\index{Composition!Of proof terms}
%   $\mvar{q}∘\mvar{p}:\mvar{a}→\mvar{c}$ to be the term in the conclusion of the
%   following deduction:
%   \begin{equation*}
%   {\prftree[r]{}
%     {\prftree[r]{}
%       {\prftree[r]{}
%         {\prftree[r]{\footnotesize refl}
%           {}
%           {Γ, \mvar{r}:\mvar{a} ⊢ \mvar{r}:\mvar{a}}}
%         {\prftree[r]{\footnotesize weak}
%           {Γ ⊢ \mvar{p}:\mvar{a}→\mvar{b}}
%           {Γ, \mvar{r}:\mvar{a} ⊢ \mvar{p}:\mvar{a}→\mvar{b}}}
%         {Γ, \mvar{r}:\mvar{a} ⊢ \apply{\mvar{p}}{\mvar{r}}:\mvar{b}}}
%       {\prftree[r]{\footnotesize weak}
%         {Γ ⊢ \mvar{q}:\mvar{b}→\mvar{c}}
%         {Γ, \mvar{r}:\mvar{a} ⊢ \mvar{q}:\mvar{b}→\mvar{c}}}
%       {Γ, \mvar{r}:\mvar{a} ⊢ \apply{\mvar{q}}{(\apply{\mvar{p}}{\mvar{r}})}:\mvar{c}}}
%     {Γ ⊢ \λ{v}{\apply{\mvar{q}}{(\apply{\mvar{p}}{v}})}:\mvar{a}→\mvar{c}}}.
%   \end{equation*}
%   In short, $\mvar{q}∘\mvar{p}\defeq \λ{v}{\apply{\mvar{q}}{(\apply{\mvar{p}}{v}})}$.
% \end{definition}

% Now suppose we have terms $\mvar{p}:\mvar{a}→\mvar{c}$ and
% $\mvar{q}:\mvar{b}→\mvar{c}$. Via the elimination rule for $∨$, there is a
% term $\apppply{\case}{\mvar{p}}{\mvar{q}}:\mvar{a}∨\mvar{b}→\mvar{c}$.
% We can rewrite the β-rules using the composition we just defined:
% \begin{align*}
%   \appply{\case}{\mvar{p}}{\mvar{q}}∘\inl &\jdeq \mvar{p} \\
%   \appply{\case}{\mvar{p}}{\mvar{q}}∘\inr &\jdeq \mvar{q}
% \end{align*}
% Similarly, given $\mvar{p}:\mvar{c}→\mvar{a}$ and $\mvar{q}:\mvar{c}→\mvar{b}$,
% the introduction rule for $∧$ guarantees a function
% $\lam{v}{(\apply{\mvar{p}}{v}, \apply{\mvar{q}}{v})}:\mvar{c}→\mvar{a}∧\mvar{b}$.
% The computation rules give us the following equalities:
% \begin{align*}
%   \pr{1}∘(\lam{v}{(\apply{\mvar{p}}{v}, \apply{\mvar{q}}{v})}) &\jdeq \mvar{p} \\
%   \pr{2}∘(\lam{v}{(\apply{\mvar{p}}{v}, \apply{\mvar{q}}{v})}) &\jdeq \mvar{q}
% \end{align*}
% (For brevity, put $f\defeq\lam{v}{(\apply{\mvar{p}}{v}, \apply{\mvar{q}}{v})}$).
% We can express these equalities using the following \define{commutative diagram}
% (skip ahead to \cref{def:commutative-diagram} for details).

% \begin{center}
%   \begin{minipage}[b]{0.47\linewidth}
%     \centering
%       \begin{tikzcd}[sep=large]
%         A \arrow[dr, "\inl"{name=I1}]\arrow[ddr, swap, "\mvar{p}"{name=F}, bend right] &
%         {} & B\arrow[dl, "\inr"{name=I2},swap] \arrow[ddl,"\mvar{q}"{name=G}, bend left] \\
%         {} & A+B \arrow[d,dashed, "\appply{\case}{\mvar{p}}{\mvar{q}}"] & {} \\
%         {} & C & {}
%         % filling in the cells
%         \arrow[phantom, shift left=0.3em, from=I1, to=F, "(\beta)"]
%         \arrow[phantom, shift right=0.3em, from=I2, to=G, "(\beta)"]
%         % \arrow[phantom, from=FG1, to=FG2, "(\eta)"]
%       \end{tikzcd}
%   \end{minipage}
%   \begin{minipage}[b]{0.47\linewidth}
%     \centering
%       \begin{tikzcd}[sep=large]
%         {} & C\arrow[ddl, bend right, swap, "\mvar{p}"{name=F}]\arrow[ddr, bend left, "\mvar{q}"{name=G}]
%               % \arrow[d,dashed, bend right, swap, "\mathsf{rec}"{name=FG1}]
%               \arrow[d,dashed, "f"{name=FG2}] & {} \\
%         {} & A\times B \arrow[dl, "\pr{1}"{name=P1}]
%                       \arrow[dr,swap, "\pr{2}"{name=P2}] & {} \\
%         A & {} & B
%         % filling in the cells
%         \arrow[phantom, shift right=0.3em, from=P1, to=F, "(\beta)"]
%         \arrow[phantom, shift left=0.3em, from=P2, to=G, "(\beta)"]
%         % \arrow[phantom, from=FG1, to=FG2, "(\eta)"]
%       \end{tikzcd}
%   \end{minipage}
% \end{center}
% In short, each of the two respective computation rules ensure that, whichever
% (directed) path you take around the perimeters of the two triangles, composing
% as you go, you get judgmentally equal results.

\section{The λ-calculus}
\label{sec:the-lambda-calculus}

In the 1930s, mathematicians and computer scientists (though they were not
called ``computer scientists'' then) sought a rigorous foundation for terms like
``effectively computable'' and ``systematic''. They sought to lay on a rigorous
logical foundation an intuitive idea of what a(n idealized) human being could
accomplish with specific (finite) instructions and arbitrarily much
paper, pen, and time. Many particular mathematical models were proposed,
including the λ-calculus (\LC{}), Turing machines, Post systems,
register machines, combinatory definability, and more \cite{sep-church-turing}.
The \textit{Church-Turing thesis} proposes that the proper definition is (any
model equivalent to) the λ-calculus.\footnote{Perhaps the strongest argument
  to accept such a thesis is the theorem that all models just mentioned are
  equivalent.}
Computer scientists widely adopt this thesis, and the λ-calculus and its
variants are of central importance to the study of computation in
general.

A comprehensive study of \LC{} is beyond the scope
of this thesis. For a tutorial on \LC{} and \STLC{} to
complement the presentation here, see \cite{lambda-lecture}. For a standard
reference on \LC{} (with an appendix on \STLC{}), see
\cite{barendregt}.

The type theory considered in \cref{chap:type-theory} is an extension of the
typed λ-calculus (\TLC{}) of \cref{subsec:more-complex-types}, which
itself an extension of the simply-typed λ-calculus (\STLC{}) of
\cref{subsec:the-simply-typed-lambda-calculus}.

\subsection{The untyped λ-calculus}
\label{subsec:the-untyped-lambda-calculus}

The formation\index{Formation!In the λ-calculus} (really, all) rules of the
λ-calculus are brilliantly simple.\footnote{Again, ignoring issues of variable
binding.} We will again need a countably infinite set of variables $v,w,\ldots$:
\begin{align*}
  {\prftree[r]{}
    {}
    {Γ ⊢ \term{v}}}
  &&
  {\prftree[r]{}
    {Γ ⊢ \term{\mvar{a}}}
    {Γ ⊢ \term{\mvar{b}}}
    {Γ ⊢ \term{\apply{\mvar{a}}{\mvar{b}}}}}
  &&
  {\prftree[r]{}
    {Γ ⊢ \mvar{a}}
    {Γ ⊢ \term{\lam{v}{\mvar{a}}}}}
\end{align*}
In short, we have variables, application of one term to another, and
\define{function abstraction}. The λ-calculus is the inspiration for our
notation of function application by juxtaposition (\cref{notation:parens}).
Accordingly, function application is left-associative.

There are no elimination rules. Judgmental equality of λ-terms is
the least congruence closed under the following computation
rule\index{Computation rules!In \LC{}} and subject to the following
unicity rule:
\begin{align*}
  {\prftree[r]{\footnotesize β}
    {Γ ⊢ \term{\mvar{a}}}
    {Γ ⊢ \term{\mvar{b}}}
    {Γ ⊢ \apply{(\λ{v}\mvar{a})}{\mvar{b}}\jdeq
      \mvar{a}[v\≔\mvar{b}]}}
  &&
  {\prftree[r]{\footnotesize η}
    {Γ ⊢ \term{\mvar{a}}}
    {Γ ⊢ \λ{v}{(\apply{\mvar{a}}{v})}\jdeq \mvar{a}}}.
\end{align*}

\begin{notation}\label[notation]{notation:beta}
  If we apply the computation rule to $\mvar{a}$ and get some term $\mvar{a}'$,
  we write $\mvar{a} \betato \mvar{a}'$. If we apply it an arbitrary but finite
  (possibly zero) amount of times to get $\mvar{a}''$, we write $\mvar{a}
  \betatoo \mvar{a}''$.
\end{notation}

\begin{example*}\label[example]{ex:calculations-in-lc}
  Under the computation rule,
  \begin{align*}
    &\appply
      {(\λ{u}{\λ{v}{\λ{w}{\appply{\mvar{x}}{v}{w}}}})}
      {(\λ{y}{y})}
      {(\λ{z}{\apply{z}{\mvar{a}}})} \\
    \≡ &\apply
      {(\λ{v}{\λ{w}{\appply{\mvar{x}}{v}{w}}})[u\≔ (\λ{y}{y})]}
      {(\λ{z}{\apply{z}{\mvar{a}}})}
    &&\quad\text{(β)} \\
    \≡ &\apply
      {(\λ{v}{\λ{w}{\appply{\mvar{x}}{v}{w}}})}
      {(\λ{z}{\apply{z}{\mvar{a}}})}
    &&\quad\text{(substitution)} \\
    \≡ & (\λ{w}{\appply{\mvar{x}}{v}{w}})[v\≔ \λ{z}{\apply{z}{\mvar{a}}}]
    &&\quad\text{(β)} \\
    \≡ & (\λ{w}{\appply{\mvar{x}}{(\λ{z}{\apply{z}{\mvar{a}}})}{w}})
    &&\quad\text{(substitution)}.
  \end{align*}
  Since $\mvar{x}$ is some arbitrary term, this expression can be reduced no
  further.
\end{example*}

% \begin{example*}\label[example]{ex:fixed-point}
%   A \define{fixed point} of a function $f$ (in \LC{},
%   \FOL{}+\ZFC, or any other formal system or programming language)
%   is an input $x$ such that $\apply{f}{x}=x$. Every term of \LC{}
%   has a fixed point (up to judgmental equality $\≡$). To see this, define
%   $θ \defeq (\λ{x}{\λ{y}{\apply{y}{(\appply{x}{x}{y})}}})$ and let
%   $𝚯\defeq \apply{θ}{θ}$. Let $\mvar{f}$ be any term of \LC{}. Then
%   \begin{align*}
%     \apply{𝚯}{f}
%     &\≡ \appply{θ}{θ}{f} \\
%     &\≡ \appply{(\λ{x}{\λ{y}{\apply{y}{(\appply{x}{x}{y})}}})}{θ}{f} \\
%     &\≡ \apply{f}{(\appply{θ}{θ}{f})} \\
%     &\≡ \apply{f}{(\apply{𝚯}{f})}
%   \end{align*}
%   The term $𝚯$ is called the \define{Turing fixed-point combinator}, and it can
%   be used to define recursive functions in \LC{} and its derivatives
%   \cite{lambda-lecture}.
% \end{example*}

We can make choices about what part of an expression to apply the computation
rule to. Consider the following computations (leaving substitution implicit):
\begin{center}
  \begin{tikzpicture}[node distance=1.5cm]
    \node[] (A) at (0, 0) {$
      \apply{(\apply{(\λ{x}{\apply{x}{x}})}
                    {(\λ{y}{\mvar{a}})})}
            {(\apply{(\λ{z}{\apply{z}{\mvar{b}}})}
                      {(\λ{w}{\mvar{c}})})}
    $};
    %%%%%%%%%%%%%%%%%%%%%%% LEFT
    \node[below of=A, xshift=-8em] (B) {$
      \apply{(\apply{(\λ{y}{\mvar{a}})}{(\λ{y}{\mvar{a}})})}
            {(\apply{(\λ{z}{\apply{z}{\mvar{b}}})}
                      {(\λ{w}{\mvar{c}})})}
    $};
    \node[below of=B] (C) {$
      \apply{\mvar{a}}
            {(\apply{(\λ{z}{\apply{z}{\mvar{b}}})}
                      {(\λ{w}{\mvar{c}})})}
    $};
    \node[below of=C] (D) {$
      \apply{\mvar{a}}
            {(\apply{{(\λ{w}{\mvar{c}})}}{\mvar{b}})}
    $};
    %%%%%%%%%%%%%%%%%%%%%%% RIGHT
    \node[below of=A, xshift=8em] (F) {$
      \apply{(\apply{(\λ{x}{\apply{x}{x}})}
                    {(\λ{y}{\mvar{a}})})}
            {(\apply{(\λ{w}{\mvar{c}})}{\mvar{b}})}
    $};
    \node[below of=F] (G) {$
      \apply{(\apply{(\λ{x}{\apply{x}{x}})}
                    {(\λ{y}{\mvar{a}})})}
            {\mvar{c}}
    $};
    \node[below of=G] (H) {$
      \apply{(\apply{(\λ{y}{\mvar{a}})}{(\λ{y}{\mvar{a}})})}
            {\mvar{c}}
    $};
    %%%%%%%%%%%%%%%%%%%%%%% LAST
    \node[below of=A, yshift=-10em] (Z) {$
      \apply{\mvar{a}}{\mvar{c}}
    $};
    %%%%%%%%%%%%%%%%%%%%%%% ARROWS
    %%%%% LEFT
    \draw (A) edge [->] node[above] {\scriptsize β} (B);
    \draw (B) edge [->] node[left] {\scriptsize β} (C);
    \draw (C) edge [->] node[left] {\scriptsize β} (D);
    \draw (D) edge [->] node[below] {\scriptsize β} (Z);
    %%%%% RIGHT
    \draw (A) edge [->] node[above] {\scriptsize β} (F);
    \draw (F) edge [->] node[right] {\scriptsize β} (G);
    \draw (G) edge [->] node[right] {\scriptsize β} (H);
    \draw (H) edge [->] node[below] {\scriptsize β} (Z);
  \end{tikzpicture}
\end{center}
In the left column, we reduce the leftmost expression first, whereas in the
right column, we reduce the rightmost expression first. In this example, they
result in the same term, but there is \textit{a priori} no reason to think this
is so in general.

\begin{theorem}[Church-Rosser]\label[theorem]{thm:church-rosser}
  The order in which computation rules are applied to a term of \LC{}
  doesn't matter. More specifically, if $\mvar{a}\betatoo\mvar{b}$ and
  $\mvar{a}\betatoo\mvar{c}$, then there is a term $\mvar{d}$ such that
  $\mvar{b}\betatoo\mvar{d}$ and $\mvar{c}\betatoo \mvar{d}$.
  Diagrammatically,
  \begin{center}
    \begin{tikzcd}[sep=small, every arrow/.append style={two heads, "β"}]
      {} &
      \mvar{a}
        \arrow[dl, swap]
        \arrow[dr] &
      {} \\
      \mvar{b}
        \arrow[dr, dashed, swap] &
      {} &
      \mvar{c}
        \arrow[dl, dashed] \\
      {} & \mvar{d} & {}
    \end{tikzcd}
  \end{center}
\end{theorem}

Note that the Church-Rosser Theorem doesn't claim judgmental equality
$\≡$,\footnote{If it did, it would be trivial; the β-rules preserve judgmental
equality.} but actual \textit{syntactic identity}, they compute to the exact
same term.

\begin{definition}\label[definition]{def:normal-form}
	If $\mvar{a}$ is a term of \LC{} such that there does not exist
  a term $\mvar{a}'$ with $\mvar{a}\betato \mvar{a}'$, then
  $\mvar{a}$ is in \define{(β-)normal form}\index{Normal form}.
  If $\mvar{b}\betatoo\mvar{b}'$ and $\mvar{b}'$ is in normal form, then
  $\mvar{b}'$ is the normal form of $\mvar{b}$.
\end{definition}

The use of the definite article in \cref{def:normal-form} is justified by
Church-Rosser: every term has at most one β-normal form. However, not every
term has a β-normal form.

\begin{example*}\label[example]{ex:non-terminating}
  Consider the term $⊥\≔ \λ{v}{\appply{v}{v}{v}}$. Using the computation rule,
  \begin{align*}
    \apply{⊥}{⊥}
    &\≡ \apply{(\λ{v}{\appply{v}{v}{v}})}{⊥}
    &&\quad\text{(Definition)} \\
    &\≡ (\appply{v}{v}{v})[v \defeq ⊥]
    &&\quad\text{(β)} \\
    &\≡ \appply{⊥}{⊥}{⊥}
    &&\quad\text{(Substitution)} \\
    &\≡ \apppply{⊥}{⊥}{⊥}{⊥}
    &&\quad\text{(…)} \\
    &\≡ \appppply{⊥}{⊥}{⊥}{⊥}{⊥} \\
    &\≡ ⋯
  \end{align*}
  Not only are there terms that don't really ``progress'' when we try to
  ``compute'', but some that actually \textit{expand} under rules meant to
  \textit{reduce} them, that are judgmentally equal to infinitely many other
  terms!
\end{example*}

While an ability to loop indefinitely is essential to Turing-Completeness (the
property of being equivalent to a Turing machine model of computation), we
might ask if there is another system which expresses many of the same
ideas and computations, but with a computation rule that always ``terminates''.

\subsection{The simply-typed λ-calculus}
\label{subsec:the-simply-typed-lambda-calculus}

In order to fix the problem raised in \cref{ex:non-terminating}, one can add
\define{types}\index{Types!Simply-typed λ-calculus} to \LC{} to get
\STLC{}, the simply-typed λ-calculus. We consider two new judgments
of the form ``$\type{\mvar{t}}$'' and ``$\mvar{a}:\mvar{t}$'', read
``\mvar{t} is a type'' and ``\mvar{a} has type \mvar{t}'', respectively (``$:$''
is a symbol of the meta-language, it is just shorthand for writing a judgment).
In writing these judgments, we implicitly assume that all expressions are
well-formed terms. As a first approximation, a type can be thought of as a
description of a program's behavior.

This language begins with additional formation rules
\begin{align*}
  {\prftree[r]{}
    {}
    {Γ ⊢ \type{\groundtype_i}}}
  &&
  {\prftree[r]{}
    {Γ ⊢ \type{\mvar{s}}}
    {Γ ⊢ \type{\mvar{t}}}
    {Γ ⊢ \type{\mvar{s}→\mvar{t}}}}
\end{align*}
The $\groundtype_i$ are called \define{ground types}, \define{base types}, or
\define{atomic types}.\footnote{There can be as many ground types as necessary
  or only one, it doesn't fundamentally change the character of the resulting
  theory.}
The term $\mvar{a}→\mvar{b}$ is the type of functions from $\mvar{a}$ to
$\mvar{b}$. The introduction
rule\index{Introduction rules!In \STLC{}} tells us how to form
elements of these types:
\begin{align*}
  {\prftree[r]{}
    {Γ, \mvar{a}:\mvar{s} ⊢ \mvar{b}:\mvar{t}}
    {Γ ⊢ \λ{v}{\mvar{b}}:\mvar{s}→\mvar{t}}}
    % {Γ ⊢ \λ{v}{\mvar{b}}[\mvar{a}\defeq v]:\mvar{s}→\mvar{t}}}
\end{align*}
Introduction rules for some ground type(s) are reasonable additional axioms, but
they are not necessary for the development of the theory.\footnote{This might
look like
\begin{align*}
  {\prftree[r]{}
    {}
    {Γ ⊢ z:\groundtype_0}}.
\end{align*}
which would assert the existence of an element $z$ of the 0\textsuperscript{th}
ground type $\groundtype_0$.
} An elimination rule balances this introduction:
\begin{align*}
  {\prftree[r]{}
    {Γ ⊢ \mvar{a}:\mvar{s}→\mvar{t}}
    {Γ ⊢ \mvar{b}:\mvar{s}}
    {Γ ⊢ \apply{\mvar{a}}{\mvar{b}}:\mvar{t}}}
\end{align*}
The computation rules\index{Computation rules!In \STLC{}} and
unicity rules remain much the same, only now with some restrictions on the types
of the expressions involved:
\begin{align*}
  {\prftree[r]{\footnotesize β}
    {Γ, \mvar{a}:\mvar{s} ⊢ \mvar{b}:\mvar{t}}
    {Γ ⊢ \mvar{c}:\mvar{s}}
    {Γ ⊢ \apply{(\λ{v}\mvar{b})}{\mvar{c}}\jdeq
      \mvar{b}[\mvar{a}\≔\mvar{c}]}}
  &&
  {\prftree[r]{\footnotesize η}
    {Γ ⊢ \term{\mvar{a}}}
    {Γ ⊢ \λ{v}{(\apply{\mvar{a}}{v})}\jdeq \mvar{a}}}.
\end{align*}
Variables may stand in for terms of arbitrary type.

\begin{theorem}\label[theorem]{thm:strong-normalization}
  \STLC{} is \define{strongly normalizing}; every term has exactly
  one normal form.
\end{theorem}

\begin{example}
	Recall the problematic term of \cref{ex:non-terminating}:
  $⊥\≔ \λ{v}{\apply{v}{v}}$. Can the introduction rules assign this term a type?
  To determine the overall type of $⊥$, we must determine the type of its
  argument. Its argument gets applied to a term, so it must have a function type
  $\mvar{a} → \mvar{b}$. However, since it is applied to
  itself, it must have type $(\mvar{a} → \mvar{b}) → \mvar{b}$. But now it still
  can't be applied to itself, so it must have type
  $((\mvar{a} → \mvar{b}) → \mvar{b}) → \mvar{b}$. Of course, this problem
  persists no matter how many layers are added, and leads us to the following
  conclusion: no well-typed term may be applied to itself.
\end{example}

\subsection{More complex types}
\label{subsec:more-complex-types}

The ontology of \STLC{} is severely limited: it can only express
ground type(s) and functions between them. This section develops
various \define{type formers}, which build up more complex types
from simpler ones (e.g.\ $→$ from \STLC{}). In
\cref{sec:propositions-and-types}, we will see that these types conform to
intuitions from \cref{sec:ipl}.

We will specifically consider four additional types (plus $→$), formed via the
following rules\index{Formation rules!In \TLC{}}:
\begin{gatherjot}
  \prftree[r]{}{}{Γ⊢\type{⊤}}
  \qquad
  \prftree[r]{}{}{Γ⊢\type{⊥}}
  \qquad
  \prftree[r]{}
    {\prftree[r, noline]{}
      {Γ ⊢ \type{\mvar{a}}}
      {Γ ⊢ \type{\mvar{b}}}}
    {Γ⊢\type{\mvar{a}×\mvar{b}}}
  \qquad
  \prftree[r]{}
    {\prftree[r, noline]{}
      {Γ ⊢ \type{\mvar{a}}}
      {Γ ⊢ \type{\mvar{b}}}}
    {Γ⊢\type{\mvar{a}+\mvar{b}}}
\end{gatherjot}

Perhaps unsurprisingly, the type $⊥$ is empty, it has no elements.
The type $⊤$ has one element:\footnote{This rule only really says we have one
  known way to construct an element of $⊤$, but the η-rules for this type
  justify the claim that this is the only such element.}
\begin{equation*}
  \prftree[r]{}{}{Γ ⊢ \unitelem:⊤}.
\end{equation*}
The type $\mvar{a}×\mvar{b}$ contains (ordered) pairs, with one element from
$\mvar{a}$ and one from $\mvar{b}$:
\begin{equation*}
  \prftree[r]{}
    {Γ ⊢ \type{\mvar{a}}}{Γ ⊢ \type{\mvar{b}}}
    {Γ ⊢ \mvar{p}:\mvar{a}}{Γ ⊢ \mvar{q}:\mvar{b}}
    {Γ ⊢ (\mvar{p},\mvar{q}):\mvar{a}×\mvar{b}}.
\end{equation*}
The type $\mvar{a}+\mvar{b}$ is sometimes called ``option'' or ``either'': it
contains tagged elements of $\mvar{a}$ or $\mvar{b}$:
\begin{align*}
  \prftree[r]{}
    {\prftree[r, noline]{}
      {Γ ⊢ \type{\mvar{a}}}
      {Γ ⊢ \type{\mvar{b}}}}
    {Γ ⊢ \mvar{p}:\mvar{a}}
    {Γ ⊢ \apply{\inl}{\mvar{p}}:\mvar{a}+\mvar{b}},
  &&
  \prftree[r]{}
    {\prftree[r, noline]{}
      {Γ ⊢ \type{\mvar{a}}}
      {Γ ⊢ \type{\mvar{b}}}}
    {Γ ⊢ \mvar{p}:\mvar{b}}
    {Γ ⊢ \apply{\inr}{\mvar{p}}:\mvar{a}+\mvar{b}}.
\end{align*}

The elimination rules\index{Elimination rules!In \TLC{}}
act as one might expect, given the reuse of the notation from \cref{sec:ipl} for
the introduction rules:
\begin{gatherjot}
  \prftree[r]{}
    {\prftree[r, noline]{}
      {Γ ⊢ \type{\mvar{a}}}
      {Γ ⊢ \type{\mvar{b}}}}
    {Γ ⊢ \mvar{p}:\mvar{a}×\mvar{b}}
    {Γ ⊢ \appr{1}\mvar{p}:\mvar{a}}
  \qquad
  \prftree[r]{}
    {\prftree[r, noline]{}
      {Γ ⊢ \type{\mvar{a}}}
      {Γ ⊢ \type{\mvar{b}}}}
    {Γ ⊢ \mvar{p}:\mvar{a}×\mvar{b}}
    {Γ ⊢ \appr{2}\mvar{p}:\mvar{b}} \\
  \prftree[r]{}
    {\prftree[r, noline]{}
      {\prftree[r, noline]{}
        {Γ ⊢ \type{\mvar{a}}}
        {Γ ⊢ \type{\mvar{b}}}}
      {Γ ⊢ \type{\mvar{c}}}}
    {\prftree[r, noline]{}
      {Γ ⊢ \mvar{p}:\mvar{a}→\mvar{c}}
      {Γ ⊢ \mvar{q}:\mvar{b}→\mvar{c}}}
    {Γ ⊢ \mvar{r}:\mvar{a}+\mvar{b}}
    {Γ ⊢ \apppply{\rec_{+}}{\mvar{p}}{\mvar{q}}{\mvar{r}}:\mvar{c}}
  \qquad
  \prftree[r]{}
    {Γ ⊢ \type{\mvar{a}}}
    {Γ ⊢ \mvar{p}:⊥}
    {Γ ⊢ \apply{\rec_{⊥}}{\mvar{p}}:\mvar{a}}.
\end{gatherjot}
Just as with the formation, introduction, and elimination rules,
the computation and unicity rules are identical to those in \IPL{}
(modulo notational differences), and so are omitted here.

Examples of (somewhat) practical programs that can be written in an extension
of this theory abound in \cref{chap:type-theory}.

\subsection{Propositions and Types}
\label{sec:propositions-and-types}

What just happened? \Cref{sec:ipl} defined a logico-deductive system, capable of
expressing many familiar proofs of sentential logic (with some differences
stemming from its constructivist nature). \Cref{sec:the-lambda-calculus} took a
sharp left turn, describing a formalism for capturing a notion of algorithmic
computability. However, the system obtained by adding a few reasonable features
for describing more complicated data to \STLC{} looked not merely
analogous, but \textit{identical to} \IPL{} (again, modulo simple
notational changes, see \cref{tab:curry-howard}).

While perhaps less surprising due to the notational conveniences of hindsight,
this observation has startling and deep consequences. It is known as the
Curry-Howard correspondence, or the propositions-as-types interpretation. Many
of its applications stem from the fact that type-checking (the process of
determining whether or not $\mvar{x}$ has a given type $t$ in a given context
$Γ$) is decidable: it can be performed mechanically, preferably by a computer.
This is why languages such as \software{Haskell} and \software{ML} have
\textit{type systems} inspired by \TLC{}: the compiler uses types to reason
about properties of the code in question.

Most importantly for us, proofs of \IPL{} can be encoded into \TLC{}. (More
accurately, the proofs in \IPL{} \textit{are} terms of \TLC{}.) 
This is the basis for a kind of program called a ``proof assistant'', which
verifies proofs written in a formal langauge. The ability to verify complex
arguments by computer (partially) solves many of the difficulties hinted
at in the introduction.

\begin{sidewaystable}
  \centering
  \begin{tabular}{l | l | l | l | l}
    \ZFC & \IPL & \UTT & Category theory & Homotopy theory \\ \hline
    Set
      & Proposition
      & Type
      & Object
      & Space \\
    Element
      & Proof
      & Term/program
      & ``Element'' ($⊤→X$)
      & Point \\
    Product ($A×B$)
      & Conjunction ($a∧b$)
      & Pair type (${a}×{b}$)
      & Product (${a}×{b}$)
      & Product space (${A}×{B}$) \\
    Disjoint union (${A}\amalg {B}$)
      & Disjunction (${a}∨{b}$)
      & Sum type ($a+b$)
      & Coprod.\ type ($A+B$)
      & Coprod.\ space ($A+B$) \\
    Empty set ($\emptyset$)
      & Falsehood
      & Empty type ($\emptytype$)
      & Initial object
      & Empty space \\
    Singleton set
      & Truth
      & Unit type ($\unittype$)
      & Final object
      & Singeton space \\
    Function ($f:A→B$)
      & Impl.\ ($p:a→b$)
      & Function ($f:A→B$)
      & Morphism ($f:A→B$)
      & Continuous function \\
    Equality ($A=B$)
      & Judg.\ eq.\ ($A\≡ B$)
      & Judg.\ eq.\ ($A\≡ B$)
      & Equality ($A=B$)
      & Equality ($A=B$) \\
    Path $f:[0,1]→X$
      &
      & Prop. eq.\ ($\propeq{A}{a}{b}$)
      & Equality ($A=B$)
      & Path \\
    {}
      & {}
      & Weak equiv.\ $\weq{A}{B}$
      & Isomorphism $A≅B$
      & Equivalence $A\simeq B$ \\
    Universal q.\ ($∀x.Px$)
      & {}
      & Π-type ($\prod_{x:A}\apply{P}{x}$)
      & {}
      & Section \\
    Existential q.\ ($∃x.Px$)
      & {}
      & Σ-type ($\sum_{x:A}\apply{P}{x}$)
      & {}
      & Total space
  \end{tabular}
  \caption{\label{tab:curry-howard}A table of metaphors, including but not
    limited to the Curry-Howard-Lambek-Voevodsky correspondence.}
\end{sidewaystable}

\subsubsection{Judgmental equality revisited}
\label{subsubsec:judgmental-equality}

The Curry-Howard correspondence suggests a new reading of judgmental equality
(``$\≡$'') of proof terms. It might also be considered a kind of
\textit{computational equality}. In fact, in a way that can be made quite
precise, $x\≡ y:A$ if and only if $x$ and $y$ have the same βη-normal form
(i.e.\ $x\betatoo z$ and $y\betatoo z$ for some $z$).\footnote{This also makes
clear that ``$\≡$'' is an equivalence relation.}

\end{document}
