\documentclass[a5paper]{article}
% Theorems, proofs, and other mathematical environments

\usepackage{amsthm}

\newtheoremstyle{mine} % name
  {7pt} % space above
  {7pt} % space below
  {} % body font
  {} % indent amount
  {\bfseries} % theorem head font
  {.} % punctuation after theorem head
  {.5em} % space after theorem head
  {} % theorem head spec (can be left empty, meaning no)

\theoremstyle{mine} % these are the non-italicized
\newtheorem{theorem}{Theorem}[subsection]
\newtheorem*{theorem*}{Theorem}
\newtheorem{lemma}[theorem]{Lemma}
\newtheorem*{lemma*}{Lemma}
\newtheorem{proposition}[theorem]{Proposition}
\newtheorem*{proposition*}{Proposition}
\newtheorem{corollary}[theorem]{Corollary}
\newtheorem*{corollary*}{Corollary}
\newtheorem{eqenv}[theorem]{Equation}
\newtheorem*{eqenv*}{Equation}
\newtheorem{remark}[theorem]{Remark}
\newtheorem*{remark*}{Remark}
\newtheorem{example}[theorem]{Example}
\newtheorem*{example*}{Example}
\newtheorem{definition}[theorem]{Definition}
\newtheorem*{definition*}{Definition}
% New environments. Syntax: \newenvironment{beforecommands}{aftercommands}
% environment for notecards. Portable accross tex files
\newenvironment{card}{\begin{framed}\begin{minipage}[t][3in][t]{5in}\noindent}{\end{minipage}\end{framed}}
\newenvironment{note}{\paragraph{Note:}}{}

% for the Anki notecard style
\newenvironment{field}{}{}
\newenvironment{conclusion}{}{}
\newenvironment{premises}{\begin{enumerate}[label=\alph*.]}{\end{enumerate}}
%% formatting

\usepackage{fontspec}
% EB Garamond (Initials - just for fun) - body text, old style, serif
% Nimbus Roman - serifed
% URW Gothic - geometric, title text
% Libre Caslon Text - absurdly well spaced
% FreeSerif - plays pretty nicely with math, looks like Times
% Tinos - good alternative to FreeSerif, less TNR looking
% \setmainfont[Ligatures=TeX]{Tinos}
% \fontspec [Path = ../tex-preamble/fonts/, 
%            UprightFont = *-regular,
%            BoldFont    = *-bold,
%            ItalicFont  = *-italic]
%            {texgyrepagella}
\setmainfont[ Path=../tex-preamble/fonts/
            , UprightFont     = *-regular
            , BoldFont        = *-bold
            , ItalicFont      = *-italic
            , BoldItalicFont  = *-bolditalic
            ]{texgyrepagella}

\usepackage{datetime2} % use yyyy--mm--dd date with \DTMtoday
\usepackage{geometry}
\geometry{
  lmargin=4cm,
  rmargin=4cm,
  tmargin=3cm,
  bmargin=3cm,
}

\renewcommand{\arraystretch}{1.2} % Make tables a little bigger
% \usepackage{indentfirst}          % indent first \par after sections
% \usepackage{parskip}              % don't indent anything
\usepackage{enumitem}             % enumerate with A), b., c) styles
\usepackage{hyperref}             % linked table of contents
\hypersetup{
    colorlinks,
    citecolor=black,
    filecolor=black,
    linkcolor=black,
    urlcolor=black
}
\usepackage[noabbrev, capitalize]{cleveref}
\crefname{notation}{Notation}{Notations}
\Crefname{notation}{Notation}{Notations}
\crefname{rule}{Rule}{Rule}
\Crefname{rule}{Rule}{Rule}

% graphics
\usepackage{graphicx}
\DeclareGraphicsExtensions{.pdf,.png,.jpg}
\usepackage{float} % image positioning
%% math symbols, notation, etc
\usepackage{amssymb}       % math characters
\usepackage{mathtools}     % improvement on amsmath

% Use better math fonts when available
\IfFileExists{../tex-preamble/fonts/latinmodern-math.otf}
  {\newcommand{\fontpath}{../tex-preamble/fonts}}
  {\newcommand{\fontpath}{../../tex-preamble/fonts}}

\IfFileExists{\fontpath}{
  \usepackage{unicode-math}         % use OTF math fonts
  \AtBeginDocument{\let\phi\varphi} % varphi with unicode-math
  %\AtBeginDocument{\let\epsilon\varepsilon}
  % \setmathfont[Path=\fontpath]{latinmodern-math.otf}
  \setmathfont[Path=\fontpath]{Asana-Math.otf}
  \setmathfont[Path=\fontpath,range=\setminus]{Asana-Math.otf}
  \setmathfont[Path=\fontpath,range=\mathbb]{texgyretermes-math.otf}
  \setmathfont[Path=\fontpath,range=\mathcal]{latinmodern-math.otf}

  % fix xmapsto. only works with text above.
  % https://github.com/wspr/unicode-math/issues/197
  \ExplSyntaxOn
  \RenewDocumentCommand \xmapsto { m } {
    \mathrel {
      \exp_after:wN \Uoverdelimiter \cs:w sym \__um_symfont_tl \cs_end: "21A6 {
        \mkern 5mu \scan_stop: #1 \mkern 5mu \scan_stop:
      }
    }
  }
  \ExplSyntaxOff
}{}

% general stuff
\newcommand{\units}[1]{\;\mathrm{#1}}
\newcommand{\dd}{\,\mathrm{d}}
\newcommand{\ov}{\overline}
\newcommand{\paren}[1]{\left(#1\right)}
\newcommand{\braces}[1]{\left\{#1\right\}}
\newcommand{\brackets}[1]{\left[#1\right]}
\newcommand{\ceiling}[1]{\left\lceil{} #1 \right\rceil}
\newcommand{\floor}[1]{\left\lfloor{} #1 \right\rfloor}
\newcommand{\abs}[1]{\left| #1 \right|}
\newcommand{\e}[1]{\cdot 10^{#1}}
% \newcommand{\emf}{\mathcal{E}}
\newcommand{\contradiction}{\Rightarrow\Leftarrow}
\newcommand{\inv}[1]{\frac{1}{#1}}
\newcommand{\tif}{\text{ if }}
\newcommand{\totherwise}{\text{ otherwise }}
% \newcommand{\mydef}{\textbf{Definition:}}
% complex numbers
\renewcommand{\Im}{\operatorname{Im}}
\renewcommand{\Re}{\operatorname{Re}}

% operators/functions
\DeclareMathOperator{\ord}{ord}
% \DeclareMathOperator{\image}{image}
\DeclareMathOperator{\Int}{Int}
\DeclareMathOperator{\Ext}{Ext}
\DeclareMathOperator{\Lim}{Lim}
\DeclareMathOperator{\Bd}{Bd}
\DeclareMathOperator{\lcm}{lcm}
\DeclareMathOperator{\proj}{proj}
\DeclareMathOperator{\xor}{xor}
\DeclareMathOperator{\spanf}{span}
\DeclareMathOperator{\id}{id}
% linear
\DeclareMathOperator{\col}{Col}
\DeclareMathOperator{\row}{Row}
\DeclareMathOperator{\nul}{Nul}
\DeclareMathOperator{\rank}{rank}

% set theory
\newcommand{\xlongmapsto}[1]{\xmapsto{\,\,\, #1\,\,\,}}
\newcommand{\xlongrightarrow}[1]{\xrightarrow{\,\,\, #1\,\,\,}}
\newcommand{\union}{\mathop{\bigcup}}
\newcommand{\disunion}{\mathop{\dot\bigcup}}
\newcommand{\intersection}{\mathop{\bigcap}}
\newcommand{\powerset}{\mathcal{P}}

% analysis & calculus
\usepackage{esint} % \oiint, etc
\newcommand{\evalat}[2]{{\bigg|_{#1}^{#2}}}
\newcommand{\ext}[1]{{#1}^{ext}}  % extension of function
\DeclareMathOperator{\determinant}{det}
\DeclareMathOperator{\oball}{B}
\DeclareMathOperator{\cball}{\ov{B}}
\DeclareMathOperator{\sphere}{S}
\DeclareMathOperator{\divergence}{div}
\DeclareMathOperator{\curl}{curl}

% vectors
\newcommand{\ihat}{\hat{\imath}}
\newcommand{\jhat}{\hat{\jmath}}
\newcommand{\khat}{\hat{k}}
\newcommand{\lvec}[1]{\overrightarrow{#1}}
\newcommand{\len}[1]{\left\| #1 \right\|}

% topology
\newcommand{\RP}{\ensuremath{\R\hspace{-0.07em}\mathrm{P}}}

% categories
\newcommand{\I}{\ensuremath{\mathbf{I}}} % "interval groupoid"
\newcommand{\Cat}{\ensuremath{\mathcal{C}\mathrm{at}}}
\newcommand{\ComGrp}{\ensuremath{\mathcal{C}\mathrm{om}\mathcal{G}\mathrm{rp}}}
\newcommand{\AbGrp}{\ensuremath{\mathcal{A}\mathrm{b}\mathcal{G}\mathrm{rp}}}
\newcommand{\Grp}{\ensuremath{\mathcal{G}\mathrm{rp}}}
\newcommand{\Grpd}{\ensuremath{\mathcal{G}\mathrm{rpd}}}
\newcommand{\Mod}{\ensuremath{\mathcal{M}\mathrm{od}}}
\newcommand{\Man}{\ensuremath{\mathcal{M}\mathrm{an}}}
\newcommand{\Metric}{\ensuremath{\mathcal{M}\mathrm{etric}}}
\newcommand{\Set}{\ensuremath{\mathcal{S}\mathrm{et}}}
\newcommand{\Top}{\ensuremath{\mathcal{T}\!\mathrm{op}}}
\newcommand{\homoTop}{\ensuremath{\mathrm{homo}\mathcal{T}\mathrm{op}}}
\newcommand{\Vect}{\ensuremath{\mathcal{V}\mathrm{ect}}}
\renewcommand{\phi}{\varphi}

% number theory
\newcommand{\modulus}[1]{\; \left(\mathrm{mod}\; #1\right)}
\newcommand{\crash}[2]{\left(\frac{#1}{#2}\right)}
\newcommand{\QED}{\begin{flushright}QED\end{flushright}}

% make letters in different fonts (bold, blackboard, caligraphic, fraktur)
% https://tex.stackexchange.com/questions/156196/newcommand-with-variable-name
\newcommand{\mkltr}[1]{%
    \expandafter\newcommand\csname bf#1\endcsname{\ensuremath{\mathbb #1}}
    \expandafter\newcommand\csname bb#1\endcsname{\ensuremath{\mathbb #1}}
    \expandafter\newcommand\csname ca#1\endcsname{\ensuremath{\mathcal #1}}
    \expandafter\newcommand\csname fr#1\endcsname{\ensuremath{\mathfrak #1}}
}%
\mkltr A \mkltr B \mkltr C \mkltr D \mkltr E \mkltr F \mkltr G \mkltr H
\mkltr I \mkltr J \mkltr K \mkltr L \mkltr M \mkltr N \mkltr O \mkltr P
\mkltr Q \mkltr R \mkltr S \mkltr T \mkltr U \mkltr V \mkltr W \mkltr X
\mkltr Y \mkltr Z

\newcommand{\zn}{\mathbf{Z}_n}
\newcommand{\zp}{\mathbf{Z}_p}
\newcommand{\N}{\ensuremath{\mathbf{N}}}
\newcommand{\Z}{\ensuremath{\mathbf{Z}}}
\newcommand{\Q}{\ensuremath{\mathbf{Q}}}
\newcommand{\R}{\ensuremath{\mathbf{R}}}
\newcommand{\C}{\ensuremath{\mathbf{C}}}

% algebra
\newcommand{\quotient}[2]{{\raisebox{.2em}{$#1$}\left/\raisebox{-.2em}{$#2$}\right.}}
\newcommand{\GL}{\mathrm{GL}}     % general linear group
\newcommand{\SL}{\mathrm{SL}}     % special linear group
\renewcommand{\S}{\mathfrak{S}}   % symmetric group
\DeclareMathOperator{\ann}{ann}   % the annihilator of a submodule
\DeclareMathOperator{\Obj}{Obj}   % objects of a category
\DeclareMathOperator{\Hom}{Hom}   % homomorphisms between modules
\DeclareMathOperator{\Mor}{Mor}   % categorical morphisms
\DeclareMathOperator{\characteristic}{char} % ring characteristic
\DeclareMathOperator{\Tor}{Tor}   % torsion elements
\DeclareMathOperator{\Stab}{Stab} % stabilizer
\DeclareMathOperator{\Orb}{Orb}   % orbit
\DeclareMathOperator{\Aut}{Aut}   % automorphisms
\DeclareMathOperator{\End}{End}   % endomorphisms
\DeclareMathOperator{\Mat}{Mat}   % matrix ring
\DeclareMathOperator{\Tr}{Trace}  % trace of a matrix
\DeclareMathOperator{\im}{im}     % image

% statistics
\DeclareMathOperator{\Var}{Var}

% nicer empty set
\let\oldemptyset\emptyset{}
\let\emptyset\varnothing{}
% set up our custom colors - these are based on "tao-yang"
\usepackage{color}
\usepackage[dvipsnames]{xcolor}
% text and background colors
\definecolor{bg}{HTML}{F1F1F1}
\definecolor{fg}{HTML}{090909}
\pagecolor{bg}
\color{fg}
\geometry{
  lmargin=1.5cm,
  rmargin=1.5cm,
  tmargin=2cm,
  bmargin=2cm
}
\usepackage{tikz}
\usetikzlibrary{cd}
\begin{document}
\title{The Fundamental Groupoid}
\author{Langston Barrett}
\date{Fall 2017}
\maketitle

\begin{definition*}[Model]
	A model is a functor which is a sort of ``homomorphism of categories''. 
\end{definition*}


\begin{definition*}[Path]
	A path in a space $(X,\tau)$ of length $r$ is a continuous function
  $a:[0,r]\to X$. 
\end{definition*}


\begin{definition*}[Category of Paths]
	For a topological space $X$, the category $PX$ of paths on $X$ has as objects
  the points of $X$ and for $x,y\in X$, $PX(x,y)$ is the set of paths from $x$
  to $y$. Composition of paths $p,q$ is written $p+q$. There is a zero path
  $0_x$ which serves as the identity.
\end{definition*}


\begin{definition*}[Full subcategory, wide subcategory]
	A subcategory $D$ of $C$ is full if for all objects $x,y$ of $D$,
  $\Hom_D(x,y)=\Hom_C(x,y)$. A subcategory is called wide if $\Obj(D)=\Obj(C)$.
\end{definition*}


\begin{example*}
	Full subcategories of $\Top$ include Hausdorff spaces, metrizable spaces, and
  compact spaces. Wide subcategories of $\Top$ include those with only open
  maps, only closed maps, or only isomorphisms.
\end{example*}


\begin{definition*}[Groupoid]
	A groupoid is a category in which every morphism is an isomorphism.
\end{definition*}


\begin{example*}
	The category of paths $PX$ is not a groupoid: if $g\in\Hom_{PX}(x,y)$ is not
  the zero path, then there is no $f$ such that $fg=0_x$.
\end{example*}


\begin{definition*}[Homotopy rel endpoints]
	If $a,b\in\Hom_{PX}(x,y)$ are paths of length $r$, then a homotopy rel
  endpoints of length 1 from $a$ to $b$ is a continuous map
  $H:[0,q]\times[0,r]$ such that
  \begin{align*}
    \begin{split}
      H(0,t) &= a(s) \\
      H(1,t) &= b(s)
    \end{split}
    \begin{split}
      H(s,0) &= x \\
      H(s,1) &= y
    \end{split}
  \end{align*}
  for all $s\in [0,r]$ and $t\in [0,1]$.
\end{definition*}


Note that homotopy rel endpoints defines an equivalence relation on
$\Hom_{PX}(x,y)$, and that for every homotopy of length other than 1, there is
also a homotopy of length 1.


\begin{definition*}[The fundamental groupoid]
	The fundamental groupoid $\Pi(X)$ of a space $(X,\tau)$ is the category with
  points of $X$ as objects and equivalence classes under homotopy of paths from
  $x\in X$ to $y\in X$ as morphisms $\Hom_{\Pi(X)}(x,y)$. Composition is 
  given by path concatenation.
\end{definition*}


\begin{example*}
	If the connected components of $X$ consist of single points, then
  \begin{equation*}
    \Hom_{\Pi(X)}(x,y) = \begin{cases}
      \emptyset &\text{ if }x\neq y \\
      \braces{\id_x} &\text{ if }x=y
    \end{cases}
  \end{equation*}
  A groupoid with this property is called discrete.

  For any convex subset $X$ of a normed vector space $V$, the straight line
  homotopy gives an equivalence between all paths in $PX(x,y)$. Therefore,
  $\Hom_{\Pi(X)}(x,y)$ has exactly one object for all $x,y\in X$. A groupoid
  with this property is called 1-connected and a tree groupoid. If $\Pi(X)$ is a
  tree groupoid, then $X$ is path-connected. If every path-component of $X$ is
  1-connected, then $X$ is simply connected.
\end{example*}


\begin{definition}[Subgroupoid]
	A subgroupoid is a subcategory of a groupoid that is also a groupoid.
\end{definition}


If $G$ is a groupoid and $x$ is an object of $G$, there is a full subgroupoid
$G(x)$ which has $x$ as its only object. This is just a group (called the object
or vertex group)!


\begin{definition}[Totally disconnected groupoid]
	A groupoid $G$ is totally disconnected if
  \begin{equation*}
    \Hom_G(x,y)=\emptyset\qquad\text{when}\qquad x\neq y
  \end{equation*}
\end{definition}


\begin{theorem*}
	If $G$ is a groupoid, all initial objects are terminal, and all terminal
  objects are initial. If $G$ has an initial or terminal object, all
  objects in $G$ are isomorphic.
\end{theorem*}
\begin{proof}
  Let $t\in\Obj(G)$ be terminal. Then for all $g\in\Obj(G)$, there is a unique arrow in $\Hom_G(g,t)$. Since $G$ is a groupoid, there is a bijection
  $\Hom_G(g,t)\cong\Hom_G(t,g)$ between arrows and their inverses. Therefore,
  there is also a single unique arrow in $\Hom_G(t,g)$, making $t$ initial in $G$.
\end{proof}


\begin{definition*}[Groupoid morphism]
	A groupoid morphism is a functor between groupoids.
\end{definition*}


\begin{proposition*}
  Functor preserve retractions, sections, and isomorphisms.
\end{proposition*}


\begin{proposition*}
  For all spaces $X,Y\in\Obj(\Top)$,
  \begin{itemize}
    \item continuous maps $X\to Y$ between spaces induce functors $PX\to PY$
      between path categories (via composition).
    \item $P$ is a functor $P:\Top\to\Cat$
    \item There is a functor $PX\to \Pi(X)$ which sends maps to their homotopy
      equivalence classes
    \item $\Pi_1$ is a functor $\Top\to\Grpd$
  \end{itemize}
\end{proposition*}


\begin{corollary*}
  If $X\cong Y$ as spaces, then $\Pi_1(X)\cong\Pi_1(X)$ as groupoids.
\end{corollary*}


\begin{proposition*}
  The fundamental group is a functor from pointed topological spaces $\Top_*$ to
  $\Grp$.
\end{proposition*}


\begin{definition*}[$\Grpd$]
  The category $\Grpd$ has groupoids as objects and groupoid morphisms as
  morphisms. 
\end{definition*}


\begin{definition*}[Cartesian closed category]
  A category $C$ is Cartesian closed if
  \begin{itemize}
    \itemsep0em
    \item It has a terminal object
    \item It has all finite products
    \item Any two objects $Y,Z\in\Obj(C)$ have an exponential $Z^Y$
  \end{itemize}
  where in a locally small category, the third condition is equivalent to
  requiring a bijection
  \begin{equation*}
    \Hom_C(X\times Y, Z)\cong \Hom_C(X,Z^Y)
  \end{equation*}
\end{definition*}


\begin{proposition*}
	The category $\Grpd$ of groupoids is cartesian closed.
\end{proposition*}


\begin{definition*}
	The groupoid $\I$ has two objects (denoted $0,1$) and one (iso)morphism
  $\iota:0\leftrightarrow 1$.
\end{definition*}


\begin{proposition*}
	If $X,Y$ are topological spaces, then
  \begin{align*}
    \Pi_1(X\times Y)\cong \Pi_1(X)\times\Pi_1(Y)
    &&\text{and}&&
    \Pi_1(X\amalg Y)\cong \Pi_1(X)\amalg\Pi_1(Y)
  \end{align*}
  (naturally) as groupoids.
\end{proposition*}


\begin{definition*}
	A homotopy of length $q$ is a map $F:X\times[0,q]\to Y$. The initial map of
  $F$ is $F_0(x)\coloneqq F(x,0)$ and the final map is $F_q(x)\coloneqq F(x,q)$.
  We write $F:F_0\simeq F_q$.
\end{definition*}


\begin{proposition*}
	Homotopy is an equivalence relation on maps $X\to Y$.
\end{proposition*}


\begin{definition*}
	If $f:X\to Y$ and $g:Y\to X$ are continuous with $f\circ g\simeq \id_Y$ and
  $g\circ f\simeq \id_X$, then we say $f$ and $g$ are homotopy inverses and $X$
  and $Y$ are homotopy equivalent (or of the same homotopy type).
\end{definition*}


\begin{definition*}
	A space is contractible if it is homotopy equivalent to a point. Equivalently,
  a space is contractible if its identity function is homotopic to a constant map.
\end{definition*}


\begin{proposition*}
	If $f,g:X\to Y$ are homotopic continuous functions, then the induced functors
  of fundamental groupoids $\Pi_1(f),\Pi_1(g):\Pi_1(X)\to\Pi_1(Y)$ are
  equivalent, i.e.\ there is a natural transformation $\eta:F\Rightarrow G$ such
  that the following diagram commutes for all $x,y\in \Pi_1(X)$ and
  $h\in\Hom_{\Pi_1(X)}(x,y)$:
  \begin{center}
    \begin{tikzcd}[sep=large]
      \Pi_1(f)(x)
        \arrow[r, "\Pi_1(f)(h)"]
        \arrow[d, <->, "\eta_x", "\eta_x^{-1}" swap] &
      \Pi_1(f)(y) 
        \arrow[d, <->, "\eta_y", "\eta_y^{-1}" swap] \\
      \Pi_1(g)(x)
        \arrow[r, "\Pi_1(g)(h)"] &
      \Pi_1(g)(y)
    \end{tikzcd}
  \end{center}
\end{proposition*}


\begin{corollary*}
	If $f:X\to Y$ is a homotopy equivalence, then $\Pi_1(X)$  and $\Pi_1(Y)$ are
  equivalent as groupoids.
\end{corollary*}


\begin{proposition*}
	If $F:C\to D$ is an equivalence of categories, then
  $F:\Hom_C(x,y)\to \Hom_D(F(x),F(y))$ is a bijection for all $x,y\in C$.
\end{proposition*}

\end{document}