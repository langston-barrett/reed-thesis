% requires logic.tex
% some of this is based on HoTT/HoTT

\newcommand{\ttfun}[1]{\ensuremath{\mathsf{#1}}}  % type-theoretic functions

% recurors
\newcommand{\case}{\ttfun{case}}
\newcommand{\rec}{\ttfun{rec}}

\newcommand{\emptytype}{\ensuremath{\mathbf{0}}}  % empty type
\newcommand{\unittype}{\ensuremath{\mathbf{1}}}   % unit type
\newcommand{\unitelem}{\ensuremath{1_{\unittype}}} % sole inhabitant of the unit type
\newcommand{\booltype}{\ensuremath{\mathbf{2}}}   % two element type
\newcommand{\btrue}{1_{\booltype}}                 % inhabitant of two element type
\newcommand{\bfalse}{0_{\booltype}}                % inhabitant of two element type
\newcommand{\inl}{\ttfun{inl}}
\newcommand{\inr}{\ttfun{inr}}
\newcommand{\universe}{\ensuremath{\mathcal{U}}}         % universe type
\newcommand{\universei}[1]{\ensuremath{\mathcal{U}_{#1}}}  % ith universe

\newcommand{\sigmat}[2]{\ensuremath{\sum_{\paren*{#1}}#2}}    % Σ-type
\newcommand{\∑}{\sigmat}                                     % Σ-type
\newcommand{\dpair}[2]{\ensuremath{\paren{#1\hspace{0.1em};#2}}} % Σ-pair
\newcommand{\pr}[1]{\ttfun{pr}_{#1}}
\newcommand{\pit}[2]{\ensuremath{\prod_{\paren*{#1}}#2}}      % Π-type
\newcommand{\∏}{\pit}      % Π-type
\newcommand{\refl}[1]{\ttfun{refl}_{#1}}      % reflexivity

%%% Natural numbers
\newcommand{\suc}{\mathsf{succ}}
\newcommand{\add}{\mathsf{add}}

%%% Lists
\newcommand{\List}[1]{\ttfun{List}(#1)}
\newcommand{\nil}{\ttfun{nil}}
\newcommand{\cons}{\ttfun{cons}}

%%% Function extensionality
\newcommand{\funext}{\mathsf{funext}}
\newcommand{\happly}{\mathsf{happly}}

\newcommand{\isContr}{\ttfun{isContr}}

%%% Relations
\newcommand{\jdeq}{\equiv}
\newcommand{\defeq}{\vcentcolon\jdeq}
\newcommand{\propeq}[3]{#2 =_{#1} #3}
\newcommand{\weq}[2]{#1 \simeq #2}
\newcommand{\homot}[2]{#1 \sim #2}

\newcommand{\appr}[2]{\apply{\pr{#1}}{#2}} % a combination of apply and pr

\newcommand{\ap}[2]{\mathsf{ap}_{#1}{\paren{#2}}} % action on paths
\newcommand{\ua}{\mathsf{ua}} % univalence axiom

%%% Transport (covariant) %%%
% \newcommand{\trans}[2]{\ensuremath{{#1}_{*}\mathopen{}\left({#2}\right)\mathclose{}}\xspace}
% \let\Trans\trans
% \newcommand{\transf}[1]{\ensuremath{{#1}_{*}}\xspace} % Without argument
\newcommand{\transpor}[2]{\ttfun{transport}^{#1}\paren{#2}}
\newcommand{\transport}[3]{\ttfun{transport}^{#1}\paren{#2,#3}}
\newcommand{\transportname}{\ttfun{transport}}