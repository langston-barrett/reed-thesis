%% math symbols, notation, etc
\usepackage{amssymb}       % math characters
\usepackage{mathtools}     % improvement on amsmath

\usepackage{unicode-math}         % use OTF math fonts
\setoperatorfont\mathup % operators should be set in math font
\AtBeginDocument{\let\setminus\minus}

% \ExplSyntaxOn % fix xmapsto. https://github.com/wspr/unicode-math/issues/197
% \RenewDocumentCommand \xmapsto { m } {
%   \mathrel {
%     \exp_after:wN \Uoverdelimiter \cs:w sym \__um_symfont_tl \cs_end: "21A6 {
%       \mkern 5mu \scan_stop: #1 \mkern 5mu \scan_stop:
%     }
%   }
% }
% \ExplSyntaxOff

% Tex Gyre Schola Math -  unpretentious, serif (good for \mathbb)
% Latin Modern Math -  update of Computer Modern, a little thicker (good for \mathcal)
% Asana Math -  the only font with the correct \setminus
\setmathfont{texgyrepagella-math.otf}
% \setmathfont{TeX Gyre Pagella Math}
% \setmathfont[range=\mathcal]{Latin Modern Math}

% A good, casual math font:
% \usepackage{mathpazo}

% alternative characters
\let\epsilon\varepsilon
\let\phi\varphi
\let\oldemptyset\emptyset{}
\let\emptyset\varnothing{}

% paired delimiters
\DeclarePairedDelimiterX{\Abs}[1]{\lvert}{\rvert}{#1}
\DeclarePairedDelimiterX{\Angle}[1]{\langle}{\rangle}{#1}
\DeclarePairedDelimiterX{\Braces}[1]{\{}{\}}{#1}
\DeclarePairedDelimiterX{\Brackets}[1]{[}{]}{#1}
\DeclarePairedDelimiterX{\Ceiling}[1]{\lceil}{\rceil}{#1}
\DeclarePairedDelimiterX{\Floor}[1]{\lfloor}{\rfloor}{#1}
\DeclarePairedDelimiterX{\Norm}[1]{\lVert}{\rVert}{#1}
\DeclarePairedDelimiterX{\Paren}[1]{\lparen}{\rparen}{#1}

% swap the starred and unstarred versions of paired delimiters
\makeatletter % https://goo.gl/osSmHV
\def\abs{\@ifstar{\Abs}{\Abs*}}
\def\angles{\@ifstar{\Angle}{\Angle*}}
\def\braces{\@ifstar{\Braces}{\Braces*}}
\def\brackets{\@ifstar{\Brackets}{\Brackets*}}
\def\ceiling{\@ifstar{\Ceiling}{\Ceiling*}}
\def\floor{\@ifstar{\Floor}{\Floor*}}
\def\norm{\@ifstar{\Norm}{\Norm*}}
\def\paren{\@ifstar{\Paren}{\Paren*}}
\makeatother

% operators/functions
% \DeclareMathOperator{\Im}{Im}        % complex numbers: real part
% \DeclareMathOperator{\Re}{Re}        % complex numbers: imaginary part
\DeclareMathOperator{\Var}{Var}         % statistics: variance
\DeclareMathOperator{\ord}{ord}
\DeclareMathOperator{\image}{image}     % general: image
\DeclareMathOperator{\id}{id}           % general: identity function
\DeclareMathOperator{\lcm}{lcm}         % general: least common multiple
\DeclareMathOperator{\xor}{xor}         % general: exclusive or
\DeclareMathOperator{\Int}{Int}         % topology: interior
\DeclareMathOperator{\Ext}{Ext}         % topology: exterior
\DeclareMathOperator{\Bd}{Bd}           % topology: boundary
\DeclareMathOperator{\oball}{B}         % topology: open ball
\DeclareMathOperator{\cball}{\ov{B}}    % topology: closed ball
\DeclareMathOperator{\sphere}{S}        % topology: sphere
\DeclareMathOperator{\proj}{proj}
\DeclareMathOperator{\spanf}{span}      % linear: span
\DeclareMathOperator{\col}{Col}         % linear: column space
\DeclareMathOperator{\row}{Row}         % linear: row space
\DeclareMathOperator{\nul}{Nul}         % linear: null space
\DeclareMathOperator{\rank}{rank}       % linear: rank
\DeclareMathOperator{\determinant}{det} % linear: determinant
\DeclareMathOperator{\Alt}{Alt}         % linear: alternating group/square
\DeclareMathOperator{\Sym}{Sym}         % linear: symmetric square
\DeclareMathOperator{\Tr}{Trace}        % linear: trace of a matrix
\DeclareMathOperator{\divergence}{div}  % multi: divergence
\DeclareMathOperator{\curl}{curl}       % multi: curl
\DeclareMathOperator{\characteristic}{char} % algebra: ring/field characteristic
\DeclareMathOperator{\ann}{ann}         % algebra: the annihilator of a submodule
\DeclareMathOperator{\GL}{GL}           % algebra: general linear group
\DeclareMathOperator{\SL}{SL}           % algebra: special linear group
\DeclareMathOperator{\GF}{GF}           % algebra: galois field
\DeclareMathOperator{\Tor}{Tor}         % algebra: torsion elements/functor
\DeclareMathOperator{\Stab}{Stab}       % algebra: stabilizer
\DeclareMathOperator{\Orb}{Orb}         % algebra: orbit
\DeclareMathOperator{\Mat}{Mat}         % algebra: matrix ring
\DeclareMathOperator{\Res}{Res}         % algebra: restriction of a representation
\DeclareMathOperator{\Ind}{Ind}         % algebra: induced representation
\DeclareMathOperator{\im}{im}           % algebra: image
\DeclareMathOperator{\Aut}{Aut}         % algebra/cat: automorphisms
\DeclareMathOperator{\End}{End}         % algebra/cat: endomorphisms

\newcommand{\DeclareCategory}[2]{\DeclareMathOperator{#1}{\mathbf{#2}}}
\DeclareMathOperator{\Obj}{Obj}   % categories: objects of a category
\DeclareMathOperator{\Hom}{Hom}   % categories: hom-sets
\DeclareCategory{\Set}{Set}       % categories: (small) sets
\DeclareCategory{\FinSet}{FinSet} % categories: finite sets
\DeclareCategory{\Cat}{Cat}       % categories: small categories
\DeclareCategory{\Grpd}{Grpd}     % categories: small groupoids
\DeclareCategory{\Mon}{Mon}       % categories: monoids
\DeclareCategory{\Grp}{Grp}       % categories: groups
\DeclareCategory{\AbGrp}{AbGrp}   % categories: abelian groups
\DeclareCategory{\Top}{Top}       % categories: topological spaces
\DeclareCategory{\Man}{Man}       % categories: topological manifolds
\DeclareCategory{\Mod}{-Mod}      % categories: modules over a ring
\DeclareCategory{\Met}{Metric}    % categories: metric spaces
\DeclareCategory{\Vect}{Vect}     % categories: vector spaces
\DeclareCategory{\dVect}{-Vect}   % categories: vector spaces (with dash)
\DeclareCategory{\Rep}{Rep}       % categories: representations
\DeclareCategory{\Alg}{-Alg}      % categories: functor algebras
\DeclareCategory{\Coalg}{\kern-1pt-Coalg} % categories: functor algebras
\DeclareCategory{\omegacat}{ω} % categories: functor algebras

% make letters in different fonts (bold, blackboard, caligraphic, fraktur)
% https://tex.stackexchange.com/questions/156196/newcommand-with-variable-name
\newcommand{\mkltr}[1]{%
    \expandafter\newcommand\csname bf#1\endcsname{\ensuremath{\mathbf #1}}
    \expandafter\newcommand\csname bb#1\endcsname{\ensuremath{\mathbb #1}}
    \expandafter\newcommand\csname ca#1\endcsname{\ensuremath{\mathcal #1}}
    \expandafter\newcommand\csname fr#1\endcsname{\ensuremath{\mathfrak #1}}
}%
\mkltr A \mkltr B \mkltr C \mkltr D \mkltr E \mkltr F \mkltr G \mkltr H
\mkltr I \mkltr J \mkltr K \mkltr L \mkltr M \mkltr N \mkltr O \mkltr P
\mkltr Q \mkltr R \mkltr S \mkltr T \mkltr U \mkltr V \mkltr W \mkltr X
\mkltr Y \mkltr Z

\newcommand{\N}{\ensuremath{\mathbb{N}}}
\newcommand{\Z}{\ensuremath{\mathbb{Z}}}
\newcommand{\Q}{\ensuremath{\mathbb{Q}}}
\newcommand{\R}{\ensuremath{\mathbb{R}}}
\newcommand{\C}{\ensuremath{\mathbb{C}}}

% general stuff
\newcommand{\units}[1]{\;\text{#1}}
\newcommand{\ov}{\overline}

% set theory
\newcommand{\xlongmapsto}[1]{\xmapsto{\,\,\, #1\,\,\,}}
\newcommand{\xlongrightarrow}[1]{\xrightarrow{\,\,\, #1\,\,\,}}
\newcommand{\union}{\mathop{\bigcup}}
\newcommand{\disunion}{\mathop{\dot\bigcup}}
\newcommand{\intersection}{\mathop{\bigcap}}
\newcommand{\powerset}{\mathcal{P}}

% analysis & calculus
\usepackage{esint}                   % \oiint, etc
\newcommand{\dd}{\,\mathrm{d}}       % differential operator (Leibniz notation)
\newcommand{\evalat}[2]{{\bigg|_{#1}^{#2}}}
\newcommand{\ext}[1]{{#1}^{ext}}     % extension of function

% vectors
\newcommand{\ihat}{\hat{\imath}}
\newcommand{\jhat}{\hat{\jmath}}
\newcommand{\khat}{\hat{k}}
\newcommand{\lvec}[1]{\overrightarrow{#1}}

% topology
\newcommand{\RP}{\ensuremath{\R\hspace{-0.07em}\mathrm{P}}}

% algebra
\newcommand{\quotient}[2]{{\raisebox{.2em}{$#1$}\left/\raisebox{-.2em}{$#2$}\right.}}
\renewcommand{\S}{\mathfrak{S}}   % symmetric group
\newcommand{\ab}{\mathbf{ab}}     % algebra: abelianization
\newcommand{\op}{\mathsf{op}}     % algebra: opposite

% statistics
% https://tex.stackexchange.com/questions/229023/expectation-operator#229029
% expectation and covariance
\DeclareMathOperator{\ExpOp}{Exp}
\DeclareMathOperator{\CovOp}{Cov}
\newcommand{\Exp}{\ExpOp\brackets*}
\newcommand{\Cov}{\CovOp\brackets*}
